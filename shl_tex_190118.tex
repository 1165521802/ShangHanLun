\documentclass[11pt,oneside,b5paper]{ctexbook}
\usepackage{geometry}
\usepackage{fontspec}
\usepackage{xeCJK}
\usepackage{graphicx}
%\geometry{left=1.2cm,right=1.2cm,top=1.2cm,bottom=1.2cm}
\geometry{b5paper,centering,scale=0.9}
%\setlength{\parskip}{6pt plus 1pt minus 1pt}
\setmainfont[Path=ttf/]{TH-Tshyn-P0}
\setCJKmainfont[Path=ttf/]{TH-Tshyn-P0}
\setCJKsansfont[Path=ttf/]{TH-Tshyn-P0}
\setCJKmonofont[Path=ttf/]{TH-Tshyn-P0}
\title{傷寒雜病論彙校本}
\author{張仲景}
\date{\today}
\begin{document}
\maketitle
\tableofcontents
\begin{flushleft}
\part{傷寒}

\chapter{辨脉法}

问曰。脉有陰陽。何谓也。\\
答曰。凡脉大。浮。數。動。滑。此名陽也。脉沈。{\CJKfontspec[Path=ttf/]{TH-Tshyn-P2}𬈧}。弱。弦。微。此名陰也。凡陰病見陽脉者生。陽病見陰脉者死。

问曰。脉有陽结陰结者。何以别之。\\
答曰。其脉[自]浮而數。能食。不大便。名曰陽结也。期十七日当劇。其脉[自]沈而遲。不能食。身体重。大便反堅。名曰陰结。期十四日当劇。
\footnote{宋本「不大便」下有「此为実」三字。}

问曰。病有洒淅惡寒。而復發熱者。何。\\
答曰。陰脉不足。陽往從之。陽脉不足。陰往乘之。\\
问曰。何谓陽不足。\\
答曰。假令寸口脉微。为陽不足。陰气上入陽中則洒淅惡寒。\\
问曰。何谓陰不足。\\
答曰。尺脉弱为陰不足。陽气下陷入陰中則發熱。

陽脉浮。陰脉弱。則血虛。血虛則筋急也。
\footnote{「筋急」敦煌甲本作「傷筋」,脉{\hbox{\scalebox{0.68}[1]{纟}\kern-0.35em\scalebox{0.64}[1]{巠}}}作「筋傷」。\\脉陽浮陰濡而弱。弱則血虛。血虛則傷筋。(敦煌甲)}

其脉沈者。榮气微也。

其脉浮而汗出如流珠者。衛气衰也。
\footnote{其脉浮。則汗出如流珠。衛气衰。(敦煌甲)}

榮气微者。加燒针則血留不行。更發熱而躁煩也。

脉靄靄如车盖者。名曰陽结。

脉累累如順長竿者。名曰陰结。

脉聶聶如吹榆莢者。名曰散。
\footnote{「散」敦煌甲本作「數」。}

脉潎潎如羹上肥者。陽气微。
\footnote{「微」玉函作「脱」。}

脉縈縈如蜘蛛絲者。陽气衰。

脉绵绵如[瀉]漆之绝者。亡其血。
\footnote{「瀉」字敦煌甲本无。}

脉來缓。時一止復來者。名曰结。脉來數。時一止復來者。名曰促。脉陽盛則促。陰盛則结。此皆病脉。
\footnote{「陰盛則结」敦煌甲本「结」作「缓」。}

陰陽相摶。名曰動。陽動則汗出。陰動則發熱。形冷惡寒者。三焦傷也。
\footnote{「三焦傷也」敦煌甲本作「此为進」。}

數脉見于{\CJKfontspec[Path=ttf/]{TH-Tshyn-P2}𬮦}上。[上下]无頭尾。如豆大。厥厥動搖者。名曰動。
\footnote{敦煌甲本无「上下」二字,「如豆大」作「大如大豆」。}

陽脉浮大而濡。陰脉浮大而濡。陰脉与陽脉同等者。名曰缓。

脉浮而緊者。名曰弦。脉緊者。如轉索无常。弦者。状如弓弦。按之不移。

脉弦而大。弦則为減。大則为芤。減則为寒。芤則为虛。寒虛相摶。此名为革。婦人則半產漏下。男子則亡血失精。

问曰。病有戰而汗出。因得解者。何。\\
答曰。脉浮而緊。按之反芤。此为本虛。故当戰而汗出。其人本虛。是以發戰。以脉浮。故当汗出而解。若脉浮而數。按之不芤。此本不虛。若欲自解。但汗出耳。不發戰也。

问曰。病有不戰而汗出解者。何。\\
答曰。脉大而浮數。故不戰汗出而解。

问曰。病有不戰不汗出而解者。何。\\
答曰。其脉自微。此以曾發汗。或吐。或下。或亡血。内无津液。陰陽自和。必自愈。故不戰不汗出而解。

问曰。傷寒三日。脉浮數而微。病人身涼和者。何。\\
答曰。此为欲解。解以夜半。脉浮而解者。濈然汗出。脉數而解者。必能食。脉微而解者。必大汗出。

问曰。脉病欲知愈未愈者。何以别之。\\
答曰。寸口。{\CJKfontspec[Path=ttf/]{TH-Tshyn-P2}𬮦}上。尺中三處。大小浮沈遲數同等。雖有寒熱不解者。此脉陰陽为和平。雖劇当愈。
\footnote{「數」敦煌甲本作「疾」。}

師曰。立夏得洪大脉。是其本位。其人病身体苦疼重者。須發其汗。若明日身不疼不重者。不須發汗。若汗濈濈自出者。明日便解矣。何以言之。立夏脉洪大。是其時脉。故使然也。四時仿此。

问曰。凡病欲知何時得。何時愈。\\
答曰。假令夜半得病者。明日日中愈。日中得病者。夜半愈。何以言之。日中得病。夜半愈者。以陽得陰則解也。夜半得病。明日日中愈者。以陰得陽則解也。

寸口脉浮为在表。沈为在裏。數为在腑。遲为在臓。假令脉遲。此为在臓。

趺陽脉浮而{\CJKfontspec[Path=ttf/]{TH-Tshyn-P2}𬈧}。少陰脉如{\hbox{\scalebox{0.68}[1]{纟}\kern-0.35em\scalebox{0.64}[1]{巠}}}者。其病在脾。法当下利。何以知之。若脉浮大者。气実血虛也。今趺陽脉浮而{\CJKfontspec[Path=ttf/]{TH-Tshyn-P2}𬈧}。故知脾气不足。胃气虛也。以少陰脉弦而浮纔見。此为調脉。故稱如{\hbox{\scalebox{0.68}[1]{纟}\kern-0.35em\scalebox{0.64}[1]{巠}}}。若反滑而數者。故知当溺膿也。
\footnote{「以少陰脉弦而浮纔見」敦煌甲本作「少陰脉弦沈纔見为調」,聖惠方作「少陰脉弦而沈此为調脉」。}

寸口脉浮而緊。浮則为風。緊則为寒。風則傷衛。寒則傷榮。榮衛俱病。骨節煩疼。当發其汗。

趺陽脉遲而缓。胃气如{\hbox{\scalebox{0.68}[1]{纟}\kern-0.35em\scalebox{0.64}[1]{巠}}}也。趺陽脉浮而數。浮則傷胃。數則動脾。此非本病。醫特下之所为也。榮衛内陷。其數先微。脉反但浮。其人必大便堅。气噫而除。何以言之。本數脉動脾。其數先微。故知脾气不治。大便堅。气噫而除。今脉反浮。其數改微。邪气獨留。心中則饥。邪熱不殺穀。潮熱發渴。數脉当遲缓。脉因前後度數如法。病者則饥。數脉不時。則生惡瘡。

師曰。病人脉微而{\CJKfontspec[Path=ttf/]{TH-Tshyn-P2}𬈧}者。此为醫所病也。大發其汗。又數大下之。其人亡血。病当惡寒。後乃發熱。无休止時。夏月盛熱。欲著複衣。冬月盛寒。欲裸其身。所以然者。陽微則惡寒。陰弱則發熱。此醫發其汗。令陽气微。又大下之。令陰气弱。五月之時。陽气在表。胃中虛冷。以陽气内微。不能勝冷。故欲著複衣。十一月之時。陽气在裏。胃中煩熱。以陰气内弱。不能勝熱。故欲裸其身。又陰脉遲{\CJKfontspec[Path=ttf/]{TH-Tshyn-P2}𬈧}。故知血亡血。
\footnote{「夏月」敦煌甲本、圣惠方作「五月」。}

脉浮而大。心下反堅。有熱。屬臓者。攻之。不令發汗。屬腑者。不令溲數。溲數則便堅。汗多則熱愈。汗少則便難。脉遲尚未可攻。

趺陽脉微{\CJKfontspec[Path=ttf/]{TH-Tshyn-P2}𬈧}。少陰反堅。微則下逆。{\CJKfontspec[Path=ttf/]{TH-Tshyn-P2}𬈧}則躁煩。少陰堅者。便則为難。汗出在頭。穀气为下。便難者。令微溏。不令汗出。甚者遂不得便。煩逆。鼻鳴。上竭下虛。不得復通。
\footnote{此條僅玉函、敦煌甲本有,「令微溏」敦煌甲本作「愈微溏」。}

脉浮而洪。身汗如油。喘而不休。水漿不下。形体不仁。乍靜乍亂。此为命绝。\\
问曰。上脉狀如此。未知何臓先受其災。\\
答曰。若汗出髮润。喘而不休者。肺先绝也。身如煙熏。直視搖頭者。心先绝也。唇吻反青。四肢漐習者。肝先绝也。環口黧黑。柔汗發黄者。脾先绝也。溲便遺失。狂言。目反直視者。腎先绝也。\\
又问。未知何臓陰陽先绝。\\
答曰。若陽气先绝。陰气後竭者。其人死。身色必青。若陰气先绝。陽气後竭者。其人死。身色必赤。腋下温。心下熱。
\footnote{「身汗如油」玉函作「軀汗如油」,聖惠方作「身汗如沾」,敦煌甲本作「軀反如沾」。「喘而不休」敦煌甲本作「濡而不休」。「乍靜乍亂」敦煌甲本作「乍理乍亂」。}

寸口脉浮大。醫反下之。此为大逆。浮則无血。大則为寒。寒气相摶。則为腸鳴。醫乃不知。而反饮冷水。令汗大出。水得寒气。冷必相摶。其人即饐。

趺陽脉浮。浮則为虛。浮虛相摶。故令气饐。言胃气虛竭也。脉滑則噦。此为醫咎。責虛取実。守空迫血。脉浮。鼻口燥者。必衄。
\footnote{「口」宋本作「中」。}

諸脉浮數。当發熱而洒淅惡寒。若有痛處。食饮如常者。畜積有膿。

脉浮而遲。面熱赤而戰愓者。六七日。当汗出而解。反發熱者。差遲。遲为无陽。不能作汗。其身必痒。
\footnote{「面熱赤而戰愓者」敦煌甲本作「面熱而赤戴陽」。}

脉虛者。不可吐下發汗。其面反有熱色者。为欲解。不能汗出。其身必癢。
\footnote{此條僅玉函、敦煌甲本有。}

寸口脉陰陽俱緊者。法当清邪中上。濁邪中下。清邪中上。名曰潔也。濁邪中下。名曰渾也。陰中於邪。必内慄也。表气微虛。裏气不守。故使邪中於陰。陽中於邪。必發熱。頭痛。項強。頸攣。腰痛。脛痠。所谓陽中霧露之气。故曰清邪中上。濁邪中下。陰气为慄。足膝逆冷。便溺妄出。表气微虛。裏气微急。三焦相溷。内外不通。上焦怫鬱。臓气相熏。口烂食齦。中焦不治。胃气上衝。脾气不轉。胃中为濁。榮衛不通。血凝不流。若衛气前通者。小便赤黄。与熱相摶。因熱作使。遊於{\hbox{\scalebox{0.68}[1]{纟}\kern-0.35em\scalebox{0.64}[1]{巠}}}络。出入臓腑。熱气所過。則为癰膿。若陰气前通者。陽气厥微。陰无所使。客气内入。嚏而出之。聲嗢咽塞。寒厥相追。为熱所擁。血凝自下。状如豚肝。陰陽俱厥。脾气孤弱。五液注下。下焦不阖。清便下重。令便數難。脐築湫痛。命將難全。
\footnote{「臓气相熏」敦煌甲本、聖惠方作「臓气相動」。「令便數難」敦煌甲本、聖惠方作「大便數難」。}

脉陰陽俱緊。口中气出。唇口乾燥。踡卧足冷。鼻中涕出。舌上胎滑。勿妄治也。到七日以上。其人微發熱。手足温者。此为欲解。或到八日已上。反大發熱者。此为難治。設惡寒者。必欲嘔也。腹内痛者。必欲利也。
\footnote{「七日以上」宋本、玉函作「七日以來」。}

脉陰陽俱緊。至於吐利。其脉獨不解。緊去人安。此为欲解。若脉遲。至六七日不欲食。此为晚發。水停故也。为未解。食自可者为欲解。

病六七日。手足三部脉皆至。大煩。口噤不能言。其人躁擾者。必欲解也。若脉和。其人大煩。目重。瞼内際黄者。此欲解也。

脉浮而數。浮[即]为風。數[即]为虛。風[即]为熱。虛[即]为寒。風虛相摶。則洒淅惡寒。
\footnote{玉函「洒淅惡寒」後有「而發熱也」。}

趺陽脉浮而微。浮即为虛。微即汗出。
\footnote{此條僅敦煌甲本、玉函有。}

脉浮而滑。浮[則]为陽。滑[則]为実。陽実相摶。其脉數疾。衛气失度。浮滑之脉數疾。發熱汗出者。此为不治。

傷寒。欬逆。上气。其脉散者死。谓其形損故也。
\footnote{脉散。其人形損。傷寒而欬。上气者死。}

\chapter{平脉法}

问曰。脉有三部。陰陽相乘。榮衛血气。在人体躬。呼吸出入。上下於中。因息遊布。津液流通。隨時動作。效象形容。春弦秋浮。冬沈夏洪。察色觀脉。大小不同。一時之间。變无{\hbox{\scalebox{0.68}[1]{纟}\kern-0.35em\scalebox{0.64}[1]{巠}}}常。尺寸参差。或短或長。上下乖错。或存或亡。病輒改易。進退低昂。心迷意惑。動失纪{\hbox{\scalebox{0.6}[1]{纟}\kern-0.3em\scalebox{0.63}[1]{岡}}}。願为具陳。令得分明。\\
師曰。子之所问。道之根源。脉有三部。尺寸及{\CJKfontspec[Path=ttf/]{TH-Tshyn-P2}𬮦}。榮衛流行。不失衡铨。腎沈心洪。肺浮肝弦。此自{\hbox{\scalebox{0.68}[1]{纟}\kern-0.35em\scalebox{0.64}[1]{巠}}}常。不失銖分。出入升降。漏刻周旋。水下二刻。一周循環。当復寸口。虛実見焉。變化相乘。陰陽相干。風則浮虛。寒則牢堅。沈潛水畜。支饮急弦。動則为痛。數則熱煩。設有不應。知變所缘。三部不同。病各異端。太過可怪。不及亦然。邪不空見。中必有姦。審察表裏。三焦别焉。知其所舍。消息診看。料度腑臓。獨見若神。为子條記。傳与賢人。

師曰。呼吸者。脉之頭也。初持脉。來疾去遲。此出疾入遲。名曰内虛外実也。初持脉。來遲去疾。此出遲入疾。名曰内実外虛也。

问曰。上工望而知之。中工问而知之。下工脉而知之。願闻其説。\\
師曰。病家人请云。病人若發熱。身体疼。病人自卧。師到。診其脉。沈而遲者。知其差也。何以知之。表有病者。脉当浮大。今脉反沈遲。故知愈也。

假令病人云。腹内卒痛。病人自坐。師到。脉之。浮而大者。知其差也。何以知之。若裏有病者。脉当沈而细。今脉浮大。故知愈也。

師曰。病家人來请云。病人發熱。煩極。明日師到。病人向壁卧。此熱已去也。設令脉不和。處言已愈。設令向壁卧。闻師到。不驚起而眄視。若三言三止。脉之。嚥唾者。此詐病也。設令脉自和。處言汝病大重。当須服吐下藥。鍼灸數十百處乃愈。

師持脉。病人欠者。无病也。脉之呻者。病也。言遲者。風也。搖頭言者。裏痛也。行遲者。表強也。坐而伏者。短气也。坐而下一腳者。腰痛也。裏実護腹如懷卵物者。心痛也。

師曰。伏气之病。以意候之。今月之内。欲有伏气。假令舊有伏气。当須脉之。若脉微弱者。当喉中痛似傷。非喉痹也。病人云。実咽中痛。雖尔。今復欲下利。

问曰。人病恐怖者。其脉何状。\\
師曰。脉形如循絲累累然。其面白脱色也。

问曰。人不饮。其脉何類。\\
師曰。其脉自{\CJKfontspec[Path=ttf/]{TH-Tshyn-P2}𬈧}。唇口乾燥也。

问曰。人愧者。其脉何類。\\
師曰。脉浮而面色乍白乍赤。

问曰。{\hbox{\scalebox{0.68}[1]{纟}\kern-0.35em\scalebox{0.64}[1]{巠}}}説。脉有三菽。六菽重者。何谓也。\\
師曰。脉。人以指按之。如三菽之重者。肺气也。如六菽之重者。心气也。如九菽之重者。脾气也。如十二菽之重者。肝气也。按之至骨者。腎气也。假令下利。寸口。{\CJKfontspec[Path=ttf/]{TH-Tshyn-P2}𬮦}上。尺中悉不見脉。然尺中時一小見。脉再舉頭者。腎气也。若見損脉來至。为難治。

问曰。脉有相乘。有{\hbox{\scalebox{0.6}[1]{纟}\kern-0.32em\scalebox{0.7}[1]{從}}}有横。有逆有順。何也。\\
師曰。水行乘火。金行乘木。名曰{\hbox{\scalebox{0.6}[1]{纟}\kern-0.32em\scalebox{0.7}[1]{從}}}。火行乘水。木行乘金。名曰横。水行乘金。火行乘木。名曰逆。金行乘水。木行乘火。名曰順也。

问曰。脉有殘賊。何谓也。\\
師曰。脉有弦。緊。浮。滑。沈。{\CJKfontspec[Path=ttf/]{TH-Tshyn-P2}𬈧}。此六者名曰殘賊。能为諸脉作病也。

问曰。脉有災怪。何谓也。\\
師曰。假令人病。脉得太陽。与形證相應。因为作湯。比還送湯如食頃。病人乃大吐。若下利。腹中痛。\\
師曰。我前來不見此證。今乃變異。是名災怪。\\
又问曰。何缘作此吐利。\\
答曰。或有舊時服藥。今乃發作。故名災怪耳。

问曰。東方肝脉。其形何似。\\
師曰。肝者。木也。名厥陰。其脉微弦濡弱而長。是肝脉也。肝病自得濡弱者。愈也。假令得纯弦脉者。死。何以知之。以其脉如弦直。是肝臓傷。故知死也。

问曰。南方心脉。其形何似。\\
師曰。心者。火也。名少陰。其脉洪大而長。是心脉也。心病自得洪大者。愈也。假令脉來微去大。故名反。病在裏也。脉來頭小本大者。故名覆。病在表也。上微頭小者。則汗出。下微本大者。則为{\CJKfontspec[Path=ttf/]{TH-Tshyn-P2}𬮦}格不通。不得溺。頭无汗者可治。有汗者死。

问曰。西方肺脉。其形何似。\\
師曰。肺者。金也。名大陰。其脉毛浮也。肺病自得此脉。若得缓遲者。皆愈。若得數者。則劇。何以知之。數者南方火。火剋西方金。法当癰腫。为難治也。

问曰。二月得毛浮脉。何以處言至秋当死。\\
師曰。二月之時。脉当濡弱。反得毛浮者。故知至秋死。二月肝用事。肝脉屬木。應濡弱。反得毛浮者。是肺脉也。肺屬金。金來剋木。故知至秋死。他皆仿此。

師曰。脉肥人責浮。瘦人責沈。肥人当沈。今反浮。瘦人当浮。今反沈。故責之。

師曰。寸脉下不至{\CJKfontspec[Path=ttf/]{TH-Tshyn-P2}𬮦}。为陽绝。尺脉上不至{\CJKfontspec[Path=ttf/]{TH-Tshyn-P2}𬮦}。为陰绝。此皆不治。決死也。若計其馀命死生之期。期以月節剋之也。

師曰。脉病人不病。名曰行尸。以无王气。卒眩仆不識人者。短命則死。人病脉不病。名曰内虛。以无穀神。雖困无苦。

问曰。翕奄沈。名曰滑。何谓也。\\
師曰。沈为纯陰。翕为正陽。陰陽和合。故令脉滑。{\CJKfontspec[Path=ttf/]{TH-Tshyn-P2}𬮦}尺自平。陽明脉微沈。食饮自可。少陰脉微滑。滑者。緊之浮名也。此为陰実。其人必股内汗出。陰下濕也。

问曰。曾为人所難。緊脉從何而來。\\
師曰。假令亡汗。若吐。以肺裏寒。故令脉緊也。假令欬者。坐饮冷水。故令脉緊也。假令下利。以胃中虛冷。故令脉緊也。

寸口衛气盛。名曰高。榮气盛。名曰章。高章相摶。名曰{\hbox{\scalebox{0.6}[1]{纟}\kern-0.3em\scalebox{0.63}[1]{岡}}}。衛气弱。名曰惵。榮气弱。名曰卑。惵卑相摶。名曰損。衛气和。名曰缓。榮气和。名曰遲。遲缓相摶。名曰沈。

寸口脉缓而遲。缓則陽气長。其色鲜。其顏光。其聲商。毛髮長。遲則陰气盛。骨髓生。血滿。肌肉緊薄鲜堅。陰陽相抱。榮衛俱行。剛柔相摶。名曰強也。

趺陽脉滑而緊。滑者胃气実。緊者脾气強。持実擊強。痛還自傷。以手把刃。坐作瘡也。

寸口脉浮而大。浮为虛。大为実。在尺为{\CJKfontspec[Path=ttf/]{TH-Tshyn-P2}𬮦}。在寸为格。{\CJKfontspec[Path=ttf/]{TH-Tshyn-P2}𬮦}則不得小便。格則吐逆。

趺陽脉伏而{\CJKfontspec[Path=ttf/]{TH-Tshyn-P2}𬈧}。伏則吐逆。水穀不化。{\CJKfontspec[Path=ttf/]{TH-Tshyn-P2}𬈧}則食不得入。名曰{\CJKfontspec[Path=ttf/]{TH-Tshyn-P2}𬮦}格。

脉浮而大。浮为風虛。大为气強。風气相摶。必成癮疹。身体为痒。痒者名泄風。久久为痂癩。

寸口脉弱而遲。弱者衛气微。遲者榮中寒。榮为血。血寒則發熱。衛为气。气微者心内饥。饥而虛滿不能食也。

趺陽脉大而緊者。当即下利。为難治。

寸口脉弱而缓。弱者陽气不足。缓者胃气有馀。噫而吞酸。食卒不下。气填於膈上也。

趺陽脉緊而浮。浮为气。緊为寒。浮为腹滿。緊为绞痛。浮緊相摶。腸鳴而轉。轉即气動。膈气乃下。少陰脉不出。其陰腫大而虛也。

寸口脉微而{\CJKfontspec[Path=ttf/]{TH-Tshyn-P2}𬈧}。微者衛气不行。{\CJKfontspec[Path=ttf/]{TH-Tshyn-P2}𬈧}者榮气不足。榮衛不能相將。三焦无所仰。身体痹不仁。榮气不足。則煩疼。口難言。衛气虛。則惡寒數欠。三焦不歸其部。上焦不歸者。噫而酢吞。中焦不歸者。不能消穀引食。下焦不歸者。則遺溲。

趺陽脉沈而數。沈为実。數消穀。緊者病難治。

寸口脉微而{\CJKfontspec[Path=ttf/]{TH-Tshyn-P2}𬈧}。微者衛气衰。{\CJKfontspec[Path=ttf/]{TH-Tshyn-P2}𬈧}者榮气不足。衛气衰。面色黄。榮气不足。面色青。榮为根。衛为葉。榮衛俱微。則根葉枯槁。而寒慄欬逆。唾腥吐涎沫也。

趺陽脉浮而芤。浮者衛气衰。芤者榮气傷。其身体瘦。肌肉甲错。浮芤相摶。宗气衰微。四屬斷绝。

寸口脉微而缓。微者衛气疏。疏則其膚空。缓者胃气実。実則穀消而水化也。穀入於胃。脉道乃行。而入於{\hbox{\scalebox{0.68}[1]{纟}\kern-0.35em\scalebox{0.64}[1]{巠}}}。其血乃成。榮盛則其膚必疏。三焦绝{\hbox{\scalebox{0.68}[1]{纟}\kern-0.35em\scalebox{0.64}[1]{巠}}}。名曰血崩。

趺陽脉微而緊。緊为寒。微則为虛。微緊相摶。則为短气。

少陰脉弱而{\CJKfontspec[Path=ttf/]{TH-Tshyn-P2}𬈧}。弱者微煩。{\CJKfontspec[Path=ttf/]{TH-Tshyn-P2}𬈧}者厥逆。

趺陽脉不出。脾不上下。身冷膚堅。

少陰脉不至。腎气微。少精血。奔气促迫。上入胸膈。宗气反聚。血结心下。陽气退下。熱歸陰股。与陰相動。令身不仁。此为尸厥。当刺期门。巨阙。

寸口脉微。尺脉緊。其人虛損多汗。知陰常在。绝不見陽也。

寸口諸微亡陽。諸濡亡血。諸弱發熱。諸緊为寒。諸乘寒者。則为厥。鬱冒不仁。以胃无穀气。脾{\CJKfontspec[Path=ttf/]{TH-Tshyn-P2}𬈧}不通。口急不能言。戰而慄也。

问曰。濡弱何以反適十一頭。\\
師曰。五臓六腑相乘。故令十一。

问曰。何以知乘腑。何以知乘臓。\\
師曰。諸陽浮數为乘腑。諸陰遲{\CJKfontspec[Path=ttf/]{TH-Tshyn-P2}𬈧}为乘臓也。

\chapter{傷寒例}

陰陽大論云。春气温和。夏气暑熱。秋气清涼。冬气冷冽。此則四時正气之序也。冬時嚴寒。万類深藏。君子固密。則不傷於寒。觸冒之者。乃名傷寒耳。其傷於四時之气。皆能为病。以傷寒为毒者。以其最成殺厉之气也。

中而即病者。名曰傷寒。不即病者。寒毒藏於肌膚。至春變为温病。至夏變为暑病。暑病者。熱極重於温也。是以辛苦之人。春夏多温熱病。皆由冬時觸寒所致。非時行之气也。

凡時行者。春時應暖而復大寒。夏時應大熱而反大涼。秋時應涼而反大熱。冬時應寒而反大温。此非其時而有其气。是以一歲之中。長幼之病多相似者。此則時行之气也。

夫欲候知四時正气为病。及時行疫气之法。皆当按斗曆占之。九月霜降節後。宜漸寒。向冬大寒。至正月雨水節後。宜解也。所以谓之雨水者。以冰雪解而为雨水故也。至驚蟄二月節後。气漸和暖。向夏大熱。至秋便涼。

從霜降以後。至春分以前。凡有觸冒霜露。体中寒即病者。谓之傷寒也。九月十月。寒气尚微。为病則輕。十一月十二月。寒冽已嚴。为病則重。正月二月。寒漸將解。为病亦輕。此以冬時不調。適有傷寒之人。即为病也。其冬有非節之暖者。名曰冬温。冬温之毒。与傷寒大異。冬温復有先後。更相重沓。亦有輕重。为治不同。證如後章。

從立春節後。其中无暴大寒。又不冰雪。而有人壯熱为病者。此屬春時陽气。發於冬時伏寒。變为温病。

從春分以後至秋分節前。天有暴寒者。皆为時行寒疫也。三月四月。或有暴寒。其時陽气尚弱。为寒所折。病熱猶輕。五月六月。陽气已盛。为寒所折。病熱則重。七月八月。陽气已衰。为寒所折。病熱亦微。其病与温及暑病相似。但治有殊耳。

十五日得一气。於四時之中。一時有六气。四六名为二十四气也。然气候亦有應至而不至。或有未應至而至者。或有至而太過者。皆成病气也。但天地動靜。陰陽鼓擊者。各正一气耳。是以彼春之暖。为夏之暑。彼秋之忿。为冬之怒。是故冬至之後。一陽爻升。一陰爻降也。夏至之後。一陽气下。一陰气上也。斯則冬夏二至。陰陽合也。春秋二分。陰陽離也。

陰陽交易。人變病焉。此君子春夏養陽。秋冬養陰。順天地之剛柔也。小人觸冒。必嬰暴疹。須知毒烈之气留在何{\hbox{\scalebox{0.68}[1]{纟}\kern-0.35em\scalebox{0.64}[1]{巠}}}。而發何病。詳而取之。是以春傷於風。夏必飱泄。夏傷於暑。秋必病瘧。秋傷於濕。冬必咳嗽。冬傷於寒。春必病温。此必然之道。可不審明之。

傷寒之病。逐日淺深。以施方治。今世人傷寒。或始不早治。或治不對病。或日數久淹。困乃告醫。醫人又不依次第而治之。則不中病。皆宜臨時消息制方。无不效也。今搜採仲景舊論。錄其證候。診脉。聲色。對病真方有神驗者。擬防世急也。

又土地温涼。高下不同。物性剛柔。飡居亦異。是黄帝興四方之问。岐伯舉四治之能。以訓後賢。{\CJKfontspec[Path=ttf/]{TH-Tshyn-P2}𫔭}其未悟者。臨病之工。宜須兩審也。

凡傷於寒則为病熱。熱雖甚不死。若兩感於寒而病者必死。

尺寸俱浮者。太陽受病也。当一二日發。以其脉上連風府。故頭項痛。腰脊強。

尺寸俱長者。陽明受病也。当二三日發。以其脉俠鼻。络於目。故身熱。目疼。鼻乾。不得卧。

尺寸俱弦者。少陽受病也。当三四日發。以其脉循脇络於耳。故胸脇痛而耳聋。此三{\hbox{\scalebox{0.68}[1]{纟}\kern-0.35em\scalebox{0.64}[1]{巠}}}皆受病。未入於府者。可汗而已。

尺寸俱沈细者。太陰受病也。当四五日發。以其脉布胃中络於嗌。故腹滿而嗌乾。

尺寸俱沈者。少陰受病也。当五六日發。以其脉貫腎络於肺。繫舌本。故口燥舌乾而渴。

尺寸俱微缓者。厥陰受病也。当六七日發。以其脉循陰器络於肝。故煩滿而囊缩。此三{\hbox{\scalebox{0.68}[1]{纟}\kern-0.35em\scalebox{0.64}[1]{巠}}}皆受病。已入於府。可下而已。

若兩感於寒者。一日太陽受之。即与少陰俱病。則頭痛。口乾。煩滿而渴。二日陽明受之。即与太陰俱病。則腹滿。身熱。不欲食。譫語。三日少陽受之。即与厥陰俱病。則耳聋。囊缩而厥。水漿不入。不知人者。六日死。若三陰三陽。五臓六腑皆受病。則榮衛不行。腑臓不通。則死矣。其不兩感於寒。更不傳{\hbox{\scalebox{0.68}[1]{纟}\kern-0.35em\scalebox{0.64}[1]{巠}}}。不加異气者。至七日太陽病衰。頭痛少愈也。八日陽明病衰。身熱少歇也。九日少陽病衰。耳聋微闻也。十日太陰病衰。腹減如故。則思饮食。十一日少陰病衰。渴止舌乾。已而嚏也。十二日厥陰病衰。囊{\hbox{\scalebox{0.6}[1]{纟}\kern-0.32em\scalebox{0.7}[1]{從}}}。少腹微下。大气皆去。病人精神爽慧也。若過十三日以上不间。尺寸陷者。大危。若更感異气變为他病者。当依舊壞證病而治之。

若脉陰陽俱盛。重感於寒者。變为温瘧。陽脉浮滑。陰脉濡弱者。更遇於風。變为風温。陽脉洪數。陰脉実大者。遇温熱。變为温毒。温毒为病最重也。陽脉濡弱。陰脉弦緊者。更遇温气。變为温疫。以此冬傷於寒。發为温病。脉之變證。方治如説。

凡人有疾。不時即治。隱忍冀差。以成痼疾。小兒女子。益以滋甚。時气不和。便当早言。尋其邪由。及在腠理。以時治之。罕有不愈者。患人忍之。數日乃説。邪气入藏。則難可制。此为家有患。備慮之要。

凡作湯藥。不可避晨夜。覺病須臾。即宜便治。不等早晚。則易愈矣。若或差遲。病即傳變。雖欲除治。必難为力。服藥正如方法。{\hbox{\scalebox{0.6}[1]{纟}\kern-0.32em\scalebox{0.7}[1]{從}}}意違師。不須治之。

凡傷寒之病。多從風寒得之。始表中風寒。入裏則不消矣。未有温覆而当不消散者。[若病]不察證治。擬欲攻之。猶当先解表。乃可下之。若表已解而内不消。大滿。大実。腹堅者。必内有燥屎。自可徐徐下之。雖{\hbox{\scalebox{0.68}[1]{纟}\kern-0.35em\scalebox{0.64}[1]{巠}}}四五日。不能为害也。若[病]不宜下而強攻之。内虛熱入。[則为]協熱遂利。煩燥諸變。不可勝數。輕者困篤。重者必死。
\footnote{宋本「乃可下之」後有「若表已解而内不消非大滿猶生寒熱則病不除」。}

世上之士。但務彼翕習之榮。而莫見此傾危之敗。惟明者居然能護其本。近取諸身。夫何遠之有焉。

凡發汗。温服湯藥。其方雖言日三服。若病劇不解。当促其间。可半日中盡三服。若与病相阻。即便有所覺。重病者。一日一夜当晬時觀之。如服一剂。病證猶在。故当復作本湯服之。至有不肯汗出。服三剂乃解。若汗不出者。死病也。

凡得時气病。至五六日。而渴欲饮水。饮不能多。不当与也。何者?以腹中熱尚少。不能消之。便更与人作病也。至七八日。大渴欲饮水者。猶当依證与之。与之常令不足。勿極意也。言能饮一斗。与五升。若饮而腹滿。小便不利。若喘若噦。不可与之。忽然大汗出。是为自愈也。

凡得病。反能饮水。此为欲愈之病。其不曉病者。但闻病饮水自愈。小渴者乃強与饮之。因成其禍。不可復數。

凡得病。厥脉動數。服湯藥更遲。脉浮大減小。初躁後靜。此皆愈證也。

凡治温病。可刺五十九穴。又身之穴三百六十有五。其三十九穴灸之有害。七十九穴刺之为災。并中髓也。

凡脉四損。三日死。平人四息。病人脉一至。名曰四損。脉五損。一日死。平人五息。病人脉一至。名曰五損。脉六損。一時死。平人六息。病人脉一至。名曰六損。

脉盛身寒。得之傷寒。脉虛身熱。得之傷暑。脉陰陽俱盛。大汗出不解者死。脉陰陽俱虛。熱不止者死。脉至乍疏乍數者死。脉至如轉索者。其日死。譫言妄語。身微熱。脉浮大。手足温者生。逆冷。脉沈细者。不過一日死矣。此以前是傷寒熱病證候也。

\chapter{辨太陽病}

太陽之为病。頭項強痛而惡寒。1

太陽病。其脉浮。1

太陽病。發熱。汗出。惡風。脉缓者。为中風。2

太陽中風。發熱而惡寒。0

太陽病。或已發熱。或未發熱。必惡寒。体痛。嘔逆。脉陰陽俱緊者。为傷寒。3

傷寒一日。太陽脉弱。至四日。太陰脉大。0

傷寒一日。太陽受之。脉若靜者。为不傳。頗欲吐。或躁煩。脉數急者。乃为傳。4

傷寒[二三日]。陽明少陽證不見者。为不傳。5
\footnote{「陽明少陽證」除宋本外其它版本均作「其二陽證」。}

傷寒三日。陽明脉大[者。为欲傳]。186

傷寒三日。少陽脉小者。为欲已。271

太陽病三四日。不吐下。見芤乃汗之。0

太陽病。發熱而渴。不惡寒者。为温病。若發汗已。身灼熱者。为風温。風温为病。脉陰陽俱浮。自汗出。身重。多眠睡。鼻息必鼾。語言難出。若被下者。小便不利。直視。失溲。若被火者。微發黄[色]。劇則如驚癇。時瘈瘲。若火熏之。一逆尚引日。再逆促命期。6

病有發熱而惡寒者。發於陽也。不熱而惡寒者。發於陰也。發於陽者七日愈。發於陰者六日愈。以陽數七。陰數六故也。7

太陽病。頭痛。至七日自当愈。以其{\hbox{\scalebox{0.68}[1]{纟}\kern-0.35em\scalebox{0.64}[1]{巠}}}盡故也。若欲作再{\hbox{\scalebox{0.68}[1]{纟}\kern-0.35em\scalebox{0.64}[1]{巠}}}者。当针足陽明。使{\hbox{\scalebox{0.68}[1]{纟}\kern-0.35em\scalebox{0.64}[1]{巠}}}不傳則愈。8
\footnote{「自当愈」宋本、千金翼作「以上自愈者」。}

太陽病欲解時。從巳盡未。9

風家。表解而不了了者。十二日愈。10

病人身大熱。反欲得近衣者。熱在皮膚。寒在骨髓也。身大寒。反不欲近衣者。寒在皮膚。熱在骨髓也。11

太陽中風。[脉]陽浮而陰弱。陽浮者熱自發。陰弱者汗自出。啬啬惡寒。淅淅惡風。翕翕發熱。鼻鳴。乾嘔。桂枝湯主之。12

\begin{itemize}
\item 惡寒。鼻鳴。乾嘔者。外邪之侯也。桂枝湯主之。脉浮弱。或浮數。而惡寒者。證雖不具。亦用此方。脉數浮弱。盖桂枝湯之脉狀也。(方機)
\end{itemize}

太陽病。發熱。汗出。此为榮弱衛強。故使汗出。欲救邪風。宜桂枝湯。95

太陽病。頭痛。發熱。汗出。惡風。桂枝湯主之。13

\begin{itemize}
\item 桂枝湯,治上衝、頭痛、發熱、汗出、惡風者。(方極)
\item 頭痛。發熱。汗出。惡風者。正證也。頭痛一證。亦当投此方矣。若由欬嗽。嘔逆。而頭痛者。非此方之所治也。(方機)
\end{itemize}

太陽病。項背強几几。反汗出。惡風。桂枝[加葛根]湯主之。14

\begin{itemize}
\item 桂枝加葛根湯。治桂枝湯證而項背強急者。(方極)
\end{itemize}

太陽病。下之。其气上衝者。可与桂枝湯。不衝者。不可与之。15

太陽病三日。已發汗吐下温针而不解。此为壞病。桂枝湯不復中与也。觀其脉證。知犯何逆。隨證治之。16

桂枝湯本为解肌。若其人脉浮緊。發熱。无汗。不可与也。常須識此。勿令误也。16

酒客不可与桂枝湯。得之則嘔。以酒客不喜甘故也。17

喘家作桂枝湯。加厚朴杏人佳。18

服桂枝湯吐者。其後必吐膿血。19

太陽病。發汗。遂漏不止。其人惡風。小便難。四肢微急。難以屈伸。桂枝加附子湯主之。20

\begin{itemize}
\item 桂枝加附子湯。治桂枝湯證而惡寒。或肢節微痛者。(方極)
\end{itemize}

太陽病。下之。脉促。胸滿者。桂枝去芍藥湯主之。若微[惡]寒者。桂枝去芍藥加附子湯主之。21.22

太陽病。得之八九日。如瘧狀。發熱。惡寒。熱多寒少。其人不嘔。清便{\hbox{\scalebox{0.6}[1]{纟}\kern-0.32em\scalebox{0.7}[1]{賣}}}自可。一日再三發。脉微缓者。为欲愈也。脉微而惡寒者。此为陰陽俱虛。不可復[吐下]發汗也。面反有熱色者。未欲解也。以其不能得汗出。身必癢。宜桂枝麻黄各半湯。23
\footnote{「{\hbox{\scalebox{0.6}[1]{纟}\kern-0.32em\scalebox{0.7}[1]{賣}}}自可」玉函作「自調」。「面反有熱色」聖惠方作「面色赤有熱」。}

太陽病。初服桂枝湯。反煩不解者。当先刺風池風府。郤与桂枝湯即愈。24

服桂枝湯。大汗出。若脉[但]洪大者。与桂枝湯。若形如瘧。一日再發者。汗出便解。宜桂枝二麻黄一湯。25

服桂枝湯。大汗出。大煩渴不解。若脉洪大。与白虎[加人参]湯。26

\begin{itemize}
\item 白虎加人参湯。治白虎湯證而心下痞堅者。(方極)
\item 白虎加人参湯。治霍亂吐瀉之後。大熱。煩躁。大渴引饮。心下痞堅。脉洪大者。(類聚方廣義)
\item 白虎加人参湯。治消渴。脉洪數。晝夜引饮不歇。心下痞堅。夜间肢体煩熱更甚。肌肉日消铄者。(類聚方廣義)
\item 白虎加人参湯。治瘧病。大熱如煅。譫語。煩躁。汗出淋漓。心下痞堅。渴饮无度者。(類聚方廣義)
\end{itemize}

太陽病。發熱。惡寒。熱多寒少。脉微弱者。此无陽也。不可[復]發汗。[宜桂枝二越脾一湯。]27

服桂枝湯。[或]下之。仍頭項強痛。翕翕發熱。无汗。心下滿。微痛。小便不利。桂枝去桂加茯苓白术湯主之。28

傷寒。脉浮。自汗出。小便數。心煩。微惡寒。腳攣急。反与桂枝湯。欲攻其表。得之便厥。咽乾。煩躁。吐逆者。当作甘草乾薑湯。以復其陽。若厥愈。足温者。更作芍藥甘草湯与之。其腳即伸。若胃气不和。譫語者。少与[調胃]承气湯。若重發汗。復加燒针者。四逆湯主之。29

\begin{itemize}
\item 甘草乾薑湯。治厥而煩躁。多涎唾者。(方極)
\item 甘草乾薑湯。治足厥。咽中燥。煩躁。吐逆者。吐下後厥逆。煩躁。不可如何者。吐涎沫。不欬。遺溺。小便數者。兼用礞石滾痰丸。(方機)
\item 芍藥甘草湯。治拘攣急迫者。(方極)
\item 芍藥甘草湯。治腳攣急。兼用應{\hbox{\scalebox{0.7}[1]{钅}\kern-0.4em\scalebox{0.7}[1]{童}}}散。紫丸。(方機)
\item 芍藥甘草湯。治腹中攣急而痛者。小兒夜啼不止。腹中攣急甚者。亦有奇效。(類聚方廣義)
\item 芍藥甘草湯。止腹痛如神。脉遲为寒。加乾薑。脉洪为熱。加黄連。(醫學心悟)
\item 調胃承气湯。治実而不滿者。腹如仰瓦。腹中轉失气。有燥糞。不大便而譫語。堅実之證宜用之。(醫壘元戎)
\item 調胃承气湯。治傷寒。發狂。煩躁。面赤。脉実。(衛生寶鑒)
\item 調胃承气湯。治熱留胃中發斑。及服熱藥過多亦發斑。({\hbox{\scalebox{0.68}[1]{纟}\kern-0.35em\scalebox{0.64}[1]{巠}}}驗良方)
\item 調胃承气湯。治中熱。大便不通。咽喉腫痛。或口舌生瘡。(口齒類要)
\item 調胃承气湯。治消中。渴而饮食多。(试效方)
\item 破棺丹。治瘡瘍。熱極。汗多。大渴。便祕。譫語。發狂。(外科樞要)
\item 調胃承气湯。治大黄甘草湯證而実者。(方極)
\item 調胃承气湯。治因汗吐下而譫語者。發汗後。熱而大便不通者。服下剂。下利不止。心煩或譫語者。吐下之後。心下嗢嗢欲吐。大便溏。腹微滿。鬱鬱微煩者。吐後腹脹滿者。(方機)
\item 痘瘡。麻疹。癰疽。疔毒。内攻衝心。大熱。譫語。煩躁。闷亂。舌上燥裂。不大便。或下利。或大便綠色者。宜調胃承气湯。(類聚方廣義)
\item 牙齒疼痛。齒齦腫痛。齲齒枯折。口臭等。其人平日多大便祕闭而衝逆。宜調胃承气湯。(類聚方廣義)
\item 胃反。膈噎。胸腹痛。或煩滿。腹中有塊。咽喉乾燥。鬱熱便祕者。消渴。五心煩熱。肌肉燥瘠。腹中凝结。二便不利者。皆宜調胃承气湯。或为兼用方亦良。(類聚方廣義)
\item 四逆湯。治傷寒陰證。唇青。面黑。身背強痛。四肢厥冷。及諸虛沈寒。(醫林集要)
\item 四逆湯。治五臟中寒。口噤。四肢強直。失音不語。或猝然昏闷。手足厥冷者。(济生方)
\item 四逆湯。治四肢厥逆。身体疼痛。下利清穀。或小便清利者。(方極)
\item 四逆湯。治手足厥冷者。下利清穀者。腹拘急。四肢厥冷。下利。惡寒者。大汗出。熱不去。拘急。四肢厥冷者。下利。腹脹滿。身体疼痛者。(方機)
\item 四逆湯。治霍亂吐利甚者。及所谓暴瀉症。急者死不崇朝。若仓皇失措。拟议误策。毙人於非命。其罪何归。醫人当平素讨究讲明。以济急靖難。可参考大汗出熱不去云云(356條)以下諸章。(類聚方廣義)
\item 四逆湯。救厥之主方也。然傷寒熱结在裏者。中風猝倒。痰涎沸涌者。霍亂未吐下。内猶有毒者。老人食鬱。及諸猝病闭塞不{\CJKfontspec[Path=ttf/]{TH-Tshyn-P2}𫔭}者。{\hbox{\scalebox{0.6}[1]{纟}\kern-0.32em\scalebox{0.7}[1]{從}}}令全身厥冷。冷汗脉微。能审其證。以白虎。瀉心。承气。紫丸。備急。走馬之類。解其结。通其闭。則厥冷不治自復。若误認为脱證。遽用四逆。玄武。猶如救{\hbox{\scalebox{0.68}[1]{纟}\kern-0.35em\scalebox{0.64}[1]{巠}}}引足。庸工殺人。常当坐此。呜呼。方伎虽小。死生系焉。存亡由焉。自非高才卓识。難探其理致矣。(類聚方廣義)
\end{itemize}

太陽病。項背強几几。无汗。惡風。葛根湯主之。31

\begin{itemize}
\item 項背強几几者。谓自腰部沿脊柱兩侧。上至後頭结节。其肌肉有強直性痙攣也。故病者若訴肩凝。或訴腰背攣痛時。可以指頭沿上述肌肉之横徑而強按壓之。倘觸知其凝结攣急。同時病人訴疼痛者。即可斷为項背強几几。百无一失。然此證之存在。有不自覺者。亦有自覺而難以明確觸知者。是当详细问觸。参外證脉象以決之。(湯本求真)
\item 葛根湯。治項背強急。發熱。惡風。或喘。或身疼者。(方極)
\item 葛根湯。治項背強而无汗。惡寒者。兼用應{\hbox{\scalebox{0.7}[1]{钅}\kern-0.4em\scalebox{0.7}[1]{童}}}散。二陽合病。下利者。痙病。无汗。小便反少。气上衝於胸。口噤不能語言者。兼用紫丸。(方機)
\item 痘瘡自初熱至點見。投葛根湯。兼用紫丸下之一度。自起脹至貫膿。葛根加桔梗湯主之。自落痂以後。葛根加大黄湯主之。若惡寒劇。起脹甚而一身腫脹或疼痛者。葛根加术附湯主之。兼用紫丸。若腫脹甚者。兼用桃花散。寒戰咬牙而下利者。兼用紫丸。俱加术附湯。(方機)
\item 頭瘡。葛根加大黄湯主之。(方機)
\item 小瘡。葛根加梓葉湯主之。兼用桃花散。以蓖麻子擦之。毒劇者。以梅肉攻之。(方機)
\item 諸頑腫。惡腫。葛根加术附湯主之。(方機)
\item 葛根湯。治瘰癧。便毒。瘍疔之類。瘰癧兼用七寶丸。梅肉日投亦可。便毒。瘍疔。兼以梅肉攻之。伯州散朝五分夕五分。酒送下。(方機)
\item 治疳瘡。兼七寶或梅肉之類選用。(方機)
\item 凡諸有膿則加桔梗。若痛劇則加术附。(方機)
\item 世俗所谓小兒赤游風丹毒類。皆葛根加术附湯主之。兼用紫丸攻之。(方機)
\item 葛根湯主治項背強急。故能治驚癇。破傷風。產後感冒猝痙。痘瘡初起等。角弓反張。上竄搐搦。身体強直者。宜隨證兼用熊膽。紫丸。参連湯。瀉心湯等。(類聚方廣義)
\item 葛根湯。治麻疹初起。惡寒。發熱。頭項強痛。惡寒。脉浮數。或乾嘔。下利者。若熱熾。咽喉刺激。心胸煩闷者。兼用黄連解毒湯。(類聚方廣義)
\item 疫痢初起。發熱。惡寒。脉數者。当先用葛根湯温服發汗。若嘔者。以葛根加半夏湯取汗後。撰用大柴胡湯。厚朴三七物湯。大小承气湯。調胃承气湯。桃人承气湯。大黄牡丹湯。大黄附子湯。各隨證處之。以疏蕩裏熱宿毒。(類聚方廣義)
\item 咽喉腫痛。時毒痄腮。疫眼焮熱腫痛。項背強急。發熱惡寒。脉浮数者。擇加桔梗。大黄。石膏。或兼用應{\hbox{\scalebox{0.7}[1]{钅}\kern-0.4em\scalebox{0.7}[1]{童}}}散。再造散。瀉心湯。黄連解毒湯等。(類聚方廣義)
\item 癰疽初起。壯熱憎寒。脉数者。以葛根湯發汗後。轉用加术附湯。促其酿膿。膿成者。可速入针。若心胸煩闷。鬱熱便祕者。兼用瀉心湯。大柴胡湯。(類聚方廣義)
\end{itemize}

太陽与陽明合病。而自利。[不嘔者。]葛根湯主之。不下利。但嘔者。葛根加半夏湯主之。32.33\footnote{「而自利」宋本作「者必自下利」,脉{\hbox{\scalebox{0.68}[1]{纟}\kern-0.35em\scalebox{0.64}[1]{巠}}}此三字下有「不嘔者」三字。}

\begin{itemize}
\item 今驗之方藥。葛根湯但治太陽證兼下利者。若有陽明證輒不效。(傷寒論今釋)
\end{itemize}

太陽病。桂枝證。醫反下之。遂利不止。脉促者。表未解也。喘而汗出者。宜葛根黄連[黄芩]湯。34

\begin{itemize}
\item 葛根芩連湯。治項背強急。心悸而下利者。(方極)
\item 葛根芩連湯。治下利。喘而汗出者。項背強。汗出下利者。兼用紫丸。淵雷按。吉益氏谓葛根主治項背強。故云尔。然本方之重用葛根。乃取其输運津液。減少腸中水分以止利。且今病毒仍向外解。故其量特重。非为項強而用之。固不必有項強證矣。(方機)
\item 下利初發。用桂枝湯葛根湯之類。表證虽解。脉益促。熱猶盛者。可用葛根芩連湯。小儿痢疾。熱熾而不需下剂者。用此多效。(方輿輗)
\item 葛根芩連湯。移治滯下。有表證而未要攻下者甚效。
\item 葛根芩連湯。治平日項背強急。心胸痞塞。神思悒鬱不舒暢者。或加大黄。(類聚方廣義)
\item 項背強急。心下痞塞。胸中冤熱。眼目牙齿疼痛。或口舌腫痛腐烂者。加大黄則其效速。(類聚方廣義)
\item 此方治表邪内陷之下利有效。尾洲之醫師用於小儿疫痢。屢有效云。余用於小儿之下利。{\hbox{\scalebox{0.68}[1]{纟}\kern-0.35em\scalebox{0.64}[1]{巠}}}驗亦多。此方之喘。乃熱勢内壅所致。非主證也。(方函口訣)
\end{itemize}

太陽病。頭痛。發熱。身疼。腰痛。骨節疼痛。惡風。无汗而喘。麻黄湯主之。35

\begin{itemize}
\item 麻黄湯。治喘而无汗。頭痛。發熱。惡寒。身体疼痛者。(方極)
\item 頭痛。發熱。身疼。腰痛。骨节疼痛。惡风。无汗而喘者。是其正證也。又治喘而胸满者。服發汗剂。而不汗却衄者。(方機)
\item 猝中风。痰涎涌盛。不省人事。心下坚。身大熱。脉浮大者。以白散或瓜蒂取吐下后。有可用麻黄湯者。(類聚方廣義)
\item 初生儿。有時時發熱。鼻塞不通。不能哺乳者。用麻黄湯即愈。(類聚方廣義)
\item 麻黄湯。治痘瘡見点時。身熱如灼。表鬱難發。及大熱。煩躁而喘。不起脹者。(類聚方廣義)
\item 治麻疹。脉浮數。發熱。身疼。腰痛。喘欬。表壅不出齐者。(類聚方廣義)
\item 麻黄湯。治哮喘痰潮。聲音不出。抬肩滾肚而不得卧。惡寒。發熱。冷汗如油者。合生姜半夏湯用之。立效。按。哮喘症大抵年一二發或五六發。又有每月一二發者。其發必因外感過食。由外感而來者。宜麻黄湯。麻杏甘石湯。大青龙湯等。因饮食或大便不利而發者。先以陷胸丸。紫丸等取吐下。疏蕩宿滯后。用對證方为佳。湯本氏云。余之{\hbox{\scalebox{0.68}[1]{纟}\kern-0.35em\scalebox{0.64}[1]{巠}}}驗。由饮食或大便不利而發者。多宜用大柴胡湯。桃人承气湯。大黄牡丹皮湯之一方乃至三方者。其需陷胸丸。紫丸者。乃極稀有。(類聚方廣義)
\end{itemize}

太陽与陽明合病。喘而胸滿者。不可下。宜麻黄湯。36

太陽病。十日已去。脉浮细而嗜卧者。外已解也。設胸滿脇痛者。与小柴胡湯。脉[但]浮者。与麻黄湯。37
\footnote{「脉但浮」脉{\hbox{\scalebox{0.68}[1]{纟}\kern-0.35em\scalebox{0.64}[1]{巠}}}、玉函作「脉浮」。}

太陽中風。脉浮緊。發熱。惡寒。身体疼痛。不汗出而煩躁者。大青龙湯主之。若脉微弱。汗出。惡風者。不可服之。服之則厥。筋愓肉{\CJKfontspec[Path=ttf/]{TH-Tshyn-P2}𥆧}。此为逆也。38
\footnote{「煩躁者」脉{\hbox{\scalebox{0.68}[1]{纟}\kern-0.35em\scalebox{0.64}[1]{巠}}}、玉函作「煩躁頭痛」。}

\begin{itemize}
\item 大青龙湯。治喘及欬嗽。渴欲饮水。上衝。或身疼。惡風寒者。(方極)
\item 大青龙湯。治發熱惡寒。身疼痛。不汗出。煩躁者。脉浮缓。發熱身重。乍有輕時者。頭痛劇。四肢惰痛。發熱而汗不出者。(方機)
\item 大青龙湯。治麻疹。脉浮紧。寒熱頭眩。身体疼痛。喘欬咽痛。汗不出而煩躁者。(類聚方廣義)
\item 大青龙湯。治眼目疼痛。流泪不止。赤脉怒张。云翳四围。或眉棱骨疼痛。或頭疼耳痛者。又治烂瞼風。涕泪稠黏。痒痛甚者。俱加车前子为佳。兼以黄連解毒湯加枯矾。頻頻洗蒸。每夜臨卧服應{\hbox{\scalebox{0.7}[1]{钅}\kern-0.4em\scalebox{0.7}[1]{童}}}散。每五日十日可与紫丸五分或一钱下之。(類聚方廣義)
\item 大青龙湯。治雷頭風。發熱惡寒。頭腦劇痛如裂。每夜不能眠者。若心下痞。胸膈煩熱者。兼服瀉心湯。黄連解毒湯。若胸膈有饮。心中滿。肩背強急者。当以瓜蒂散吐之。(類聚方廣義)
\item 風眼症(即淋菌性结膜炎角膜炎)。暴發劇痛者。不早救治。則眼球破裂進出。尤为極險至急之症。急用紫丸一钱或一钱五分。取峻瀉数行。大势已解之後。可用大青龙湯。更隨其腹診。兼用大承气湯。大黄硝石湯。瀉心湯。桃人承气湯等。(類聚方廣義)
\item 大青龙湯。治小儿赤游丹毒。大熱煩渴。驚惕。或痰喘壅盛者。兼用紫丸或龙葵丸。(類聚方廣義)
\item 急驚風。痰涎沸涌。直视口噤者。当先用熊胆紫丸走馬湯等取吐下後。大熱。煩躁。喘鳴。搐搦不止者。宜以大青龙湯發汗。(類聚方廣義)
\end{itemize}

傷寒。脉浮缓。身不疼。但重。或有輕時。无少陰證者。大青龙湯發之。39\footnote{「或有輕時」除聖惠方外其它版本均作「乍有輕時」。}

\begin{itemize}
\item 水之为病。其脉沈小。屬少陰。浮者为風。无水。虛脹者为气。水。發其汗即已。脉沈者。宜麻黄附子湯。浮者。宜杏子湯。(金匱)
\item 饮水流行。歸於四肢。当汗出而不汗出。身体疼重。謂之溢饮。病溢饮者。当發其汗。大青龙湯主之。(金匱)
\end{itemize}

傷寒。表不解。心下有水气。乾嘔。發熱而欬。或渴。或利。或噎。或小便不利。少腹滿。或[微]喘。小青龙湯主之。40

傷寒。心下有水气。欬而微喘。發熱。不渴。服湯已而渴者。此寒去。为欲解。小青龙湯主之。41

太陽病。外證未解。脉浮弱者。当以汗解。宜桂枝湯。42

太陽病。下之。微喘者。表未解故也。桂枝[加厚朴杏人]湯主之。43
\footnote{「桂枝加厚朴杏人湯」一云「麻黄湯」。}

太陽病。外證未解者。不可下。下之为逆。欲解外者。宜桂枝湯。44
\footnote{「欲解外者宜桂枝湯」脉{\hbox{\scalebox{0.68}[1]{纟}\kern-0.35em\scalebox{0.64}[1]{巠}}}无。}

太陽病。先發汗不解而下之。其脉浮者不愈。浮为在外。而反下之。故令不愈。今脉浮。故在外。当解其外則愈。宜桂枝湯。45
\footnote{太陽病。下之不愈。其脉浮者为在外。汗之則愈。宜桂枝湯。45(聖惠方)}

太陽病。脉浮緊。无汗。發熱。身疼痛。八九日不解。表證{\hbox{\scalebox{0.6}[1]{纟}\kern-0.32em\scalebox{0.7}[1]{賣}}}在。此当發其汗。服藥已。微除。其人發煩目暝。劇者必衄。衄乃解。所以然者。陽气重故也。麻黄湯主之。46

太陽病。脉浮緊。發熱。身无汗。自衄者愈。47

二陽并病。太陽初得病時。發其汗。汗先出[。復]不徹。因轉屬陽明。{\hbox{\scalebox{0.6}[1]{纟}\kern-0.32em\scalebox{0.7}[1]{賣}}}自微汗出。不惡寒。若太陽病證不罷者。不可下。下之为逆。如此者可小發汗。設面色缘缘正赤者。陽气怫鬱不得越。当解之熏之。当汗不汗。其人躁煩。不知痛處。乍在腹中。乍在四肢。按之不可得。其人短气。但坐。以汗出不徹故也。更發汗則愈。何以知汗出不徹。以脉{\CJKfontspec[Path=ttf/]{TH-Tshyn-P2}𬈧}故知之。48

脉浮數者。法当汗出而愈。若下之。身体重。心悸者。不可發汗。当自汗出而解。所以然者。尺中脉微。此裏虛。須表裏実。津液和。即自汗出愈。49

脉浮而緊。法当身体疼痛。当以汗解之。假令尺中脉遲者。不可發汗。何以知然。以榮气不足。血少故也。50\footnote{凡脉尺中遲。不可發汗。榮衛不足。血少故也。(聖惠方)50}

脉浮者。病在表。可發汗。宜麻黄湯。51
\footnote{「麻黄湯」一云「桂枝湯」。}

[太陽病。]脉浮而數者。可發汗。宜麻黄湯。52
\footnote{「太陽病」三字宋本、千金翼无。「麻黄湯」一云「桂枝湯」。}

病常自汗出者。此为榮气和。衛气不和也。榮行脉中。衛行脉外。復發其汗。衛和則愈。宜桂枝湯。53
\footnote{「此为榮气和衛气不和也」玉函作「此为營气与衛气不和也」,脉{\hbox{\scalebox{0.68}[1]{纟}\kern-0.35em\scalebox{0.64}[1]{巠}}}作「此为榮气和榮气和而外不解此衛不和也」,千金作「此为榮气和榮气和而外不解此为衛不和也」,千金翼作「此为榮气和衛气不和故也」,聖惠方作「此为榮气和衛气不和」,宋本作「此为榮气和榮气和者外不谐以衛气不共榮气谐和故尔」。}

病人臓无他病。時發熱。自汗出而不愈者。此衛气不和也。先其時發汗則愈。宜桂枝湯。54

傷寒。脉浮緊。不發汗。因致衄者。宜麻黄湯。55

[寸口]脉浮而緊。浮則为風。緊則为寒。風則傷衛。寒則傷榮。榮衛俱病。骨節煩疼。当發其汗。宜麻黄湯。0

傷寒。不大便六七日。頭痛。有熱。与承气湯。其小便清者。此为不在裏。{\hbox{\scalebox{0.6}[1]{纟}\kern-0.32em\scalebox{0.7}[1]{賣}}}在表也。当發其汗。頭痛者必衄。宜桂枝湯。56
\footnote{玉函此條出現兩次,「与承气湯」四字上分别有「未可」、「不可」二字。「其小便清者」脉{\hbox{\scalebox{0.68}[1]{纟}\kern-0.35em\scalebox{0.64}[1]{巠}}}、千金翼作「其大便反青」,玉函作「其小便反清」,外臺作「其人小便反清者」。}

傷寒。發汗已解。半日許復煩。其脉浮數。可復發汗。宜桂枝湯。57

凡病。或發汗。或吐。或下。或亡血。无津液而陰陽自和者。必自愈。58

大下後。復發汗。其人小便不利。此亡津液。勿治之。得小便利。必自愈。59

下之後。復發汗。必振寒。脉微细。所以然者。内外俱虛故也。60

下之後。復發汗。晝日煩躁不得眠。夜而安靜。不嘔。不渴。无表證。脉沈微。身无大熱。乾薑附子湯主之。61

\begin{itemize}
\item 乾薑附子湯。治下利。煩躁而厥者。(類聚方廣義)
\end{itemize}

發汗後。身体疼痛。脉沈遲者。桂枝加芍藥生薑人参湯主之。62

\begin{itemize}
\item 桂枝加芍藥生薑人参湯。治桂枝湯證。而心下痞堅。身疼痛。及嘔者。(類聚方廣義)
\item 桂枝加芍藥生薑人参湯。治桂枝湯證。而心下痞堅。或拘攣。及嘔者。(方極)
\end{itemize}

發汗後。不可更行桂枝湯。汗出而喘。无大熱者。可与麻杏甘石湯。63

\begin{itemize}
\item 麻杏甘石湯。治甘草麻黄湯證(喘急迫。或自汗。或不汗)而欬。煩渴者。(方極)
\item 麻杏甘石湯。治汗出而喘。熱伏者。又治喘息而渴者。兼用礞石滾痰丸或姑洗丸。(方機)
\item 麻杏甘石湯。治冬月欬嗽。寒痰结於咽喉。語聲不出者。此寒气客於會厭。故猝然而喑也。(張氏醫通)
\item 用小青龙湯後。表解而喘猶盛者。水熱相结也。麻杏甘石湯主之。(方輿輗)
\item 麻杏甘石湯。治喘欬不止。面目浮腫。咽乾。口渴。或胸痛者。兼用礞石滾痰丸。姑洗丸。(類聚方廣義)
\item 哮喘。胸中如火。气逆涎潮。太息。呻吟。聲如拽锯。鼻流清涕。心下{\CJKfontspec[Path=ttf/]{TH-Tshyn-P2}𩊅}塞。巨裏動如奔馬者。宜麻杏甘石湯。当須痰融聲出後。以陷胸丸。紫丸之類疏導之。(類聚方廣義)
\item 麻杏甘石湯。治肺癰。發熱。喘欬。脉浮數。臭痰膿血。渴欲饮水者。宜加桔梗。時以白散攻之。(類聚方廣義)
\item 麻杏甘石湯与麻黄湯有表裏之異。以汗出而喘为目的。熱在肉裏。上熏肺部者。非麻石之力不解。故此方与越脾湯皆云无大熱也。(方函口訣)
\end{itemize}

發汗過多。其人叉手自冒心。心下悸。欲得按者。桂枝甘草湯主之。64

\begin{itemize}
\item 桂枝甘草湯。治上衝急迫者。然比諸桂枝加桂湯。衝逆猶輕。(類聚方廣義)
\end{itemize}

發汗後。其人脐下悸。欲作奔豚。苓桂甘棗湯主之。65

\begin{itemize}
\item 苓桂甘棗湯。治脐下悸。而攣急。上衝者。(類聚方廣義)
\item 苓桂甘棗湯。治脐下悸者。奔豚迫於心胸。短气急迫者。兼用紫丸。(方機)
\end{itemize}

發汗後。腹脹滿者。厚朴[生薑半夏甘草人参]湯主之。66

\begin{itemize}
\item 按。当有吐逆證。(類聚方廣義)
\item 厚朴生薑半夏甘草人参湯。治胸腹滿而嘔者。(類聚方廣義)
\item 厚朴生薑半夏甘草人参湯。治霍亂吐瀉後。腹猶滿痛。有嘔气者。(類聚方廣義)
\item 厚朴生薑半夏甘草人参湯。治胃虛嘔逆。痞滿不食。(張氏醫通)
\end{itemize}

傷寒吐下發汗後。心下逆滿。气上衝胸。起則頭眩。脉沈緊。發汗則動{\hbox{\scalebox{0.68}[1]{纟}\kern-0.35em\scalebox{0.64}[1]{巠}}}。身为振搖。苓桂术甘湯主之。67
\footnote{「吐下發汗後」玉函作「若吐若下若發汗後」,宋本作「若吐若下後」。「白术」脉{\hbox{\scalebox{0.68}[1]{纟}\kern-0.35em\scalebox{0.64}[1]{巠}}}作「术」。}

\begin{itemize}
\item 苓桂术甘湯。治心下悸。上衝。起則頭眩。小便不利者。(類聚方廣義)
\item 苓桂术甘湯。治饮家。眼目生雲翳。昏暗疼痛。上衝。頭眩。眵泪多者。加芣苢。尤有奇效。当以心下逆滿證为目的。兼用應{\hbox{\scalebox{0.7}[1]{钅}\kern-0.4em\scalebox{0.7}[1]{童}}}散。時以紫丸。十棗湯等攻之。雀目證亦有效。(類聚方廣義)
\end{itemize}

發汗不解。反惡寒者。虛故也。芍藥甘草附子湯主之。不惡寒。但熱者。実也。当和胃气。宜調胃承气湯。68.70
\footnote{「調胃承气湯」除宋本外其它版本均作「小承气湯」。}

發汗或下之。病仍不解。煩躁。茯苓四逆湯主之。69
\footnote{發汗吐下以後。不解。煩躁。茯苓四逆湯主之。(脉{\hbox{\scalebox{0.68}[1]{纟}\kern-0.35em\scalebox{0.64}[1]{巠}}}。玉函。千金翼)69}

太陽病發汗後。大汗出。胃中乾。煩躁不得眠。其人欲饮水。当稍饮之。令胃气和即愈。若脉浮。小便不利。微熱。消渴者。五苓散主之。71
\footnote{「其人欲饮水当稍饮之」宋本作「欲得饮水者少少与饮之」。}

\begin{itemize}
\item 五苓散。治消渴。小便不利。或渴欲饮水。水入則吐者。(類聚方廣義)
\item 霍亂吐下後。有厥冷。煩躁。渴饮不止而水藥共吐者。嚴禁湯水果物。每欲饮水。与五苓散。但一貼。二三次服为佳。不過三貼。嘔吐煩渴必止。吐渴共止。則必厥復而熱發。身体惰痛。仍用五苓散。則漐漐汗出。諸證脱然而愈。是五苓散。小半夏湯之别也(類聚方廣義)
\item 此方治眼患。与苓桂术甘湯略似。而彼以心下悸。心下逆滿。胸脇支滿。上衝等證为目的。此以發熱。消渴。目多眵泪。小便不利为目的。二方俱以利小便为其效也。兼用應{\hbox{\scalebox{0.7}[1]{钅}\kern-0.4em\scalebox{0.7}[1]{童}}}散。紫丸等。(類聚方廣義)
\end{itemize}

發汗已。脉浮數。煩渴者。五苓散主之。72

傷寒。汗出而渴者。五苓散主之。不渴者。茯苓甘草湯主之。73

\begin{itemize}
\item 按。当有衝逆。嘔吐證。(類聚方廣義)
\end{itemize}

中風。發熱。六七日不解而煩。有表裏證。渴欲饮水。水入則吐。此为水逆。五苓散主之。74

未持脉時。病人叉手自冒心。師因教试令欬。而不即欬者。此必兩耳聋无闻也。所以然者。以重發汗。虛故也。75

發汗後。饮水多者必喘。以水灌之亦喘。75

發汗後。水藥不得入口。为逆。[若更發汗。必吐下不止。]76

發汗吐下後。虛煩不得眠。劇者反覆顛倒。心中懊憹。梔子[豉]湯主之。若少气者。梔子甘草[豉]湯主之。若嘔者。梔子生薑[豉]湯主之。76

\begin{itemize}
\item 梔子豉湯。治心中懊憹者。此方梔子香豉二味而已。然施之其證。其效如響。(類聚方廣義)
\item 梔子豉湯。治少年房多短气。(千金方)
\item 梔子豉湯。治霍亂吐下後。心腹脹滿。(肘後方)
\item 梔子甘草豉湯。治梔子豉湯證而急迫者。(類聚方廣義)
\item 梔子生薑豉湯。治梔子豉湯證而嘔者。(類聚方廣義)
\end{itemize}

發汗或下之。煩熱。胸中窒者。梔子[豉]湯主之。77
\footnote{「胸中窒者」脉{\hbox{\scalebox{0.68}[1]{纟}\kern-0.35em\scalebox{0.64}[1]{巠}}}「窒」作「塞」,千金要方作「胸中窒气逆搶心者」。}

傷寒五六日。大下之後。身熱不去。心中结痛者。未欲解也。梔子[豉]湯主之。78

傷寒下後。煩而腹滿。卧起不安。梔子厚朴湯主之。79

\begin{itemize}
\item 梔子厚朴湯。治胸腹煩滿者。(類聚方廣義)
\end{itemize}

傷寒。醫以丸藥大下之。身熱不去。微煩。梔子乾薑湯主之。80

\begin{itemize}
\item 梔子乾薑湯。治心中微煩者。当有乾嘔證。微煩亦虛煩耳。(類聚方廣義)
\end{itemize}

凡用梔子湯證。其人微溏者。不可与服之。81
\footnote{凡用梔子湯。病人舊微溏者。不可与服之。(宋本)81}

太陽病。發汗。汗出不解。其人仍發熱。心下悸。頭眩。身{\CJKfontspec[Path=ttf/]{TH-Tshyn-P2}𥆧}動。振振欲擗地。玄武湯主之。82

咽喉乾燥者。不可發汗。83

\begin{itemize}
\item 咽喉乾燥者。上焦津液不足也。肺结核。喉頭结核。咽頭结核。皆咽喉乾燥之例。病结核者。營養不良。津液缺乏。故在禁汗之例。(傷寒論今釋)
\end{itemize}

淋家。不可發汗。發汗必便血。84

瘡家。雖身疼痛。不可發汗。汗出則痙。85

衄家。不可發汗。汗出必額上促急[而緊]。直視不能眴。不得眠。86
\footnote{「額上促急而緊」宋本作「額上陷脉急緊」,脉{\hbox{\scalebox{0.68}[1]{纟}\kern-0.35em\scalebox{0.64}[1]{巠}}}、玉函作「額陷脉上促急而緊」,千金翼作「額上促急」。}

亡血家。不可發汗。汗出則寒慄而振。87

汗家。重發汗。必恍惚心亂。小便已。陰疼。与禹馀粮丸。88
\footnote{凡失血者。不可發汗。發汗必恍惚心亂。(聖惠方)88}

病者有寒。復發汗。胃中冷。必吐蛔。89
\footnote{「吐蛔」一云「吐逆」。}

本發汗而復下之。为逆。若先發汗。治不为逆。本先下之而反汗之。为逆。若先下之。治不为逆。90

傷寒。醫下之。{\hbox{\scalebox{0.6}[1]{纟}\kern-0.32em\scalebox{0.7}[1]{賣}}}得下利。清穀不止。身体疼痛。急当救裏。後身体疼痛。清便自調。急当救表。救裏宜四逆湯。救表宜桂枝湯。91

病發熱。頭痛。脉反沈。若不差。身体疼痛。当救其裏。宜四逆湯。92

太陽病。先下而不愈。因復發汗。表裏俱虛。其人因冒。冒家当汗出自愈。所以然者。汗出表和故也。表和。然後下之。93
\footnote{「表和然後下之」玉函、千金翼作「表和故下之」,宋本、玉函作「裏未和然後復下之」。}

太陽病未解。脉陰陽俱微。必先振汗出而解。但陽[脉]微者。先汗出而解。但陰[脉]微者。先下之而解。汗之宜桂枝湯。下之宜[調胃]承气湯。94
\footnote{「調胃承气湯」一云「大柴胡湯」。}

血弱气盡。腠理{\CJKfontspec[Path=ttf/]{TH-Tshyn-P2}𫔭}。邪气因入。与正气相摶。结於脇下。正邪分爭。往來寒熱。休作有時。默默不欲饮食。臓腑相連。其痛必下。邪高痛下。故使嘔也。小柴胡湯主之。服柴胡湯已而渴者。屬陽明。以法治之。97

得病六七日。脉遲浮弱。惡風寒。手足温。醫再三下之。不能食。其人脇下滿[痛]。面目及身黄。頸項強。小便難。与柴胡湯後必下重。本渴。饮水而嘔。柴胡[湯]不復中与也。食穀者噦。98

傷寒五六日。中風。往來寒熱。胸脇苦滿。默默不欲饮食。心煩。喜嘔。或胸中煩而不嘔。或渴。或腹中痛。或脇下痞堅。或心下悸。小便不利。或不渴。外有微熱。或欬。小柴胡湯主之。96
\footnote{「傷寒五六日中風往來寒熱」脉{\hbox{\scalebox{0.68}[1]{纟}\kern-0.35em\scalebox{0.64}[1]{巠}}}、玉函作「中風往來寒熱傷寒五六日以後」,玉函此條重出作「中風五六日傷寒往來寒熱」。}

\begin{itemize}
\item 小柴胡湯。治胸脇苦滿。往來寒熱。心下痞堅而嘔者。(類聚方廣義)
\item 柴胡諸方。皆能治瘧。要当以胸脇苦滿證为目的。(類聚方廣義)
\item 初生兒。時時无故發熱。胸悸。或吐乳者。稱之變蒸熱。宜小柴胡湯。(類聚方廣義)
\item 傷寒愈後。唯有耳中啾啾不安。或耳聋累月不復者。可長服此方。(類聚方廣義)
\item 小柴胡湯。治胸脇苦滿。或往來寒熱。或嘔者。淵雷按。当有心下痞堅證。(方極)
\item 小柴胡湯。治往來寒熱。胸脇苦滿。默默不欲饮食。心煩。喜嘔者。胸滿脇痛者。身熱。惡風。頸項強。脇下滿。或渴。或微嘔者。又治脇下逆滿。鬱鬱不欲饮食。或嘔者。兼用應{\hbox{\scalebox{0.7}[1]{钅}\kern-0.4em\scalebox{0.7}[1]{童}}}散。發潮熱。胸脇滿而嘔者。兼用消塊。寒熱發作有時。胸脇苦滿。有{\hbox{\scalebox{0.68}[1]{纟}\kern-0.35em\scalebox{0.64}[1]{巠}}}水之變者。兼用應{\hbox{\scalebox{0.7}[1]{钅}\kern-0.4em\scalebox{0.7}[1]{童}}}散。產婦四肢苦煩熱。頭痛。胸脇滿者。兼用解毒散。產婦鬱冒。寒熱往來。嘔而不能食。大便堅。或盜汗出者。兼用消塊或應{\hbox{\scalebox{0.7}[1]{钅}\kern-0.4em\scalebox{0.7}[1]{童}}}散。發熱。大便溏。小便自可。胸滿者。兼用消塊。發黄色。腹痛而嘔。或胸脇滿而渴者。兼用應{\hbox{\scalebox{0.7}[1]{钅}\kern-0.4em\scalebox{0.7}[1]{童}}}散。脇下硬滿。不大便而嘔者。兼用消塊。(方機)
\item 小柴胡湯以胸脇苦滿为主證。診察之法。令病人仰卧。醫以指頭從其肋骨弓下。沿前胸壁裏面向胸腔按壓。觸知一種抵抗物。而病人覺壓痛。此即小柴胡湯之腹證。然則胸脇苦滿云者。当是肝脾胰三臓之腫脹硬结矣。然肝脾胰并无異狀而肋骨弓下仍有抵抗物觸知者。臨床上所見甚多。是必有他種{\CJKfontspec[Path=ttf/]{TH-Tshyn-P2}𬮦}係。以理推知。殆该部淋巴腺之腫脹硬结也。何則。凡以肋骨弓下抵抗物为主證而用小柴胡湯。治腦病。五官器病。咽喉病。呼吸器病。胸膜病。心臟病。胃腸病。以及肝脾胰腎子宮等病。其病漸癒。則抵抗物亦從而消缩。據{\hbox{\scalebox{0.68}[1]{纟}\kern-0.35em\scalebox{0.64}[1]{巠}}}驗之事実以推其病理。除淋巴系统外。无可説明。盖因上述諸臟器中。一臓乃至數臓之原發病變。其毒害性物質由淋巴及淋巴管之媒介達於隔膜上下。引起该部淋巴结之{\hbox{\scalebox{0.6}[1]{纟}\kern-0.32em\scalebox{0.7}[1]{賣}}}發病變。使之腫脹硬结也。仲師創立小柴胡湯。使原發{\hbox{\scalebox{0.6}[1]{纟}\kern-0.32em\scalebox{0.7}[1]{賣}}}發諸病同時俱治。而以{\hbox{\scalebox{0.6}[1]{纟}\kern-0.32em\scalebox{0.7}[1]{賣}}}發的胸脇苦滿为主證者。取其易於觸知故也。(湯本求真)
\item 默默。昏昏之意。非靜默也。(喻嘉言)
\end{itemize}

傷寒四五日。身体熱。惡風。頸項強。脇下滿。手足温而渴。小柴胡湯主之。99

傷寒。陽脉{\CJKfontspec[Path=ttf/]{TH-Tshyn-P2}𬈧}。陰脉弦。法当腹中急痛。先与小建中湯。不差者。与小柴胡湯。100
\footnote{「小柴胡湯」聖惠方作「大柴胡湯」。}

\begin{itemize}
\item 小建中湯。治裏急。腹皮拘急。及急痛者。(類聚方廣義)
\item 腹中急痛或拘攣者。此其正證也。兼用應{\hbox{\scalebox{0.7}[1]{钅}\kern-0.4em\scalebox{0.7}[1]{童}}}散。若有外闭之證。則非此方之所主治也。(方機)
\item 小建中湯。治產後苦少腹痛。(千金方)
\item 小建中湯治腹痛如神。然腹痛按之便痛。重按卻不甚痛。此只是气痛。重按愈痛而堅者。当自有積也。气痛不可下。下之愈甚。此虛寒證也。此藥偏治腹中虛寒。補血。尤治腹痛。(蘇沈良方)
\end{itemize}

傷寒中風。有柴胡證。但見一證便是。不必悉具。101

凡柴胡湯證而下之。柴胡證不罷者。復与柴胡湯。必蒸蒸而振。卻發熱汗出而解。101

傷寒二三日。心中悸而煩者。小建中湯主之。102

太陽病。過{\hbox{\scalebox{0.68}[1]{纟}\kern-0.35em\scalebox{0.64}[1]{巠}}}十馀日。反再三下之。後四五日。柴胡證{\hbox{\scalebox{0.6}[1]{纟}\kern-0.32em\scalebox{0.7}[1]{賣}}}在。先与小柴胡湯。嘔不止。心下急。其人鬱鬱微煩者。为未解。与大柴胡湯下之則愈。103
\footnote{「嘔不止心下急」除宋本外其它版本均作「嘔止小安」。}

傷寒十三日不解。胸脇滿而嘔。日晡所發潮熱[。已]而微利。此本当柴胡湯下之。不得利。今反利者。知醫以丸藥下之。非其治也。潮熱者。実也。先宜服小柴胡湯。以解其外。後以柴胡加芒消湯主之。104

\begin{itemize}
\item 柴胡加芒硝湯。治小柴胡湯證而苦滿難解者。(方極)
\item 小柴胡湯證而有堅塊者。柴胡加芒硝湯主之。(類聚方廣義)
\end{itemize}

傷寒十三日。過{\hbox{\scalebox{0.68}[1]{纟}\kern-0.35em\scalebox{0.64}[1]{巠}}}。譫語者。内有熱也。当以湯下之。若小便利者。大便当堅而反利。脉調和者。知醫以丸藥下之。非其治也。若自利者。脉当微厥。今反和者。此为内実也。[調胃]承气湯主之。105

太陽病不解。熱结膀胱。其人如狂。血自下。下之即愈。其外不解者。尚未可攻。当先解其外。[宜桂枝湯。]外解已。[但]少腹急结者。乃可攻之。宜桃人承气湯。106
\footnote{「下之即愈」除脉{\hbox{\scalebox{0.68}[1]{纟}\kern-0.35em\scalebox{0.64}[1]{巠}}}外其它本皆作「下者即愈」或「下者愈」。}

\begin{itemize}
\item 師雖曰熱结膀胱。又稱少腹急结。以余多年{\hbox{\scalebox{0.68}[1]{纟}\kern-0.35em\scalebox{0.64}[1]{巠}}}驗。此结当不在膀胱部位。而在下行结腸部位。以指尖沿下行结腸之横徑。向腹底擦過而強按壓之。觸知堅结物。病人訴急痛。此即少腹急结之正證也。急结之大小廣狹長短。種種无定。時或上追於左季脇上及心下部。致上半身之疾。又或下降於左腸骨窩及膀胱部。致下半身之疾。診察之際。必須细意周到也。(湯本求真)
\item 桃人承气湯。治產後惡露不下。喘脹欲死。服之十瘥十。({\hbox{\scalebox{0.65}[1]{纟}\kern-0.35em\scalebox{0.68}[1]{悤}}}病論)
\item 婦人月事沈滯。數月不行。肌肉不減。内{\hbox{\scalebox{0.68}[1]{纟}\kern-0.35em\scalebox{0.64}[1]{巠}}}曰。此名为瘕为沈也。沈者。月事沈滯不行也。急宜服桃人承气湯加当歸。大作剂料服。不過三服立愈。後用四物湯補之。(儒门事親)
\item 桃人承气湯。治血證。小腹急结。上衝者。(方極)
\item 桃人承气湯。治小腹急结。如狂者。胞衣不下。气急息迫者。產後小腹堅痛。惡露不盡。或不大便而煩躁。或譫語者。痢病。小腹急痛者。(方機)
\item 桃人承气湯。治產後惡露{\CJKfontspec[Path=ttf/]{TH-Tshyn-P2}𬈧}滯。脐腹大痛者。胎死腹中。胞衣不出。血暈等諸證亦佳。(方輿輗)
\item 婦人久患頭痛。諸藥不效者。与桃人承气湯。兼用桃花散。(青州治譚)
\item 桃人承气湯。治痢疾身熱。腹中拘急。口乾唇燥。舌色殷红。便膿血者。(類聚方廣義)
\item 桃人承气湯。治血行不利。上衝心悸。小腹拘急。四肢堅痹或痼冷者。(類聚方廣義)
\item 桃人承气湯。治{\hbox{\scalebox{0.68}[1]{纟}\kern-0.35em\scalebox{0.64}[1]{巠}}}水不調。上衝甚。眼中生厚膜。或赤脉怒起。瞼胞赤烂。或齲齒疼痛。小腹急结者。(類聚方廣義)
\item 桃人承气湯。治{\hbox{\scalebox{0.68}[1]{纟}\kern-0.35em\scalebox{0.64}[1]{巠}}}闭。上逆。發狂者。(類聚方廣義)
\item 桃人承气湯。治產後惡露不下。小腹凝结。上衝急迫。心胸不安者。凡產後諸患。多惡露不盡所致。早用此方为佳。(類聚方廣義)
\item 淋家。小腹急结。痛連腰腿。莖中疼痛。小便涓滴不通者。非利水剂所能治。用桃人承气湯。二便快利。痛苦立除。小便癃闭。小腹急结而痛者。打扑疼痛。不能轉側。二便闭澀者亦良。(類聚方廣義)
\item 桃人承气湯,为桂枝茯苓丸證之嚴重者,如実熱之瘀血證伴有急迫上衝症候,或產後惡露極多、胎盤殘留、下肢靜脉血栓重症者等適用本方,有便祕傾向者亦可。應注意必須方与證相合,而非任何人均適用。(矢數道明)
\end{itemize}

傷寒八九日。下之。胸滿。煩。驚。小便不利。譫語。一身[盡重。]不可轉側。柴胡加龙骨牡蛎湯主之。107

傷寒。腹滿。譫語。寸口脉浮而緊。此为肝乘脾。名曰{\hbox{\scalebox{0.6}[1]{纟}\kern-0.32em\scalebox{0.7}[1]{從}}}。当刺期门。108

傷寒。發熱。啬啬惡寒。其人大渴。欲饮水者。其腹必滿。自汗出。小便利。其病欲解。此为肝乘肺。名曰横。当刺期门。109

傷寒。脉浮。醫以火迫劫之。亡陽。[必]驚狂。卧起不安。桂枝去芍藥加蜀漆牡蛎龙骨救逆湯主之。112
\footnote{傷寒。脉浮。而以火逼劫。汗即亡陽。必驚狂。卧起不安。(聖惠方)112}
\begin{itemize}
\item 桂枝去芍藥加蜀漆牡蛎龙骨救逆湯。治桂枝去芍藥湯證而胸腹動劇者。(方極)
\item 驚狂。卧起不安者。或火逆。煩躁。胸腹動劇者。及瘧疾而有上衝者。桂枝去芍藥加蜀漆牡蛎龙骨救逆湯主之。俱兼用紫丸。若有胸脇苦滿之證。則别有主治矣。(方機)
\item 不寐之人。徹夜不得一瞑目。及五六夜必發狂。可恐也。当輒服此方。蜀漆能去心腹之邪積也。淵雷按。徹夜不得眠。即所谓卧起不安。故本方治之。須知仲景書所舉症候。为用藥處方之標準。推而廣之。可以泛應變化无方之病情。(方輿輗)
\item 此證驚狂。卧起不安。由於衝气上逆。胸腹脐下動劇。故用桂枝以降衝逆。用龙牡蜀漆以镇動气。本草谓蜀漆主胸中痰结吐逆。亦因衝气而痰饮上逆也。(傷寒論今釋)
\end{itemize}

傷寒。其脉不弦緊而弱[。弱]者必渴。被火必譫語。[弱者。發熱。脉浮。解之当汗出愈。]113

太陽病。以火熏之。不得汗。其人必躁。到{\hbox{\scalebox{0.68}[1]{纟}\kern-0.35em\scalebox{0.64}[1]{巠}}}不解。必清血。114
\footnote{太陽病。以火蒸之。不得汗者。其人必燥结。若不结。必下清血。其脉躁者。必發黄也。(聖惠方)114}

脉浮。熱甚。而反灸之。此为実。実以虛治。因火而動。咽燥。必吐血。115 

微數之脉。慎不可灸。因火为邪。則为煩逆。追虛逐実。血散脉中。火气雖微。内攻有力。焦骨傷筋。血難復也。116
\footnote{凡微數之脉。不可灸。因熱为邪。必致煩逆。内有損骨傷筋血枯之患。(聖惠方)116}

脉浮。当以汗解。而反灸之。邪无從出。因火而盛。病從腰以下必重而痹。此为火逆。若欲自解。当先煩。煩乃有汗。隨汗而解。何以知之。脉浮。故知汗出当解。116
\footnote{脉当以汗解。反以灸之。邪无所去。因火而盛。病当必重。此为逆治。若欲解者。当發其汗而解也。(聖惠方)116}

燒针令其汗。针處被寒。核起而赤者。必發奔豚。气從少腹上衝心者。灸其核上各一壯。与桂枝加桂湯。117

\begin{itemize}
\item 桂枝加桂湯。治桂枝湯證而上衝劇者。(類聚方廣義)
\end{itemize}

火逆。下之。因燒针。煩躁者。桂枝甘草龙骨牡蛎湯主之。118

傷寒。加温针必驚。119

太陽病。当惡寒。發熱。今自汗出。反不惡寒。發熱。{\CJKfontspec[Path=ttf/]{TH-Tshyn-P2}𬮦}上脉细數。此醫吐之過也。一二日吐之者。腹中饥。口不能食。三四日吐之者。不喜糜粥。欲食冷食。朝食暮吐。此醫吐之所致也。此为小逆。120

病人脉數。數为熱。当消穀引食。而反吐者。以醫發其汗。令陽气微。膈气虛。脉則为數。數为客熱。不能消穀。胃中虛冷。故吐也。122

太陽病。過{\hbox{\scalebox{0.68}[1]{纟}\kern-0.35em\scalebox{0.64}[1]{巠}}}十馀日。心下嗢嗢欲吐而胸中痛。大便反溏。腹微滿。鬱鬱微煩。先[此]時自極吐下者。与[調胃]承气湯。若不尔者不可与。但欲嘔。胸中痛。微溏者。此非柴胡湯證。以嘔。故知極吐下也。123

太陽病六七日。表證{\hbox{\scalebox{0.6}[1]{纟}\kern-0.32em\scalebox{0.7}[1]{賣}}}在。脉微而沈。反不结胸。其人發狂。此熱在下焦。少腹当堅滿。小便自利者。下血乃愈。所以然者。以太陽隨{\hbox{\scalebox{0.68}[1]{纟}\kern-0.35em\scalebox{0.64}[1]{巠}}}。瘀熱在裏故也。抵当湯主之。124

太陽病。身黄。脉沈结。少腹堅。小便不利者。为无血也。小便自利。其人如狂者。血證谛也。抵当湯主之。125

傷寒。有熱。少腹滿。應小便不利。今反利者。为有血也。当下之。宜扺当丸。126
\footnote{「当下之」三字下宋本、玉函有「不可馀藥」四字。}

问曰。病有结胸。有臓结。其狀何如。\\
答曰。按之痛。寸口脉浮。{\CJKfontspec[Path=ttf/]{TH-Tshyn-P2}𬮦}上自沈。为结胸。\\
问曰。何谓臓结。\\
答曰。如结胸狀。饮食如故。時時下利。寸口脉浮。{\CJKfontspec[Path=ttf/]{TH-Tshyn-P2}𬮦}上细沈而緊。为臓结。舌上白胎滑者。難治。128.129

臓结无陽證。不往來寒熱。其人反靜。舌上胎滑者。不可攻也。130

病發於陽而反下之。熱入因作结胸。病發於陰而反下之。因作痞。所以成结胸者。以下之太早故也。131

结胸者。項亦強。如柔痙狀。下之則和。宜大陷胸丸。131

结胸證。脉浮大者。不可下。下之則死。132

结胸證悉具而煩躁者死。133

太陽病。脉浮而動數。浮則为風。數則为熱。動則为痛。數則为虛。頭痛。發熱。微盜汗出。而反惡寒者。表未解也。醫反下之。動數變遲。膈内拒痛。胃中空虛。客气動膈。短气。躁煩。心中懊憹。陽气内陷。心下因堅。則为结胸。大陷胸湯主之。若不结胸。但頭汗出。馀處无汗。齐頸而還。小便不利。身必發黄。134
\footnote{「膈内拒痛」脉{\hbox{\scalebox{0.68}[1]{纟}\kern-0.35em\scalebox{0.64}[1]{巠}}}、千金翼作「頭痛即眩」,玉函作「頭痛則眩」。}

傷寒六七日。结胸熱実。其脉沈緊。心下痛。按之如石堅。大陷胸湯主之。135

傷寒十馀日。熱结在裏。復往來寒熱者。与大柴胡湯。但结胸。无大熱者。此为水结在胸脇。[但]頭微汗出者。与大陷胸湯。136

太陽病。重發汗而復下之。不大便五六日。舌上燥而渴。日晡所小有潮熱。從心下至少腹堅滿而痛不可近。大陷胸湯主之。137

小结胸者。正在心下。按之則痛。其脉浮滑。小陷胸湯主之。138

\begin{itemize}
\item 小陷胸湯。治心下结痛。气喘而闷者。(内臺方議)
\item 小陷胸湯。治食積(即急性胃炎)。痰壅滯而喘急。为末和丸服之。(丹溪心法)
\item 小陷胸湯。治结胸有痰饮之變者。兼用礞石滾痰丸。姑洗。或紫丸。龜背。
\end{itemize}

太陽病二三日。[终]不能卧。但欲起者。心下必结。脉微弱者。此本寒也。而反下之。利止者。必结胸。未止者。四五日復下之。此[作]挾熱利也。139
\footnote{「此本寒也」宋本作「此本有寒分也」,外臺作「本有久寒也」。}

病在陽。当以汗解。反以水潠之或灌之。其熱被劫不得去。[弥更]益煩。皮上粟起。意欲饮水。反不渴。宜服文蛤散。若不差。与五苓散。若寒実结胸。无熱證者。与三物白散。141
\footnote{「其熱被劫不得去」脉{\hbox{\scalebox{0.68}[1]{纟}\kern-0.35em\scalebox{0.64}[1]{巠}}}、千金翼、玉函、外臺作「其熱卻不得去」。}

太陽与少陽并病。頭項強痛。或眩冒。時如结胸。心下痞堅者。当刺大椎第一间。肺腧。肝腧。慎不可發汗。發汗則譫語。譫語則脉弦。譫語五日不止者。当刺期门。142

婦人中風。發熱。惡寒。{\hbox{\scalebox{0.68}[1]{纟}\kern-0.35em\scalebox{0.64}[1]{巠}}}水適來。得之七八日。熱除。脉遲。身涼。胸脇下滿。如结胸狀。譫語。此为熱入血室。当刺期门。隨其虛実而取之。143
\footnote{「隨其虛実而取之」宋本、玉函作「隨其実而瀉之」。}

婦人中風七八日。{\hbox{\scalebox{0.6}[1]{纟}\kern-0.32em\scalebox{0.7}[1]{賣}}}得寒熱。發作有時。{\hbox{\scalebox{0.68}[1]{纟}\kern-0.35em\scalebox{0.64}[1]{巠}}}水適斷。此为熱入血室。其血必结。故使如瘧狀。發作有時。小柴胡湯主之。144

\begin{itemize}
\item 適斷者。謂{\hbox{\scalebox{0.68}[1]{纟}\kern-0.35em\scalebox{0.64}[1]{巠}}}行中得病而斷者也。(類聚方廣義)
\item 婦人在蓐得風。盖四肢苦煩熱。皆自發露所为。若頭痛。与小柴胡湯。頭不痛。但煩熱。与三物黄芩湯。(千金方)
\end{itemize}

婦人傷寒。發熱。{\hbox{\scalebox{0.68}[1]{纟}\kern-0.35em\scalebox{0.64}[1]{巠}}}水適來。晝日明了。暮則譫語。如見鬼狀。此为熱入血室。无犯胃气及上二焦。必自愈。145
\footnote{「了了」宋本、玉函作「明瞭」。「二焦」脉{\hbox{\scalebox{0.68}[1]{纟}\kern-0.35em\scalebox{0.64}[1]{巠}}}作「三焦」。}

傷寒六七日。發熱。微惡寒。肢節煩疼。微嘔。心下支结。外證未去者。柴胡桂枝湯主之。146

發汗多。亡陽。狂語者。不可下。[可]与柴胡桂枝湯。和其榮衛。以通津液。後自愈。

傷寒五六日。已發汗而復下之。胸脇滿。微结。小便不利。渴而不嘔。但頭汗出。往來寒熱。心煩。此为未解。柴胡桂枝乾薑湯主之。147
\footnote{傷寒六日。已發汗及下之。其人胸脇滿。大腸微结。小腸不利而不嘔。但頭汗出。往來寒熱而煩。此为未解。宜小柴胡桂枝湯。(聖惠方)147}

傷寒五六日。頭汗出。微惡寒。手足冷。心下滿。口不欲食。大便堅。其脉细。此为陽微结。必有表。復有裏。沈亦为病在裏。汗出为陽微。假令纯陰结。不得有外證。悉入在裏。此为半在外。半在裏。脉雖沈緊。不得为少陰病。所以然者。陰不得有汗。今頭汗出。故知非少陰也。可与[小]柴胡湯。設不了了者。得屎而解。148

傷寒五六日。嘔而發熱。柴胡湯證具。而以他藥下之。柴胡證仍在者。復与柴胡湯。此雖已下。不为逆也。必蒸蒸而振。卻發熱汗出而解。若心下滿而堅痛者。此为结胸。宜大陷胸湯。若但滿而不痛者。此为痞。柴胡[湯]不復中与也。宜半夏瀉心湯。149

\begin{itemize}
\item 半夏瀉心湯。治心下堅。腹中雷鳴者。(方極)
\item 半夏瀉心湯。治心下痞堅。腹中雷鳴者。嘔而腸鳴。心下痞堅者。俱兼用太蔟。心中煩悸。或怒。或悲伤者。兼用紫丸。(方機)
\item 休息痢。世皆以为難治。盖亦穢物不盡耳。宜服篤落丸。兼用半夏瀉心湯之類。(芳翁醫談)
\item 下利如休息。而无膿血。唯水瀉。時或自止則腹脹。瀉則爽然。而日漸羸惫。面色痿黄。惡心吞酸。時腹自痛者。与半夏瀉心湯。兼用篤落丸为佳。且宜長服。(芳翁醫談)
\item 痢疾腹痛。嘔而心下痞堅。或便膿血者。及饮食湯藥下腹。每漉漉有聲而轉泄者。可选用半夏瀉心湯。甘草瀉心湯。生薑瀉心湯三方。(類聚方廣義)
\item 半夏瀉心湯。治疝瘕積聚。痛浸心胸。心下痞堅。惡心嘔吐。腸鳴或下利者。若大便祕者。兼用消塊丸或陷胸丸。(類聚方廣義)
\item 半夏瀉心湯。主饮邪并结。心下痞堅者。故饮或滯饮之痞堅者不效。因饮邪并结。致嘔吐或哕逆或下利者。皆运用之。有特效。千金冀加附子。即附子瀉心湯之意。乃温散饮邪之成法也。淵雷按。胃炎之富有黏液或有停水者。古人谓之痰饮。此方治胃腸之炎症。故浅田氏云尔。惟兩醫所谓胃炎者。不皆是痰饮。古人所谓痰饮者。不皆是胃炎。不可不知。(方函口訣)
\end{itemize}

太陽与少陽并病而反下之。[成]结胸。心下堅。下利不[復]止。水漿不[肯]下。其人[必]心煩。150

脉浮緊而反下之。緊反入裏。則作痞。按之自濡。但气痞耳。151

太陽中風。下利。嘔逆。表解乃可攻之。其人漐漐汗出。發作有時。頭痛。心下痞堅滿。引脇下痛。乾嘔。短气。汗出。不惡寒。此为表解。裏未和。十棗湯主之。152

\begin{itemize}
\item 朱雀湯。疗久病癖饮。停痰不消。在胸膈上液液。時頭眩痛。苦攣。眼睛身体手足十指甲盡黄。亦疗脇下支滿。饮輒引脇下痛。(外臺)
\item 十棗湯。治病在胸腹。掣痛者。(方極)
\item 十棗湯。治頭痛。心下痞堅。引脇下痛。乾嘔。汗出者。欬煩。胸中痛者。胸背掣痛。不得息者。(方機)
\item 陳无擇三因方以十棗湯藥为末。用棗肉和丸。以治水气。四肢浮腫。上气喘急。大小便不通。盖善变通者也。(汪氏)
\item 十棗湯。兼下水腫腹脹。并酒食積。肠垢積滯。痃癖堅積。蓄熱暴痛。瘧气久不已。或表之正气与邪熱并甚於里。熱極似陰。反寒战。表气入裏。陽厥極深。脉微而绝。并風熱燥甚。结於下焦。大小便不通。実熱腰痛。及小儿熱结。乳癖積熱。作發風潮搐。斑疹熱毒。不能了绝者。(宣明論)
\end{itemize}

太陽病。醫發其汗。遂發熱。惡寒。復下之。則心下痞。此表裏俱虛。陰陽气并竭。无陽則陰獨。復加燒针。因胸煩。面色青黄。膚{\CJKfontspec[Path=ttf/]{TH-Tshyn-P2}𥆧}者。難治。今色微黄。手足温者。易愈。153

心下痞。按之濡。其脉{\CJKfontspec[Path=ttf/]{TH-Tshyn-P2}𬮦}上浮者。大黄[黄連]瀉心湯主之。154

心下痞。而復惡寒。汗出者。附子瀉心湯主之。155

\begin{itemize}
\item 附子瀉心湯。治瀉心湯證而惡寒者。(方極)
\item 中風猝倒者最難治。与附子瀉心湯。间得效。然亦多死。(芳翁醫談)
\item 附子瀉心湯。治瀉心湯證而但欲寐甚者。可以饮食与藥同進而睡。又。手足微冷等證。亦宜此方。(方輿輗)
\item 老人停食。瞀闷暈倒。不省人事。心下滿。四肢厥冷。面无血色。额上冷汗。脉伏如绝。其狀仿佛中風者。谓之食鬱食厥。宜附子瀉心湯。淵雷按。急性胃炎。中醫名曰傷食。時醫例用山楂。雞内金。神曲。麦芽等藥。古方則以芩連为主。諸瀉心湯之證是也。用山楂等藥。不過防止胃内容物之發酵腐敗。必須芩連。方能消除炎症。因發炎部必充血故也。古方時方之优劣。於此可見一斑。(類聚方廣義)
\end{itemize}

本以下之。故心下痞。与瀉心湯。痞不解。其人渴而口燥[煩]。小便不利者。五苓散主之。156

傷寒。汗出。解之後。胃中不和。心下痞堅。乾噫食臭。脇下有水气。腹中雷鳴而利。生薑瀉心湯主之。157

\begin{itemize}
\item 若乾噫食臭。腹中雷鳴。下利或嘔吐者。生薑瀉心湯主之。(方機)
\item 生薑瀉心湯。治半夏瀉心湯證而嘔者。(方極)
\item 生薑瀉心湯。治猝癇。乾嘔。(二神傳)
\item 凡患噫气乾嘔。或嘈雜吞酸。或平日饮食每觉惡心煩闷。水饮升降於脇下者。其人多心下痞硬。或脐上有塊。長服此方。灸五椎至十一椎及章门。日数百壯。兼用消塊丸。硝石。大圓等。自然有效。(類聚方廣義)
\end{itemize}

傷寒中風。醫反下之。其人下利日數十行。穀不化。腹中雷鳴。心下痞堅而滿。乾嘔。心煩。不得安。醫見心下痞。谓病不盡。復下之。其痞益甚。此非结熱。但以胃中虛。客气上逆。故使之堅。甘草瀉心湯主之。158

\begin{itemize}
\item 甘草瀉心湯。治半夏瀉心湯證。而心煩不得安者。(方極)
\item 甘草瀉心湯。治下利不止。乾嘔。心煩者。默默欲眠。目不得闭。起卧不安。不欲饮食。惡闻食臭者。(方機)
\item 甘草瀉心湯不過於半夏瀉心湯方内更加甘草一兩。而其所主治大不同。曰下利日数十行。穀不化。曰乾嘔。心煩。不得安。曰默默欲眠。目不得闭。卧起不安。此皆急迫所使然。故以甘草为君藥。(類聚方廣義)
\item 慢驚風有宜甘草瀉心湯者。(類聚方廣義)
\item 甘草瀉心湯。主胃中不和之下利。故以穀不化。雷鳴。下利为目的。若非穀不化而雷鳴下利者。理中四逆所主也。外台作水穀不化。与清穀異文。可从。又用於产後之口糜瀉。有奇效。此等處。芩連反有健胃之效。(方函山訣)
\item 甘草瀉心湯。治走馬牙疳。特有奇驗。(温知醫談)
\end{itemize}

傷寒。服湯藥。下利不止。心下痞堅。服瀉心湯已。復以他藥下之。利不止。醫以理中与之。利益甚。理中者。理中焦。此利在下焦。赤石脂禹馀粮湯主之。若不止者。当利小便。159

傷寒。吐下[後]發汗。虛煩。脉甚微。八九日。心下痞堅。脇下痛。气上衝咽喉。眩冒。{\hbox{\scalebox{0.68}[1]{纟}\kern-0.35em\scalebox{0.64}[1]{巠}}}脉動愓者。久而成痿。160

傷寒。發汗[或]吐[或]下。解後。心下痞堅。噫气不除者。旋復代赭湯主之。161

\begin{itemize}
\item 本方与三瀉心同主痞堅。而三瀉心重在雷鳴。本方則重在噫气。三瀉心为急性胃腸炎。故用芩連。本方为慢性。故不用。昔賢谓瀉心虛実相伴。本方纯乎虛。有以也。旋復花。代赭石。今人用以治痰。可知此證亦多黏液。凡有黏膜器官之炎症。西醫名卡他。谓滲出黏液。其它器官有黏液時。不致甚苦。胃多黏液則大礙消化。故藥治必先滌除之。(傷寒論今釋)
\item 旋復代赭湯。治生薑瀉心湯證而一等重者。(方函口訣)
\item 病解后。痞堅。噫气。不下利者。旋復代赭湯。下利者。生薑瀉心湯。今用于嘔吐諸證。大便祕结者效。下利不止。嘔吐宿水者亦效。既宜于祕结。又宜于下利。妙在不拘表里。又治哕逆属水饮者。(醫學{\hbox{\scalebox{0.6}[1]{纟}\kern-0.3em\scalebox{0.63}[1]{岡}}}目)
\item 有旋復代赭湯證。其人或欬逆气虛者先服四逆湯。胃寒者先服理中丸。次服旋復代赭湯为良。(活人書)
\end{itemize}

下後。不可更行桂枝湯。汗出而喘。无大熱者。可与麻杏甘石湯。162

太陽病。外證未除而數下之。遂挾熱而利。利下不止。心下痞堅。表裏不解。桂枝人参湯主之。163

傷寒。大下後。復發汗。心下痞。惡寒者。表未解也。不可攻痞。当先解表。表解乃可攻痞。解表宜桂枝湯。攻痞宜大黄黄連瀉心湯。164

傷寒。發熱。汗出。不解。心中痞堅。嘔吐。下利。大柴胡湯主之。165

病如桂枝證。頭不痛。項不強。寸口脉微浮。胸中痞堅。气上衝咽喉。不得息者。此为胸有寒。当吐之。宜瓜蒂散。166

傷寒或吐或下後。七八日不解。熱结在裏。表裏俱熱。時時惡風。大渴。舌上乾燥而煩。欲饮水數升。白虎[加人参]湯主之。168
\footnote{傷寒六日不解。熱结在裏。但熱。時時惡風。大渴。舌乾。煩躁。宜白虎湯。(聖惠方)168}

傷寒。无大熱。口燥渴。心煩。背微惡寒。白虎[加人参]湯主之。169

傷寒。脉浮。發熱。无汗。其表不解者。不可与白虎湯。渴欲饮水。无表證者。白虎[加人参]湯主之。170

太陽与少陽合病。自下利者。与黄芩湯。若嘔者。与黄芩加半夏生薑湯。172

傷寒。胸中有熱。胃中有邪气。腹中痛。欲嘔吐。黄連湯主之。173

傷寒八九日。風濕相摶。身体疼煩。不能自轉側。不嘔。不渴。脉浮虛而{\CJKfontspec[Path=ttf/]{TH-Tshyn-P2}𬈧}者。桂枝附子湯主之。若其人大便堅。小便自利者。[白]术附子湯主之。174

風濕相摶。骨節疼煩。掣痛。不得屈伸。近之則痛劇。汗出。短气。小便不利。惡風。不欲去衣。或身微腫。甘草附子湯主之。175

傷寒。脉浮滑。此以表有熱。裏有寒。白虎湯主之。176

傷寒。脉结代。心動悸。炙甘草湯主之。177

\chapter{辨陽明病}

陽明之为病。胃家実是也。180

问曰。病有太陽陽明。有正陽陽明。有微陽陽明。何谓也。\\
答曰。太陽陽明者。脾约是也。正陽陽明者。胃家実是也。微陽陽明者。發汗或利小便。胃中燥。大便難是也。179
\footnote{「微陽」宋本作「少陽」。宋本「胃中燥」下有「煩実」二字。}

问曰。何缘得陽明病。\\
答曰。太陽病。發汗或下之。亡其津液。胃中乾燥。因轉屬陽明。不更衣。内実。大便難者。为陽明病也。181

问曰。陽明病外證云何。\\
答曰。身熱。汗出。不惡寒。[但]反惡熱。182
\footnote{陽明病外證。身熱。汗出。而不惡寒。但惡熱。宜柴胡湯。(聖惠方)182}

问曰。病有得之一日。不發熱而惡寒者。何。\\
答曰。然雖一日。惡寒自罷。即汗出。惡熱也。183\\
问曰。惡寒何故自罷。\\
答曰。陽明居中。主土。万物所歸。无所復傳。始雖惡寒。二日自止。此为陽明病也。184

本太陽。初得病時。發其汗。汗先出[。復]不徹。因轉屬陽明。185
\footnote{太陽病而發汗。汗雖出。復不解。不解者。轉屬陽明也。宜麻黄湯。(聖惠方)185}

傷寒。發熱。无汗。嘔不能食。而反汗出濈濈然。是为轉屬陽明。185

傷寒。脉浮而缓。手足自温。是为系在太陰。太陰[身]当發黄。若小便自利者。不能發黄。至七八日。大便堅者。为屬陽明。187

傷寒轉系陽明者。其人濈濈然微汗出也。188

陽明中風。口苦。咽乾。腹滿。微喘。發熱。惡寒。脉浮緊。[若]下之。則腹滿。小便難也。189

陽明病。能食为中風。不能食为中寒。190

陽明病。中寒。不能食。小便不利。手足濈然汗出。此欲作堅瘕。必大便頭堅後溏。所以然者。胃中冷。水穀不别故也。191
\footnote{陽明中寒。不能食。小便不利。手足濈然汗出。欲作堅症也。所以然者。胃中水穀不化故也。宜桃人承气湯。(聖惠方)191}

陽明病。初欲食。小便反不利。大便自調。其人骨節疼。翕翕如有熱狀。奄然發狂。濈然汗出而解。此为水不勝穀气。与汗共并。脉緊則愈。192

陽明病欲解時。從申盡戍。193

陽明病。不能食。下之不解。攻其熱必噦。所以然者。胃中虛冷故也。[以其人本虛。故攻其熱必噦。]194
\footnote{陽明病。能食。下之不解。其人不能食。攻其熱必噦者。胃中虛冷也。宜半夏湯。(聖惠方)194}

陽明病。脉遲。食難用饱。饱即發煩。頭眩。必小便難。此欲作穀疸。雖下之。腹滿如故。所以然者。脉遲故也。195
\footnote{陽明病。脉遲。發熱。頭眩。小便難。此欲作榖疸。下之必腹滿。宜柴胡湯。(聖惠方)195}

陽明病当多汗。而反无汗。其身如虫行皮中狀。此以久虛故也。196

陽明病。反无汗。但小便利。二三日。嘔而欬。手足厥者。其人頭必痛。若不嘔。不欬。手足不厥者。頭不痛。197

陽明病。但頭眩。不惡寒。故能食而欬。其人咽必痛。若不欬者。咽不痛。198

陽明病。脉浮而緊者。必潮熱。發作有時。但浮者。必盜汗出。201

陽明病。无汗。小便不利。心中懊憹者。必發黄。199
\footnote{陽明病。无汗。小便不利。心中熱壅。必發黄也。宜茵陳湯。(聖惠方)199}

陽明病。被火。額上微汗出。小便不利者。必發黄。200
\footnote{陽明病。被火灸。其額上微有汗出。小便不利者。必發黄也。宜茵陳湯。(聖惠方)200}

陽明病。口燥。但欲漱水。不欲咽者。必衄。202

陽明病。本自汗出。醫復重發汗。病已差。其人微煩不了了者。此必大便堅故也。以亡津液。胃中乾燥。故令其堅。当问其小便日幾行。若本日三四行。今日再行者。知必大便不久出。今为小便數少。津液当還入胃中。故知不久必大便也。203

夫病陽多者熱。下之則堅。汗出多極。發其汗亦堅。
\footnote{夫病陽多熱。下之則堅。汗出多極。發其汗亦堅。(玉函)\\欬而小便利。若失小便。不可攻其表。汗出則厥逆冷。汗出多極。發其汗亦堅。(脉{\hbox{\scalebox{0.68}[1]{纟}\kern-0.35em\scalebox{0.64}[1]{巠}}})}

傷寒。嘔多。雖有陽明證。不可攻之。204

陽明病。心下堅滿者。不可攻之。攻之遂利不止者死。利止者愈。205

陽明病。面合赤色者。不可攻之。[攻之]必發熱。色黄。小便不利也。206

陽明病。不吐下而心煩者。可与[調胃]承气湯。207

陽明病。脉遲。雖汗出。不惡寒。其身必重。短气。腹滿而喘。有潮熱者。此外欲解。可攻裏也。若手足濈然汗出者。此大便已堅。[大]承气湯主之。若汗多。微發熱。惡寒者。外未解也。其熱不潮。未可与承气湯。若腹大滿而不大便者。可与小承气湯。微和其胃气。勿令至大下。208

陽明病。潮熱。大便微堅者。可与大承气湯。不堅者。不可与之。若不大便六七日。恐有燥屎。欲知之法。可少与小承气湯。若腹中轉失气者。此有燥屎。乃可攻之。若不轉失气者。此但頭堅後溏。不可攻之。攻之必腹滿。不能食也。欲饮水者。与水即噦。其後發熱者。必大便復堅而少也。以小承气湯和之。若不轉失气者。慎不可攻之。209
\footnote{陽明病。有潮熱。大便堅。可与承气湯。若有结燥。乃可徐徐攻之。若无壅滯。不可攻之。攻之者。必腹滿。不能食。欲饮水者即噦。其候發熱。必腹堅脹。宜与小承气湯。(聖惠方)209}

夫実則譫語。虛則鄭聲。鄭聲者。重語也。直視。譫語。喘滿者死。下利者亦死。210
\footnote{「鄭聲者重語也」外臺作「鄭聲重語也」五字寫作雙行小字注釋。}

發汗多。重發汗。亡其陽。[若]譫語。脉短者死。脉自和者不死。211

傷寒吐下後不解。不大便五六日。至十馀日。日脯所發潮熱。不惡寒。獨語。如見鬼神之狀。若劇者。發則不識人。循衣妄撮。怵惕不安。微喘。直視。脉弦者生。{\CJKfontspec[Path=ttf/]{TH-Tshyn-P2}𬈧}者死。微者但發熱。譫語。[大]承气湯主之。若一服利。止後服。212

陽明病。其人多汗。津液外出。胃中燥。大便必堅。堅則譫語。[小]承气湯主之。[若一服譫語止。莫復服。]213

陽明病。譫語。發潮熱。脉滑疾者。[小]承气湯主之。因与承气湯一升。腹中轉失气者。復与一升。若不轉失气者。勿更与之。明日又不大便。脉反微{\CJKfontspec[Path=ttf/]{TH-Tshyn-P2}𬈧}者。此为裏虛。为難治。不可復与承气湯。214
\footnote{「譫語」玉函、千金翼、聖惠方作「譫語妄言」。}

陽明病。譫語。有潮熱。反不能食者。[胃中]必有燥屎五六枚。若能食者。但堅耳。[大]承气湯主之。215
\footnote{「胃中」二字除宋本外其它版本均无。「大承气湯主之」宋本作「宜大承气湯下之」。}

陽明病。下血。譫語者。此为熱入血室。但頭汗出。当刺期门。隨其実而瀉之。濈然汗出則愈。216

汗出。譫語者。以有燥屎在胃中。此風也。[須下者。]過{\hbox{\scalebox{0.68}[1]{纟}\kern-0.35em\scalebox{0.64}[1]{巠}}}乃可下之。下之若早。語言必亂。以表虛裏実故也。下之則愈。宜[大]承气湯。217
\footnote{「大承气湯」一云「大柴胡湯」。}

傷寒四五日。脉沈而喘滿。沈为在裏。反發其汗。津液越出。大便为難。表虛裏実。久則譫語。218

三陽合病。腹滿。身重。難以轉側。口不仁。面垢。譫語。遺溺。發汗則譫語[甚]。下之則額上生汗。手足厥冷。自汗。白虎湯主之。219
\footnote{「自汗」宋本作「若自汗出者」。}

二陽并病。太陽證罷。但發潮熱。手足漐漐汗出。大便難而譫語者。下之則愈。宜[大]承气湯。220

陽明病。脉浮而緊。咽乾。口苦。腹滿而喘。發熱。汗出。不惡寒。反惡熱。身重。若發汗則躁。心憒憒。反譫語。若加温针。必怵惕。煩躁。不得眠。若下之。則胃中空虛。客气動膈。心中懊憹。舌上胎者。梔子[豉]湯主之。若渴欲饮水。口乾舌燥者。白虎[加人参]湯主之。若脉浮。發熱。渴欲饮水。小便不利者。豬苓湯主之。221.222.223
\footnote{陽明病。脉浮。咽乾。口苦。腹滿。汗出而喘。不惡寒。反惡熱。心躁。譫語。不得眠。胃虛。客熱。舌燥。宜梔子湯。(聖惠方)221}

陽明病。汗出多而渴者。不可与豬苓湯。以汗多。胃中燥。豬苓湯復利其小便故也。224
\footnote{陽明病。汗出多而渴者。不可与豬苓湯。汗多者。胃中燥也。汗少者。宜与豬苓湯利其小便。(聖惠方)224}

脉浮而遲。表熱裏寒。下利清穀者。四逆湯主之。225

[陽明病。]若胃中虛冷。不能食者。饮水即噦。226

脉浮。發熱。口乾。鼻燥。能食者。則衄。227
\footnote{脉浮。發熱。口鼻中燥。能食者。必衄。宜黄芩湯。(聖惠方)227}

陽明病。下之。其外有熱。手足温。不结胸。心中懊憹。饥不能食。但頭汗出。梔子[豉]湯主之。228

陽明病。發潮熱。大便溏。小便自可。胸脇滿不去。小柴胡湯主之。229
\footnote{陽明病。發潮熱。大便溏。小便自利。胸脇煩滿不止。宜小柴胡湯。(聖惠方)229}

陽明病。脇下堅滿。不大便而嘔。舌上白胎者。可与小柴胡湯。上焦得通。津液得下。胃气因和。身濈然汗出而解。230
\footnote{陽明病。脇下堅滿。大便祕而嘔。口燥。宜柴胡湯。(聖惠方)230}

陽明中風。脉弦浮大而短气。腹都滿。脇下及心痛。久按之。气不通。鼻乾。不得汗。嗜卧。一身及目悉黄。小便難。有潮熱。時時噦。耳前後腫。刺之小差。外不解。病過十日。脉{\hbox{\scalebox{0.6}[1]{纟}\kern-0.32em\scalebox{0.7}[1]{賣}}}浮者。与[小]柴胡湯。脉但浮。无馀證者。与麻黄湯。不溺。腹滿加噦者。不治。231.232
\footnote{陽明病。中風。其脉浮大。短气心痛。鼻乾嗜卧。不得汗。一身悉黄。小便難。有潮熱而噦。耳前後腫。刺之雖小差。外若不解。宜柴胡湯。(聖惠方)231.232}

陽明病。自汗出。若發汗。小便自利者。此为[津液]内竭。雖堅。不可攻之。当須自欲大便。宜蜜煎。導而通之。若土瓜根及豬膽汁。皆可以導。233

陽明病。脉遲。汗出多。微惡寒者。表未解也。可發汗。宜桂枝湯。234

陽明病。脉浮。无汗。其人必喘。發汗則愈。宜麻黄湯。235
\footnote{「其人必喘」宋本作「而喘者」。}

陽明病。發熱。汗出者。此为熱越。不能發黄也。但頭汗出。身无汗。齐頸而還。小便不利。渴引水漿者。此为瘀熱在裏。身必發黄。茵陳[蒿]湯主之。236
\footnote{「熱越」聖惠方作「熱退」。}

陽明證。其人喜忘。必有畜血。所以然者。本有久瘀血。故令喜忘。屎雖堅。大便反易。其色必黑。抵当湯主之。237

陽明病。下之。心中懊憹而煩。胃中有燥屎者。可攻。其人腹微滿。頭堅後溏者。不可攻之。若有燥屎者。宜[大]承气湯。238

病者五六日不大便。{\hbox{\scalebox{0.6}[1]{纟}\kern-0.3em\scalebox{0.63}[1]{堯}}}脐痛。躁煩。發作有時。此有燥屎。故使不大便也。239

病者煩熱。汗出即解。復如瘧狀。日晡所發[熱]者。屬陽明。脉実者。当下之。脉浮虛者。当發其汗。下之宜[大]承气湯。發汗宜桂枝湯。240
\footnote{「大承气湯」一云「大柴胡湯」。\\汗出後則暫解。日晡則復發。脉実者。当宜下之。(聖惠方)240}

大下後。六七日不大便。煩不解。腹滿痛者。此有燥屎。所以然者。本有宿食故也。宜[大]承气湯。241
\footnote{「此有燥屎所以然者」玉函作「有燥屎者」四字。}

病者小便不利。大便乍難乍易。時有微熱。喘冒。不能卧者。有燥屎故也。宜[大]承气湯。242
\footnote{「喘冒」玉函、千金翼作「怫鬱」。}

食穀欲嘔者。屬陽明也。[吳]茱萸湯主之。得湯反劇者。屬上焦。243

\begin{itemize}
\item 得湯反劇者。更与此方。則嘔气自止。但一帖藥。二三次服之为佳。(類聚方廣義)
\item 噦逆。有宜吳茱萸湯者。按。外臺曰。療食後醋咽多噫。(類聚方廣義)
\item 霍亂。不吐不下。心腹劇痛欲死者。先用備急丸或紫丸。繼投此方則无不吐者。吐則无不下者。已得快吐下。則苦楚脱然而除。其效至速。不可不知。(類聚方廣義)
\item 吳茱萸湯。治人食畢噫醋及醋心。(肘後方)
\item 吳茱萸湯。治胸滿。心下痞硬。嘔者。(方極)
\item 吳茱萸湯。治食穀欲嘔者。方意以气逆为主證。又治吐利。手足厥冷。煩躁者。乾嘔。吐涎沫。頭痛者。兼用礞石滾痰丸。嘔而胸滿者。兼用紫丸。腳气上攻而嘔者。兼用紫丸。若水腫而嘔者。非此湯之所治也。(方機)
\end{itemize}

太陽病。寸[口]缓。{\CJKfontspec[Path=ttf/]{TH-Tshyn-P2}𬮦}[上小]浮。尺[中]弱。其人發熱。汗出。復惡寒。不嘔。但心下痞者。此为醫下之故也。若不下。其人不惡寒而渴者。此轉屬陽明也。小便數者。大便必堅。不更衣十日。无所苦也。[渴]欲饮水者。少少与之。但以法救之。渴者。宜五苓散。244

脉陽微而汗出少者。为自和。汗出多者。为太過。陽脉実。因發其汗。出多者。亦为太過。太過者。陽绝於内。亡津液。大便因堅。245

脉浮而芤。浮則为陽。芤則为陰。浮芤相摶。胃气生熱。其陽則绝。246

跌陽脉浮而{\CJKfontspec[Path=ttf/]{TH-Tshyn-P2}𬈧}。浮則胃气強。{\CJKfontspec[Path=ttf/]{TH-Tshyn-P2}𬈧}則小便數。浮{\CJKfontspec[Path=ttf/]{TH-Tshyn-P2}𬈧}相摶。大便則堅。其脾为约。麻子人丸主之。247

太陽病三日。發汗不解。蒸蒸發熱者。屬胃也。[調胃]承气湯主之。248

傷寒吐後。腹脹滿者。与[調胃]承气湯。249

太陽病吐下發汗後。微煩。小便數。大便因堅。可与小承气湯。和之則愈。250

得病二三日。脉弱。无太陽柴胡證。煩躁。心下堅。至四日。雖能食。以[小]承气湯少与。微和之。令小安。至六日。与承气湯一升。若不大便六七日。小便少者。雖不大便。但頭堅後溏。未定成堅。攻之必溏。当須小便利。屎定堅。乃可攻之。宜[大]承气湯。251
\footnote{「雖不大便」宋本作「雖不受食」,玉函作「雖不能食」。}

傷寒六七日。目中不了了。睛不和。无表裏證。大便難。身微熱者。此为実也。急下之。宜[大]承气湯。252
\footnote{傷寒六七日。目中瞳子不明。无外證。大便難。微熱者。此为実。宜急下之。(聖惠方)252\\「大承气湯」一云「大柴胡湯」。}

陽明病。發熱。汗多者。急下之。宜[大]承气湯。253
\footnote{「大承气湯」一云「大柴胡湯」。}

發汗不解。腹滿痛者。急下之。宜[大]承气湯。254
\footnote{傷寒病。腹中滿痛者为実。当宜下之。(聖惠方)254\\「大承气湯」一云「大柴胡湯」。}

腹滿不減。減不足言。当下之。宜[大]承气湯。255
\footnote{「大承气湯」一云「大柴胡湯」。}

陽明与少陽合病而利。脉不負者为順。負者。失也。互相克賊。名为負。256

脉滑而數者。有宿食也。当下之。宜[大]承气湯。256
\footnote{傷寒。脉數而滑者。有宿食。当下之則愈。(聖惠方)256\\脉數而滑者。実也。此有宿食也。当下之。宜大承气湯。(金匱要略)256\\「大承气湯」一云「大柴胡湯」。}

病人无表裏證。發熱七八日。雖脉浮數。可下之。[宜大柴胡湯。]假令下已。脉數不解。合熱則消穀善饥。至六七日。不大便者。有瘀血。宜抵当湯。若脉數不解而下不止。必挾熱。便膿血。257.258

傷寒。發汗已。身目为黄。所以然者。寒濕相摶。在裏不解故也。[以为非瘀熱而不可下。当於寒湿中求之。]259

傷寒七八日。身黄如橘[子色]。小便不利。腹微滿者。茵陳[蒿]湯主之。260

傷寒。身黄。發熱。梔子蘗皮湯主之。261

傷寒。瘀熱在裏。身必發黄。麻黄連軺赤小豆湯主之。262

\chapter{辨少陽病}

少陽之为病。口苦。咽乾。目眩也。263
\footnote{傷寒三日。少陽受病。口苦乾燥目眩。宜柴胡湯。(聖惠方)263}

少陽中風。兩耳无所闻。目赤。胸中滿而煩者。不可吐下。吐下則悸而驚。264

傷寒。脉弦细。頭痛。發熱者。屬少陽。少陽不可發汗。發汗則譫語。此屬胃。胃和即愈。胃不和。煩而悸。265

太陽病不解。轉入少陽。脇下堅滿。乾嘔。不能食。往來寒熱。尚未吐下。脉沈緊者。可与小柴胡湯。若己吐下發汗温针。譫語。柴胡湯證罷。此为壞病。知犯何逆。以法治之。266.267

三陽合病。脉浮大。上{\CJKfontspec[Path=ttf/]{TH-Tshyn-P2}𬮦}上。但欲寐。目合則汗。268

傷寒六七日。无大熱。其人躁煩。此为陽去入陰故也。269

傷寒三日。三陽为盡。三陰当受邪。其人反能食。而不嘔。此为三陰不受邪也。270

傷寒三日。少陽脉小者。[为]欲已。271

少陽病欲解時。從寅盡辰。272

\chapter{辨太陰病}

太陰之为病。腹滿而吐。食不下。下之益甚。時腹自痛。胸下结堅。273
\footnote{太陰之为病。腹滿而吐。食不下。自利益甚。時腹自痛。若下之。必胸下结硬。(宋本)273}

太陰病。脉浮者。可發汗。宜桂枝湯。276

太陰中風。四肢煩疼。[脉]陽微陰{\CJKfontspec[Path=ttf/]{TH-Tshyn-P2}𬈧}而長者。为欲愈。274

太陰病欲解時。從亥盡丑。275

自利不渴者。屬太陰。以其臓有寒故也。当温之。宜四逆輩。277

傷寒。脉浮而缓。手足自温者。系在太陰。太陰当發身黄。若小便自利者。不能發黄。至七八日。雖暴煩。下利日十馀行。必自止。所以自止者。脾家実。腐穢已去故也。278
\footnote{傷寒。脉浮而缓。手足自温。是为系在太陰。小便不利。其人当發黄。宜茵陳湯。太陰病不解。雖暴煩。下利十馀行而自止。所以自止者。脾家実。腐穢已去故也。宜橘皮湯。278(聖惠方)}

[本]太陽病。醫反下之。因尔腹滿時痛者。屬太陰也。桂枝加芍藥湯主之。大実痛者。桂枝加大黄湯主之。279
\footnote{太陰病。下之後。腹滿時痛。宜桂心芍藥湯。若太実腹痛者。宜承气湯下之。(聖惠方)279}

\begin{itemize}
\item 桂枝加芍藥湯。治桂枝湯證而腹拘攣甚者。(類聚方廣義)
\item 桂枝加芍藥湯。加附子。名桂枝加芍藥附子湯。治桂枝加芍藥湯證而惡寒者。又治腰腳攣急。冷痛。惡寒者。(類聚方廣義)
\end{itemize}

太陰为病。脉弱。其人{\hbox{\scalebox{0.6}[1]{纟}\kern-0.32em\scalebox{0.7}[1]{賣}}}自便利。設当行大黄芍藥者。宜減之。以其人胃气弱。易動故也。280

\chapter{辨少陰病}

少陰之为病。脉微细。但欲寐。281

少陰病。欲吐不吐。心煩。但欲寐。五六日。自利而渴者。屬少陰。虛故引水自救。若小便色白者。少陰病形悉具。所以然者。以下焦虛寒。不能制水。故白也。282
\footnote{傷寒五日。少陰受病。其脉微细。但欲寐。其人欲吐而不煩。五日自利而渴者。屬陰虛。故引水以自救。小便白而利者。下焦有虛寒。故不能制水。而小便白也。宜龙骨牡蛎湯。(聖惠方)281.282\\「欲吐不吐心煩」千金翼作「欲吐而不煩」。}

病人脉陰陽俱緊。反汗出者。亡陽也。此屬少陰。法当咽痛而復吐利。283

少陰病。欬而下利。譫語者。被火气劫故也。小便必難。以強責少陰汗也。284
\footnote{少陰病。欬而下利。譫語。是为心臓有積熱故也。小便必難。宜服豬苓湯。(聖惠方)284}

少陰病。脉细沈數。病为在裏。不可發汗。285

少陰病。脉微。不可發汗。亡陽故也。陽已虛。尺中弱{\CJKfontspec[Path=ttf/]{TH-Tshyn-P2}𬈧}者。復不可下之。286

少陰病。脉緊。至七八日。自下利。脉暴微。手足反温。脉緊反去。此为欲解。雖煩。下利。必自愈。287

少陰病。下利。若利自止。惡寒而踡。手足温者。可治。288

少陰病。惡寒而踡。時自煩。欲去衣被者。可治。289

少陰中風。脉陽微陰浮者。为欲愈。290

少陰病欲解時。從子盡寅。291

少陰病八九日。一身手足盡熱者。以熱在膀胱。必便血也。293

少陰病。吐。利。手足不逆[冷]。反發熱者。不死。脉不至者。灸少陰七壯。292

少陰病。但厥。无汗。而強發之。必動其血。未知從何道出。或從口鼻。或從[耳]目出。是为下厥上竭。为難治。294

少陰病。惡寒。身踡而利。手足逆[冷]者。不治。295

少陰病。下利止而頭眩。時時自冒者死。297

少陰病。吐。利。躁煩。四逆者死。296

少陰病。四逆。惡寒而踡。脉不至。不煩而躁者死。298

少陰病六七日。息高者死。299

少陰病。脉微细沈。但欲卧。汗出不煩。自欲吐。[至]五六日。自利。復煩燥。不得卧寐者死。300

少陰病。始得之。反發熱。脉沈者。麻黄细辛附子湯主之。301

少陰病。得之二三日。麻黄附子甘草湯微發汗。以二三日无證。故微發汗。302
\footnote{「无證」玉函作「无裏證」。}

少陰病。得之二三日以上。心中煩。不得卧。黄連阿膠湯主之。303

少陰病。得之一二日。口中和。其背惡寒者。当灸之。附子湯主之。304
\footnote{少陰病。其人雖裏和。其病惡寒者。宜灸之。(聖惠方)304}

少陰病。身体痛。手足寒。骨節痛。脉沈者。附子湯主之。305

少陰病。下利。便膿血。桃花湯主之。306

少陰病二三日至四五日。腹痛。小便不利。下利不止。便膿血。桃花湯主之。307

少陰病。下利。便膿血者。可刺。308

少陰病。吐利。手足逆。煩躁欲死者。[吳]茱萸湯主之。309

少陰病。下利。咽痛。胸滿。心煩。豬膚湯主之。309

少陰病二三日。咽痛者。可与甘草湯。不差。与桔梗湯。311

少陰病。咽中傷。生瘡。不能語言。聲不出。苦酒湯主之。312

少陰病。咽中痛。半夏散及湯主之。313

少陰病。下利。白通湯主之。314

少陰病。下利。脉微。服白通湯。利不止。厥逆。无脉。乾[嘔。]煩者。白通加豬膽汁湯主之。服湯脉暴出者死。微{\hbox{\scalebox{0.6}[1]{纟}\kern-0.32em\scalebox{0.7}[1]{賣}}}者生。315
\footnote{少陰病。下利。服白通湯。止後。厥逆。无脉。煩躁者。宜白通豬苓湯。其脉暴出者死。微微{\hbox{\scalebox{0.6}[1]{纟}\kern-0.32em\scalebox{0.7}[1]{賣}}}出者生。(聖惠方)315\\「利不止」脉{\hbox{\scalebox{0.68}[1]{纟}\kern-0.35em\scalebox{0.64}[1]{巠}}}作「下利止」,聖惠方作「止後」。}

少陰病。二三日不已。至四五日。腹痛。小便不利。四肢沈重疼痛而利。此为有水气。其人或欬。或小便[自]利。或下利。或嘔。玄武湯主之。316

少陰病。下利清穀。裏寒外熱。手足厥逆。脉微欲绝。身反不惡寒。其人面赤。或腹痛。或乾嘔。或咽痛。或利止而脉不出。通脉四逆湯主之。317
\footnote{「身反不惡寒」千金翼、聖惠方作「身反惡寒」。「或利止而脉不出」聖惠方作「或時利止而脉不出者」。}

少陰病。四逆。其人或欬。或悸。或小便不利。或腹中痛。或泄利下重。四逆散主之。318

少陰病。下利六七日。欬而嘔。渴。心煩不得眠。豬苓湯主之。319
\footnote{少陰病。下利。咳而嘔。煩渴。不得眠臥。宜豬苓湯。(聖惠方)319}

少陰病。得之二三日。口燥。咽乾者。急下之。宜[大]承气湯。320

少陰病。下利清水。色青者。心下必痛。口乾燥者。可下之。宜[大]承气湯。321
\footnote{「可下之」玉函、聖惠方、外臺作「急下之」。}

少陰病六七日。腹滿。不大便者。急下之。宜[大]承气湯。322

少陰病。脉沈者。急温之。宜四逆湯。323

少陰病。其人饮食入則吐。心中嗢嗢欲吐。復不能吐。始得之。手足寒。脉弦遲。此胸中実。不可下也。当吐之。若膈上有寒饮。乾嘔者。不可吐。当温之。宜四逆湯。324

少陰病。下利。脉微{\CJKfontspec[Path=ttf/]{TH-Tshyn-P2}𬈧}。嘔而汗出。必數更衣。反少者。当温其上。灸之。325

\chapter{辨厥陰病}

厥陰之为病。消渴。气上撞[心]。心中疼熱。饥而不欲食。甚者食則吐[蛔]。下之利不止。326

厥陰中風。其脉微浮为欲愈。不浮为未愈。327

厥陰病欲解時。從丑盡卯。328

\begin{itemize}
\item 下文三百三十五條云。期之旦日夜半愈。夜半是子上。既非丑。更非卯。矛盾若斯。不足为法明矣。(傷寒論今釋)
\end{itemize}

厥陰病。渴欲饮水者。少少与之即愈。329
\footnote{傷寒六日。渴欲饮水者。宜豬苓湯。(聖惠方)329}

\chapter{辨厥利嘔噦}

諸四逆。厥者。不可下之。虛家亦然。330
\footnote{諸四逆。厥者。不可吐之。虛家亦然。330}

傷寒。先厥後發熱而利者。必自止。見厥復利。331

\begin{itemize}
\item 厥利并作。其後厥止而發熱者。利必自止。熱止復厥。則又下利。舊注皆如此解。盖據次條及三百三十七條而言。然於原文而字者字。頗不穩貼。且厥熱互發之病。実未之見也。故本篇厥熱諸條。皆不可強解。(傷寒論今釋)
\end{itemize}

傷寒。始發熱六日。厥反九日而利。凡厥利者。当不能食。今反能食。恐为除中。食以黍餅。不發熱者。知胃气尚在。必愈。恐暴熱來出而復去也。後日脉之。其熱{\hbox{\scalebox{0.6}[1]{纟}\kern-0.32em\scalebox{0.7}[1]{賣}}}在者。期之旦日夜半愈。所以然者。本發熱六日。厥反九日。復發熱三日。并前六日。亦为九日。与厥相應。故期之旦日夜半愈。後三日脉之而脉數。其熱不罷者。此为熱气有馀。必發癰膿。332

\begin{itemize}
\item 未嘗目驗此種病。古人醫案中亦未有此種病。猶是纸上空談耳。山田氏云。上三條系後人之言。当刪之。(傷寒論今釋)
\end{itemize}

傷寒。脉遲六七日。而反与黄芩湯徹其熱。脉遲为寒。而与黄芩湯復除其熱。腹中應冷。当不能食。今反能食。此为除中。必死。333

傷寒。先厥後發熱。下利必自止。而反汗出。咽中痛者。其喉为痹。發熱。无汗。而利必自止。若不止。必便膿血。便膿血者。其喉不痹。334

傷寒一二日至四五日。厥者。必發熱。前熱者後必厥。厥深者熱亦深。厥微者熱亦微。厥應下之。而反發汗者。必口傷烂赤。335
\footnote{「前熱者後必厥」脉{\hbox{\scalebox{0.68}[1]{纟}\kern-0.35em\scalebox{0.64}[1]{巠}}}、玉函、千金翼作「前厥者後必熱」。}

凡厥者。陰陽气不相順接。便为厥。厥者。手足逆冷是也。337

傷寒。病厥五日。熱亦五日。設六日当復厥。不厥者自愈。厥终不過五日。以熱五日。故知自愈。336

傷寒。脉微而厥。至七八日。膚冷。其人躁无暫安時者。此为臓厥。非蛔厥也。蛔厥者。其人当吐蛔。今病者靜。而復時煩。此为臓寒。蛔上入膈。故煩。須臾復止。得食而嘔。又煩者。蛔闻食臭出。其人常自吐蛔。蛔厥者。烏梅丸主之。338

傷寒。熱少。厥微。指頭寒。默默不欲食。煩躁。數日。小便利。色白者。此熱除也。欲得食。其病为愈。若厥而嘔。胸脇煩滿者。其後必便血。339

病者手足厥冷。言我不结胸。小腹滿。按之痛。此冷结在膀胱{\CJKfontspec[Path=ttf/]{TH-Tshyn-P2}𬮦}元也。340
\footnote{「小腹」千金翼作「少腹」。}

傷寒。發熱四日。厥反三日。復[發]熱四日。厥少熱多。其病当愈。四日至七日。熱不除者。必便膿血。341

傷寒。厥四日。熱反三日。復厥五日。其病为進。寒多熱少。陽气退。故为進。342

傷寒六七日。脉微。手足厥[冷]。煩躁。灸其厥陰。厥不還者死。343
\footnote{「脉微」千金翼、聖惠方作「其脉數」。}

傷寒。[發熱。]下利。厥逆。躁不得卧者死。344
\footnote{「發熱」二字宋本、玉函有,脉{\hbox{\scalebox{0.68}[1]{纟}\kern-0.35em\scalebox{0.64}[1]{巠}}}、千金翼无。}

傷寒。發熱。下利至[甚。]厥不止者死。345

傷寒六七日。不利。忽發熱而利。其人汗出不止者死。有陰无陽故也。346
\footnote{傷寒。厥逆六七日。不利。便發熱而利者生。其人汗出。利不止者死。但有陰无陽故也。(脉{\hbox{\scalebox{0.68}[1]{纟}\kern-0.35em\scalebox{0.64}[1]{巠}}})346\\傷寒六七日。不利。便發熱而利。其人汗出不止者死。有陰无陽故也。(千金翼。宋本)346}

傷寒五六日。不结胸。腹濡。脉虛。復厥者。不可下。此为亡血。[下之]死。347

傷寒。發熱而厥七日。下利者。为難治。348

傷寒。脉促。手足厥逆。可灸之。349

傷寒。脉滑而厥者。裏有熱也。白虎湯主之。350

手足厥寒。脉细欲绝者。当歸四逆湯主之。若其人内有久寒者。当歸四逆加吳茱萸生薑湯主之。351.352

大汗出。熱不去。内拘急。四肢疼。又下利。厥逆而惡寒。四逆湯主之。353
\footnote{「又下利」脉{\hbox{\scalebox{0.68}[1]{纟}\kern-0.35em\scalebox{0.64}[1]{巠}}}作「下利」,千金翼作「若下利」。}

大汗出。若大下利而厥冷者。四逆湯主之。354

病者手足厥冷。脉乍緊。邪结在胸中。心下滿而煩。饥不能食。病在胸中。当吐之。宜瓜蒂散。355

傷寒。厥而心下悸。宜先治水。当与茯苓甘草湯。卻治其厥。不尔。水漬入胃。必作利也。356

\begin{itemize}
\item 茯苓甘草湯。治心下悸。上衝而嘔者。(類聚方廣義)
\item 心下悸。大率屬癇与饮。此方加龙骨牡蛎绝妙。又。此症有致不寐者。酸棗湯。歸脾湯皆不能治。余用此方。屢奏奇效。(方輿輗)
\end{itemize}

傷寒六七日。大下後。[寸]脉沈遲。手足厥逆。下部脉不至。咽喉不利。唾膿血。泄利不止者。为難治。麻黄升麻湯主之。357

傷寒四五日。腹中痛。若轉气下趨少腹者。为欲自利也。358

傷寒。本自寒下。醫復吐下之。寒格。更逆吐下。食入即出。乾薑黄芩黄連人参湯主之。359

下利。有微熱而渴。脉弱者自愈。360

下利。脉數。有微熱。汗出者。自愈。設[脉]復緊。为未解。361
\footnote{下利。脉數。若微發熱。汗出者。自愈。設脉復緊。为未解。(千金翼)361\\「設脉復緊」除千金翼外其它版本均作「設復緊」。}

下利。手足厥[冷]。无脉。[当灸其厥陰。]灸之不温[而脉不還]。反微喘者死。少陰負趺陽者为順。362

下利。寸脉反浮數。尺中自{\CJKfontspec[Path=ttf/]{TH-Tshyn-P2}𬈧}者。必清膿血。363

下利清穀。不可攻表。汗出必脹滿。364

下利。脉沈弦者下重。脉大者为未止。脉微弱數者为欲自止。雖發熱。不死。365

下利。脉沈而遲。其人面少赤。身有微熱。下利清穀者。必鬱冒。汗出而解。其人必微厥。所以然者。其面戴陽。下虛故也。366

下利。脉反數而渴者。今自愈。設不差。必清膿血。以有熱故也。367

下利後。脉绝。手足厥[冷]。晬時脉還。手足温者生。不還[不温]者死。368

傷寒。下利日十馀行。脉反実者死。369

下利清穀。裏寒外熱。汗出而厥。通脉四逆湯主之。370

熱利下重者。白頭翁湯主之。371

下利。欲饮水者。为有熱也。白頭翁湯主之。373

下利。腹[脹]滿。身体疼痛者。先温其裏。乃攻其表。温裏宜四逆湯。攻表宜桂枝湯。372

下利。譫語者。有燥屎也。宜[小]承气湯。374

下利後更煩。按之心下濡者。为虛煩也。梔子[豉]湯主之。375

嘔家有癰膿。不可治嘔。膿盡自愈。376

嘔而發熱者。小柴胡湯主之。379

嘔而脉弱。小便復利。身有微熱。見厥者。難治。四逆湯主之。377

乾嘔。吐涎沫。頭痛者。吳茱萸湯主之。378

傷寒。大吐大下之。極虛。復極汗者。其人外气怫鬱。復与之水。以發其汗。因得噦。所以然者。胃中寒冷故也。380

傷寒。噦而腹滿。視其前後。知何部不利。利之即愈。381

\chapter{辨霍亂}

问曰。病有霍亂者何。\\
答曰。嘔吐而利。此为霍亂。382

问曰。病發熱。頭痛。身疼。惡寒。吐利者。当屬何病。\\
答曰。当为霍亂。霍亂吐利止。復更發熱也。383
\footnote{「霍亂吐利止」千金翼作「霍亂吐下利止」,玉函作「吐下利止」,宋本作「霍亂自吐下又利止」。}

傷寒。其脉微{\CJKfontspec[Path=ttf/]{TH-Tshyn-P2}𬈧}。本是霍亂。今是傷寒。卻四五日。至陰{\hbox{\scalebox{0.68}[1]{纟}\kern-0.35em\scalebox{0.64}[1]{巠}}}上。轉入陰。必利。本素嘔下利者不治。若其人似欲大便。但反失气而不利者。此屬陽明。便必堅。十三日愈。所以然者。{\hbox{\scalebox{0.68}[1]{纟}\kern-0.35em\scalebox{0.64}[1]{巠}}}盡故也。384
\footnote{「必利」玉函、千金翼作「当利」,脉{\hbox{\scalebox{0.68}[1]{纟}\kern-0.35em\scalebox{0.64}[1]{巠}}}作「必吐利」。}

下利後当便堅。堅則能食者愈。今反不能食。到後{\hbox{\scalebox{0.68}[1]{纟}\kern-0.35em\scalebox{0.64}[1]{巠}}}中。頗能食。復過一{\hbox{\scalebox{0.68}[1]{纟}\kern-0.35em\scalebox{0.64}[1]{巠}}}能食。過之一日当愈。若不愈。不屬陽明也。384

惡寒。脉微。而復利。利止必亡血。四逆加人参湯主之。(千金翼)385
\footnote{「利止必亡血」宋本、玉函作「利止亡血也」。}

霍亂。頭痛。發熱。身疼痛。熱多。欲饮水者。五苓散主之。寒多。不用水者。理中湯主之。386
\footnote{「理中湯」宋本作「理中丸」。}

吐利止而身痛不休者。当消息和解其外。宜桂枝湯小和之。387

吐利。汗出。發熱。惡寒。四肢拘急。手足厥冷。四逆湯主之。388

既吐且利。小便復利。而大汗出。下利清穀。裏寒外熱。脉微欲绝。四逆湯主之。388

吐已下斷。汗出而厥。四肢拘急不解。脉微欲绝。通脉四逆加豬膽汁湯主之。390

吐利發汗後。其人脉平。小煩者。以新虛不勝穀气故也。391

\chapter{辨陰易病已後勞復}

\footnote{「陰易」除千金翼外其它版本均作「陰陽易」。}

傷寒陰易之为病。其人身体重。少气。少腹裏急。或引陰中拘攣。熱上衝胸。頭重不欲舉。眼中生眵。[眼胞赤。]膝脛拘急。燒裩散主之。392
\footnote{「眼中生眵」除千金要方外其它版本均作「眼中生花」。「眼胞赤」三字宋本、千金要方、外臺均无,千金翼作「痂胞赤」。}

大病差後勞復者。枳実梔子湯主之。393

傷寒差已後。更發熱者。小柴胡湯主之。脉浮者。以汗解之。脉沈実者。以下解之。394

大病差後。從腰以下有水气者。牡蛎澤瀉散主之。395

傷寒解後。虛羸少气。气逆欲吐。竹葉石膏湯主之。397

大病差後。其人喜唾。久不了了者。胃上有寒。当温之。宜理中丸。396
\footnote{「胃上有寒」千金翼作「胸上有寒」。}

病人脉已解。而日暮微煩者。以病新差。人強与穀。脾胃气尚弱。不能消穀。故令微煩。損穀即愈。398

\chapter{發汗吐下後}

發汗後。身熱。又重發汗。胃中虛冷。必反吐也。0

大下後。口燥者。裏虛故也。0

\chapter{可与不可}

\section{不可發汗}

咽中闭塞。不可發汗。發汗即吐血。气微绝。厥冷。
\footnote{凡咽中闭塞。不可發汗。(聖惠方)}

厥[而脉緊]。不可發汗。發汗即聲亂。咽嘶。舌萎。聲不能出。

冬時。不可發汗。發汗必吐利。口中烂。生瘡。0
\footnote{凡積熱在臓。不宜發汗。汗則必吐。口中烂。生瘡。(聖惠方)}

欬而小便利。若失小便者。不可攻表。汗出則厥。逆冷。
\footnote{欬嗽小便利者。不可攻表。汗出即逆。(聖惠方)}

太陽病。發其汗。因致痙。
\footnote{太陽病。發汗太多。因致痙。(宋本。金匱)}

\section{可發汗}

大法。春夏宜發汗。

凡發汗。欲令手足皆周至。汗出漐漐然。一時间許益佳。不可令如水流離。若病不解。当重發汗。汗多則亡陽。陽虛不得重發汗也。

凡服湯發汗。中病便止。不必盡剂也。

凡云可發汗而无湯者。丸散亦可用。然不如湯隨證良驗。

凡脉浮者。病在外。可發汗。0

陽明病。脉浮虛者。可發汗。(千金翼)0
\footnote{陽明病。脉浮數者。可發汗。(聖惠方)0}

\section{可吐}

大法。春宜吐。

凡服湯吐。中病便止。不必盡剂也。

病胸上諸実。胸中鬱鬱而痛。不能食。欲使人按之。而反有涎唾。下利日十馀行。其脉反遲。寸口[脉]微滑。此可吐之。吐之利即止。
\footnote{夫胸心滿実。胸中鬱鬱而痛。不能食。多涎唾。下利。其脉遲反逆。寸口脉數。此可吐也。(聖惠方)}

宿食在上脘。宜吐之。
\footnote{宿食在上管。宜吐之。(脉{\hbox{\scalebox{0.68}[1]{纟}\kern-0.35em\scalebox{0.64}[1]{巠}}}。宋本)\\夫宿食在胃管。宜吐之。(聖惠方)}

\section{不可下}

咽中闭塞。不可下。下之則上輕下重。水漿不下。卧則欲踡。身体急痛。下利日數十行。

諸外実。不可下。下之則發微熱。亡脉則厥。当脐握熱。

諸虛。不可下。下之則渴。引水者易愈。惡水者劇。

脉數者。久數不止。止則邪结。正气不能復。正气卻结於臓。故邪气浮之。与皮毛相得。脉數者。不可下。下之必煩。利不止。

脉浮大。應發汗。醫反下之。此为大逆。

太陽与少陽并病。心下痞堅。頸項強而眩。[当刺大椎第一间。肺腧。肝腧。]不可下。171
\footnote{太陽与少陽并病。頭項強痛。或眩冒。時如结胸。心下痞堅。当刺大椎第一间。肺腧。肝腧。慎不可發汗。發汗即譫語。譫語則脉弦。譫語五日不止。当刺期门。142}

病欲吐者。不可下。

\section{可下}

大法。秋宜下。

凡可下者。用湯勝丸散。

凡服湯下。中病則止。不必盡剂也。

下利。三部脉皆平。按其心下。堅者。急下之。宜[大]承气湯。
\footnote{傷寒下痢。三部脉皆和。按其心下堅。宜急下之。(聖惠方)}

下利。脉遲而滑者。[内]実也。利未欲止。当下之。宜[大]承气湯。

问曰。人病有宿食。何以别之。\\
師曰。寸口脉浮大。按之反{\CJKfontspec[Path=ttf/]{TH-Tshyn-P2}𬈧}。尺中亦微而{\CJKfontspec[Path=ttf/]{TH-Tshyn-P2}𬈧}。故知有宿食。当下之。宜[大]承气湯。

下利。不欲食者。有宿食[也]。当下之。宜[大]承气湯。

下利[已]差。至其[年月日]時復發。此为病不盡。当復下之。宜[大]承气湯。

病腹中滿痛者。为実。当下之。宜大承气湯。
\footnote{病腹中滿痛者。为実。当下之。宜大柴胡湯。}

脉雙弦而遲。心下堅。脉大而緊者。陽中有陰也。可下之。宜[大]承气湯。

\section{可温}

大法。冬宜服温熱藥及灸。
\footnote{大法。冬宜熱藥。(聖惠方)}

下利。脉遲緊。为痛未欲止。当温之。得冷者。滿而便腸垢。0
\footnote{下利。脉遲緊。为痛未止。(聖惠方)}

下利。脉浮大者。此为虛。以強下之故也。当温之。与水必噦。[宜当歸四逆湯。]0
\footnote{下利。脉浮大者。此皆为虛。宜温之。(聖惠方)}

下利。欲食者。当温之。0

\section{可火}

下利。穀道中痛。当温之。宜灸枳実。或熬鹽等熨之。0
\footnote{凡下利後。下部中痛。当温之。宜炒枳実。若熬鹽等熨之。(聖惠方)}

\section{不可刺}

大怒勿刺。[已刺勿怒。]新内勿刺。[已刺勿内。]大勞勿刺。[已刺勿勞。]大醉勿刺。[已刺勿醉。]大饱勿刺。[已刺勿饱。大饥勿刺。已刺勿饥。]大渴勿刺。[已刺勿渴。]大驚勿刺。勿刺熇熇之熱。勿刺漉漉之汗。勿刺渾渾之脉。[身熱甚。陰陽皆爭者。勿刺也。其可刺者。急取之。不汗則泄。所谓勿刺者。有死徵也。]勿刺病与脉相逆者。上工刺未生。其次刺未盛。其次刺已衰。工逆此者。是谓伐形。0

\section{可刺}

婦人傷寒。懷娠。腹滿。不得小便。從腰以下重。如有水气狀。懷娠七月。太陰当養不養。此心气実。当刺。瀉勞宮及{\CJKfontspec[Path=ttf/]{TH-Tshyn-P2}𬮦}元。小便利則愈。(脉{\hbox{\scalebox{0.68}[1]{纟}\kern-0.35em\scalebox{0.64}[1]{巠}}}。千金翼)0
\footnote{婦人傷寒。懷娠。腹滿。不得大便。從腰以下重。如有水气狀。懷娠七月。太陰当養不養。此心气実。当刺。瀉勞宮及{\CJKfontspec[Path=ttf/]{TH-Tshyn-P2}𬮦}元。小便利則愈。(玉函)0}

傷寒。喉痹。刺手少陰。少陰在腕当小指後動脉是也。针入三分補之。0

\section{不可水}

下利。其脉浮大。此为虛。以強下之故也。設脉浮革。因尔腸鳴。当温之。与水即噦。0

太陽病。小便利者。为水多。心下必悸。0

\section{可水}

嘔吐而病在膈上。急思水者。与五苓散饮之。即可饮水也。0
\footnote{嘔吐而病在膈上。後必思水者。与五苓散饮之。水亦得也。(脉{\hbox{\scalebox{0.68}[1]{纟}\kern-0.35em\scalebox{0.64}[1]{巠}}}。玉函)0\\若嘔吐。熱在膈上。思水者。与五苓散。即可饮水也。(聖惠方)0}

\chapter{阙文(據宋本補)}

太陽病二日。反躁。反熨其背。而大汗出。大熱入胃。胃中水竭。躁煩必發譫語。十馀日。振慄自下利者。此为欲解也。故其汗從腰已下不得汗。欲小便不得。反嘔。欲失溲。足下惡風。大便硬。小便当數而反不數及多。大便已。頭卓然而痛。其人足心必熱。穀气下流故也。110

太陽病。中風。以火劫發汗。邪風被火熱。血气流溢。失其常度。兩陽相熏灼。其身發黄。陽盛則欲衄。陰虛則小便難。陰陽俱虛竭。身体則枯燥。但頭汗出。齐頸而還。腹滿微喘。口乾咽烂。或不大便。久則譫語。甚者至噦。手足躁擾。捻衣摸床。小便利者。其人可治。111

太陽病。吐之。但太陽病当惡寒。今反不惡寒。不欲近衣。此为吐之内煩也。121

太陽病。小便利者。以饮水多。必心下悸。小便少者。必苦裏急也。127

太陽病。下之。其脉促。不结胸者。此为欲解也。脉浮者。必结胸也。脉緊者。必咽痛。脉弦者。必兩脇拘急。脉细數者。頭痛未止。脉沈緊者。必欲嘔。脉沈滑者。協熱利。脉浮滑者。必下血。140

病脇下素有痞。連在脐傍。痛引少腹。入陰筋者。此名臓结。死。167

\part{雜病}

\chapter{臓腑經络先後}

\footnote{「问曰上工治未病」一段。吳本在臓腑{\hbox{\scalebox{0.68}[1]{纟}\kern-0.35em\scalebox{0.64}[1]{巠}}}络先後篇之前。未入正文。且此段文字都是无用不実之词。故刪。}

问曰。病人有气色見於面部。願闻其説。\\
師曰。鼻頭色青。腹中痛。苦冷者死。鼻頭色微黑者。有水气。色黄者。胸上有寒。色白者。亡血也。設微赤。非時者死。其目正圓者。痙。不治。又色青为痛。色黑为勞。色赤为風。色黄者便難。色鲜明者有留饮。

師曰。病人語聲寂然。喜驚呼者。骨節间病。語聲喑喑然不徹者。心膈间病。語聲啾啾然细而長者。頭中病。

師曰。息搖肩者。心中堅。息引胸中上气者。欬。息張口短气者。肺痿唾沫。

師曰。吸而微數。其病在中焦。実也。当下之即愈。虛者不治。在上焦者。其吸促。在下焦者。其吸遠。此皆難治。呼吸動搖振振者。不治。

師曰。寸口脉動者。因其王時而動。假令肝王色青。四時各隨其色。肝色青而反色白。非其時色脉。皆当病。

问曰。有未至而至。有至而不至。有至而不去。有至而太過。何谓也。\\
師曰。冬至之後。甲子夜半少陽起。少陽之時陽始生。天得温和。以未得甲子。天因温和。此为未至而至也。以得甲子而天未温和。此为至而不至也。以得甲子而天大寒不解。此为至而不去也。以得甲子而天温如盛夏五六月時。此为至而太過也。

師曰。病人脉浮者在前。其病在表。浮者在後。其病在裏。腰痛背強不能行。必短气而極也。

问曰。{\hbox{\scalebox{0.68}[1]{纟}\kern-0.35em\scalebox{0.64}[1]{巠}}}云。厥陽獨行。何谓也。\\
師曰。此为有陽无陰。故稱厥陽。

问曰。寸脉沈大而滑。沈則为実。滑則为气。実气相摶。血气入臓即死。入腑即愈。此为卒厥。何谓也。\\
師曰。唇口青。身冷。为入臓即死。如身和。汗自出。为入腑即愈。

问曰。脉脱。入臓即死。入腑即愈。何谓也。\\
師曰。非为一病。百病皆然。譬如浸淫瘡。從口起流向四肢者。可治。從四肢流來入口者。不可治。[諸]病在外者可治。入裏者即死。

问曰。陽病十八。何谓也。\\
師曰。頭痛。項。腰。脊。臂。腳掣痛。\\
问曰。陰病十八。何谓也。\\
師曰。欬。上气。喘。噦。咽。腸鳴。脹滿。心痛。拘急。五臓病各有十八。合为九十病。人又有六微。微有十八病。合为一百八病。五勞。七傷。六極。婦人三十六病。不在其中。清邪居上。濁邪居下。大邪中表。小邪中裏。䅽饪之邪。從口入者。宿食也。五邪中人。各有法度。風中於前。寒中於暮。濕傷於下。霧傷於上。風令脉浮。寒令脉急。霧傷皮腠。濕流{\CJKfontspec[Path=ttf/]{TH-Tshyn-P2}𬮦}節。食傷脾胃。極寒傷{\hbox{\scalebox{0.68}[1]{纟}\kern-0.35em\scalebox{0.64}[1]{巠}}}。極熱傷络。

问曰。病有急当救裏救表者。何谓也。\\
師曰。病。醫下之。{\hbox{\scalebox{0.6}[1]{纟}\kern-0.32em\scalebox{0.7}[1]{賣}}}得下利。清穀不止。身体疼痛者。急当救裏。後身体疼痛。清便自調者。急当救表也。

夫病痼疾。加以卒病。当先治其卒病。後乃治其痼疾也。

師曰。五臓病各有所得者愈。五臓病各有所惡。各隨其所不喜者为病。病者素不應食。而反暴思之。必發熱也。

夫諸病在臓。欲攻之。当隨其所得而攻之。如渴者。与豬苓湯。馀皆仿此。

\chapter{痙濕暍}

太陽病。發熱。无汗。反惡寒者。名曰剛痙。

太陽病。發熱。汗出。不惡寒者。名曰柔痙。

\begin{itemize}
\item 按。葛根湯證即剛痙。桂枝加葛根湯證即柔痙。
\item 痙以強急得名。乃賅腦脊髓膜炎。破傷風諸病而言。巢源。千金所載可考也。千金云。太陽中風。重感於寒濕。則變痙也。痙者。口噤不{\CJKfontspec[Path=ttf/]{TH-Tshyn-P2}𫔭}。背強而直。如發癇之狀。搖頭。馬鳴。腰反折。
\end{itemize}

太陽病。發熱。脉沈而细者。名曰痙。[为難治。]

\begin{itemize}
\item 太陽病。發熱。脉沈而细者。麻黄附子细辛湯。麻黄附子甘草湯所主。未为難治。今曰痙曰難治者。以其有頭項強急。口噤。背反張之證。非兩感傷寒也。夫曰太陽。則病尚初起。病初起即項背勁強。脉沈而细者。乃惡性腦脊髓膜炎。致命極速。故曰難治。(金匱要略今釋)
\end{itemize}

太陽病。發汗太多。因致痙。
\footnote{脉{\hbox{\scalebox{0.68}[1]{纟}\kern-0.35em\scalebox{0.64}[1]{巠}}}、玉函、千金翼「發汗太多」作「發其汗」。}

瘡家。雖身疼痛。不可發汗。汗出則痙。

病者身熱足寒。頸項強急。惡寒。時頭熱。面赤。目赤。獨頭動搖。卒口噤。背反張者。痙病也。
\footnote{「目赤」脉{\hbox{\scalebox{0.68}[1]{纟}\kern-0.35em\scalebox{0.64}[1]{巠}}}、玉函、千金翼均作「目脉赤」。}

痙病。發其汗者。寒濕相得。其表益虛。即惡寒甚。發其汗已。其脉如蛇。暴腹脹大者。为欲解。脉如故。反伏弦者。痙

夫風病。下之則痙。復發汗。必拘急。

夫痙脉。按之緊而弦。直上下行。

\begin{itemize}
\item 張缩血管之神{\hbox{\scalebox{0.68}[1]{纟}\kern-0.35em\scalebox{0.64}[1]{巠}}}出自脊髓。脊髓病。故血管为之痙孿也。
\end{itemize}

痙病有灸瘡。難治。

\begin{itemize}
\item 有灸創者。有破傷風之可能。故難治。
\end{itemize}

瘡家。雖身疼痛。不可發汗。汗出則痙。

太陽病。其證備。身体強。几几然。脉反沈遲。此为痙。栝蔞桂枝湯主之。

\begin{itemize}
\item 此即上條所谓柔痙也。几几。強直貌。尋常熱病。頭痛。發熱。汗出。惡風。而項背之肌肉因津液衰少而勁強者。栝蔞桂枝湯及桂枝加葛根湯俱效。然此條身体強而脉沈遲。明是腦脊髓膜炎。腦脊髓膜炎之脉搏。初起多甚沈遲。瀕死則數。因迷走神{\hbox{\scalebox{0.68}[1]{纟}\kern-0.35em\scalebox{0.64}[1]{巠}}}始則興奮。终則麻痹故也。此非栝樓桂枝湯所能治。金匱误矣。(金匱要略今釋)
\item 栝蔞桂枝湯。治桂枝湯證而渴者。此方当有葛根。(方極)
\end{itemize}

太陽病。无汗而小便反少。气上衝胸。口噤不得語。欲作剛痙。葛根湯主之。

\begin{itemize}
\item 无病之人。有汗時小便必少。有汗時小便必多。因人身水分之排泄有一定限度。故盈於此者必拙於彼也。今无汗而小便反少。是津液不足。分泌失職之候。云气上衝胸。口噤不得語。又云欲作剛痙。則是剛痙之發。咀嚼肌最先痙孿。此乃破傷風之特徵。非葛根湯所能治也。合前條觀之。柔痙似專指腦脊髓膜炎。剛痙似專指破傷風。二病雖以痙孿为主證。然与尋常熱病之項背強急者大異。金匱用葛根剂。误矣。(金匱要略今釋)
\end{itemize}

[剛]痙为病。胸滿。口噤。卧不著席。腳攣急。其人必齘齒。可与大承气湯。

太陽病。{\CJKfontspec[Path=ttf/]{TH-Tshyn-P2}𬮦}節疼痛而煩。脉沈而细者。此名濕痹。濕痹之候。其人小便不利。大便反快。但当利其小便。
\footnote{太陽病。{\CJKfontspec[Path=ttf/]{TH-Tshyn-P2}𬮦}節疼煩。脉沈缓者。为中濕。(脉{\hbox{\scalebox{0.68}[1]{纟}\kern-0.35em\scalebox{0.64}[1]{巠}}}。玉函)}

濕家之为病。一身盡疼。發熱。身色如熏黄也。

濕家。其人但頭汗出。背強。欲得被覆向火。若下之早則噦。或胸滿。小便[不]利。舌上如胎。以丹田有熱。胸上有寒。渴欲得饮而不能饮。則口燥[煩]也。

濕家下之。額上汗出。微喘。小便利者死。下利不止者亦死。

问曰。風濕相摶。一身盡疼痛。法当汗出而解。值天陰雨不止。師云。此可發汗。汗之病不愈者。何也。\\
答曰。發其汗。汗大出者。但風气去。濕气{\hbox{\scalebox{0.6}[1]{纟}\kern-0.32em\scalebox{0.7}[1]{賣}}}在。是故不愈。若治風濕者。發其汗。但微微似欲出汗者。則風濕俱去也。

濕家。病身疼。發熱。面黄而喘。頭痛。鼻塞而煩。其脉大。自能饮食。腹中和。无病。病在頭。中寒濕。故鼻塞。内藥鼻中即愈。

濕家身煩疼。可与麻黄加术湯。發其汗为宜。慎不可以火攻之。

\begin{itemize}
\item 麻黄加术湯。治寒濕。身体煩疼。无汗。惡寒。發熱者。(三因方)
\item 麻黄加术湯。治麻黄湯證而小便不利者。(方極)
\item 麻黄加术湯。治麻黄湯證而一身浮腫。小便不利者。隨證加附子。(類聚方廣義)
\item 婦人体弱者。妊娠终每患水腫。与越脾加术湯。木防己湯等往往墜胎。宜麻黄加术湯。或合葵子茯苓散亦良。(類聚方廣義)
\item 山行冒瘴霧。或入窟穴中。或於曲室混堂。諸濕气熱气鬱於處。昏倒气绝者。可連服大剂麻黄加术湯即甦。按。據尾臺氏説。則此方可治碳酸中毒。(類聚方廣義)
\item 术分赤白始於名醫别錄。仲景書本但稱术。後人輒加白字。别錄之赤术。即今之蒼术。此方意在使濕從汗解。則宜蒼术。(金匱要略今釋)
\end{itemize}

病者一身盡疼。發熱。日晡所劇者。此名風濕。此病傷於汗出当風。或久傷取冷所致也。可与麻杏薏甘湯。

病者一身盡疼。發熱。日晡所劇者。名風濕。此病傷於汗出当風。或久傷取冷所致也。可与麻杏薏甘湯。

\begin{itemize}
\item 麻杏薏甘湯之證。较之麻黄加术湯。濕邪所滯稍深。因用薏苡等品欤。余曾應用於梅毒。痛痹等。(方輿輗)
\item 麻杏薏甘湯。治孕婦浮腫而喘欬息迫。或身体麻痹。或疼痛者。(類聚方廣義)
\item 麻杏薏甘湯。治肺癰初起。惡寒。息迫。欬嗽不止。面目浮腫。濁唾臭痰。胸痛者。当迨其精气未脱。与白散交用。蕩滌邪穢。則易於平復。(類聚方廣義)
\item 麻杏薏甘湯。治風濕痛風。發熱。劇痛而{\CJKfontspec[Path=ttf/]{TH-Tshyn-P2}𬮦}節腫起者。雖證加术附。奇效。(類聚方廣義)
\end{itemize}

風濕。脉浮。身重。汗出。惡風者。防己黄耆湯主之。

\begin{itemize}
\item 防己黄耆湯方。分量。煮法非古。今從外臺祕要。(類聚方廣義)
\item 防己黄耆湯方,分量、煮服法{\hbox{\scalebox{0.68}[1]{纟}\kern-0.35em\scalebox{0.64}[1]{巠}}}後人篡改,千金風痹门所載当是金匱原方。(金匱要略今釋)
\item 防己黄耆湯。治水病身重。汗出。惡風。小便不利者。(方極)
\item 防己黄耆湯。治風毒腫。附骨疽。穿踝疽。稠膿已歇。稀膿不止。或痛。或不痛。身体瘦削。或見浮腫者。若惡寒。或下利。盜汗者。更加附子为佳。兼用伯州散。應{\hbox{\scalebox{0.7}[1]{钅}\kern-0.4em\scalebox{0.7}[1]{童}}}散。七寶等。凡附骨疽久不治。或治而復發者以毒之根蒂不除也。若此者。宜劀{\CJKfontspec[Path=ttf/]{TH-Tshyn-P2}𫔭}傷口。抉剔以除盡病根。无不治者。(類聚方廣義)
\end{itemize}

傷寒八九日。風濕相摶。身体疼煩。不能自轉側。不嘔。不渴。脉浮虛而{\CJKfontspec[Path=ttf/]{TH-Tshyn-P2}𬈧}者。桂枝附子湯主之。若其人大便堅。小便自利者。术附子湯主之。

\begin{itemize}
\item 桂枝附子湯。治桂枝湯去芍藥證而身体疼痛。不能自轉側者。(方極)
\end{itemize}

風濕相摶。骨節疼煩。掣痛。不得屈伸。近之則痛劇。汗出短气。小便不利。惡風。不欲去衣。或身微腫者。甘草附子湯主之。

太陽中熱者。暍是也。其人汗出。惡寒。身熱而渴。白虎[加人参]湯主之。

太陽中暍。身熱疼重而脉微弱。此以夏月傷冷水。水行皮膚中所致也。瓜蒂湯主之。

太陽中暍。發熱。惡寒。身重而疼痛。其脉弦细芤遲。小便已。洒洒然毛聳。手足逆冷。小有勞。身即熱。口{\CJKfontspec[Path=ttf/]{TH-Tshyn-P2}𫔭}。前板齒燥。若發其汗。則惡寒甚。加温针。則發熱甚。數下之。則淋甚。

\chapter{百合狐\protect{\CJKfontspec[Path=ttf/]{TH-Tshyn-P2}𧌒}陰陽毒}

論曰。百合病者。百脉一宗。悉致其病也。意欲食復不能食。常默默。欲卧不能卧。欲行不能行。饮食或有美時。或有不用闻食臭時。如寒无寒。如熱无熱。口苦。小便赤。諸藥不能治。得藥則劇吐利。如有神靈者。身形如和。其脉微數。每溺時頭痛者。六十日乃愈。若溺時頭不痛。淅然者。四十日愈。若溺快然。但頭眩者。二十日愈。其證或未病而預見。或病四五日而出。或病二十日或一月微見者。各隨證治之。

百合病發汗後者。百合知母湯主之。

百合病下之後者。[百合]滑石代赭湯主之。

百合病吐之後者。百合雞子湯主之。

百合病。不{\hbox{\scalebox{0.68}[1]{纟}\kern-0.35em\scalebox{0.64}[1]{巠}}}吐下發汗。病形如初者。百合地黄湯主之。

百合病。一月不解。變成渴者。百合洗方主之。

百合病。渴不差者。栝蔞牡蛎散主之。

百合病。變發熱者。百合滑石散主之。

百合病。見於陰者。以陽法救之。見於陽者。以陰法救之。見陽攻陰。復發其汗。此为逆。見陰攻陽。乃復下之。此亦为逆。

狐{\CJKfontspec[Path=ttf/]{TH-Tshyn-P2}𧌒}之为病。狀如傷寒。默默欲眠。目不得闭。卧起不安。蚀於喉为{\CJKfontspec[Path=ttf/]{TH-Tshyn-P2}𧌒}。蚀於陰为狐。不欲饮食。惡闻食臭。其面目乍赤。乍黑。乍白。蚀於上部則聲喝。甘草瀉心湯主之。蚀於下部則咽乾。苦参湯洗之。蚀於肛者。雄黄熏之。

\begin{itemize}
\item 嘗見麻疹被寒涼遏抑。不得透發。致蚀烂肛门以死者。(金匱要略今釋)
\end{itemize}

病者脉數。无熱。微煩。默默。但欲卧。汗出。初得之三四日。目赤如鳩眼。七八日目四眥黑。若能食者。膿已成也。赤[小]豆当歸散主之。

陽毒之为病。面赤斑斑如锦文。咽喉痛。唾膿血。五日可治。七日不可治。[升麻鱉甲湯主之。]
\footnote{「升麻鱉甲湯主之」吳本无。}

陰毒之为病。面目青。身痛如被杖。咽喉痛。五日可治。七日不可治。升麻鱉甲湯去雄黄蜀椒主之。
\footnote{「如被杖」吳本作「狀如被打」。}

\chapter{瘧}

師曰。瘧脉自弦。弦數者多熱。弦遲者多寒。弦小緊者下之差。弦遲者可温之。弦緊者可發汗。针灸也。浮大者可吐之。弦數者風發也。以饮食消息止之。
\footnote{「風發」吳本作「風疾」。}

问曰。瘧以月一日發。当以十五日愈。設不差。当月盡解。如其不差。当如何。\\
師曰。此结为癥瘕。名曰瘧母。急治之。宜鱉甲煎丸。

師曰。陰气孤绝。陽气獨發。則熱而少气煩寃。手足熱而欲嘔。名曰癉瘧。若但熱不寒者。邪气内藏於心。外舍分肉之间。令人消铄脱肉。
\footnote{「煩寃」吳本作「煩滿」。}

温瘧者。其脉如平。身无寒。但熱。骨節疼煩。時嘔。白虎加桂枝湯主之。

瘧多寒者。名曰牡瘧。蜀漆散主之。

附方

牡蛎湯。治牡瘧。

瘧病發渴者。与小柴胡去半夏加栝樓湯。

柴胡桂薑湯。此方治寒多微有熱。或但寒不熱。服一剂如神。故錄之。
\footnote{「此方」至「錄之」二十一字为小字注文。}

\chapter{中風歷節}

夫風之为病。当半身不遂。或但臂不遂者。此为痹。脉微而數。中風使然。

寸口脉浮而緊。緊則为寒。浮則为虛。寒虛相摶。邪在皮膚。浮者血虛。络脉空虛。賊邪不瀉。或左或右。邪气反缓。正气即急。正气引邪。喎僻不遂。邪在於络。肌膚不仁。邪在於{\hbox{\scalebox{0.68}[1]{纟}\kern-0.35em\scalebox{0.64}[1]{巠}}}。即重不勝。邪入於腑。即不識人。邪入於臓。舌即難言。口吐涎。

大風。四肢煩重。心中惡寒不足者。侯氏黑散主之。

寸口脉遲而缓。遲則为寒。缓則为虛。榮缓則为亡血。衛缓則为中風。邪气中{\hbox{\scalebox{0.68}[1]{纟}\kern-0.35em\scalebox{0.64}[1]{巠}}}。則身痒而癮疹。心气不足。邪气入中。則胸滿而短气。
\footnote{此條吳本无。}

風引湯。除熱。主癱癇。
\footnote{此條鄧本作「風引湯除熱癱癇」,吳本作「風引除熱主癱癇湯方」。}

病如狂狀。妄行。獨語不休。惡寒熱。其脉浮。防己地黄湯主之。

頭風摩散方。

寸口脉沈而弱。沈即主骨。弱即主筋。沈即为腎。弱即为肝。汗出入水中。如水傷心。歷節黄汗出。故曰歷節。

跌陽脉浮而滑。滑則穀气実。浮則汗自出。

少陰脉浮而弱。浮則为風。弱則血不足。風血相摶。即疼痛如掣。

盛人脉{\CJKfontspec[Path=ttf/]{TH-Tshyn-P2}𬈧}小。短气。自汗出。歷節疼。不可屈伸。此皆饮酒。汗出当風所致。

諸肢節疼痛。身体魁瘰。腳腫如脱。頭眩短气。嗢嗢欲吐。桂枝芍藥知母湯主之。

\begin{itemize}
\item 桂枝芍藥知母湯。用於慢性風濕性{\CJKfontspec[Path=ttf/]{TH-Tshyn-P2}𬮦}節疾病。以身体瘦弱。{\CJKfontspec[Path=ttf/]{TH-Tshyn-P2}𬮦}節腫脹如樹瘤为指徵。惡心欲吐這一症狀。就我個人{\hbox{\scalebox{0.68}[1]{纟}\kern-0.35em\scalebox{0.64}[1]{巠}}}驗來説。多不具備。沒必要列为應用指徵。(金匱要略研究)
\end{itemize}

味酸則傷筋。筋傷則缓。名曰泄。鹹則傷骨。骨傷則痿。名曰枯。枯泄相摶。名曰斷泄。榮气不通。衛不獨行。榮衛俱微。三焦无所御。四屬斷绝。身体羸瘦。獨足腫大。黄汗出。脛冷。假令發熱。便为歷節也。
\footnote{此條吳本无。}

病歷節。疼痛。不可屈伸。烏頭湯主之。

\begin{itemize}
\item 應用该方時。一定不能忘記使用蜂蜜。蜂蜜具有使藥物吸收缓慢。預防烏頭中毒的作用。條文中的煎服法是一種慎重的方法。(金匱要略研究)
\end{itemize}

烏頭湯。治腳气。疼痛。不可屈伸。
\footnote{此條吳本无。}

礬石湯。治腳气衝心。

附方

{\hbox{\scalebox{0.6}[1]{纟}\kern-0.32em\scalebox{0.7}[1]{賣}}}命湯。治中風痱。身体不能自收。口不能言。冒昧不知痛處。或拘急不得轉側。

三黄湯。治中風。手足拘急。百節疼痛。煩熱心亂。惡寒。{\hbox{\scalebox{0.68}[1]{纟}\kern-0.35em\scalebox{0.64}[1]{巠}}}日不欲饮食。

术附子湯。治風虛頭重眩。苦極不知食味。暖肌補中。益精气。

崔氏八味丸。治腳气上入。少腹不仁。

越婢加术湯。治肉極熱則身体津脱。腠理{\CJKfontspec[Path=ttf/]{TH-Tshyn-P2}𫔭}。汗大泄。厉風气。下焦腳弱。
\footnote{「肉」俞本作「内」。}

\chapter{血痹虛勞}

问曰。血痹病從何得之。\\
師曰。夫尊榮人。骨弱肌膚盛。重因疲勞汗出。卧不時動搖。加被微風。遂得之。[形如風狀。]但以脉自微{\CJKfontspec[Path=ttf/]{TH-Tshyn-P2}𬈧}。在寸口。{\CJKfontspec[Path=ttf/]{TH-Tshyn-P2}𬮦}上小緊。宜针引陽气。令脉和緊去則愈。

血痹。陰陽俱微。寸口{\CJKfontspec[Path=ttf/]{TH-Tshyn-P2}𬮦}上微。尺中小緊。外證身体不仁。如風痹狀。黄耆桂枝五物湯主之。

\begin{itemize}
\item 血痹者。末梢知覺神{\hbox{\scalebox{0.68}[1]{纟}\kern-0.35em\scalebox{0.64}[1]{巠}}}麻痹也。(金匱要略今釋)
\item 黄耆桂枝五物湯。治桂枝湯證而嘔。身体不仁。不急迫者。(方極)
\item 方極但就藥味之去加言之。於本文之證无所考。此證雖用桂枝。无衝逆之證而有麻痹不仁之外證。亦无發嘔之候。非以嘔而增加生薑也。(和久田寅叔)
\end{itemize}

夫男子平人。脉大为勞。極虛亦为勞。

\begin{itemize}
\item 凡慢性病。見營養不良。機能衰減之證者。古人统稱虛勞。如腎上腺病。遺精病。前列腺漏。
\end{itemize}

男子面色薄者。主渴及亡血。卒喘悸。脉浮者。裏虛也。

男子脉虛沈弦。无寒熱。短气。裏急。小便不利。面色白。時目瞑。兼衄。少腹滿。此为勞使之然。

勞之为病。其脉浮大。手足煩。春夏劇。秋冬差。陰寒精自出。痠削不能行。

男子脉浮弱而{\CJKfontspec[Path=ttf/]{TH-Tshyn-P2}𬈧}。为无子。精清泠。
\footnote{「精清泠」鄧本作「精气清冷」。}

夫失精家。少腹弦急。陰頭寒。目眩。髮落。脉極虛芤遲。为清穀。亡血。失精。脉得諸芤動微緊。男子失精。女子夢交。桂枝加龙骨牡蛎湯主之。天雄散亦主之。

\begin{itemize}
\item 此條言遺精之證治也。
\end{itemize}

男子平人。脉虛弱细微者。善盜汗也。

人年五六十。其病脉大者。痹俠背行。苦腸鳴。馬刀俠癭者。皆为勞得之。

脉沈小遲。名脱气。其人疾行則喘喝。手足逆寒。腹滿。甚則溏泄。食不消化也。

脉弦而大。弦則为減。大則为芤。減則为寒。芤則为虛。虛寒相摶。此名为革。婦人則半產漏下。男子則亡血失精。

虛勞。裏急。悸。衄。腹中痛。夢失精。四肢痠疼。手足煩熱。咽乾口燥。小建中湯主之。

\begin{itemize}
\item 衄。失精。下血之人。腹中攣急。或痛。手足煩熱者。衄兼用黄連解毒丸。下血兼用應{\hbox{\scalebox{0.7}[1]{钅}\kern-0.4em\scalebox{0.7}[1]{童}}}散。(方機)
\item 產婦。手足煩熱。咽乾口燥。腹中拘攣者。兼用應{\hbox{\scalebox{0.7}[1]{钅}\kern-0.4em\scalebox{0.7}[1]{童}}}散。若有塊者。兼用夷則丸。(方機)
\item 此證余每用黄耆建中湯。其效勝於小建中湯。(類聚方廣義)
\end{itemize}

虛勞。裏急。諸不足。黄耆建中湯主之。

虛勞。腰痛。少腹拘急。小便不利者。八味腎气丸主之。

\begin{itemize}
\item 古醫書所言腎病。多是内分泌疾患。而{\CJKfontspec[Path=ttf/]{TH-Tshyn-P2}𬮦}係腎上腺者十之八九。又以腰部。少腹部为腎之領域。腎又与膀胱为表裏。故藥方能治腰痛。少腹拘急。小便不利者。名曰腎气丸。(金匱要略今釋)
\end{itemize}

虛勞。諸不足。風气百疾。薯蕷丸主之。

虛勞。虛煩。不得眠。酸棗湯主之。
\footnote{虛勞。煩。悸。不得眠。酸棗湯主之。(千金方)}

\begin{itemize}
\item 酸棗人湯。治煩躁。不得眠者。(方極)
\item 酸棗人湯治煩而不得眠者。煩。悸而眠不寤者。(方機)
\item 酸棗人湯。治虛勞。煩。悸。不得眠者
\end{itemize}

五勞。虛極。羸瘦。腹滿。不能饮食。食傷。憂傷。饮傷。房室傷。饥傷。勞傷。{\hbox{\scalebox{0.68}[1]{纟}\kern-0.35em\scalebox{0.64}[1]{巠}}}络榮衛气傷。内有乾血。肌膚甲错。兩目黯黑。缓中補虛。大黄䗪虫丸主之。

附方

虛勞不足。汗出而闷。脉结。心悸。行動如常。不出百日。危急者。十一日死。炙甘草湯主之。

獺肝散。治冷勞。又主鬼疰。一门相染。

\chapter{肺痿肺癰欬嗽上气}

问曰。熱在上焦者。因欬为肺痿。肺痿之病。何從得之。\\
師曰。或從汗出。或從嘔吐。或從消渴。小便利數。[或從便難。]又被快藥下利。重亡津液。故得之。
\footnote{「或從便難」吳本无。}

问曰。寸口脉數。其人欬。口中反有濁唾涎沫者何。\\
師曰。[此]为肺痿之病。若口中辟辟燥。欬即胸中隱隱痛。脉反滑數。此为肺癰。欬唾濃血。脉數虛者为肺痿。數実者为肺癰。

问曰。病欬逆。脉之何以知此为肺癰。当有膿血。吐之則死。其脉何類。\\
師曰。寸口脉微而數。微則为風。數則为熱。微則汗出。數則惡寒。風中於衛。呼气不入。熱過於榮。吸而不出。風傷皮毛。熱傷血脉。風舍於肺。其人則欬。口乾。喘滿。咽燥。不渴。時唾濁沫。時時振寒。熱之所過。血为凝滯。畜结癰膿。吐如米粥。始萌可救。膿成則死。

上气。面浮腫。肩息。其脉浮大。不治。又加利尤甚。

上气。喘而躁者。屬肺脹。欲作風水。發汗則愈。

肺痿。吐涎沫而不欬者。其人不渴。必遺溺。小便數。所以然者。以上虛不能制下故也。此为肺中冷。必眩。多涎唾。甘草乾薑湯以温之。若服湯已渴者。屬消渴。
\footnote{「温之若服湯已渴者屬消渴」吳本作「温其病」。}

欬而上气。喉中水雞聲。射干麻黄湯主之。

欬逆上气。時時唾濁。但坐不得卧。皂莢丸主之。
\footnote{欬逆。气上衝。唾濁。但坐不得卧。皂莢丸主之。(吳本)}

欬而脉浮者。厚朴麻黄湯主之。脉沈者。澤漆湯主之。
\footnote{「欬而」吳本作「上气」。}

大逆上气。咽喉不利。止逆下气者。麦门冬湯主之。

肺癰。喘不得卧。葶藶大棗瀉肺湯主之。

欬而胸滿。振寒。脉數。咽乾。不渴。時出濁唾腥臭。久久吐膿如米粥者。为肺癰。桔梗湯主之。

欬而上气。此为肺脹。其人喘。目如脱狀。脉浮大者。越脾加半夏湯主之。
\footnote{「欬而上气」吳本作「欬逆倚息」。}

肺脹。欬而上气。煩躁而喘。脉浮者。心下有水。小青龙加石膏湯主之。
\footnote{「煩躁」鄧本作「煩燥」。}

附方

肺痿。涎唾多。心中嗢嗢液液者。炙甘草湯主之。

肺痿。欬唾涎沫不止。咽燥而渴。生薑甘草湯主之。

肺痿。吐涎沫。桂枝去芍藥加皂莢湯主之。

欬而胸滿。振寒。脉數。咽乾。不渴。時出濁唾腥臭。久久吐膿如米粥者。为肺癰。桔梗白散主之。

葦莖湯。治欬。有微熱。煩滿。胸中甲错。是为肺癰。

肺癰。胸滿脹。一身面目浮腫。鼻塞。清涕出。不闻香臭酸辛。欬逆上气。喘鳴迫塞。葶藶大棗瀉肺湯主之。

欬而上气。肺脹。其脉浮。心下有水气。脇下痛引缺盆。小青龙加石膏湯主之。

\chapter{奔豚气}

\begin{itemize}
\item 奔豚系一種發作性疾病。患者多系中年男女。發作時。先於少腹蚱结成瘕塊而作痛。塊漸大。痛亦漸劇。同時气從少腹上衝至心胸。其人困苦欲死。俯仰坐卧。饮食呼吸。无一而可。既而衝气漸降。痛漸減。塊亦漸小。终至痛止塊消。健好如常人。当其發作之時。一若
\end{itemize}

師曰。病有奔豚。有吐膿。有驚怖。有火邪。此四部病。皆從驚發得之。

師曰。奔豚病。從少腹起。上衝咽喉。發作欲死。復還止。皆從驚恐得之。

奔豚。气上衝胸。腹痛。往來寒熱。奔豚湯主之。

發汗後。燒针令其汗。针處被寒。核起而赤者。必發奔豚。气從少腹上衝心者。灸其核上各一壯。与桂枝加桂湯。

發汗後。其人脐下悸者。欲作奔豚。苓桂甘棗湯主之。

\chapter{胸痹心痛短气}

師曰。夫脉当取太過不及。陽微陰弦。即胸痹而痛。所以然者。責其極虛也。今陽虛。知在上焦。所以胸痹心痛者。以其陰弦故也。

平人无寒熱。短气不足以息者。実也。

胸痹之病。喘息欬唾。胸背痛。短气。寸口脉沈而遲。{\CJKfontspec[Path=ttf/]{TH-Tshyn-P2}𬮦}上小緊數。栝蔞薤白白酒湯主之。

胸痹不得卧。心痛徹背者。栝蔞薤白半夏湯主之。

胸痹。心中痞。留气结在胸。胸滿。脇下逆搶心。枳実薤白桂枝湯主之。理中湯亦主之。

胸痹。胸中气塞。短气。茯苓杏人甘草湯主之。橘[皮]枳[実生]薑湯亦主之。

胸痹缓急者。薏苡[人]附子散主之。

心中痞。諸逆。心懸痛。桂枝生薑枳実湯主之。

心痛徹背。背痛徹心。烏頭赤石脂丸主之。

九痛丸。治九重心痛。

\chapter{腹滿寒疝宿食}

趺陽脉微弦。法当腹滿。不滿者必便難。兩胠疼痛。此虛寒從下上也。当以温藥服之。

病者腹滿。按之不痛[者]为虛。痛者为実。可下之。舌黄未下者。下之黄自去。
\footnote{傷寒。腹滿。按之不痛者为虛。痛者为実。当下之。舌黄未下者。下之黄自去。宜大承气湯。(玉函)}

腹滿時減。復如故。此为寒。当与温藥。

病者痿黄。躁而不渴。胸中寒実而利不止者死。

寸口脉弦者。即脇下拘急而痛。其人啬啬惡寒也。

夫中寒家。喜欠。其人清涕出。發熱。色和者。善嚏。

中寒。其人下利。以裏虛也。欲嚏不能。此人肚中寒。

夫瘦人{\hbox{\scalebox{0.6}[1]{纟}\kern-0.3em\scalebox{0.63}[1]{堯}}}脐痛。必有風冷。穀气不行。而反下之。其气必衝。不衝者。心下則痞。

病腹滿。發熱十日。脉浮而數。饮食如故。厚朴七物湯主之。

腹中寒气。雷鳴。切痛。胸脇逆滿。嘔吐。附子粳米湯主之。

痛而闭者。厚朴三物湯主之。
\footnote{「痛而闭者」吳本作「腹滿脉數」。}

按之心下滿痛者。此为実也。当下之。宜大柴胡湯主之。
\footnote{「按之心下」吳本作「病腹中」。}

腹滿不減。減不足言。当須下之。宜大承气湯。

心胸中大寒痛。嘔。不能饮食。腹中寒。上衝皮起。出見有頭足。上下痛而不可觸近。大建中湯主之。

脇下偏痛。發熱。其脉緊弦。此寒也。以温藥下之。宜大黄附子湯。

寒气厥逆。赤丸主之。

腹痛。脉弦而緊。弦則衛气不行。[衛气不行]即惡寒。緊則不欲食。邪正相摶。即为寒疝。寒疝遶脐痛。若發則白汗出。手足厥冷。其脉沈弦者。大烏頭煎主之。

寒疝。腹中痛。及脇痛裏急者。当歸生薑羊肉湯主之。

寒疝。腹中痛。逆冷。手足不仁。若身疼痛。灸刺諸藥不能治。抵当烏頭桂枝湯主之。

其脉數而緊乃弦。狀如弓弦。按之不移。脉數弦者。当下其寒。脉雙弦而遲者。必心下堅。脉大而緊者。陽中有陰。可下之。
\footnote{「雙弦」鄧本作「緊大」。}

附方

烏頭湯治寒疝。腹中绞痛。賊風入攻五臟。拘急不得轉側。發作有時。使人陰缩。手足厥逆。

寒疝。腹中痛者。柴胡桂枝湯主之。(吳本)
\footnote{柴胡桂枝湯方。治心腹猝中痛者。(鄧本)}

卒疝。走馬湯主之。(吳本)
\footnote{走馬湯。治中惡。心痛。腹脹。大便不通。(鄧本)}

问曰。人病有宿食。何以别之。\\
師曰。寸口脉浮而大。按之反{\CJKfontspec[Path=ttf/]{TH-Tshyn-P2}𬈧}。尺中亦微而{\CJKfontspec[Path=ttf/]{TH-Tshyn-P2}𬈧}。故知有宿食。大承气湯主之。

脉緊如轉索无常者。有宿食也。

脉緊。頭痛。風寒。腹中有宿食不化也。

脉數而滑者。実也。此有宿食。下之愈。宜大承气湯。

下利。不欲食者。有宿食也。当下之。宜大承气湯。

宿食在上脘。当吐之。宜瓜蒂散。

\chapter{五臓風寒積聚}

肺中風者。口燥而喘。身運而重。冒而腫脹。

\begin{itemize}
\item 運即眩暈之暈。喘为肺臟疾患必見之證。身運而重及冒。皆因碳氧之交換不足所致。乃呼吸障礙之结果。腫脹則鬱血性水腫也。此條頗似肺气腫之證。(金匱要略今釋)
\end{itemize}

肺中寒者。吐濁涕。

肺死臓。浮之虛。按之弱如葱葉。下无根者。死。

肝中風者。頭目{\CJKfontspec[Path=ttf/]{TH-Tshyn-P2}𥆧}。兩脇痛。行常傴。令人嗜甘。

肝中寒者。兩臂不舉。舌本燥。喜太息。胸中痛。不得轉側。食則吐而汗出也。

肝死臓。浮之弱。按之如索不來。或曲如蛇行者。死。

肝著。其人常欲蹈其胸上。先未苦時。但欲饮熱。旋復花湯主之。

心中風者。翕翕發熱。不能起。心中饥[而欲食]。食即嘔吐。
\footnote{「心中饥」下吳本有「而欲食」三字。}

心中寒者。其人苦病心如噉蒜狀。劇者心痛徹背。背痛徹心。譬如蠱注。其脉浮者。自吐乃愈。

心傷者。其人勞倦即頭面赤而下重。心中痛而自煩。發熱。当脐跳。其脉弦。此为心臓傷所致也。

心死臓。浮之実如麻豆。按之益躁疾者。死。

邪哭使魂魄不安者。血气少也。血气少者。屬於心。心气虛者。其人則畏。合目欲眠。夢遠行而精神離散。魂魄妄行。陰气衰者为癲。陽气衰者为狂。

脾中風者。翕翕發熱。形如醉人。腹中煩重。皮肉{\CJKfontspec[Path=ttf/]{TH-Tshyn-P2}𥆧}{\CJKfontspec[Path=ttf/]{TH-Tshyn-P2}𥆧}而短气。

脾死臓。浮之大堅。按之如覆杯潔潔。狀如搖者死。

趺陽脉浮而{\CJKfontspec[Path=ttf/]{TH-Tshyn-P2}𬈧}。浮則胃气強。{\CJKfontspec[Path=ttf/]{TH-Tshyn-P2}𬈧}則小便數。浮{\CJKfontspec[Path=ttf/]{TH-Tshyn-P2}𬈧}相摶。大便則堅。其脾为约。麻子人丸主之。

腎著之病。其人身体重。腰中冷。如坐水中。形如水狀。反不渴。小便自利。饮食如故。病屬下焦。身勞汗出。衣裏冷濕。久久得之。腰以下冷痛。腹重如帶五千錢。甘薑苓术湯主之。

腎死臓。浮之堅。按之亂如轉丸。益下入尺中者死。

问曰。三焦竭部。上焦竭善噫。何谓也。\\
師曰。上焦受中焦气未和。不能消穀。故令噫耳。下焦竭。即遺溺失便。其气不和。不能自禁制。不須治。久自愈。

師曰。熱在上焦者。因欬为肺痿。熱在中焦者。則为堅。熱在下焦者。則溺血。亦令淋祕不通。大腸有寒者。多鶩溏。有熱者。便腸垢。小腸有寒者。其人下重便血。有熱者。必痔。

问曰。病有積。有聚。有䅽气。何谓也。\\
師曰。積者。臓病也。终不移。聚者。腑病也。發作有時。展轉痛移。为可治。䅽气者。脇下痛。按之則愈。復發为䅽气。諸積大法。脉來细而附骨者。乃積也。寸口。積在胸中。微出寸口。積在喉中。{\CJKfontspec[Path=ttf/]{TH-Tshyn-P2}𬮦}上。積在脐傍。上{\CJKfontspec[Path=ttf/]{TH-Tshyn-P2}𬮦}上。積在心下。微下{\CJKfontspec[Path=ttf/]{TH-Tshyn-P2}𬮦}。積在少腹。尺中。積在气衝。脉出左。積在左。脉出右。積在右。脉兩出。積在中央。各以其部處之。

\chapter{痰饮欬嗽}

问曰。夫饮有四。何谓也。\\
師曰。有痰饮。有懸饮。有溢饮。有支饮。

问曰。四饮何以为異。\\
師曰。其人素盛今瘦。水走腸间。瀝瀝有聲。谓之痰饮。饮後水流在脇下。欬唾引痛。谓之懸饮。饮水流行。歸於四肢。当汗出而不汗出。身体疼重。谓之溢饮。欬逆倚息。短气不得卧。其形如腫。谓之支饮。

水在心。心下堅築。短气。惡水。不欲饮。

水在肺。吐涎沫。欲饮水。

水在脾。少气身重。

水在肝。脇下支滿。嚏而痛。

水在腎。心下悸。

夫心下有留饮。其人背寒冷如手大。

留饮者。脇下痛引缺盆。欬嗽則輒已。

胸中有留饮。其人短气而渴。四肢歷節痛。脉沈者。有留饮。

膈上病痰。滿。喘。欬。吐。發則寒熱。背痛。腰疼。目泣自出。其人振振身{\CJKfontspec[Path=ttf/]{TH-Tshyn-P2}𥆧}劇。必有伏饮。
\footnote{「病痰」吳本作「之病」。}

夫病人饮水多。必暴喘滿。凡食少饮多。水停心下。甚者則悸。微者短气。

脉雙弦者。寒也。皆大下後善虛。脉偏弦者。饮也。

肺饮不弦。但苦喘。短气。

支饮。亦喘而不能卧。加短气。其脉平也。

病痰饮者。当以温藥和之。

心下有痰饮。胸脇支滿。目胘。苓桂术甘湯主之。

夫短气。有微饮。当從小便去之。苓桂术甘湯主之。腎气丸亦主之。

病者脉伏。其人欲自利。利反快。雖利。心下{\hbox{\scalebox{0.6}[1]{纟}\kern-0.32em\scalebox{0.7}[1]{賣}}}堅滿。此为留饮欲去故也。甘遂半夏湯主之。

脉浮而细滑。傷饮。

脉弦數。有寒饮。冬夏難治。

脉沈而弦者。懸饮内痛。

病懸饮者。十棗湯主之。

病溢饮者。当發其汗。大青龙湯主之。小青龙湯亦主之。

\begin{itemize}
\item 按。此證当以大青龙湯發汗。曰小青龙湯主之者誤也。(類聚方廣義)
\end{itemize}

膈间支饮。其人喘滿。心下痞堅。面色黎黑。其脉沈緊。得之數十日。醫吐下之不愈。木防己湯主之。虛者即愈。実者三日復發。復与不愈者。宜去石膏加茯苓芒硝湯。

心下有支饮。其人苦冒眩。澤瀉湯主之。

支饮。胸滿者。厚朴大黄湯主之。

支饮。不得息。葶藶大棗瀉肺湯主之。

嘔家本渴。渴者为欲解。今反不渴。心下有支饮故也。小半夏湯主之。

腹滿。口舌乾燥。此腸间有水气。己椒藶黄丸主之。

卒嘔吐。心下痞。膈间有水。眩悸者。[小]半夏加茯苓湯主之。

假令瘦人脐下有悸。吐涎沫而癲眩。此水也。五苓散主之。

附方

茯苓饮。治心胸中有停痰宿水。自吐出水後。心胸间虛。气滿。不能食。消痰气。令能食。

欬家。其脉弦。为有水。十棗湯主之。

夫有支饮家。欬煩。胸中痛者。不卒死。至一百日[或]一歲。宜十棗湯。

久欬數歲。其脉弱者可治。実大數者死。其脉虛者必苦冒。其人本有支饮在胸中故也。治屬饮家。

欬逆倚息。[不得卧。]小青龙湯主之。

青龙湯下已。多唾。口燥。寸脉沈。尺脉微。手足厥逆。气從少腹上衝胸咽。手足痹。其面翕熱如醉狀。因復下流陰股。小便難。時復冒者。与茯苓桂枝五味甘草湯。治其气衝。

衝气即低。而反更欬。胸滿者。用桂苓五味甘草湯。去桂加乾薑。细辛。以治其欬滿。

欬滿即止。而復更渴。衝气復發者。以细辛。乾薑为熱藥也。服之当遂渴。而渴反止者。为支饮也。支饮者。法当冒。冒者必嘔。嘔者復内半夏。以去其水。

水去。嘔止。其人形腫者。可内麻黄。以其欲逐痹。故不内麻黄。乃内杏人也。若逆而内麻黄者。必厥。所以然者。以其人血虛。麻黄發其陽故也。(吳本)
\footnote{水去。嘔止。其人形腫者。加杏人主之。其證應内麻黄。以其人遂痹。故不内之。若逆而内之者。必厥。所以然者。以其人血虛。麻黄發其陽故也。(鄧本)}

若面熱如醉[狀者]。此为胃熱上衝熏其面。加大黄以利之。
\footnote{「加大黄以利之」吴本作「加大黄湯和之」。}

先渴後嘔。为水停心下。此屬饮家。小半夏[加]茯苓湯主之。

\chapter{消渴小便[不]利淋}

厥陰之为病。消渴。气上衝心。心中疼熱。饥而不欲食。食即吐。下之不肯止。

寸口脉浮而遲。浮即为虛。遲即为勞。虛則衛气不足。勞則榮气竭。跌陽脉浮而數。浮即为气。數即消穀而大[便]堅。气盛則溲數。數數即堅。堅數相摶。即为消渴。
\footnote{「大堅」吳本作「矢堅」。}

男子消渴。小便反多。以饮一斗。小便一斗。腎气丸主之。

脉浮。小便不利。微熱。消渴者。宜利小便。發汗。五苓散主之。

渴欲饮水。水入則吐者。名曰水逆。五苓散主之。

渴欲饮水不止者。文蛤散主之。

淋之为病。小便如粟狀。小腹弦急。痛引脐中。

趺陽脉數。胃中有熱。即消穀引食。大便必堅。小便即數。

淋家不可發汗。發汗則必便血。

小便不利者。有水气。其人若渴。栝樓瞿麦丸主之。
\footnote{「若渴」徐本作「苦渴」。}

小便不利。蒲灰散主之。滑石白魚散。茯苓戎鹽湯并主之。

渴欲饮水。口乾舌燥者。白虎加人参湯主之。

脉浮。發熱。渴欲饮水。小便不利者。豬苓湯主之。

\chapter{水气}

師曰。病有風水。有皮水。有正水。有石水。有黄汗。風水。其脉自浮。外證骨節疼痛。惡風。皮水。其脉亦浮。外證胕腫。按之沒指。不惡風。其腹如鼓。不渴。当發其汗。正水。其脉沈遲。外證自喘。石水。其脉自沈。外證腹滿。不喘。黄汗。其脉沈遲。身發熱。胸滿。四肢頭面腫。久不愈。必致癰膿。

脉浮而洪。浮則为風。洪則为气。風气相擊。身体洪腫。汗出則愈。惡風則虛。此为風水。不惡風者。小便通利。上焦有寒。其口多涎。此为黄汗。(吳本)
\footnote{脉浮而洪。浮則为風。洪則为气。風气相摶。風強則为癮疹。身体为痒。痒为泄風。久为痂癩。气強則为水。難以俛仰。風气相擊。身体洪腫。汗出則愈。惡風則虛。此为風水。不惡風者。小便通利。上焦有寒。其口多涎。此为黄汗。(鄧本)\\脉浮而大。浮为風虛。大为气強。風气相摶。必成癮疹。身体为痒。痒者名泄風。久久为痂癩。(平脉法)}

寸口脉沈滑者。中有水气。面目腫大。有熱。名曰風水。視人之目窠上微擁。如蠶新卧起狀。其頸脉動。時時欬。按其手足上。陷而不起者。風水。

太陽病。脉浮而緊。法当骨節疼痛。反不痛。身体反重而痠。其人不渴。汗出即愈。此为風水。惡寒者。此为極虛。發汗得之。渴而不惡寒者。此为皮水。身腫而冷。狀如周痹。胸中窒。不能食。反聚痛。暮躁不得眠。此为黄汗。痛在骨節。欬而喘。不渴者。此为脾脹。其狀如腫。發汗即愈。然諸病此者。渴而下利。小便數者。皆不可發汗。

皮水者。一身面目洪腫。其脉沈。小便不利。故令病水。假如小便自利。此亡津液。故令渴也。越脾加术湯主之。

趺陽脉当伏。今反緊。本自有寒。疝瘕。腹中痛。醫反下之。下之即胸滿短气。

趺陽脉当伏。今反數。本自有熱。消穀。小便數。今反不利。此欲作水。

寸口脉浮而遲。浮脉則熱。遲脉則潛。熱潛相摶。名曰沈。趺陽脉浮而數。浮脉即熱。數脉即止。熱止相摶。名曰伏。沈伏相摶。名曰水。沈則络脉虛。伏則小便難。虛難相摶。水走皮膚。即为水矣。

寸口脉弦而緊。弦則衛气不行。[衛气不行]即惡寒。水不沾流。走於腸间。

少陰脉緊而沈。緊則为痛。沈則为水。小便即難。脉得諸沈。当責有水。身体腫重。水病脉出者死。

夫水病人。目下有卧蠶。面目鲜澤。脉伏。其人消渴。病水腹水。小便不利。其脉沈绝者。有水。可下之。

问曰。病下利後。渴饮水。小便不利。腹滿因腫者。何也。\\
答曰。此法当病水。若小便自利及汗出者。自当愈。

心水者。其身重而少气。不得卧。煩而躁。其人陰腫。

肝水者。其腹大。不能自轉則。脇下腹痛。時時津液微生。小便{\hbox{\scalebox{0.6}[1]{纟}\kern-0.32em\scalebox{0.7}[1]{賣}}}通。

肺水者。其身腫。小便難。時時鴨溏。

脾水者。其腹大。四肢苦重。津液不生。但苦少气。小便難。

腎水者。其腹大。脐腫。腰痛。不得溺。陰下濕如牛鼻上汗。其足逆冷。面反瘦。

師曰。諸有水者。腰以下腫。当利小便。腰以上腫。当發汗乃愈。

師曰。寸口脉沈而遲。沈則为水。遲則为寒。寒水相摶。趺陽脉伏。水穀不化。脾气衰則鶩溏。胃气衰則身腫。少陽脉卑。少陰脉细。男子則小便不利。婦人則{\hbox{\scalebox{0.68}[1]{纟}\kern-0.35em\scalebox{0.64}[1]{巠}}}水不通。{\hbox{\scalebox{0.68}[1]{纟}\kern-0.35em\scalebox{0.64}[1]{巠}}}为血。血不利則为水。名曰血分。
\footnote{「少陽脉卑」吳本无。}

问曰。病者苦水。面目身体四肢皆腫。小便不利。脉之不言水。反言胸中痛。气上衝咽。狀如炙肉。当微欬喘。審如師言。其脉何類。\\
師曰。寸口脉沈而緊。沈为水。緊为寒。沈緊相摶。结在{\CJKfontspec[Path=ttf/]{TH-Tshyn-P2}𬮦}元。始時当微。年盛不覺。陽衰之後。榮衛相干。陽損陰盛。结寒微動。腎气上衝。喉咽塞噎。脇下急痛。醫以为留饮。而大下之。气擊不去。其病不除。後重吐之。胃家虛煩。咽燥欲饮水。小便不利。水穀不化。面目手足浮腫。又与葶藶丸下水。当時如小差。食饮過度。腫復如前。胸脇苦痛。象若奔豚。其水揚溢。則浮欬喘逆。当先攻擊衛气令止。乃治欬。欬止。其喘自差。先治新病。病当在後。

風水。脉浮。身重。汗出。惡風者。防己黄耆湯主之。腹痛加芍藥。

風水。惡風。一身悉腫。脉浮。不渴。{\hbox{\scalebox{0.6}[1]{纟}\kern-0.32em\scalebox{0.7}[1]{賣}}}自汗出。无大熱。越脾湯主之。

皮水为病。四肢腫。水气在皮膚中。四肢聶聶動者。防己茯苓湯主之。

裏水。越脾加术湯主之。甘草麻黄湯亦主之。

水之为病。其脉沈小。屬少陰。浮者为風。无水。虛脹者为气。水。發其汗即已。脉沈者。宜麻黄附子湯。浮者。宜杏子湯。
\footnote{杏子湯方書中未見,可能是大青龙湯。}

厥而皮水者。蒲灰散主之。

问曰。黄汗之为病。身体腫。發熱。汗出而渴。狀如風水。汗沾衣。色正黄如檗汁。脉自沈。何從得之。\\
師曰。以汗出入水中浴。水從汗孔入得之。

黄汗。黄耆芍藥桂枝苦酒湯主之。

黄汗之病。兩脛自冷。假令發熱。此屬歷節。食已汗出。又身常暮[卧]盜汗出者。此勞气也。若汗出已。反發熱者。久久其身必甲错。發熱不止者。必生惡瘡。若身重。汗出已輒輕者。久久必身{\CJKfontspec[Path=ttf/]{TH-Tshyn-P2}𥆧}。即胸中痛。又從腰以上必汗出。下无汗。腰髖弛痛。如有物在皮中狀。劇者不能食。身疼重。煩躁。小便不利。此为黄汗。桂枝加黄耆湯主之。

師曰。寸口脉遲而{\CJKfontspec[Path=ttf/]{TH-Tshyn-P2}𬈧}。遲則为寒。{\CJKfontspec[Path=ttf/]{TH-Tshyn-P2}𬈧}为血不足。趺陽脉微而遲。微則为气。遲則为寒。寒气不足。則手足逆冷。手足逆冷。則榮衛不利。榮衛不利。則腹滿脇鳴相逐。气轉膀胱。榮衛俱勞。陽气不通即身冷。陰气不通即骨痛。陽前通則惡寒。陰前通則痹不仁。陰陽相得。其气乃行。大气一轉。其气乃散。実則失气。虛則遺溺。名曰气分。

气分。心下堅。大如盤。邊如旋杯。水饮所作。桂枝去芍藥加麻[黄细]辛附子湯主之。

心下堅。大如盤。邊如旋盤。水饮所作。枳[実]术湯主之。

附方

夫風水。脉浮为在表。其人或頭汗出。表无他病。病者但下重。故知從腰以上为和。腰以下当腫及陰。難以屈伸。防己黄耆湯主之。

\chapter{黄疸}

寸口脉浮而缓。浮則为風。缓則为痹。痹非中風。四肢苦煩。脾色必黄。瘀熱以行。

趺陽脉緊而數。數則为熱。熱則消穀。緊則为寒。食即为滿。尺脉浮为傷腎。趺陽脉緊为傷脾。風寒相摶。食穀即眩。穀气不消。胃中苦濁。濁气下流。小便不通。陰被其寒。熱流膀胱。身体盡黄。名曰穀疸。額上黑。微汗出。手足中熱。薄暮即發。膀胱急。小便自利。名曰女勞疸。腹如水狀。不治。心中懊憹而熱。不能食。時欲吐。名曰酒疸。

陽明病。脉遲者。食難用饱。饱則發煩。頭眩。必小便難。此欲作穀疸。雖下之。腹滿如故。所以然者。脉遲故也。

夫病酒黄疸。必小便不利。其候心中熱。足下熱。是其證也。

酒黄疸者。或无熱。靖言了[了]。腹滿欲吐。鼻燥。其脉浮者。先吐之。沈弦者。先下之。

酒疸。心中熱。欲嘔者。吐之愈。

酒疸下之。久久为黑疸。目青。面黑。心中如噉蒜虀狀。大便正黑。皮膚爪之不仁。其脉浮弱。雖黑。微黄。故知之。

師曰。病黄疸。發熱。煩喘。胸滿。口燥者。以病發時火劫其汗。兩熱所得。然黄家所得。從濕得之。一身盡發熱而黄。肚熱。熱在裏。当下之。

脉沈。渴欲饮水。小便不利者。皆發黄。
\footnote{「脉沈」吳本作「脉浮」。}

腹滿。舌痿黄燥。不得睡。屬黄家。

黄疸之病。当以十八日为期。治之十日以上差。反劇。为難治。

疸而渴者。其疸難治。疸而不渴者。其疸可治。發於陰部。其人必嘔。[發於]陽部。其人振寒而發熱也。

穀疸之为病。寒熱不食。食即頭眩。心胸不安。久久發黄。为穀疸。茵陳蒿湯主之。

黄家。日晡所發熱。而反惡寒。此为女勞得之。膀胱急。少腹滿。身盡黄。額上黑。足下熱。因作黑疸。其腹脹如水狀。大便必黑。時溏。此女勞之病。非水也。腹滿者難治。硝石礬石散主之。

酒黄疸。心中懊憹。或熱痛。梔子[枳実豉]大黄湯主之。

諸病黄家。但利其小便。假令脉浮。当以汗解之。宜桂枝加黄耆湯主之。

諸黄。豬膏髮煎主之。

黄疸病。茵陳五苓散主之。

黄疸。腹滿。小便不利而赤。自汗出。此为表和裏実。当下之。宜大黄[黄蘗梔子]硝石湯。

黄疸病。小便色不變。欲自利。腹滿而喘。不可除熱。熱除必噦。噦者。小半夏湯主之。

諸黄。腹痛而嘔者。宜柴胡湯。

男子黄。小便自利。当与虛勞小建中湯。

附方

諸黄。瓜蒂湯主之。

黄疸。麻黄淳酒湯主之。

\chapter{驚悸吐衄下血胸滿瘀血}

寸口脉動而弱。動即为驚。弱即为悸。

師曰。尺脉浮。目睛暈黄。衄未止。暈黄去。目睛急了。知衄今止。\\
又曰。從春至夏衄者。太陽。從秋至冬衄者。陽明。

衄家不可發汗。汗出必額上陷。脉緊急。直視不能眴。不得眠。

病人面无色。无寒熱。脉沈弦者。衄。[脉]浮弱。手按之绝者。下血。煩欬者。必吐血。

夫吐血。欬逆上气。其脉數而有熱。不得卧者死。

夫酒客。欬者。必致吐血。此因極饮過度所致也。

寸口脉弦而大。弦則为減。大則为芤。減則为寒。芤則为虛。寒虛相擊。此名曰革。婦人則半產漏下。男子則亡血。

亡血不可攻其表。汗出則寒慄而振。

病人胸滿。唇痿舌青。口燥。但欲嗽水。不欲咽。无寒熱。脉微大來遲。腹不滿。其人言我滿。为有瘀血。

病者如熱狀。煩滿。口乾燥而渴。其脉反无熱。此为陰狀。是瘀血也。当下之。

火邪者。桂枝去芍藥加蜀漆牡蛎龙骨救逆湯主之。

心下悸者。半夏麻黄丸主之。

吐血不止者。柏葉湯主之。

下血。先便後血。此遠血也。黄土湯主之。

下血。先血後便。此近血也。赤小豆当歸散主之。

心气不足。吐血。衄血。瀉心湯主之。

\chapter{嘔吐噦下利}

夫嘔家有癰膿。不可治嘔。膿盡自愈。

先嘔卻渴者。此为欲解。先渴卻嘔者。为水停心下。此屬饮家。

嘔家本渴。今反不渴者。以心下有支饮故也。此屬支饮。

问曰。病人脉數。數为熱。当消數引食。而反吐者。何也。\\
師曰。以發其汗。令陽微。膈气虛。脉乃數。數为客熱。不能消穀。胃中虛冷故也。脉弦者虛也。胃气无馀。朝食暮吐。變为胃反。寒在於上。醫反下之。令脉反弦。故名曰虛。
\footnote{「胃中虛冷故也」吳本作「胃中虛冷故吐也」。}

寸口脉微而數。微則无气。无气則榮虛。榮虛則血不足。血不足則胸中冷。

趺陽脉浮而{\CJKfontspec[Path=ttf/]{TH-Tshyn-P2}𬈧}。浮則为虛。{\CJKfontspec[Path=ttf/]{TH-Tshyn-P2}𬈧}則傷脾。脾傷則不磨。朝食暮吐。暮食朝吐。宿穀不化。名曰胃反。脉緊而{\CJKfontspec[Path=ttf/]{TH-Tshyn-P2}𬈧}。其病難治。

病人欲吐者。不可下之。

噦而腹滿。視其前後。知何部不利。利之即愈。

嘔而胸滿者。茱萸湯主之。

乾嘔。吐涎沫。頭痛者。茱萸湯主之。
\footnote{「茱萸湯」吳本作「吳茱萸湯」。}

嘔而腸鳴。心下痞者。半夏瀉心湯主之。

乾嘔而利者。黄芩加半夏生薑湯主之。

諸嘔吐。穀不得下者。小半夏湯主之。

嘔吐而病在膈上。後思水者解。急与之。思水者。豬苓散主之。

嘔而脉弱。小便復利。身有微熱。見厥者難治。四逆湯主之。

嘔而發熱者。小柴胡湯主之。

胃反嘔吐者。大半夏湯主之。

食已即吐者。大黄甘草湯主之。

胃反。吐而渴欲饮水者。茯苓澤瀉湯主之。

吐後。渴欲得水而貪饮者。文蛤湯主之。兼主微風。脉緊。頭痛。

乾嘔。吐逆。吐涎沫。半夏乾薑散主之。

病人胸中似喘不喘。似嘔不嘔。似噦不噦。徹心中憒憒然无奈。生薑半夏湯主之。

乾嘔。噦。若手足厥者。橘皮湯主之。

噦逆者。橘皮竹茹湯主之。

夫六腑气绝於外者。手足寒。上气。腳缩。五臓气绝於内者。利不禁。下甚者。手足不仁。

下利。脉沈弦者。下重。脉大者。为未止。脉微弱數者。为欲自止。雖發熱。不死。

下利。手足厥[冷]。无脉。[当灸其厥陰。]灸之不温[而脉不還]。反微喘者死。少陰負趺陽者为順。362
\footnote{下利。手足厥冷。无脉者。灸之不温。若脉不還。反微喘者死。少陰負趺陽者。为順也。(金匱)362}

下利。有微熱而渴。脉弱者。今自愈。

下利。脉數。有微熱。汗出。今自愈。設脉緊。为未解。

下利。脉數而渴者。今自愈。設不差。必清膿血。以有熱故也。

下利。脉反弦。發熱。身汗者。自愈。

下利气者。当利其小便。

下利。寸脉反浮數。尺中自{\CJKfontspec[Path=ttf/]{TH-Tshyn-P2}𬈧}者。必清膿血。

下利清穀。不可攻其表。汗出必脹滿。

下利。脉沈而遲。其人面少赤。身有微熱。下利清穀者。必鬱冒。汗出而解。病人必微熱。所以然者。其面戴陽。下虛故也。
\footnote{「必微熱」吳本作「必微厥」。}

下利後。脉绝。手足厥[冷]。晬時脉還。手足温者生。不還[不温]者死。368
\footnote{下利後。脉绝。手足厥冷。晬時脉還。手足温者生。脉不還者死。(金匱)}

下利。腹脹滿。身体疼痛者。先温其裏。乃攻其表。温裏宜四逆湯。攻表宜桂枝湯。

下利。三部脉皆平。按之心下堅者。急下之。宜大承湯。

下利。脉遲而滑者。実也。利未欲止。急下之。宜大承气湯。

下利。脉反滑者。当有所去。下乃愈。宜大承气湯。

下利已差。至其[年月日]時復發者。此为病不盡。当復下之。宜[大]承气湯。
\footnote{下利已差。至其年月日時復發者。以病不盡故也。当下之。宜大承气湯。(金匱)}

下利。譫語者。有燥屎也。小承气湯主之。

下利。便膿血者。桃花湯主之。

熱利重下者。白頭翁湯主之。

下利後更煩。按之心下濡者。为虛煩也。梔子豉湯主之。

下利清穀。裏寒外熱。汗出而厥者。通脉四逆湯主之。

下利。肺痛。紫参湯主之。

气利。诃梨勒散主之。

附方

小承气湯。治大便不通。噦。數譫語。

乾嘔。下利。黄芩湯主之。

\chapter{瘡癰腸癰浸淫}

諸浮數脉。應当發熱。而反洒淅惡寒。若有痛處。当發其癰。

師曰。諸癰腫。欲知有膿无膿。以手掩腫上。熱者为有膿。不熱者为无膿。

腸癰之为病。其身甲错。腹皮急。按之濡。如腫狀。腹无積聚。身无熱。脉數。此为腸内有癰膿。薏苡附子敗醬散主之。

腸癰者。少腹腫痞。按之即痛如淋。小便自調。時時發熱。自汗出。復惡寒。其脉遲緊者。膿未成。可下之。当有血。脉洪數者。膿已成。不可下也。大黄牡丹湯主之。

问曰。寸口脉浮微而{\CJKfontspec[Path=ttf/]{TH-Tshyn-P2}𬈧}。法当亡血。若汗出。設不汗者云何。\\
答曰。若身有瘡。被刀斧所傷。亡血故也。

病金瘡。王不留行散主之。

浸淫瘡。從口流向四肢者。可治。從四肢流來入口者。不可治。黄連粉主之。
\footnote{「黄連粉主之」鄧本作「浸淫瘡黄連粉主之」且單獨列为一條。}

\chapter{趺蹶手指臂腫轉筋陰狐疝蛔虫}

師曰。病趺蹶。其人但能前。不能卻。刺腨入二寸。此太陽{\hbox{\scalebox{0.68}[1]{纟}\kern-0.35em\scalebox{0.64}[1]{巠}}}傷也。

病人常以手指臂腫動。此人身体{\CJKfontspec[Path=ttf/]{TH-Tshyn-P2}𥆧}{\CJKfontspec[Path=ttf/]{TH-Tshyn-P2}𥆧}者。藜蘆甘草湯主之。
\footnote{「臂腫動」吳本作「臂脛動」。}

轉筋之为病。其人臂腳直。脉上下行。微弦。轉筋入腹者。雞屎白散主之。

陰狐疝气者。偏有大小。時時上下。蜘蛛散主之。

问曰。病腹痛有虫。其脉何以别之。\\
師曰。腹中痛。其脉当沈若弦。反洪大。故有蛔虫。

蛔虫之为病。令人吐涎。心痛。發作有時。毒藥不止。甘草粉蜜湯主之。

蛔厥者。其人当吐蛔。今病者靜。而復時煩。此为臓寒。蛔上入膈。故煩。須臾復止。得食而嘔。又煩者。蛔闻食臭出。其人当自吐蛔。蛔厥者。烏梅丸主之。
\footnote{「今病者靜」吳本、鄧本均作「令病者靜」。}

\chapter{婦人妊娠病}

師曰。婦人得平脉。陰脉小弱。其人渴。不能食。无寒熱。名妊娠。桂枝湯主之。於法六十日当有此證。設有醫治逆者。卻一月。加吐下者。則绝之。
\footnote{師曰。脉婦人得平脉。陰脉小弱。其人渴。不能食。无寒熱。名为軀。桂枝湯主之。法六十日当有娠。設有醫治逆者。卻一月。加吐下者。則绝之。(吳本)}

婦人妊娠。{\hbox{\scalebox{0.68}[1]{纟}\kern-0.35em\scalebox{0.64}[1]{巠}}}斷三月而得漏下。下血四十日不止。胎欲動。在於脐上。此为妊娠。六月動者。前三月{\hbox{\scalebox{0.68}[1]{纟}\kern-0.35em\scalebox{0.64}[1]{巠}}}水利時。胎也。下血者。後斷三月。衃也。所以下血不止者。其癥不去故也。当下其癥。宜桂枝茯苓丸。(吳本)
\footnote{婦人宿有癥病。{\hbox{\scalebox{0.68}[1]{纟}\kern-0.35em\scalebox{0.64}[1]{巠}}}斷未及三月。而得漏下不止。胎動在脐上者。为癥痼害。妊娠六月動者。前三月{\hbox{\scalebox{0.68}[1]{纟}\kern-0.35em\scalebox{0.64}[1]{巠}}}水利時。胎下血者。後斷三月。衃也。所以血不止者。其癥不去故也。当下其癥。桂枝茯苓丸主之。(鄧本)}

\begin{itemize}
\item 按。当有衝逆。心下悸證。是方不唯治婦人之病方也。(類聚方廣義)
\item 此方倍加大黄为散。为兼用方。或为單用。其功勝於丸。(類聚方廣義)
\item 桂枝茯苓丸。治{\hbox{\scalebox{0.68}[1]{纟}\kern-0.35em\scalebox{0.64}[1]{巠}}}水有變。或胎動。拘攣。上衝。心下悸者。(類聚方廣義)
\item 產後惡露不盡。則諸患错出。其窮至不可救。故其治以逐瘀血为至要。宜桂枝茯苓丸。又。妊娠臨盆。用之催生尤有效。(類聚方廣義)
\item 桂枝茯苓丸。治{\hbox{\scalebox{0.68}[1]{纟}\kern-0.35em\scalebox{0.64}[1]{巠}}}水不調。時時頭痛。腹中拘攣。或手足頑痹者。加大黄煎服为佳。(類聚方廣義)
\item 桂枝茯苓丸。治每至{\hbox{\scalebox{0.68}[1]{纟}\kern-0.35em\scalebox{0.64}[1]{巠}}}期。頭重眩暈。腹中。腰。腳㽲痛者。加大黄煎服为佳。(類聚方廣義)
\item 桂枝茯苓丸。治產後已過數十日。无它異證。但時時{\hbox{\scalebox{0.6}[1]{纟}\kern-0.3em\scalebox{0.63}[1]{堯}}}脐刺痛。或痛延腰腿者。加大黄煎服为佳。(類聚方廣義)
\item 桂枝茯苓丸。治{\hbox{\scalebox{0.68}[1]{纟}\kern-0.35em\scalebox{0.64}[1]{巠}}}闭。上衝。頭痛。眼中生翳。尺脉{\hbox{\scalebox{0.6}[1]{纟}\kern-0.32em\scalebox{0.7}[1]{從}}}横。疼痛羞明。腹中拘攣者。加大黄煎服为佳。(類聚方廣義)
\item 桂枝茯苓丸。治妊婦顛仆。子死於腹中。下血不止。少腹攣痛者。用之胎即下。加大黄煎服为佳。(類聚方廣義)
\item 血淋。腸風。下血。選用桂枝茯苓丸皆有效。加大黄煎服为佳。(類聚方廣義)
\item 常用於月{\hbox{\scalebox{0.68}[1]{纟}\kern-0.35em\scalebox{0.64}[1]{巠}}}痛的處方中,對虛證用当歸芍藥散、桂枝加芍藥湯、小建中湯等。對瘀血、血実者用桂枝茯苓丸。疼痛劇烈者用桃人承气湯。炎症性而下血塊者可用折衝饮等方。(矢數道明)
\item 桂枝茯苓丸,從体貭上看与当歸芍藥散恰好相反,大致用於実證,体型屬於所謂的女丈夫型、比較胖、红顏、有瘀血腹證、下腹部有抵抗壓痛、產後惡露少、腹痛、腹脹、下肢靜脉血栓症、胎盤殘留、死胎等時應用,但本方不是任何人均可常用者。(矢數道明)
\end{itemize}

婦人懷娠六七月。脉弦。發熱。其胎愈脹。腹痛。惡寒者。少腹如扇。所以然者。子臓{\CJKfontspec[Path=ttf/]{TH-Tshyn-P2}𫔭}故也。当以附子湯温其臓。

師曰。婦人有漏下者。有半產後因{\hbox{\scalebox{0.6}[1]{纟}\kern-0.32em\scalebox{0.7}[1]{賣}}}下血都不绝者。有妊娠下血者。假令妊娠腹中痛。为胞阻。膠艾湯主之。

\begin{itemize}
\item 膠艾湯。治漏下。腹中痛。及吐血。下血者。为則按。凡治吐血。下血。諸血證者。不别男子婦人矣。(類聚方廣義)
\end{itemize}

婦人懷娠。腹中㽲痛。当歸芍藥散主之。

\begin{itemize}
\item 㽲。腹中急痛也。(廣韻)
\item 当歸芍藥散大致用於虛證体貭者,体型消瘦、膚色白、有貧血傾向、冷症、易疲倦者。本方是古方中產後最常用的處方,特别是產後腹痛、產後陣痛時最多用,相当於後世方中的芎歸調血饮,是平安无事的處方。(矢數道明)
\end{itemize}

妊娠。嘔吐不止。乾薑人参半夏丸主之。

妊娠。小便難。饮食如故。当歸貝母苦参丸主之。

妊娠。有水气。身重。小便不利。洒淅惡寒。起即頭眩。葵子茯苓散主之。

婦人妊娠。宜常服当歸散。

附方

妊娠養胎。白术散主之。

婦人傷胎。懷身。腹滿。不得小便。從腰以下重。如有水气狀。懷身七月。太陰当養不養。此心气実。当刺瀉勞宮及{\CJKfontspec[Path=ttf/]{TH-Tshyn-P2}𬮦}元。小便[微]利則愈。
\footnote{吳本「傷胎」作「傷寒」,「小便微利」作「小便利」。}

\chapter{婦人產後病}

问曰。新產婦人有三病。一者病痙。二者病鬱冒。三者大便難。何谓也。\\
師曰。新產血虛。多汗出。喜中風。故令病痙。亡血復汗。寒多。故令鬱冒。亡津液。胃燥。故大便難。

產婦鬱冒。其脉微弱。不能食。大便反堅。但頭汗出。所以然者。血虛而厥。厥而必冒。冒家欲解。必大汗出。以血虛下厥。孤陽上出。故頭汗出。所以產婦喜汗出者。亡陰血虛。陽气獨盛。故当汗出。陰陽乃復。大便堅。嘔不能食。小柴胡湯主之。

病解。能食。七八日更發熱者。此为胃実。大承气湯主之。
\footnote{「胃実」吳本作「胃熱气実」。}

產後腹中㽲痛。当歸生薑羊肉湯主之。并治腹中寒疝。虛勞不足。

產後腹痛。煩滿不得卧。枳実芍藥散主之。

師曰。產婦腹痛。法当与枳実芍藥散。假令不愈者。此为腹中有乾血著脐下。宜下瘀血湯主之。

產後七八日。无太陽證。少腹堅痛。此惡露不盡。不大便。煩躁發熱。切脉微実。再倍發熱。日晡時煩躁者。不食。食則譫語。至夜即愈。宜大承气湯主之。熱在裏。结在膀胱也。(鄧本)
\footnote{婦人產後七八日。无太陽證。少腹堅痛。此惡露不盡。不大便四五日。趺陽脉微実。再倍其人發熱。日晡所煩躁者。不食。食即譫語。利之即愈。宜大承气湯。熱在裏。结在膀胱也。(吳本)}

產後風。{\hbox{\scalebox{0.6}[1]{纟}\kern-0.32em\scalebox{0.7}[1]{賣}}}之數十日不解。頭微痛。惡寒。時時有熱。心下闷。乾嘔。汗出。雖久。陽旦證{\hbox{\scalebox{0.6}[1]{纟}\kern-0.32em\scalebox{0.7}[1]{賣}}}在耳。可与陽旦湯。

產後中風。發熱。面正赤。喘而頭痛。竹葉湯主之。

婦人乳中虛。煩亂。嘔逆。安中益气。竹皮大丸主之。

產後下利。虛極。白頭翁加甘草阿膠湯主之。

附方

婦人在草蓐得風。四肢苦煩熱。皆自發露所为。頭痛者。与小柴胡湯。頭不痛。但煩者。与三物黄芩湯。

内補当歸建中湯。治婦人產後虛羸不足。腹中刺痛不止。吸吸少气。或苦少腹拘急。攣痛引腰背。不能食饮。產後一月。日得服四五剂为善。令人強壯。

\chapter{婦人雜病}

婦人中風七八日。{\hbox{\scalebox{0.6}[1]{纟}\kern-0.32em\scalebox{0.7}[1]{賣}}}得寒熱。發作有時。{\hbox{\scalebox{0.68}[1]{纟}\kern-0.35em\scalebox{0.64}[1]{巠}}}水適斷。此为熱入血室。其血必结。故使如瘧狀。發作有時。小柴胡湯主之。144

婦人傷寒。發熱。{\hbox{\scalebox{0.68}[1]{纟}\kern-0.35em\scalebox{0.64}[1]{巠}}}水適來。晝日明了。暮則譫語。如見鬼狀。此为熱入血室。无犯胃气及上二焦。必自愈。145

婦人中風。發熱。惡寒。{\hbox{\scalebox{0.68}[1]{纟}\kern-0.35em\scalebox{0.64}[1]{巠}}}水適來。得之七八日。熱除。脉遲。身涼。胸脇下滿。如结胸狀。其人譫語。此为熱入血室。当刺期门。隨其虛実而取之。143

陽明病。下血。譫語者。此为熱入血室。但頭汗出。当刺期门。隨其実而瀉之。濈然汗出者愈。

婦人咽中如有炙臠。半夏厚朴湯主之。

婦人臓躁。喜悲傷。欲哭。象如神靈所作。數欠伸。甘[草小]麦大棗湯主之。

婦人吐涎沫。醫反下之。心下即痞。当先治其吐涎沫。宜小青龙湯。涎沫止。乃治痞。宜瀉心湯。

婦人之病。因虛積冷结气。为諸{\hbox{\scalebox{0.68}[1]{纟}\kern-0.35em\scalebox{0.64}[1]{巠}}}水斷绝。至有歷年。血寒積结。胞门寒傷。{\hbox{\scalebox{0.68}[1]{纟}\kern-0.35em\scalebox{0.64}[1]{巠}}}络凝堅。在上嘔吐涎唾。久成肺癰。形体損分。在中盤结。{\hbox{\scalebox{0.6}[1]{纟}\kern-0.3em\scalebox{0.63}[1]{堯}}}脐寒疝。或兩脇疼痛。与臓相連。或结熱[在]中。痛在{\CJKfontspec[Path=ttf/]{TH-Tshyn-P2}𬮦}元。脉數无瘡。肌若魚鱗。時著男子。非止女身。在下未多。{\hbox{\scalebox{0.68}[1]{纟}\kern-0.35em\scalebox{0.64}[1]{巠}}}候不勻。令陰掣痛。少腹惡寒。或引腰脊。下根气街。气衝急痛。膝脛疼煩。奄忽眩冒。狀如厥癲。或有憂慘。悲傷多嗔。此皆帶下。非有鬼神。久則羸瘦。脉虛多寒。三十六病。千變万端。審脉陰陽。虛実緊弦。行其针藥。治危得安。其雖同病。脉各異源。子当辯記。勿谓不然。

问曰。婦人年五十所。病下利。數十日不止。暮即發熱。少腹裏急。腹滿。手掌煩熱。唇口乾燥。何也。\\
師曰。此病屬帶下。何以故。曾{\hbox{\scalebox{0.68}[1]{纟}\kern-0.35em\scalebox{0.64}[1]{巠}}}半產。瘀血在少腹不去。何以知之。其證唇口乾燥。故知之。当以温{\hbox{\scalebox{0.68}[1]{纟}\kern-0.35em\scalebox{0.64}[1]{巠}}}湯主之。

[婦人]帶下。{\hbox{\scalebox{0.68}[1]{纟}\kern-0.35em\scalebox{0.64}[1]{巠}}}水不利。少腹滿痛。{\hbox{\scalebox{0.68}[1]{纟}\kern-0.35em\scalebox{0.64}[1]{巠}}}一月再見者。土瓜根散主之。

寸口脉弦而大。弦則为減。大則为芤。減則为寒。芤則为虛。寒虛相摶。此名曰革。婦人則半產漏下。旋復花湯主之。

婦人陷{\hbox{\scalebox{0.68}[1]{纟}\kern-0.35em\scalebox{0.64}[1]{巠}}}。漏下黑不解。膠薑湯主之。

婦人少腹滿如敦狀。小便微難而不渴。生後者。此为水与血并结在血室也。大黄甘遂湯主之。

婦人{\hbox{\scalebox{0.68}[1]{纟}\kern-0.35em\scalebox{0.64}[1]{巠}}}水不利。抵当湯主之。
\footnote{「{\hbox{\scalebox{0.68}[1]{纟}\kern-0.35em\scalebox{0.64}[1]{巠}}}水不利」鄧本作「{\hbox{\scalebox{0.68}[1]{纟}\kern-0.35em\scalebox{0.64}[1]{巠}}}水不利下」。}

婦人{\hbox{\scalebox{0.68}[1]{纟}\kern-0.35em\scalebox{0.64}[1]{巠}}}水闭不利。臓堅癖不止。中有乾血。下白物。礬石丸主之。

婦人六十二種風。及腹中血气刺痛。红藍花酒主之。

婦人腹中諸疾痛。当歸芍藥散主之。

婦人腹中痛。小建中湯主之。

问曰。婦人病。饮食如故。煩熱不得卧。而反倚息者。何也。\\
師曰。此名轉胞。不得溺也。以胞系了戾。故致此病。但利小便則愈。宜腎气丸主之。
\footnote{「主之」吳本作「以中有茯苓故也」。}

蛇床子散。温陰中坐藥。

少陰脉滑而數者。陰中即生瘡。[婦人]陰中蚀瘡烂者。狼牙湯洗之。

胃气下泄。陰吹而正喧。此穀气之実也。膏髮煎導之。

\chapter{字形}

実为与洒烂虛準术体气發時處无当粮参脉沈温產别卧麦盖泪{\CJKfontspec[Path=ttf/]{TH-Tshyn-P2}𧌒}{\CJKfontspec[Path=ttf/]{TH-Tshyn-P2}𩊅}

覺學舉

纳约结细缓缩纸纯红绝绞{\hbox{\scalebox{0.6}[1]{纟}\kern-0.32em\scalebox{0.7}[1]{從}}}{\hbox{\scalebox{0.68}[1]{纟}\kern-0.35em\scalebox{0.64}[1]{巠}}}{\hbox{\scalebox{0.6}[1]{纟}\kern-0.32em\scalebox{0.7}[1]{賣}}}{\hbox{\scalebox{0.6}[1]{纟}\kern-0.3em\scalebox{0.63}[1]{岡}}}{\hbox{\scalebox{0.65}[1]{纟}\kern-0.35em\scalebox{0.68}[1]{悤}}}{\hbox{\scalebox{0.6}[1]{纟}\kern-0.3em\scalebox{0.63}[1]{堯}}}

證許譫語設諸診訴謝辯訣記談諺謬誠靄訓認譚識説調論議

齐脐剂

腫種

针铄镇{\hbox{\scalebox{0.7}[1]{钅}\kern-0.4em\scalebox{0.7}[1]{童}}}铨错

门{\CJKfontspec[Path=ttf/]{TH-Tshyn-P2}𥆧}润闷阖闻问闭{\CJKfontspec[Path=ttf/]{TH-Tshyn-P2}𫔭}{\CJKfontspec[Path=ttf/]{TH-Tshyn-P2}𬮦}间癇

内纳

輿輗輒暈渾漸連軺載暫陳運轉輕

尔弥

饮饥饪饱馀蚀

説脱

龙聋

万厉蛎

帶滯

黄横

剛{\hbox{\scalebox{0.6}[1]{纟}\kern-0.3em\scalebox{0.63}[1]{岡}}}

長脹張

啬{\CJKfontspec[Path=ttf/]{TH-Tshyn-P2}𬈧}
\end{flushleft}
\end{document}