% Encoding: UTF-8

\documentclass[12pt,twoside,UTF8,b5paper]{ctexbook}

\usepackage{geometry}%頁面尺寸
	\geometry{left=2cm,right=2cm,top=2cm,bottom=2cm}

\usepackage{fontspec}
	\setmainfont{SimSun}
	\setsansfont{SimSun}

\usepackage[stable]{footmisc}

\usepackage{xeCJK}
	\setCJKmainfont{SimSun}
	\setCJKsansfont{SimSun}
	\setCJKmonofont{SimSun}
%	\setmainfont[Path=ttf/]{TH-Tshyn-P0}
%	\setCJKmainfont[Path=ttf/]{TH-Tshyn-P0}
%	\setCJKsansfont[Path=ttf/]{TH-Tshyn-P0}
%	\setCJKmonofont[Path=ttf/]{TH-Tshyn-P0}

\punctstyle{kaiming}
\CJKsetecglue{}
\raggedright
\setlength{\parindent}{0em}%段落中第一行的缩进量
\setlength{\parskip}{3ex}%两段落间的距离

\title{张仲景方}
\author{张仲景}
\date{\today}

\begin{document}

\maketitle
\tableofcontents%生成目录

\part{類聚方}

\section{桂枝湯}

太陽中風。[脉]陽浮而陰弱。陽浮者熱自发。陰弱者汗自出。嗇嗇惡寒。淅淅惡風。翕翕发熱。鼻鳴。乾嘔。桂枝湯主之。12

太陽病。发熱。汗出。此为榮弱衛强。故使汗出。欲救邪風。宜桂枝湯。95

太陽病。頭痛。发熱。汗出。惡風。桂枝湯主之。13

太陽病。下之。其气上衝者。可与桂枝湯。不衝者。不可与之。15

太陽病三日。已发汗吐下温針而不觧。此为壞病。桂枝湯不復中与也。觀其脉證。知犯何逆。隨證治之。16

桂枝湯本为觧肌。若其人脉浮緊。发熱。无汗。不可与也。常須識此。勿令誤也。16

酒客不可与桂枝湯。得之則嘔。以酒客不喜甘故也。17

服桂枝湯吐者。其後必吐膿血。19

太陽病。初服桂枝湯。反煩不觧者。当先刺風池風府。卻与桂枝湯即愈。24

服桂枝湯。大汗出。若脉[但]洪大者。与桂枝湯。若形如瘧。一日再发者。汗出便觧。宜桂枝二麻黄一湯。25

太陽病。外證未觧。脉浮弱者。当以汗觧。宜桂枝湯。42

太陽病。外證未觧者。不可下。下之为逆。欲觧外者。宜桂枝湯。44

太陽病。先发汗不觧而下之。其脉浮者不愈。浮为在外。而反下之。故令不愈。今脉浮。故在外。当觧其外則愈。宜桂枝湯。45

病常自汗出者。此为榮气和。衛气不和也。榮行脉中。衛行脉外。復发其汗。衛和則愈。宜桂枝湯。53

病人臓无他病。时发熱。自汗出。而不愈者。此衛气不和也。先其时发汗則愈。宜桂枝湯。54

傷寒。不大便六七日。頭痛。有熱者。与承气湯。其小便清者。此为不在裏。續在表也。当发其汗。頭痛者必衄。宜桂枝湯。56

傷寒。发汗已觧。半日許復煩。脉浮數者。可復发汗。宜桂枝湯。57

傷寒。醫下之。續得下利。清穀不止。身体疼痛。急当救裏。後身体疼痛。清便自調。急当救表。救裏宜四逆湯。救表宜桂枝湯。91

傷寒。大下後。復发汗。心下痞。惡寒者。表未觧也。不可攻痞。当先觧表。表觧乃可攻痞。觧表宜桂枝湯。攻痞宜大黄黄連瀉心湯。164

陽明病。脉遲。汗出多。微惡寒者。表未觧也。可发汗。宜桂枝湯。234

病者煩熱。汗出即觧。復如瘧狀。日晡所发者。屬陽明。脉実者。当下之。脉浮虚者。当发其汗。下之宜[大]承气湯。发汗宜桂枝湯。240

太陰病。脉浮者。可发汗。宜桂枝湯。276

下利。腹[胀]滿。身体疼痛者。先温其裏。乃攻其表。温裏宜四逆湯。攻表宜桂枝湯。372

吐利止而身痛不休者。当消息和觧其外。宜桂枝湯小和之。387

師曰。脉婦人得平脉。陰脉小弱。其人渴。不能食。无寒熱。名为軀。桂枝湯主之。法六十日当有娠。設有醫治逆者。卻一月。加吐下者。則絶之。(吳本)

師曰。婦人得平脉。陰脉小弱。其人渴。不能食。无寒熱。名妊娠。桂枝湯主之。於法六十日当有此證。設有醫治逆者。卻一月。加吐下者。則絶之。(鄧本)

產後風。續之數十日不觧。頭微痛。惡寒。时时有熱。心下悶。乾嘔。汗出。雖久。陽旦證續在耳。可与陽旦湯。

\section{桂枝加桂湯}

燒針令其汗。針処被寒。核起而赤者。必发奔豚。气從少腹上衝心者。灸其核上各一壯。与桂枝加桂湯。117

\section{桂枝加芍藥湯}

[本]太陽病。醫反下之。因尔腹滿时痛者。屬太陰。桂枝加芍藥湯主之。大実痛者。桂枝加大黄湯主之。279

\section{桂枝去芍藥湯}

太陽病。下之。脉促。胸滿者。桂枝去芍藥湯主之。若微[惡]寒者。桂枝去芍藥加附子湯主之。21.22

\section{桂枝加葛根湯}

太陽病。項背强几几。反汗出。惡風。桂枝[加葛根]湯主之。14

\section{栝蔞桂枝湯}

太陽病。其證備。身体强。几几然。脉反沈遲。此为痙。栝蔞桂枝湯主之。

\section{桂枝加黄耆湯}

黄汗之病。兩脛自冷。假令发熱。此屬歷節。食已汗出。又身常暮[卧]盜汗出者。此勞气也。若汗出已。反发熱者。久久其身必甲錯。发熱不止者。必生惡瘡。若身重。汗出已輒輕者。久久必身瞤。即胸中痛。又從腰以上必汗出。下无汗。腰髖弛痛。如有物在皮中狀。劇者不能食。身疼重。煩躁。小便不利。此为黄汗。桂枝加黄耆湯主之。

諸病黄家。但利其小便。假令脉浮。当以汗觧之。宜桂枝加黄耆湯主之。

\section{桂枝加大黄湯}

[本]太陽病。醫反下之。因尔腹滿时痛者。屬太陰。桂枝加芍藥湯主之。大実痛者。桂枝加大黄湯主之。279

\section{桂枝加芍藥生薑人参湯}

发汗後。身体疼痛。其脉沈遲。桂枝加芍藥生薑人参湯主之。62

\section{桂枝加厚朴杏仁湯}

喘家作桂枝湯。加厚朴杏仁佳。18

太陽病。下之。微喘者。表未觧故也。桂枝[加厚朴杏仁]湯主之。43

\section{烏頭桂枝湯}

寒疝。腹中痛。逆冷。手足不仁。若身疼痛。灸刺諸藥不能治。抵当烏頭桂枝湯主之。

\section{桂枝加附子湯}

太陽病。发汗。遂漏不止。其人惡風。小便難。四肢微急。難以屈伸。桂枝加附子湯主之。20

\section{桂枝去芍藥加附子湯}

太陽病。下之。脉促。胸滿者。桂枝去芍藥湯主之。若微[惡]寒者。桂枝去芍藥加附子湯主之。21.22

\section{桂枝附子湯}

傷寒八九日。風濕相摶。身体疼煩。不能自轉側。不嘔。不渴。脉浮虚而濇者。桂枝附子湯主之。若其人大便堅。小便自利者。术附子湯主之。174

\section{术附子湯}

傷寒八九日。風濕相摶。身体疼煩。不能自轉側。不嘔。不渴。脉浮虚而濇者。桂枝附子湯主之。若其人大便堅。小便自利者。术附子湯主之。174

\section{甘草附子湯}

風濕相摶。骨節疼煩。掣痛不得屈伸。近之則痛劇。汗出。短气。小便不利。惡風。不欲去衣。或身微腫。甘草附子湯主之。175

\section{桂枝去桂加茯苓术湯}

服桂枝湯。[或]下之。仍頭項强痛。翕翕发熱。无汗。心下滿。微痛。小便不利。桂枝去桂加茯苓[白]术湯主之。28

\section{桂枝去芍藥加麻黄細辛附子湯}

气分。心下堅。大如盤。邊如旋杯。水飲所作。桂枝去芍藥加麻黄細辛附子湯主之。

\section{桂枝去芍藥加皂莢湯}

肺痿。吐涎沫。桂枝去芍藥加皂莢湯主之。

\section{桂枝加龙骨牡蛎湯}

夫失精家。少腹弦急。陰頭寒。目眩。髮落。脉極虚芤遲。为清穀。亡血。失精。脉得諸芤動微緊。男子失精。女子夢交。桂枝加龙骨牡蛎湯主之。天雄散亦主之。

\section{桂枝去芍藥加蜀漆牡蛎龙骨救逆湯}

傷寒。脉浮。醫以火迫劫之。亡陽。[必]驚狂。卧起不安。桂枝去芍藥加蜀漆牡蛎龙骨救逆湯主之。112

火邪者。桂枝去芍藥加蜀漆牡蛎龙骨救逆湯主之。

\section{桂枝甘草龙骨牡蛎湯}

火逆。下之。因燒針。煩躁者。桂枝甘草龙骨牡蛎湯主之。118

\section{桂枝二麻黄一湯}

服桂枝湯。大汗出。若脉[但]洪大者。与桂枝湯。若形如瘧。一日再发者。汗出便觧。宜桂枝二麻黄一湯。25

\section{桂枝二越婢一湯}

太陽病。发熱。惡寒。熱多寒少。脉微弱者。此无陽也。不可[復]发汗。[宜桂枝二越婢一湯。]27

\section{桂枝麻黄各半湯}

太陽病。得之八九日。如瘧狀。发熱。惡寒。熱多寒少。其人不嘔。清便續自可。一日再三发。脉微緩者。为欲愈也。脉微而惡寒者。此为陰陽俱虚。不可復[吐下]发汗也。面反有熱色者。未欲觧也。以其不能得汗出。身必癢。宜桂枝麻黄各半湯。23

\section{小建中湯}

傷寒。陽脉濇。陰脉弦。法当腹中急痛。先与小建中湯。不差者。与小柴胡湯。100

傷寒二三日。心中悸而煩者。小建中湯主之。102

虚勞。裏急。悸。衄。腹中痛。夢失精。四肢痠疼。手足煩熱。咽乾口燥。小建中湯主之。

男子黄。小便自利。当与虚勞小建中湯。

婦人腹中痛。小建中湯主之。

\section{黄耆建中湯}

虚勞。裏急。諸不足。黄耆建中湯主之。

\section{黄耆桂枝五物湯}

血痹。陰陽俱微。寸口関上微。尺中小緊。外證身体不仁。如風痹狀。黄耆桂枝五物湯主之。

\section{黄耆芍藥桂枝苦酒湯}

問曰。黄汗之为病。身体腫。发熱。汗出而渴。狀如風水。汗沾衣。色正黄如檗汁。脉自沈。何從得之。\\
師曰。以汗出入水中浴。水從汗孔入得之。

黄汗。黄耆芍藥桂枝苦酒湯主之。

\section{桂枝甘草湯}

发汗過多。其人叉手自冒心。心下悸。欲得按者。桂枝甘草湯主之。64

\section{半夏散及湯}

少陰病。咽中痛。半夏散及湯主之。313

\section{桂枝人参湯}

太陽病。外證未除。而數下之。遂挾熱而利。利下不止。心下痞堅。表裏不觧。桂枝人参湯主之。163

\section{理中湯}

傷寒。服湯藥。下利不止。心下痞堅。服瀉心湯已。復以他藥下之。利不止。醫以理中与之。利益甚。理中者。理中焦。此利在下焦。赤石脂禹餘糧湯主之。復不止者。当利小便。159

霍亂。頭痛。发熱。身疼痛。熱多。欲飲水者。五苓散主之。寒多。不用水者。理中湯主之。386

大病差後。其人喜唾。久不了了者。胃上有寒。当温之。宜理中丸。396

胸痹。心中痞。留气結在胸。胸滿。脇下逆搶心。枳実薤白桂枝湯主之。理中湯亦主之。

\section{茯苓杏仁甘草湯}

胸痹。胸中气塞。短气。茯苓杏仁甘草湯主之。橘[皮]枳[実生]薑湯亦主之。

\section{茯苓戎鹽湯}

小便不利。蒲灰散主之。滑石白魚散。茯苓戎鹽湯并主之。

\section{葵子茯苓散}

妊娠。有水气。身重。小便不利。洒淅惡寒。起即頭眩。葵子茯苓散主之。

\section{甘草乾薑茯苓白术湯}

腎著之病。其人身体重。腰中冷。如坐水中。形如水狀。反不渴。小便自利。飲食如故。病屬下焦。身勞汗出。衣裏冷濕。久久得之。腰以下冷痛。腹重如帶五千錢。甘[草乾]薑[茯]苓[白]术湯主之。

\section{苓桂术甘湯}

傷寒吐下发汗後。心下逆滿。气上衝胸。起則頭眩。其脉沈緊。发汗則動經。身为振搖。苓桂术甘湯主之。67

心下有痰飲。胸脇支滿。目胘。苓桂术甘湯主之。

夫短气。有微飲。当從小便去之。苓桂术甘湯主之。腎气丸亦主之。

\section{苓桂甘棗湯}

发汗後。其人脐下悸。欲作奔豚。苓桂甘棗湯主之。65

\section{茯苓桂枝五味子甘草湯}

青龙湯下已。多唾。口燥。寸脉沈。尺脉微。手足厥逆。气從少腹上衝胸咽。手足痹。其面翕熱如醉狀。因復下流陰股。小便難。时復冒者。与茯苓桂枝五味子甘草湯。治其气衝。

\section{苓甘五味姜辛湯}

衝气即低。而反更欬。胸滿者。用桂苓五味甘草湯。去桂加乾薑細辛。以治其欬滿。

\section{}

欬滿即止。而復更渴。衝气復发者。以細辛。乾薑为熱藥也。服之当遂渴。而渴反止者。为支飲也。支飲者。法当冒。冒者必嘔。嘔者復内半夏。以去其水。

\section{}

水去。嘔止。其人形腫者。可内麻黄。以其欲逐痹。故不内麻黄。乃内杏仁也。若逆而内麻黄者。必厥。所以然者。以其人血虚。麻黄发其陽故也。(吳本)

水去。嘔止。其人形腫者。加杏仁主之。其證應内麻黄。以其人遂痹。故不内之。若逆而内之者。必厥。所以然者。以其人血虚。麻黄发其陽故也。(鄧本)

\section{}

若面熱如醉[狀者]。此为胃熱上衝熏其面。加大黄以利之。

\section{澤瀉湯}

心下有支飲。其人苦冒眩。澤瀉湯主之。

\section{茯苓澤瀉湯}

胃反。吐而渴欲飲水者。茯苓澤瀉湯主之。

\section{茯苓甘草湯}

傷寒。汗出而渴者。五苓散主之。不渴者。茯苓甘草湯主之。73

傷寒。厥而心下悸。宜先治水。当与茯苓甘草湯。卻治其厥。不尔。水漬入胃。必作利也。356

\section{五苓散}

太陽病发汗後。大汗出。胃中乾。煩躁不得眠。其人欲飲水。当稍飲之。令胃气和即愈。若脉浮。小便不利。微熱。消渴者。五苓散主之。71

发汗已。脉浮數。煩渴者。五苓散主之。72

傷寒。汗出而渴者。五苓散主之。不渴者。茯苓甘草湯主之。73

中風。发熱。六七日不觧而煩。有表裏證。渴欲飲水。水入則吐。此为水逆。五苓散主之。74

病在陽。当以汗觧。反以水潠之或灌之。其熱被劫不得去。益煩。皮上粟起。意欲飲水。反不渴。宜服文蛤散。若不差。与五苓散。若寒実結胸。无熱證者。与三物白散。141

本以下之。故心下痞。与瀉心湯。痞不觧。其人渴而口燥[煩]。小便不利者。五苓散主之。156

太陽病。寸[口]緩。関[上小]浮。尺[中]弱。其人发熱。汗出。復惡寒。不嘔。但心下痞者。此为醫下之故也。若不下。其人不惡寒而渴者。此轉屬陽明。小便數者。大便必堅。不更衣十日。无所苦也。[渴]欲飲水者。少少与之。但以法救之。渴者。宜五苓散。244

霍亂。頭痛。发熱。身疼痛。熱多。欲飲水者。五苓散主之。寒多。不用水者。理中湯主之。386

假令瘦人脐下悸。吐涎沫而癲眩。此水也。五苓散主之。

脉浮。小便不利。微熱。消渴者。宜利小便。发汗。五苓散主之。

\section{茵陳五苓散}

黄疸病。茵陳五苓散主之。

\section{豬苓湯}

陽明病。脉浮而緊。咽乾。口苦。腹滿而喘。发熱。汗出。不惡寒。反惡熱。身重。若发汗則躁。心憒憒。反譫語。若加温針。必怵惕。煩躁。不得眠。若下之。則胃中空虚。客气動膈。心中懊憹。舌上胎者。梔子[豉]湯主之。若渴欲飲水。口乾舌燥者。白虎[加人参]湯主之。若脉浮。发熱。渴欲飲水。小便不利者。豬苓湯主之。221.222.223

陽明病。汗出多而渴者。不可与豬苓湯。以汗多。胃中燥。豬苓湯復利其小便故也。224

少陰病。下利六七日。欬而嘔。渴。心煩不得眠。豬苓湯主之。319

\section{豬苓散}

嘔吐而病在膈上。後思水者觧。急与之。思水者。豬苓散主之。

\section{牡蛎澤瀉散}

大病差後。從腰以下有水气者。牡蛎澤瀉散主之。395

\section{腎气丸}

崔氏八味丸。治腳气上入。少腹不仁。

虚勞。腰痛。少腹拘急。小便不利者。八味腎气丸主之。

夫短气。有微飲。当從小便去之。苓桂术甘湯主之。腎气丸亦主之。

男子消渴。小便反多。以飲一斗。小便一斗。腎气丸主之。

問曰。婦人病。飲食如故。煩熱不得卧。而反倚息者。何也。\\
師曰。此名轉胞。不得尿也。以胞系了戾。故致此病。但利小便則愈。宜腎气丸。

\section{栝蔞瞿麦丸}

小便不利者。有水气。其人若渴。栝蔞瞿麦丸主之。

\section{麻黄湯}

太陽病。頭痛。发熱。身疼。腰痛。骨節疼痛。惡風。无汗而喘。麻黄湯主之。35

太陽与陽明合病。喘而胸滿者。不可下。宜麻黄湯。36

太陽病。十日已去。脉浮細而嗜卧者。外已觧也。設胸滿脇痛者。与小柴胡湯。脉[但]浮者。与麻黄湯。37

太陽病。脉浮緊。无汗。发熱。身疼痛。八九日不觧。表證續在。此当发其汗。服藥已。微除。其人发煩目暝。劇者必衄。衄乃觧。所以然者。陽气重故也。麻黄湯主之。46

脉浮者。病在表。可发汗。宜麻黄湯。51

[太陽病。]脉浮而數者。可发汗。宜麻黄湯。52

傷寒。脉浮緊。不发汗。因致衄者。宜麻黄湯。55

[寸口]脉浮而緊。浮則为風。緊則为寒。風則傷衛。寒則傷榮。榮衛俱病。骨節煩疼。当发其汗。宜麻黄湯。0

陽明中風。脉弦浮大而短气。腹都滿。脇下及心痛。久按之。气不通。鼻乾。不得汗。嗜卧。一身及目悉黄。小便難。有潮熱。时时噦。耳前後腫。刺之小差。外不觧。病過十日。脉續浮者。与[小]柴胡湯。脉但浮。无餘證者。与麻黄湯。若不尿。腹滿加噦者。不治。231.232

陽明病。脉浮。无汗而喘者。发汗則愈。宜麻黄湯。235

\section{麻黄加术湯}

濕家身煩疼。可与麻黄加术湯。发其汗为宜。慎不可以火攻之。

\section{甘草麻黄湯}

裏水。越婢加术湯主之。甘草麻黄湯亦主之。

\section{麻黄附子甘草湯}

少陰病。得之二三日。麻黄附子甘草湯微发汗。以二三日无證。故微发汗。302

水之为病。其脉沈小。屬少陰。浮者为風。无水。虚胀者为气。水。发其汗即已。脉沈者。宜麻黄附子湯。浮者。宜杏子湯。

\section{麻黄細辛附子湯}

少陰病。始得之。反发熱。脉沈者。麻黄細辛附子湯主之。301

\section{麻杏石甘湯}

发汗後。不可更行桂枝湯。汗出而喘。无大熱者。可与麻杏石甘湯。63

下後。不可更行桂枝湯。汗出而喘。无大熱者。可与麻杏石甘湯。162

\section{麻杏薏甘湯}

病者一身尽疼。发熱。日晡所劇者。名風濕。此病傷於汗出当風。或久傷取冷所致也。可与麻杏薏甘湯。

\section{牡蛎湯}

牡蛎湯。治牡瘧。

\section{麻黄淳酒湯}

黄疸。麻黄淳酒湯主之。

\section{半夏麻黄丸}

心下悸者。半夏麻黄丸主之。

\section{小青龙湯}

傷寒。表不觧。心下有水气。乾嘔。发熱而欬。或渴。或利。或噎。或小便不利。少腹滿。或[微]喘。小青龙湯主之。40

傷寒。心下有水气。欬而微喘。发熱。不渴。服湯已而渴者。此寒去。为欲觧。小青龙湯主之。41

病溢飲者。当发其汗。大青龙湯主之。小青龙湯亦主之。

欬逆倚息。[不得卧。]小青龙湯主之。

婦人吐涎沫。醫反下之。心下即痞。当先治其吐涎沫。宜小青龙湯。涎沫止。乃治痞。宜瀉心湯。

\section{小青龙加石膏湯}

肺胀。欬而上气。煩躁而喘。脉浮者。心下有水。小青龙加石膏湯主之。

欬而上气。肺胀。其脉浮。心下有水气。脇下痛引缺盆。小青龙加石膏湯主之。

\section{大青龙湯}

太陽中風。脉浮緊。发熱。惡寒。身体疼痛。不汗出而煩躁者。大青龙湯主之。若脉微弱。汗出。惡風者。不可服之。服之則厥。筋愓肉瞤。此为逆也。38

傷寒。脉浮緩。身不疼。但重。乍有輕时。无少陰證者。大青龙湯发之。39

病溢飲者。当发其汗。大青龙湯主之。小青龙湯亦主之。

\section{文蛤湯}

吐後。渴欲得水而貪飲者。文蛤湯主之。兼主微風。脉緊。頭痛。

\section{文蛤散}

病在陽。当以汗觧。反以水潠之或灌之。其熱被劫不得去。益煩。皮上粟起。意欲飲水。反不渴。宜服文蛤散。若不差。与五苓散。若寒実結胸。无熱證者。与三物白散。141

渴欲飲水不止者。文蛤散主之。

\section{越婢湯}

風水。惡風。一身悉腫。脉浮。不渴。續自汗出。无大熱。越婢湯主之。

\section{越婢加术湯}

裏水者。一身面目洪腫。其脉沈。小便不利。故令病水。假如小便自利。此亡津液。故令渴也。越婢加术湯主之。

裏水。越婢加术湯主之。甘草麻黄湯亦主之。

越婢加术湯。治肉極熱則身体津脱。腠理開。汗大泄。厉風气。下焦腳弱。

\section{越婢加半夏湯}

欬而上气。此为肺胀。其人喘。目如脱狀。脉浮大者。越婢加半夏湯主之。

\section{葛根湯}

太陽病。項背强几几。无汗。惡風。葛根湯主之。31

太陽与陽明合病。而自利者。葛根湯主之。不下利。但嘔者。葛根加半夏湯主之。32.33

太陽病。无汗而小便反少。气上衝胸。口噤不得語。欲作剛痙。葛根湯主之。

\section{葛根加半夏湯}

太陽与陽明合病。而自利者。葛根湯主之。不下利。但嘔者。葛根加半夏湯主之。32.33

\section{葛根黄連黄芩湯}

太陽病。桂枝證。醫反下之。遂利不止。脉促者。表未觧也。喘而汗出者。宜葛根黄連[黄芩]湯。34

\section{小柴胡湯}

太陽病。十日已去。脉浮細而嗜卧者。外已觧也。設胸滿脇痛者。与小柴胡湯。脉[但]浮者。与麻黄湯。37

血弱气尽。腠理開。邪气因入。与正气相摶。結於脇下。正邪分爭。往來寒熱。休作有时。默默不欲飲食。臓腑相連。其痛必下。邪高痛下。故使嘔也。小柴胡湯主之。服柴胡湯已而渴者。屬陽明。以法治之。97

傷寒五六日。中風。往來寒熱。胸脇苦滿。默默不欲飲食。心煩。喜嘔。或胸中煩而不嘔。或渴。或腹中痛。或脇下痞堅。或心下悸。小便不利。或不渴。外有微熱。或欬。小柴胡湯主之。96

得病六七日。脉遲浮弱。惡風寒。手足温。醫再三下之。不能食。其人脇下滿[痛]。面目及身黄。頸項强。小便難。与柴胡湯後必下重。本渴。飲水而嘔。柴胡[湯]不復中与也。食穀者噦。98

傷寒四五日。身体熱。惡風。頸項强。脇下滿。手足温而渴。小柴胡湯主之。99

傷寒。陽脉濇。陰脉弦。法当腹中急痛。先与小建中湯。不差者。与小柴胡湯。100

傷寒中風。有柴胡證。但見一證便是。不必悉具。101

凡柴胡湯證而下之。柴胡證不罷者。復与柴胡湯。必蒸蒸而振。卻发熱汗出而觧。101

太陽病。過經十餘日。反再三下之。後四五日。柴胡證續在。先与小柴胡湯。嘔不止。心下急。其人鬱鬱微煩者。为未觧。与大柴胡湯下之則愈。103

傷寒十三日不觧。胸脇滿而嘔。日晡所发潮熱[。已]而微利。此本当柴胡湯下之。不得利。今反利者。知醫以丸藥下之。非其治也。潮熱者。実也。先宜服小柴胡湯以觧其外。後以柴胡加芒硝湯主之。104

婦人中風七八日。續得寒熱。发作有时。經水適斷。此为熱入血室。其血必結。故使如瘧狀。发作有时。小柴胡湯主之。144

傷寒五六日。頭汗出。微惡寒。手足冷。心下滿。口不欲食。大便堅。其脉細。此为陽微結。必有表。復有裏。沈亦为病在裏。汗出为陽微。假令純陰結。不得有外證。悉入在裏。此为半在外半在裏。脉雖沈緊。不得为少陰病。所以然者。陰不得有汗。今頭汗出。故知非少陰也。可与[小]柴胡湯。設不了了者。得屎而觧。148

傷寒五六日。嘔而发熱。柴胡湯證具。而以他藥下之。柴胡證仍在者。復与柴胡湯。此雖已下之。不为逆。必蒸蒸而振。卻发熱汗出而觧。若心下滿而堅痛者。此为結胸。宜大陷胸湯。若但滿而不痛者。此为痞。柴胡[湯]不復中与也。宜半夏瀉心湯。149

陽明病。发潮熱。大便溏。小便自可。胸脇滿不去。小柴胡湯主之。229

陽明病。脇下堅滿。不大便而嘔。舌上白胎者。可与小柴胡湯。上焦得通。津液得下。胃气因和。身濈然汗出而觧。230

陽明中風。脉弦浮大而短气。腹都滿。脇下及心痛。久按之。气不通。鼻乾。不得汗。嗜卧。一身及目悉黄。小便難。有潮熱。时时噦。耳前後腫。刺之小差。外不觧。病過十日。脉續浮者。与[小]柴胡湯。脉但浮。无餘證者。与麻黄湯。若不尿。腹滿加噦者。不治。231.232

太陽病不觧。轉入少陽。脇下堅滿。乾嘔。不能食。往來寒熱。尚未吐下。脉沈緊者。可与小柴胡湯。若已吐下发汗温針。譫語。柴胡湯證罷。此为壞病。知犯何逆。以法治之。266.267

嘔而发熱者。小柴胡湯主之。379

傷寒差已後。更发熱者。小柴胡湯主之。脉浮者。以汗觧之。脉沈実者。以下觧之。394

諸黄。腹痛而嘔者。宜柴胡湯。

問曰。新產婦人有三病。一者病痙。二者病鬱冒。三者大便難。何谓也。\\
師曰。新產血虚。多汗出。喜中風。故令病痙。亡血復汗。寒多。故令鬱冒。亡津液。胃燥。故大便難。產婦鬱冒。其脉微弱。不能食。大便反堅。但頭汗出。所以然者。血虚而厥。厥而必冒。冒家欲觧。必大汗出。以血虚下厥。孤陽上出。故頭汗出。所以產婦喜汗出者。亡陰血虚。陽气獨盛。故当汗出。陰陽乃復。大便堅。嘔不能食。小柴胡湯主之。病觧。能食。七八日更发熱者。此为胃実。大承气湯主之。

婦人在草蓐得風。四肢苦煩熱。皆自发露所为。頭痛者。与小柴胡湯。頭不痛。但煩者。与三物黄芩湯。

\section{柴胡加芒硝湯}

傷寒十三日不觧。胸脇滿而嘔。日晡所发潮熱[。已]而微利。此本当柴胡湯下之。不得利。今反利者。知醫以丸藥下之。非其治也。潮熱者。実也。先宜服小柴胡湯以觧其外。後以柴胡加芒硝湯主之。104

\section{小柴胡去半夏加栝樓湯}

瘧病发渴者。与小柴胡去半夏加栝樓湯。

\section{柴胡桂枝湯}

傷寒六七日。发熱。微惡寒。肢節煩疼。微嘔。心下支結。外證未去者。柴胡桂枝湯主之。146

发汗多。亡陽。狂語者。不可下。[可]与柴胡桂枝湯。和其榮衛。以通津液。後自愈。

寒疝。腹中痛者。柴胡桂枝湯主之。(吳本)

柴胡桂枝湯方。治心腹卒中痛者。(鄧本)

\section{柴胡桂枝乾薑湯}

傷寒五六日。已发汗而復下之。胸脇滿。微結。小便不利。渴而不嘔。但頭汗出。往來寒熱。心煩。此为未觧。柴胡桂枝乾薑湯主之。147

柴胡桂薑湯。此方治寒多微有熱。或但寒不熱。服一剂如神。故錄之。

\section{柴胡加龙骨牡蛎湯}

傷寒八九日。下之。胸滿。煩。驚。小便不利。譫語。一身尽重。不可轉側。柴胡加龙骨牡蛎湯主之。107

\section{大柴胡湯}

太陽病。過經十餘日。反再三下之。後四五日。柴胡證續在。先与小柴胡湯。嘔不止。心下急。其人鬱鬱微煩者。为未觧。与大柴胡湯下之則愈。103

傷寒十餘日。熱結在裏。復往來寒熱者。与大柴胡湯。但結胸。无大熱者。此为水結在胸脇。[但]頭微汗出。大陷胸湯主之。136

傷寒。发熱。汗出不觧。心中痞堅。嘔吐。下利。大柴胡湯主之。165

按之心下滿痛者。此为実也。当下之。宜大柴胡湯主之。

\section{白虎湯}

傷寒。脉浮滑。此以表有熱。裏有寒。白虎湯主之。176

三陽合病。腹滿。身重。難以轉側。口不仁。面垢。譫語。遺尿。发汗則譫語[甚]。下之則額上生汗。手足厥冷。自汗。白虎湯主之。219

傷寒。脉滑而厥者。裏有熱也。白虎湯主之。350

\section{白虎加人参湯}

服桂枝湯。大汗出。大煩渴不觧。若脉洪大。与白虎[加人参]湯。26

傷寒或吐或下後。七八日不觧。熱結在裏。表裏俱熱。时时惡風。大渴。舌上乾燥而煩。欲飲水數升。白虎[加人参]湯主之。168

傷寒。无大熱。口燥渴。心煩。背微惡寒。白虎[加人参]湯主之。169

傷寒。脉浮。发熱。无汗。其表不觧。不可与白虎湯。渴欲飲水。无表證者。白虎[加人参]湯主之。170

陽明病。脉浮而緊。咽乾。口苦。腹滿而喘。发熱。汗出。不惡寒。反惡熱。身重。若发汗則躁。心憒憒。反譫語。若加温針。必怵惕。煩躁。不得眠。若下之。則胃中空虚。客气動膈。心中懊憹。舌上胎者。梔子[豉]湯主之。若渴欲飲水。口乾舌燥者。白虎[加人参]湯主之。若脉浮。发熱。渴欲飲水。小便不利者。豬苓湯主之。221.222.223

太陽中熱者。暍是也。其人汗出。惡寒。身熱而渴。白虎[加人参]湯主之。

\section{白虎加桂枝湯}

温瘧者。其脉如平。身无寒。但熱。骨節疼煩。时嘔。白虎加桂枝湯主之。

\section{小承气湯}

傷寒。不大便六七日。頭痛。有熱者。与承气湯。其小便清者。此为不在裏。續在表也。当发其汗。頭痛者必衄。宜桂枝湯。56

陽明病。脉遲。雖汗出。不惡寒。其身必重。短气。腹滿而喘。有潮熱。如此者。其外为觧。可攻其裏。若手足濈然汗出者。此大便已堅。[大]承气湯主之。若汗多。微发熱。惡寒者。为外未觧。[桂枝湯主之。]其熱不潮。未可与承气湯。若腹大滿而不大便者。可与小承气湯。微和其胃气。勿令至大下。208

陽明病。潮熱。大便微堅者。可与大承气湯。不堅者。不可与之。若不大便六七日。恐有燥屎。欲知之法。可少与小承气湯。若腹中轉失气者。此有燥屎。乃可攻之。若不轉失气者。此但頭堅後溏。不可攻之。攻之必腹滿。不能食也。欲飲水者。与水即噦。其後发熱者。必大便復堅而少也。以小承气湯和之。若不轉失气者。慎不可攻之。209

陽明病。其人多汗。津液外出。胃中燥。大便必堅。堅則譫語。[小]承气湯主之。[若一服譫語止。莫復服。]213

陽明病。譫語。发潮熱。脉滑疾者。[小]承气湯主之。因与承气湯一升。腹中轉失气者。復与一升。若不轉失气者。勿更与之。明日又不大便。脉反微濇者。此为裏虚。为難治。不可復与承气湯。214

太陽病吐下发汗後。微煩。小便數。大便因堅。可与小承气湯。和之則愈。250

得病二三日。脉弱。无太陽柴胡證。煩躁。心下堅。至四日。雖能食。以[小]承气湯少与。微和之。令小安。至六日。与承气湯一升。若不大便六七日。小便少者。雖不大便。但頭堅後溏。未定成堅。攻之必溏。当須小便利。屎定堅。乃可攻之。宜[大]承气湯。251

下利。譫語者。有燥屎也。宜[小]承气湯。374

小承气湯。治大便不通。噦。數譫語。

\section{厚朴三物湯}

痛而閉者。厚朴三物湯主之。

\section{厚朴七物湯}

病腹滿。发熱十日。脉浮而數。飲食如故。厚朴七物湯主之。

\section{大承气湯}

陽明病。脉遲。雖汗出。不惡寒。其身必重。短气。腹滿而喘。有潮熱。如此者。其外为觧。可攻其裏。若手足濈然汗出者。此大便已堅。[大]承气湯主之。若汗多。微发熱。惡寒者。为外未觧。[桂枝湯主之。]其熱不潮。未可与承气湯。若腹大滿而不大便者。可与小承气湯。微和其胃气。勿令至大下。208

陽明病。潮熱。大便微堅者。可与大承气湯。不堅者。不可与之。若不大便六七日。恐有燥屎。欲知之法。可少与小承气湯。若腹中轉失气者。此有燥屎。乃可攻之。若不轉失气者。此但頭堅後溏。不可攻之。攻之必腹滿。不能食也。欲飲水者。与水即噦。其後发熱者。必大便復堅而少也。以小承气湯和之。若不轉失气者。慎不可攻之。209

傷寒。吐下後未觧。不大便五六日。至十餘日。其人日晡所发潮熱。不惡寒。獨語。如見鬼神之狀。若劇者。发則不識人。順衣妄撮。怵惕不安。微喘。直視。脉弦者生。濇者死。[若]微者。但发熱。譫語。[大]承气湯主之。若一服利。止後服。212

陽明病。譫語。有潮熱。反不能食者。[胃中]必有燥屎五六枚。若能食者。但堅耳。[大]承气湯主之。215

汗出。譫語者。以有燥屎在胃中。此風也。[須下者。]過經乃可下之。下之若早。語言必亂。以表虚裏実故也。下之則愈。宜[大]承气湯。217

二陽并病。太陽證罷。但发潮熱。手足漐漐汗出。大便難而譫語者。下之則愈。宜[大]承气湯。220

陽明病。下之。心中懊憹而煩。胃中有燥屎者。可攻。其人腹微滿。頭堅後溏者。不可攻之。若有燥屎者。宜[大]承气湯。238

病者煩熱。汗出即觧。復如瘧狀。日晡所发者。屬陽明。脉実者。当下之。脉浮虚者。当发其汗。下之宜[大]承气湯。发汗宜桂枝湯。240

大下後。六七日不大便。煩不觧。腹滿痛者。此有燥屎。所以然者。本有宿食故也。宜[大]承气湯。241

病者小便不利。大便乍難乍易。时有微熱。喘冒。不能卧者。有燥屎故也。宜[大]承气湯。242

得病二三日。脉弱。无太陽柴胡證。煩躁。心下堅。至四日。雖能食。以[小]承气湯少与。微和之。令小安。至六日。与承气湯一升。若不大便六七日。小便少者。雖不大便。但頭堅後溏。未定成堅。攻之必溏。当須小便利。屎定堅。乃可攻之。宜[大]承气湯。251

傷寒六七日。目中不了了。睛不和。无表裏證。大便難。身微熱者。此为実也。急下之。宜[大]承气湯。252

陽明病。发熱。汗多者。急下之。宜[大]承气湯。253

发汗不觧。腹滿痛者。急下之。宜[大]承气湯。254

腹滿不減。減不足言。当下之。宜[大]承气湯。255

脉滑而數者。有宿食也。当下之。宜[大]承气湯。256

少陰病。得之二三日。口燥。咽乾者。急下之。宜[大]承气湯。320

少陰病。下利清水。色青者。心下必痛。口乾燥者。急下之。宜[大]承气湯。321

少陰病六七日。腹滿。不大便者。急下之。宜[大]承气湯。322

下利。三部脉皆平。按其心下。堅者。急下之。宜[大]承气湯。

下利。脉遲而滑者。[内]実也。利未欲止。当下之。宜[大]承气湯。

問曰。人病有宿食。何以别之。\\
師曰。寸口脉浮大。按之反濇。尺中亦微而濇。故知有宿食。当下之。宜[大]承气湯。

下利。不欲食者。有宿食也。当下之。宜[大]承气湯。

下利[已]差。至其[年月日]时復发者。此为病不尽。当復下之。宜[大]承气湯。

下利。脉反滑者。当有所去。下乃愈。宜大承气湯。

病腹中滿痛者。为実。当下之。宜大承气湯。

脉雙弦而遲。心下堅。脉大而緊者。陽中有陰也。可下之。宜[大]承气湯。

[剛]痙为病。胸滿。口噤。卧不著席。腳攣急。其人必齘齒。可与大承气湯。

問曰。新產婦人有三病。一者病痙。二者病鬱冒。三者大便難。何谓也。\\
師曰。新產血虚。多汗出。喜中風。故令病痙。亡血復汗。寒多。故令鬱冒。亡津液。胃燥。故大便難。產婦鬱冒。其脉微弱。不能食。大便反堅。但頭汗出。所以然者。血虚而厥。厥而必冒。冒家欲觧。必大汗出。以血虚下厥。孤陽上出。故頭汗出。所以產婦喜汗出者。亡陰血虚。陽气獨盛。故当汗出。陰陽乃復。大便堅。嘔不能食。小柴胡湯主之。病觧。能食。七八日更发熱者。此为胃実。大承气湯主之。

婦人產後七八日。无太陽證。少腹堅痛。此惡露不尽。不大便四五日。趺陽脉微実。再倍其人发熱。日晡所煩躁者。不食。食即譫語。利之即愈。宜大承气湯。熱在裏。結在膀胱也。(吳本)

產後七八日。无太陽證。少腹堅痛。此惡露不尽。不大便。煩躁发熱。切脉微実。再倍发熱。日晡时煩躁者。不食。食則譫語。至夜即愈。宜大承气湯主之。熱在裏。結在膀胱也。(鄧本)

\section{大黄黄連瀉心湯}

心下痞。按之濡。其脉関上浮者。大黄[黄連]瀉心湯主之。154

傷寒。大下後。復发汗。心下痞。惡寒者。表未觧也。不可攻痞。当先觧表。表觧乃可攻痞。觧表宜桂枝湯。攻痞宜大黄黄連瀉心湯。164

\section{瀉心湯}

本以下之。故心下痞。与瀉心湯。痞不觧。其人渴而口燥[煩]。小便不利者。五苓散主之。156

心气不足。吐血。衄血。瀉心湯主之。

婦人吐涎沫。醫反下之。心下即痞。当先治其吐涎沫。宜小青龙湯。涎沫止。乃治痞。宜瀉心湯。

\section{附子瀉心湯主}

心下痞。而復惡寒。汗出者。附子瀉心湯主之。155

\section{大黄附子湯}

脇下偏痛[发熱]。其脉緊弦。此寒也。以温藥下之。宜大黄附子湯。

\section{大黄甘草湯}

食已即吐者。大黄甘草湯主之。

\section{調胃承气湯}

傷寒。脉浮。自汗出。小便數。心煩。微惡寒。腳攣急。反与桂枝湯。欲攻其表。得之便厥。咽乾。煩躁。吐逆者。当作甘草乾薑湯。以復其陽。若厥愈。足温者。更作芍藥甘草湯与之。其腳即伸。若胃气不和。譫語者。少与[調胃]承气湯。若重发汗。復加燒針者。四逆湯主之。29

发汗不觧。反惡寒者。虚故也。芍藥甘草附子湯主之。不惡寒。但熱者。実也。当和胃气。宜調胃承气湯。68.70

太陽病未觧。脉陰陽俱微。必先振汗出而觧。但陽[脉]微者。先汗出而觧。但陰[脉]微者。先下之而觧。汗之宜桂枝湯。下之宜[調胃]承气湯。94

傷寒十三日。過經。譫語者。内有熱也。当以湯下之。若小便利者。大便当堅。而反[下]利。脉調和者。知醫以丸藥下之。非其治也。若自利者。脉当微厥。今反和者。此为内実也。[調胃]承气湯主之。105

太陽病。過經十餘日。心下嗢嗢欲吐。而胸中痛。大便反溏。腹微滿。鬱鬱微煩。先[此]时自極吐下者。与[調胃]承气湯。若不尔者。不可与。但欲嘔。胸中痛。微溏者。此非柴胡湯證。以嘔。故知極吐下也。123

陽明病。不吐下而心煩者。可与[調胃]承气湯。207

太陽病三日。发汗不觧。蒸蒸发熱者。屬胃也。[調胃]承气湯主之。248

傷寒吐後。腹胀滿者。与[調胃]承气湯。249

\section{橘皮大黄朴消湯}

\section{桃仁承气湯}

太陽病不觧。熱結膀胱。其人如狂。血自下。下之即愈。其外不觧者。尚未可攻。当先觧其外。[宜桂枝湯。]外觧已。[但]少腹急結者。乃可攻之。宜桃仁承气湯。106

\section{大黄牡丹湯}

腸癰者。少腹腫痞。按之即痛如淋。小便自調。时时发熱。自汗出。復惡寒。脉遲緊者。膿未成。可下之。当有血。脉洪數者。膿已成。不可下也。大黄牡丹湯主之。

\section{大黄甘遂湯}

婦人少腹滿如敦狀。小便微難而不渴。生後者。此为水与血并結在血室也。大黄甘遂湯主之。

\section{下瘀血湯}

產婦腹痛。法当与枳実芍藥散。假令不愈者。此为腹中有乾血著脐下。宜下瘀血湯主之。

\section{抵当湯}

太陽病六七日。表證續在。脉微而沈。反不結胸。其人发狂。此熱在下焦。少腹当堅滿。小便自利者。下血乃愈。所以然者。以太陽隨經。瘀熱在裏故也。抵当湯主之。124

太陽病。身黄。脉沈結。少腹堅。小便不利者。为无血也。小便自利。其人如狂者。血證諦也。抵当湯主之。125

陽明證。其人喜忘者。必有畜血。所以然者。本有久瘀血。故令喜忘。屎雖堅。大便反易。其色必黑。抵当湯主之。237

病人无表裏證。发熱七八日。雖脉浮數。可下之。宜大柴胡湯。假令下已。脉數不觧。合熱則消穀善飢。至六七日。不大便者。有瘀血。宜抵当湯。若脉數不觧而下不止。必挾熱。便膿血。257.258

婦人經水不利。抵当湯主之。

\section{抵当丸}

傷寒。有熱。少腹滿。應小便不利。今反利者。为有血也。当下之。宜抵当丸。126

\section{土瓜根散}

[婦人]帶下。經水不利。少腹滿痛。經一月再見者。土瓜根散主之。

\section{甘草湯}

少陰病二三日。咽痛者。可与甘草湯。不差者。与桔梗湯。311

\section{桔梗湯}

少陰病二三日。咽痛者。可与甘草湯。不差者。与桔梗湯。311

欬而胸滿。振寒。脉數。咽乾。不渴。时出濁唾腥臭。久久吐膿如米粥者。为肺癰。桔梗湯主之。

\section{排膿湯}

\section{芍藥甘草湯}

傷寒。脉浮。自汗出。小便數。心煩。微惡寒。腳攣急。反与桂枝湯。欲攻其表。得之便厥。咽乾。煩躁。吐逆者。当作甘草乾薑湯。以復其陽。若厥愈。足温者。更作芍藥甘草湯与之。其腳即伸。若胃气不和。譫語者。少与[調胃]承气湯。若重发汗。復加燒針者。四逆湯主之。29

\section{芍藥甘草附子湯}

发汗不觧。反惡寒者。虚故也。芍藥甘草附子湯主之。不惡寒。但熱者。実也。当和胃气。宜調胃承气湯。68.70

\section{甘遂半夏湯}

病者脉伏。其人欲自利。利反快。雖利。心下續堅滿。此为留飲欲去故也。甘遂半夏湯主之。

\section{甘麦大棗湯}

婦人臓躁。喜悲傷。欲哭。象如神靈所作。數欠伸。甘[草小]麦大棗湯主之。

\section{甘草粉蜜湯}

蛔虫之为病。令人吐涎。心痛。发作有时。毒藥不止。甘草粉蜜湯主之。

\section{生薑甘草湯}

肺痿。欬唾涎沫不止。咽燥而渴。生薑甘草湯主之。

\section{甘草乾薑湯}

傷寒。脉浮。自汗出。小便數。心煩。微惡寒。腳攣急。反与桂枝湯。欲攻其表。得之便厥。咽乾。煩躁。吐逆者。当作甘草乾薑湯。以復其陽。若厥愈。足温者。更作芍藥甘草湯与之。其腳即伸。若胃气不和。譫語者。少与[調胃]承气湯。若重发汗。復加燒針者。四逆湯主之。29

肺痿。吐涎沫而不欬者。其人不渴。必遺尿。小便數。所以然者。以上虚不能制下故也。此为肺中冷。必眩。多涎唾。甘草乾薑湯以温之。若服湯已渴者。屬消渴。

\section{乾薑附子湯}

下之後。復发汗。晝日煩躁不得眠。夜而安靜。不嘔。不渴。无表證。脉沈微。身无大熱。乾薑附子湯主之。61

\section{四逆湯}

傷寒。脉浮。自汗出。小便數。心煩。微惡寒。腳攣急。反与桂枝湯。欲攻其表。得之便厥。咽乾。煩躁。吐逆者。当作甘草乾薑湯。以復其陽。若厥愈。足温者。更作芍藥甘草湯与之。其腳即伸。若胃气不和。譫語者。少与[調胃]承气湯。若重发汗。復加燒針者。四逆湯主之。29

傷寒。醫下之。續得下利。清穀不止。身体疼痛。急当救裏。後身体疼痛。清便自調。急当救表。救裏宜四逆湯。救表宜桂枝湯。91

病发熱。頭痛。脉反沈。若不差。身体疼痛。当救其裏。宜四逆湯。92

脉浮而遲。表熱裏寒。下利清穀者。四逆湯主之。225

自利。不渴者。屬太陰。以其臓有寒故也。当温之。宜四逆輩。277

少陰病。脉沈者。急温之。宜四逆湯。323

少陰病。其人飲食入則吐。心中嗢嗢欲吐。復不能吐。始得之。手足寒。脉弦遲。此胸中実。不可下也。当吐之。若膈上有寒飲。乾嘔者。不可吐。当温之。宜四逆湯。324

大汗出。熱不去。内拘急。四肢疼。又下利。厥逆而惡寒。四逆湯主之。353

大汗出或大下利。而厥冷者。四逆湯主之。354

下利。腹[胀]滿。身体疼痛者。先温其裏。乃攻其表。温裏宜四逆湯。攻表宜桂枝湯。372

嘔而脉弱。小便復利。身有微熱。見厥者。難治。四逆湯主之。377

吐利。汗出。发熱。惡寒。四肢拘急。手足厥冷。四逆湯主之。388

既吐且利。小便復利。而大汗出。下利清穀。裏寒外熱。脉微欲絶。四逆湯主之。388

\section{通脉四逆湯}

少陰病。下利清穀。裏寒外熱。手足厥逆。脉微欲絶。身反不惡寒。其人面赤。或腹痛。或乾嘔。或咽痛。或利止而脉不出。通脉四逆湯主之。317

下利清穀。裏寒外熱。汗出而厥。通脉四逆湯主之。370

惡寒。脉微。而復利。利止必亡血。四逆加人参湯主之。385

\section{茯苓四逆湯}

发汗或下之。病仍不觧。煩躁。茯苓四逆湯主之。69

\section{通脉四逆加豬膽汁湯}

吐已下斷。汗出而厥。四肢拘急不觧。脉微欲絶。通脉四逆加豬膽汁湯主之。390

\section{白通湯}

少陰病。下利。白通湯主之。314

少陰病。下利。脉微。服白通湯。利不止。厥逆。无脉。乾嘔。煩者。白通加豬膽汁湯主之。服湯脉暴出者死。微[微]續[出]者生。315

\section{白通加豬膽汁湯}

少陰病。下利。脉微。服白通湯。利不止。厥逆。无脉。乾嘔。煩者。白通加豬膽汁湯主之。服湯脉暴出者死。微[微]續[出]者生。315

\section{玄武湯}

太陽病。发汗。汗出不觧。其人仍发熱。心下悸。頭眩。身瞤動。振振欲擗地。玄武湯主之。82

少陰病。二三日不已。至四五日。腹痛。小便不利。四肢沈重疼痛而利。此为有水气。其人或欬。或小便[自]利。或不利。或嘔。玄武湯主之。316

\section{附子湯}

少陰病。得之一二日。口中和。其背惡寒者。当灸之。附子湯主之。304

少陰病。身体痛。手足寒。骨節痛。脉沈者。附子湯主之。305

\section{附子粳米湯}

腹中寒气。雷鳴。切痛。胸脇逆滿。嘔吐。附子粳米湯主之。

\section{赤丸}

寒气厥逆。赤丸主之。

\section{大烏頭煎}

腹痛。脉弦而緊。弦則衛气不行。[衛气不行]即惡寒。緊則不欲食。邪正相摶。即为寒疝。寒疝遶脐痛。若发則白汗出。手足厥冷。其脉沈弦者。大烏頭煎主之。

\section{烏頭湯}

病歷節。疼痛。不可屈伸。烏頭湯主之。

烏頭湯。治腳气。疼痛。不可屈伸。

烏頭湯。治寒疝。腹中絞痛。賊風入攻五臟。拘急不得轉側。发作有时。使人陰縮。手足厥逆。

\section{薏苡附子散}

胸痹緩急者。薏苡附子散主之。

\section{薏苡附子敗醬散}

腸癰之为病。其身甲錯。腹皮急。按之濡。如腫狀。腹无積聚。身无熱。脉數。此为腸内有癰膿。薏苡附子敗醬散主之。

\section{天雄散}

夫失精家。少腹弦急。陰頭寒。目眩。髮落。脉極虚芤遲。为清穀。亡血。失精。脉得諸芤動微緊。男子失精。女子夢交。桂枝加龙骨牡蛎湯主之。天雄散亦主之。

\section{蜀漆散}

瘧多寒者。名曰牡瘧。蜀漆散主之。

\section{梔子豉湯}

发汗吐下後。虚煩不得眠。若劇者。反覆顛倒。心中懊憹。梔子[豉]湯主之。若少气者。梔子甘草[豉]湯主之。若嘔者。梔子生薑[豉]湯主之。76

发汗或下之。煩熱。胸中窒者。梔子[豉]湯主之。77

傷寒五六日。大下之後。身熱不去。心中結痛者。未欲觧也。梔子[豉]湯主之。78

凡用梔子湯證。其人微溏者。不可与服之。81

陽明病。脉浮而緊。咽乾。口苦。腹滿而喘。发熱。汗出。不惡寒。反惡熱。身重。若发汗則躁。心憒憒。反譫語。若加温針。必怵惕。煩躁。不得眠。若下之。則胃中空虚。客气動膈。心中懊憹。舌上胎者。梔子[豉]湯主之。若渴欲飲水。口乾舌燥者。白虎[加人参]湯主之。若脉浮。发熱。渴欲飲水。小便不利者。豬苓湯主之。221.222.223

陽明病。下之。其外有熱。手足温。不結胸。心中懊憹。飢不能食。但頭汗出。梔子[豉]湯主之。228

下利後更煩。按之心下濡者。为虚煩也。梔子[豉]湯主之。375

\section{梔子甘草豉湯}

发汗吐下後。虚煩不得眠。若劇者。反覆顛倒。心中懊憹。梔子[豉]湯主之。若少气者。梔子甘草[豉]湯主之。若嘔者。梔子生薑[豉]湯主之。76

\section{梔子生薑豉湯}

发汗吐下後。虚煩不得眠。若劇者。反覆顛倒。心中懊憹。梔子[豉]湯主之。若少气者。梔子甘草[豉]湯主之。若嘔者。梔子生薑[豉]湯主之。76

\section{枳実梔子湯}

大病差後勞復者。枳実梔子湯主之。393

\section{梔子枳実豉大黄湯}

酒黄疸。心中懊憹。或熱痛。梔子[枳実豉]大黄湯主之。

\section{大黄黄蘗梔子硝石湯}

黄疸。腹滿。小便不利而赤。自汗出。此为表和裏実。当下之。宜大黄[黄蘗梔子]硝石湯。

\section{茵陳蒿湯}

陽明病。发熱。汗出者。此为熱越。不能发黄也。但頭汗出。身无汗。齐頸而還。小便不利。渴引水漿者。此为瘀熱在裏。身必发黄。茵陳[蒿]湯主之。236

傷寒七八日。身黄如橘。小便不利。腹微滿者。茵陳[蒿]湯主之。260

穀疸之为病。寒熱不食。食即頭眩。心胸不安。久久发黄。为穀疸。茵陳蒿湯主之。

\section{梔子蘗皮湯}

傷寒。身黄。发熱。梔子蘗皮湯主之。261

\section{梔子厚朴湯}

傷寒下後。煩而腹滿。卧起不安。梔子厚朴湯主之。79

\section{梔子乾薑湯}

傷寒。醫以丸藥大下之。身熱不去。微煩。梔子乾薑湯主之。80

\section{酸棗人湯}

虚勞。虚煩。不得眠。酸棗湯主之。

\section{大陷胸湯}

太陽病。脉浮而動數。浮則为風。數則为熱。動則为痛。數則为虚。頭痛。发熱。微盜汗出。而反惡寒者。表未觧也。醫反下之。動數變遲。膈内拒痛。胃中空虚。客气動膈。短气。躁煩。心中懊憹。陽气内陷。心下因堅。則为結胸。大陷胸湯主之。若不結胸。但頭汗出。餘処无汗。齐頸而還。小便不利。身必发黄。134

傷寒六七日。結胸熱実。其脉沈緊。心下痛。按之如石堅。大陷胸湯主之。135

傷寒十餘日。熱結在裏。復往來寒熱者。与大柴胡湯。但結胸。无大熱者。此为水結在胸脇。[但]頭微汗出。大陷胸湯主之。136

太陽病。重发汗而復下之。不大便五六日。舌上燥而渴。日晡所小有潮熱。從心下至少腹堅滿而痛不可近。大陷胸湯主之。137

傷寒五六日。嘔而发熱。柴胡湯證具。而以他藥下之。柴胡證仍在者。復与柴胡湯。此雖已下之。不为逆。必蒸蒸而振。卻发熱汗出而觧。若心下滿而堅痛者。此为結胸。宜大陷胸湯。若但滿而不痛者。此为痞。柴胡[湯]不復中与也。宜半夏瀉心湯。149

\section{大陷胸丸}

結胸者。項亦强。如柔痙狀。下之則和。宜大陷胸丸。131

\section{小陷胸湯}

小結胸者。正在心下。按之則痛。其脉浮滑。小陷胸湯主之。138

\section{枳実薤白桂枝湯}

胸痹。心中痞。留气結在胸。胸滿。脇下逆搶心。枳実薤白桂枝湯主之。理中湯亦主之。

\section{栝蔞薤白白酒湯}

胸痹之病。喘息欬唾。胸背痛。短气。寸口脉沈而遲。関上小緊數。栝蔞薤白白酒湯主之。

\section{栝蔞薤白半夏湯}

胸痹不得卧。心痛徹背者。栝蔞薤白半夏湯主之。

\section{瓜蒂散}

病如桂枝證。頭不痛。項不强。寸[口]脉微浮。胸中痞堅。气上衝咽喉。不得息。此为胸有寒。当吐之。宜瓜蒂散。166

病者手足厥冷。脉乍緊。邪結在胸中。心下滿而煩。飢不能食。病在胸中。当吐之。宜瓜蒂散。355

宿食在上脘。当吐之。宜瓜蒂散。

\section{小半夏湯}

嘔家本渴。渴者为欲觧。今反不渴。心下有支飲故也。小半夏湯主之。

黄疸病。小便色不變。欲自利。腹滿而喘。不可除熱。熱除必噦。噦者。小半夏湯主之。

諸嘔吐。穀不得下者。小半夏湯主之。

\section{小半夏加茯苓湯}

卒嘔吐。心下痞。膈間有水。眩悸者。[小]半夏加茯苓湯主之。

先渴後嘔。为水停心下。此屬飲家。小半夏[加]茯苓湯主之。

\section{大半夏湯}

胃反嘔吐者。大半夏湯主之。

\section{生薑半夏湯}

病人胸中似喘不喘。似嘔不嘔。似噦不噦。徹心中憒憒然无奈。生薑半夏湯主之。

\section{苦酒湯}

少陰病。咽中傷。生瘡。不能語言。聲不出者。苦酒湯主之。312

\section{半夏厚朴湯}

婦人咽中如有炙臠。半夏厚朴湯主之。

\section{半夏乾薑散}

乾嘔。吐逆。吐涎沫。半夏乾薑散主之。

\section{厚朴生薑半夏甘草人参湯}

发汗後。腹胀滿者。厚朴[生薑半夏甘草人参]湯主之。66

\section{乾薑人参半夏丸}

妊娠。嘔吐不止。乾薑人参半夏丸主之。

\section{乾薑黄芩黄連人参湯}

傷寒。本自寒下。醫復吐下之。寒格。更逆吐下。食入即出。乾薑黄芩黄連人参湯主之。359

\section{黄連湯}

傷寒。胸中有熱。胃中有邪气。腹中痛。欲嘔吐。黄連湯主之。173

\section{半夏瀉心湯}

傷寒五六日。嘔而发熱。柴胡湯證具。而以他藥下之。柴胡證仍在者。復与柴胡湯。此雖已下之。不为逆。必蒸蒸而振。卻发熱汗出而觧。若心下滿而堅痛者。此为結胸。宜大陷胸湯。若但滿而不痛者。此为痞。柴胡[湯]不復中与也。宜半夏瀉心湯。149

嘔而腸鳴。心下痞者。半夏瀉心湯主之。

\section{甘草瀉心湯}

傷寒中風。醫反下之。其人下利日數十行。穀不化。腹中雷鳴。心下痞堅而滿。乾嘔。心煩。不能得安。醫見心下痞。谓病不尽。復下之。其痞益甚。此非結熱。但以胃中虚。客气上逆。故使之堅。甘草瀉心湯主之。158

狐惑之为病。狀如傷寒。默默欲眠。目不得閉。卧起不安。蝕於喉为惑。蝕於陰为狐。不欲飲食。惡聞食臭。其面目乍赤。乍黑。乍白。蝕於上部則聲喝。甘草瀉心湯主之。蝕於下部則咽乾。苦参湯洗之。蝕於肛者。雄黄熏之。

\section{生薑瀉心湯}

傷寒。汗出。觧之後。胃中不和。心下痞堅。乾噫食臭。脇下有水气。腹中雷鳴。下利者。生薑瀉心湯主之。157

\section{旋復代赭湯}

傷寒。发汗[或]吐[或]下。觧後。心下痞堅。噫气不除者。旋復代赭湯主之。161

\section{吳茱萸湯}

食穀欲嘔者。屬陽明。[吳]茱萸湯主之。得湯反劇者。屬上焦。243

少陰病。吐利。手足逆[冷]。煩躁欲死者。[吳]茱萸湯主之。309

乾嘔。吐涎沫。頭痛者。吳茱萸湯主之。378

嘔而胸滿者。吳茱萸湯主之。

\section{大建中湯}

心胸中大寒痛。嘔。不能飲食。腹中寒。上衝皮起。出見有頭足。上下痛而不可觸近。大建中湯主之。

\section{黄連阿膠湯}

少陰病。得之二三日以上。心中煩。不得卧。黄連阿膠湯主之。303

\section{黄芩湯}

太陽与少陽合病。自下利者。与黄芩湯。若嘔者。与黄芩加半夏生薑湯。172

\section{黄芩加半夏生薑湯}

太陽与少陽合病。自下利者。与黄芩湯。若嘔者。与黄芩加半夏生薑湯。172

乾嘔而利者。黄芩加半夏生薑湯主之。

\section{黄芩湯}

乾嘔。下利。黄芩湯主之。

\section{三物黄芩湯}

婦人在草蓐得風。四肢苦煩熱。皆自发露所为。頭痛者。与小柴胡湯。頭不痛。但煩者。与三物黄芩湯。

\section{白頭翁湯}

熱利下重者。白頭翁湯主之。371

下利。欲飲水者。为有熱也。白頭翁湯主之。373

\section{白頭翁加甘草阿膠湯}

產後下利。虚極。白頭翁加甘草阿膠湯主之。

\section{木防己湯}

膈間支飲。其人喘滿。心下痞堅。面色黎黑。其脉沈緊。得之數十日。醫吐下之不愈。木防己湯主之。虚者即愈。実者三日復发。復与不愈者。宜去石膏加茯苓芒硝湯。

\section{木防己去石膏加茯苓芒硝湯}

膈間支飲。其人喘滿。心下痞堅。面色黎黑。其脉沈緊。得之數十日。醫吐下之不愈。木防己湯主之。虚者即愈。実者三日復发。復与不愈者。宜去石膏加茯苓芒硝湯。

\section{防己茯苓湯}

皮水为病。四肢腫。水气在皮膚中。四肢聶聶動者。防己茯苓湯主之。

\section{防己黄耆湯}

風濕。脉浮。身重。汗出。惡風者。防己黄耆湯主之。

風水。脉浮。身重。汗出。惡風者。防己黄耆湯主之。腹痛加芍藥。

夫風水。脉浮为在表。其人或頭汗出。表无他病。病者但下重。故知從腰以上为和。腰以下当腫及陰。難以屈伸。防己黄耆湯主之。

\section{枳実术湯}

心下堅。大如盤。邊如旋盤。水飲所作。枳[実]术湯主之。

\section{枳実芍藥散}

產後腹痛。煩滿不得卧。枳実芍藥散主之。

產婦腹痛。法当与枳実芍藥散。假令不愈者。此为腹中有乾血著脐下。宜下瘀血湯主之。

\section{排膿散}

\section{桂枝生薑枳実湯}

心中痞。諸逆。心懸痛。桂枝生薑枳実湯主之。

\section{橘皮枳実生薑湯}

胸痹。胸中气塞。短气。茯苓杏仁甘草湯主之。橘[皮]枳[実生]薑湯亦主之。

\section{橘皮湯}

乾嘔。噦。若手足厥者。橘皮湯主之。

\section{橘皮竹茹湯}

噦逆者。橘皮竹茹湯主之。

\section{茯苓飲}

茯苓飲。治心胸中有停痰宿水。自吐出水後。心胸間虚。气滿。不能食。消痰气。令能食。

\section{桂枝茯苓丸}

婦人妊娠。經斷三月而得漏下。下血四十日不止。胎欲動。在於脐上。此为妊娠。六月動者。前三月經水利时。胎也。下血者。後斷三月。衃也。所以下血不止者。其癥不去故也。当下其癥。宜桂枝茯苓丸。(吳本)

婦人宿有癥病。經斷未及三月。而得漏下不止。胎動在脐上者。为癥痼害。妊娠六月動者。前三月經水利时。胎下血者。後斷三月。衃也。所以血不止者。其癥不去故也。当下其癥。桂枝茯苓丸主之。(鄧本)

\section{膠艾湯}

師曰。婦人有漏下者。有半產後因續下血都不絶者。有妊娠下血者。假令妊娠腹中痛。为胞阻。膠艾湯主之。

\section{赤石脂禹餘糧湯}

傷寒。服湯藥。下利不止。心下痞堅。服瀉心湯已。復以他藥下之。利不止。醫以理中与之。利益甚。理中者。理中焦。此利在下焦。赤石脂禹餘糧湯主之。復不止者。当利小便。159

\section{桃花湯}

少陰病。下利。便膿血。桃花湯主之。306

少陰病二三日至四五日。腹痛。小便不利。下利不止。便膿血。桃花湯主之。307

下利。便膿血者。桃花湯主之。

\section{蜜煎}

陽明病。自汗出。若发汗。小便自利者。此为[津液]内竭。雖堅。不可攻之。当須自欲大便。宜蜜煎。導而通之。若土瓜根及豬膽汁。皆可以導。233

\section{麻子仁丸}

跌陽脉浮而濇。浮則胃气强。濇則小便數。浮濇相摶。大便則堅。其脾为約。麻子仁丸主之。247

\section{己椒藶黄丸}

腹滿。口舌乾燥。此腸間有水气。己椒藶黄丸主之。

\section{葶藶大棗瀉肺湯}

肺癰。喘不得卧。葶藶大棗瀉肺湯主之。

肺癰。胸滿胀。一身面目浮腫。鼻塞。清涕出。不聞香臭酸辛。欬逆上气。喘鳴迫塞。葶藶大棗瀉肺湯主之。

支飲。不得息。葶藶大棗瀉肺湯主之。

\section{十棗湯}

太陽中風。下利。嘔逆。表觧乃可攻之。其人漐漐汗出。发作有时。頭痛。心下痞堅滿。引脇下痛。乾嘔。短气。汗出。不惡寒。此为表觧。裏未和。十棗湯主之。152

病懸飲者。十棗湯主之。

欬家。其脉弦。为有水。十棗湯主之。

夫有支飲家。欬煩。胸中痛者。不卒死。至一百日[或]一歲。宜十棗湯。

\section{桔梗白散}

病在陽。当以汗觧。反以水潠之或灌之。其熱被劫不得去。益煩。皮上粟起。意欲飲水。反不渴。宜服文蛤散。若不差。与五苓散。若寒実結胸。无熱證者。与三物白散。141

欬而胸滿。振寒。脉數。咽乾。不渴。时出濁唾腥臭。久久吐膿如米粥者。为肺癰。桔梗白散主之。

\section{走馬湯}

卒疝。走馬湯主之。(吳本)

走馬湯。治中惡。心痛。腹胀。大便不通。(鄧本)

\section{備急丸}

\section{礬石湯}

礬石湯。治腳气衝心。

\section{硝石礬石散}

黄家。日晡所发熱。而反惡寒。此为女勞得之。膀胱急。少腹滿。身尽黄。額上黑。足下熱。因作黑疸。其腹胀如水狀。大便必黑。时溏。此女勞之病。非水也。腹滿者難治。硝石礬石散主之。

\section{礬石丸}

婦人經水閉不利。臓堅癖不止。中有乾血。下白物。礬石丸主之。

\section{蛇床子散}

蛇床子散。温陰中坐藥。

\section{竹葉石膏湯}

傷寒觧後。虚羸少气。气逆欲吐。竹葉石膏湯主之。397

\section{麦門冬湯}

大逆上气。咽喉不利。止逆下气者。麦門冬湯主之。

\section{雄黄}

狐惑之为病。狀如傷寒。默默欲眠。目不得閉。卧起不安。蝕於喉为惑。蝕於陰为狐。不欲飲食。惡聞食臭。其面目乍赤。乍黑。乍白。蝕於上部則聲喝。甘草瀉心湯主之。蝕於下部則咽乾。苦参湯洗之。蝕於肛者。雄黄熏之。

\section{頭風摩散}

頭風摩散方。

\section{皂莢丸}

欬逆。气上衝。唾濁。但坐不得卧。皂莢丸主之。(吳本)

欬逆上气。时时唾濁。但坐不得卧。皂莢丸主之。(鄧本)

\section{葦莖湯}

葦莖湯。治欬。有微熱。煩滿。胸中甲錯。是为肺癰。

\section{当歸生薑羊肉湯}

寒疝。腹中痛。及脇痛裏急者。当歸生薑羊肉湯主之。

產後腹中㽲痛。当歸生薑羊肉湯主之。并治腹中寒疝。虚勞不足。

\section{蒲灰散}

小便不利。蒲灰散主之。滑石白魚散。茯苓戎鹽湯并主之。

厥而皮水者。蒲灰散主之。

\section{滑石白魚散}

小便不利。蒲灰散主之。滑石白魚散。茯苓戎鹽湯并主之。

\section{豬膏髮煎}

諸黄。豬膏髮煎主之。

胃气下泄。陰吹而正喧。此穀气之実也。膏髮煎導之。

\section{蘗葉湯}

吐血不止者。蘗葉湯主之。

\section{黄土湯}

下血。先便後血。此遠血也。黄土湯主之。

\section{雞屎白散}

轉筋之为病。其人臂腳直。脉上下行。微弦。轉筋入腹者。雞屎白散主之。

\section{蜘蛛散}

陰狐疝气者。偏有大小。时时上下。蜘蛛散主之。

\section{当歸芍藥散}

婦人懷娠。腹中㽲痛。当歸芍藥散主之。

婦人腹中諸疾痛。当歸芍藥散主之。

\section{当歸貝母苦参丸}

妊娠。小便難。飲食如故。当歸貝母苦参丸主之。

\section{狼牙湯}

少陰脉滑而數者。陰中即生瘡。[婦人]陰中蝕瘡爛者。狼牙湯洗之。

\section{小兒疳虫蝕齒方}

\section{炙甘草湯}

傷寒。脉結代。心動悸。炙甘草湯主之。177

傷寒。脉結代。心中驚悸。炙甘草湯主之。(玉函)177

虚勞不足。汗出而悶。脉結。心悸。行動如常。不出百日。危急者。十一日死。炙甘草湯主之。

肺痿。涎唾多。心中嗢嗢液液者。炙甘草湯主之。

\section{当歸四逆湯}

手足厥寒。脉細欲絶者。当歸四逆湯主之。若其人内有久寒者。当歸四逆加吳茱萸生薑湯主之。351.352

\section{当歸四逆加吳茱萸生薑湯}

手足厥寒。脉細欲絶者。当歸四逆湯主之。若其人内有久寒者。当歸四逆加吳茱萸生薑湯主之。351.352

\section{麻黄連軺赤小豆湯}

傷寒。瘀熱在裏。身必发黄。麻黄連軺赤小豆湯主之。262

\section{四逆散}

少陰病。四逆。其人或欬。或悸。或小便不利。或腹中痛。或泄利下重。四逆散主之。318

\section{射干麻黄湯}

欬而上气。喉中水雞聲。射干麻黄湯主之。

\section{桂枝芍藥知母湯}

諸肢節疼痛。身体魁瘰。腳腫如脱。頭眩短气。嗢嗢欲吐。桂枝芍藥知母湯主之。

\section{当歸建中湯}

内補当歸建中湯。治婦人產後虚羸不足。腹中刺痛不止。吸吸少气。或苦少腹拘急。攣痛引腰背。不能食飲。產後一月。日得服四五剂为善。令人强壯。

\section{續命湯}

續命湯。治中風痱。身体不能自收。口不能言。冒昧不知痛処。或拘急不得轉側。

\section{烏梅丸}

傷寒。脉微而厥。至七八日。膚冷。其人躁无暫安时者。此为臓厥。非蛔厥也。蛔厥者。其人当吐蛔。今病者靜。而復时煩。此为臓寒。蛔上入膈。故煩。須臾復止。得食而嘔。又煩者。蛔聞食臭出。其人常自吐蛔。蛔厥者。烏梅丸主之。338

\section{大黄䗪虫丸}

五勞。虚極。羸瘦。腹滿。不能飲食。食傷。憂傷。飲傷。房室傷。飢傷。勞傷。經絡榮衛气傷。内有乾血。肌膚甲錯。兩目黯黑。緩中補虚。大黄䗪虫丸主之。

\section{麻黄升麻湯}

傷寒六七日。大下後。[寸]脉沈遲。手足厥逆。下部脉不至。咽喉不利。唾膿血。泄利不止者。为難治。麻黄升麻湯主之。357

\section{禹餘糧丸}

汗家。重发汗。必恍惚心亂。小便已。陰疼。与禹餘糧丸。88

\section{豬膚湯}

少陰病。下利。咽痛。胸滿。心煩。豬膚湯主之。310

\section{燒裩散}

傷寒陰易之为病。其人身体重。少气。少腹裏急。或引陰中拘攣。熱上衝胸。頭重不欲舉。眼中生眵。[眼胞赤。]膝脛拘急。燒裩散主之。392

\section{瓜蒂湯}

太陽中暍。身熱疼重而脉微弱。此以夏月傷冷水。水行皮膚中所致也。瓜蒂湯主之。

\section{百合知母湯}

百合病发汗後。更发者。百合知母湯主之。

\section{百合滑石代赭湯}

百合病下之後。更发者。百合滑石代赭湯主之。

\section{百合雞子湯}

百合病吐之後。更发者。百合雞子湯主之。

\section{百合地黄湯}

百合病。不經发汗吐下。病形如初者。百合地黄湯主之。

\section{百合洗方}

百合病。經月不觧。變成渴者。百合洗方主之。

\section{栝蔞牡蛎散}

百合病。渴不差者。栝蔞牡蛎散主之。

\section{百合滑石散}

百合病。變发熱者。百合滑石散主之。

\section{}

百合病。變腹中滿痛者。但取百合根隨多少熬令黄色搗篩为散飲服方寸匕。日三。滿消痛止。

\section{赤小豆当歸散}

病者脉數。无熱。微煩。默默。但欲卧。汗出。初得之三四日。目赤如鳩眼。七八日目四眥黑。若能食者。膿已成也。赤[小]豆当歸散主之。

\section{升麻鱉甲湯}

陽毒之为病。面赤斑斑如锦文。咽喉痛。唾膿血。五日可治。七日不可治。升麻鱉甲湯主之。

\section{}

陰毒之为病。面目青。身痛如被杖。咽喉痛。五日可治。七日不可治。升麻鱉甲湯去雄黄蜀椒主之。

\section{鱉甲煎丸}

問曰。瘧以月一日发。当以十五日愈。設不差。当月尽觧。如其不差。当如何。\\
師曰。此結为癥瘕。名曰瘧母。急治之。宜鱉甲煎丸。

\section{侯氏黑散}

大風。四肢煩重。心中惡寒不足者。侯氏黑散主之。

\section{風引湯}

風引湯。除熱。主癱癇。

\section{防己地黄湯}

病如狂狀。妄行。獨語不休。惡寒熱。其脉浮。防己地黄湯主之。

\section{三黄湯}

三黄湯。治中風。手足拘急。百節疼痛。煩熱心亂。惡寒。經日不欲飲食。

\section{术附子湯}

术附子湯。治風虚頭重眩。苦極不知食味。暖肌補中。益精气。

\section{薯蕷丸}

虚勞。諸不足。風气百疾。薯蕷丸主之。

\section{獺肝散}

獺肝散。治冷勞。又主鬼疰。一門相染。

\section{澤漆湯}

欬而脉浮者。厚朴麻黄湯主之。脉沈者。澤漆湯主之。

\section{奔豚湯}

奔豚。气上衝胸。腹痛。往來寒熱。奔豚湯主之。

\section{烏頭赤石脂丸}

心痛徹背。背痛徹心。烏頭赤石脂丸主之。

\section{九痛丸}

九痛丸。治九種心痛。

\section{温經湯}

問曰。婦人年五十所。病下血。數十日不止。暮即发熱。少腹裏急。腹滿。手掌煩熱。唇口乾燥。何也。\\
師曰。此病屬帶下。何以故。曾經半產。瘀血在少腹不去。何以知之。其證唇口乾燥。故知之。当以温經湯主之。

\section{旋復花湯}

肝著。其人常欲蹈其胸上。先未苦时。但欲飲熱。旋復花湯主之。

寸口脉弦而大。弦則为減。大則为芤。減則为寒。芤則为虚。寒虚相摶。此名曰革。婦人則半產漏下。旋復花湯主之。

\section{厚朴大黄湯}

支飲。胸滿者。厚朴大黄湯主之。

\section{瓜蒂湯}

諸黄。瓜蒂湯主之。

\section{赤小豆当歸散}

下血。先血後便。此近血也。赤小豆当歸散主之。

\section{紫参湯}

下利。肺痛。紫参湯主之。

\section{诃梨勒散}

气利。诃梨勒散主之。

\section{王不留行散}

病金瘡。王不留行散主之。

\section{黄連粉}

浸淫瘡。從口流向四肢者。可治。從四肢流來入口者。不可治。黄連粉主之。

\section{藜蘆甘草湯}

病人常以手指臂腫動。此人身体瞤瞤者。藜蘆甘草湯主之。

\section{附子湯}

婦人懷娠六七月。脉弦。发熱。其胎愈胀。腹痛。惡寒者。少腹如扇。所以然者。子臓開故也。当以附子湯温其臓。

\section{当歸散}

婦人妊娠。宜常服当歸散。

\section{白术散}

妊娠養胎。白术散主之。

\section{竹葉湯}

產後中風。发熱。面正赤。喘而頭痛。竹葉湯主之。

\section{竹皮大丸}

婦人乳中虚。煩亂。嘔逆。安中益气。竹皮大丸主之。

\section{膠薑湯}

婦人陷經。漏下黑不觧。膠薑湯主之。

\section{紅藍花酒}

婦人六十二種風。及腹中血气刺痛。紅藍花酒主之。

\part{傷寒論}

\chapter{辨脉法}

問曰。脉有陰陽。何谓也。\\
答曰。凡脉大。浮。數。動。滑。此名陽也。脉沈。濇。弱。弦。微。此名陰也。凡陰病見陽脉者生。陽病見陰脉者死。

問曰。脉有陽結陰結者。何以别之。\\
答曰。其脉[自]浮而數。能食。不大便。名曰陽結也。期十七日当劇。其脉[自]沈而遲。不能食。身体重。大便反堅。名曰陰結。期十四日当劇。
	\footnote{宋本「不大便」下有「此为実」三字。}

問曰。病有洒淅惡寒。而復发熱者。何。\\
答曰。陰脉不足。陽往從之。陽脉不足。陰往乘之。\\
問曰。何谓陽不足。\\
答曰。假令寸口脉微。为陽不足。陰气上入陽中則洒淅惡寒。\\
問曰。何谓陰不足。\\
答曰。尺脉弱为陰不足。陽气下陷入陰中則发熱。\\

陽脉浮。陰脉弱。則血虚。血虚則筋急也。
	\footnote{「筋急」敦煌甲本作「傷筋」,脉經作「筋傷」。}

*脉陽浮陰濡而弱。弱則血虚。血虚則傷筋。(敦煌甲)

其脉沈者。榮气微也。

其脉浮。而汗出如流珠者。衛气衰也。

*其脉浮。則汗出如流珠。衛气衰。(敦煌甲)

榮气微者。加燒針則血留不行。更发熱而躁煩也。

脉靄靄如车盖者。名曰陽結。

脉累累如順长竿者。名曰陰結。

脉聶聶如吹榆莢者。名曰散。
	\footnote{「散」敦煌甲本作「數」。}

脉潎潎如羹上肥者。陽气微。
	\footnote{「微」玉函作「脱」。}

脉縈縈如蜘蛛絲者。陽气衰。

脉绵绵如[瀉]漆之絶者。亡其血。
	\footnote{「瀉」字敦煌甲本无。}

脉來緩。时一止復來者。名曰結。脉來數。时一止復來者。名曰促。脉陽盛則促。陰盛則結。此皆病脉。
	\footnote{「陰盛則結」敦煌甲本「結」作「緩」。}

陰陽相摶。名曰動。陽動則汗出。陰動則发熱。形冷惡寒者。三焦傷也。
	\footnote{「三焦傷也」敦煌甲本作「此为進」。}

數脉見于関上。[上下]无頭尾。如豆大。厥厥動搖者。名曰動。
	\footnote{敦煌甲本无「上下」二字,「如豆大」作「大如大豆」。}

陽脉浮大而濡。陰脉浮大而濡。陰脉与陽脉同等者。名曰緩。

脉浮而緊者。名曰弦。脉緊者。如轉索无常。弦者。状如弓弦。按之不移。

脉弦而大。弦則为減。大則为芤。減則为寒。芤則为虚。寒虚相摶。此名为革。婦人則半產漏下。男子則亡血失精。

問曰。病有戰而汗出。因得觧者。何。\\
答曰。脉浮而緊。按之反芤。此为本虚。故当戰而汗出。其人本虚。是以发戰。以脉浮。故当汗出而觧。若脉浮而數。按之不芤。此本不虚。若欲自觧。但汗出耳。不发戰也。

問曰。病有不戰而汗出觧者。何。\\
答曰。脉大而浮數。故不戰汗出而觧。

問曰。病有不戰不汗出而觧者。何。
答曰。其脉自微。此以曾发汗。或吐。或下。或亡血。内无津液。陰陽自和。必自愈。故不戰不汗出而觧。

問曰。傷寒三日。脉浮數而微。病人身涼和者。何。\\
答曰。此为欲觧。觧以夜半。脉浮而觧者。濈然汗出。脉數而觧者。必能食。脉微而觧者。必大汗出。

問曰。脉病欲知愈未愈者。何以别之。\\
答曰。寸口。関上。尺中三処。大小浮沈遲數同等。雖有寒熱不觧者。此脉陰陽为和平。雖劇当愈。
	\footnote{「數」敦煌甲本作「疾」。}

師曰。立夏得洪大脉。是其本位。其人病身体苦疼重者。須发其汗。若明日身不疼不重者。不須发汗。若汗濈濈自出者。明日便觧矣。何以言之。立夏脉洪大。是其时脉。故使然也。四时仿此。

問曰。凡病欲知何时得。何时愈。\\
答曰。假令夜半得病者。明日日中愈。日中得病者。夜半愈。何以言之。日中得病。夜半愈者。以陽得陰則觧也。夜半得病。明日日中愈者。以陰得陽則觧也。

寸口脉浮为在表。沈为在裏。數为在腑。遲为在臓。假令脉遲。此为在臓。

趺陽脉浮而濇。少陰脉如經者。其病在脾。法当下利。何以知之。若脉浮大者。气実血虚也。今趺陽脉浮而濇。故知脾气不足。胃气虚也。以少陰脉弦而浮纔見。此为調脉。故稱如經。若反滑而數者。故知当尿膿也。
	\footnote{「以少陰脉弦而浮纔見」敦煌甲本作「少陰脉弦沈纔見为調」,聖惠方作「少陰脉弦而沈此为調脉」。}

寸口脉浮而緊。浮則为風。緊則为寒。風則傷衛。寒則傷榮。榮衛俱病。骨節煩疼。当发其汗。

趺陽脉遲而緩。胃气如經也。趺陽脉浮而數。浮則傷胃。數則動脾。此非本病。醫特下之所为也。榮衛内陷。其數先微。脉反但浮。其人必大便堅。气噫而除。何以言之。本數脉動脾。其數先微。故知脾气不治。大便堅。气噫而除。今脉反浮。其數改微。邪气獨留。心中則飢。邪熱不殺穀。潮熱发渴。數脉当遲緩。脉因前後度數如法。病者則飢。數脉不时。則生惡瘡。

師曰。病人脉微而濇者。此为醫所病也。大发其汗。又數大下之。其人亡血。病当惡寒。後乃发熱。无休止时。夏月盛熱。欲著複衣。冬月盛寒。欲裸其身。所以然者。陽微則惡寒。陰弱則发熱。此醫发其汗。令陽气微。又大下之。令陰气弱。五月之时。陽气在表。胃中虚冷。以陽气内微。不能勝冷。故欲著複衣。十一月之时。陽气在裏。胃中煩熱。以陰气内弱。不能勝熱。故欲裸其身。又陰脉遲濇。故知血亡血。
	\footnote{「夏月」敦煌甲本、圣惠方作「五月」。}

脉浮而大。心下反堅。有熱。屬臓者。攻之。不令发汗。屬腑者。不令溲數。溲數則便堅。汗多則熱愈。汗少則便難。脉遲尚未可攻。

趺陽脉微濇。少陰反堅。微則下逆。濇則躁煩。少陰堅者。便則为難。汗出在頭。穀气为下。便難者。令微溏。不令汗出。甚者遂不得便。煩逆。鼻鳴。上竭下虚。不得復通。
	\footnote{此條僅玉函、敦煌甲本有,「令微溏」敦煌甲本作「愈微溏」。}

脉浮而洪。身汗如油。喘而不休。水漿不下。形体不仁。乍靜乍亂。此为命絶。\\
問曰。上脉狀如此。未知何臓先受其災。\\
答曰。若汗出髮潤。喘而不休者。肺先絶也。身如煙熏。直視搖頭者。心先絶也。唇吻反青。四肢漐習者。肝先絶也。環口黧黑。柔汗发黄者。脾先絶也。溲便遺失。狂言。目反直視者。腎先絶也。\\
又問。未知何臓陰陽先絶。\\
答曰。若陽气先絶。陰气後竭者。其人死。身色必青。若陰气先絶。陽气後竭者。其人死。身色必赤。腋下温。心下熱。
	\footnote{「身汗如油」玉函作「軀汗如油」,聖惠方作「身汗如沾」,敦煌甲本作「軀反如沾」。\\「喘而不休」敦煌甲本作「濡而不休」。\\「乍靜乍亂」敦煌甲本作「乍理乍亂」。}

寸口脉浮大。醫反下之。此为大逆。浮則无血。大則为寒。寒气相摶。則为腸鳴。醫乃不知。而反飲冷水。令汗大出。水得寒气。冷必相摶。其人即饐。

趺陽脉浮。浮則为虚。浮虚相摶。故令气饐。言胃气虚竭也。脉滑則噦。此为醫咎。責虚取実。守空迫血。脉浮。鼻口燥者。必衄。
	\footnote{「鼻口燥」宋本作「鼻中燥」。}

諸脉浮數。当发熱而洒淅惡寒。若有痛処。食飲如常者。畜積有膿。

脉浮而遲。面熱赤而戰愓者。六七日。当汗出而觧。反发熱者。差遲。遲为无陽。不能作汗。其身必癢。
	\footnote{「面熱赤而戰愓者」敦煌甲本作「面熱而赤戴陽」。}

脉虚者。不可吐下发汗。其面反有熱色者。为欲觧。不能汗出。其身必癢。
	\footnote{此條僅玉函、敦煌甲本有。}

寸口脉陰陽俱緊者。法当清邪中上。濁邪中下。清邪中上。名曰潔也。濁邪中下。名曰渾也。陰中於邪。必内慄也。表气微虚。裏气不守。故使邪中於陰。陽中於邪。必发熱。頭痛。項强。頸攣。腰痛。脛痠。所谓陽中霧露之气。故曰清邪中上。濁邪中下。陰气为慄。足膝逆冷。便尿妄出。表气微虚。裏气微急。三焦相溷。内外不通。上焦怫鬱。臓气相熏。口爛食齦。中焦不治。胃气上衝。脾气不轉。胃中为濁。榮衛不通。血凝不流。若衛气前通者。小便赤黄。与熱相摶。因熱作使。遊於經絡。出入臓腑。熱气所過。則为癰膿。若陰气前通者。陽气厥微。陰无所使。客气内入。嚏而出之。聲嗢咽塞。寒厥相追。为熱所擁。血凝自下。状如豚肝。陰陽俱厥。脾气孤弱。五液注下。下焦不闔。清便下重。令便數難。脐築湫痛。命將難全。
	\footnote{「臓气相熏」敦煌甲本、聖惠方作「臓气相動」。\\「令便數難」敦煌甲本、聖惠方作「大便數難」。}

脉陰陽俱緊。口中气出。唇口乾燥。踡卧足冷。鼻中涕出。舌上胎滑。勿妄治也。到七日以上。其人微发熱。手足温者。此为欲觧。或到八日已上。反大发熱者。此为難治。設惡寒者。必欲嘔也。腹内痛者。必欲利也。
	\footnote{「七日以上」宋本、玉函作「七日以來」。}

脉陰陽俱緊。至於吐利。其脉獨不觧。緊去人安。此为欲觧。若脉遲。至六七日不欲食。此为晚发。水停故也。为未觧。食自可者为欲觧。病六七日。手足三部脉皆至。大煩。口噤不能言。其人躁擾者。必欲觧也。若脉和。其人大煩。目重瞼内際黄者。此欲觧也。
	\footnote{「瞼」各本均作「臉」,錢超塵認为應作「瞼」,據改。}

脉浮而數。浮为風。數为虚。風为熱。虚为寒。風虚相摶。則洒淅惡寒。
	\footnote{玉函「洒淅惡寒」後有「而发熱也」。}

趺陽脉浮而微。浮即为虚。微即汗出。
	\footnote{此條僅敦煌甲本、玉函有。}

脉浮而滑。浮[則]为陽。滑[則]为実。陽実相摶。其脉數疾。衛气失度。浮滑之脉數疾。发熱汗出者。此为不治。

傷寒。欬逆。上气。其脉散者死。谓其形損故也。

*脉散。其人形損。傷寒而欬。上气者死。

\chapter{平脉法}

問曰。脉有三部。陰陽相乘。榮衛血气。在人体躬。呼吸出入。上下於中。因息遊布。津液流通。隨时動作。效象形容。春弦秋浮。冬沈夏洪。察色觀脉。大小不同。一时之間。變无經常。尺寸参差。或短或长。上下乖錯。或存或亡。病輒改易。進退低昂。心迷意惑。動失纪綱。願为具陳。令得分明。\\
師曰。子之所問。道之根源。脉有三部。尺寸及関。榮衛流行。不失衡銓。腎沈心洪。肺浮肝弦。此自經常。不失銖分。出入升降。漏刻周旋。水下二刻。一周循環。当復寸口。虚実見焉。變化相乘。陰陽相干。風則浮虚。寒則牢堅。沈潛水畜。支飲急弦。動則为痛。數則熱煩。設有不應。知變所緣。三部不同。病各異端。太過可怪。不及亦然。邪不空見。中必有姦。審察表裏。三焦别焉。知其所舍。消息診看。料度腑臓。獨見若神。为子條記。傳与賢人。

師曰。呼吸者。脉之頭也。初持脉。來疾去遲。此出疾入遲。名曰内虚外実也。初持脉。來遲去疾。此出遲入疾。名曰内実外虚也。

問曰。上工望而知之。中工問而知之。下工脉而知之。願聞其説。\\
師曰。病家人请云。病人若发熱。身体疼。病人自卧。師到。診其脉。沈而遲者。知其差也。何以知之。表有病者。脉当浮大。今脉反沈遲。故知愈也。

假令病人云。腹内卒痛。病人自坐。師到。脉之。浮而大者。知其差也。何以知之。若裏有病者。脉当沈而細。今脉浮大。故知愈也。

師曰。病家人來请云。病人发熱。煩極。明日師到。病人向壁卧。此熱已去也。設令脉不和。処言已愈。設令向壁卧。聞師到。不驚起而眄視。若三言三止。脉之。嚥唾者。此詐病也。設令脉自和。処言汝病大重。当須服吐下藥。鍼灸數十百処乃愈。

師持脉。病人欠者。无病也。脉之呻者。病也。言遲者。風也。搖頭言者。裏痛也。行遲者。表强也。坐而伏者。短气也。坐而下一腳者。腰痛也。裏実護腹如懷卵物者。心痛也。

師曰。伏气之病。以意候之。今月之内。欲有伏气。假令舊有伏气。当須脉之。若脉微弱者。当喉中痛似傷。非喉痹也。病人云。実咽中痛。雖尔。今復欲下利。

問曰。人病恐怖者。其脉何状。\\
師曰。脉形如循絲累累然。其面白脱色也。

問曰。人不飲。其脉何類。\\
師曰。其脉自濇。唇口乾燥也。

問曰。人愧者。其脉何類。\\
師曰。脉浮而面色乍白乍赤。

問曰。經説。脉有三菽。六菽重者。何谓也。\\
師曰。脉。人以指按之。如三菽之重者。肺气也。如六菽之重者。心气也。如九菽之重者。脾气也。如十二菽之重者。肝气也。按之至骨者。腎气也。假令下利。寸口。関上。尺中悉不見脉。然尺中时一小見。脉再舉頭者。腎气也。若見損脉來至。为難治。

問曰。脉有相乘。有縱有横。有逆有順。何也。
師曰。水行乘火。金行乘木。名曰縱。火行乘水。木行乘金。名曰横。水行乘金。火行乘木。名曰逆。金行乘水。木行乘火。名曰順也。

問曰。脉有殘賊。何谓也。\\
師曰。脉有弦。緊。浮。滑。沈。濇。此六者名曰殘賊。能为諸脉作病也。

問曰。脉有災怪。何谓也。\\
師曰。假令人病。脉得太陽。与形證相應。因为作湯。比還送湯如食頃。病人乃大吐。若下利。腹中痛。\\
師曰。我前來不見此證。今乃變異。是名災怪。\\
又問曰。何緣作此吐利。\\
答曰。或有舊时服藥。今乃发作。故名災怪耳。

問曰。東方肝脉。其形何似。\\
師曰。肝者。木也。名厥陰。其脉微弦濡弱而长。是肝脉也。肝病自得濡弱者。愈也。假令得純弦脉者。死。何以知之。以其脉如弦直。是肝臓傷。故知死也。

問曰。南方心脉。其形何似。\\
師曰。心者。火也。名少陰。其脉洪大而长。是心脉也。心病自得洪大者。愈也。假令脉來微去大。故名反。病在裏也。脉來頭小本大者。故名覆。病在表也。上微頭小者。則汗出。下微本大者。則为関格不通。不得尿。頭无汗者可治。有汗者死。

問曰。西方肺脉。其形何似。\\
師曰。肺者。金也。名大陰。其脉毛浮也。肺病自得此脉。若得緩遲者。皆愈。若得數者。則劇。何以知之。數者南方火。火剋西方金。法当癰腫。为難治也。

問曰。二月得毛浮脉。何以処言至秋当死。\\
師曰。二月之时。脉当濡弱。反得毛浮者。故知至秋死。二月肝用事。肝脉屬木。應濡弱。反得毛浮者。是肺脉也。肺屬金。金來剋木。故知至秋死。他皆仿此。

師曰。脉肥人責浮。瘦人責沈。肥人当沈。今反浮。瘦人当浮。今反沈。故責之。

師曰。寸脉下不至関。为陽絶。尺脉上不至関。为陰絶。此皆不治。決死也。若計其餘命死生之期。期以月節剋之也。

師曰。脉病人不病。名曰行尸。以无王气。卒眩仆不識人者。短命則死。人病脉不病。名曰内虚。以无穀神。雖困无苦。

問曰。翕奄沈。名曰滑。何谓也。沈为純陰。翕为正陽。陰陽和合。故令脉滑。関尺自平。陽明脉微沈。食飲自可。少陰脉微滑。滑者緊之浮名也。此为陰実。其人必股内汗出。陰下濕也。

問曰。曾为人所難。緊脉從何而來。\\
師曰。假令亡汗。若吐。以肺裏寒。故令脉緊也。假令欬者。坐飲冷水。故令脉緊也。假令下利。以胃中虚冷。故令脉緊也。

寸口衛气盛。名曰高。榮气盛。名曰章。高章相摶。名曰綱。衛气弱。名曰惵。榮气弱。名曰卑。惵卑相摶。名曰損。衛气和。名曰緩。榮气和。名曰遲。遲緩相摶。名曰沈。

寸口脉緩而遲。緩則陽气长。其色鲜。其顏光。其聲商。毛髮长。遲則陰气盛。骨髓生。血滿。肌肉緊薄鲜堅。陰陽相抱。榮衛俱行。剛柔相摶。名曰强也。

趺陽脉滑而緊。滑者胃气実。緊者脾气强。持実擊强。痛還自傷。以手把刃。坐作瘡也。

寸口脉浮而大。浮为虚。大为実。在尺为関。在寸为格。関則不得小便。格則吐逆。

趺陽脉伏而濇。伏則吐逆。水穀不化。濇則食不得入。名曰関格。

脉浮而大。浮为風虚。大为气强。風气相摶。必成癮疹。身体为癢。癢者名泄風。久久为痂癩。

寸口脉弱而遲。弱者衛气微。遲者榮中寒。榮为血。血寒則发熱。衛为气。气微者心内飢。飢而虚滿不能食也。

趺陽脉大而緊者。当即下利。为難治。

寸口脉弱而緩。弱者陽气不足。緩者胃气有餘。噫而吞酸。食卒不下。气填於膈上也。

趺陽脉緊而浮。浮为气。緊为寒。浮为腹滿。緊为絞痛。浮緊相摶。腸鳴而轉。轉即气動。膈气乃下。少陰脉不出。其陰腫大而虚也。

寸口脉微而濇。微者衛气不行。濇者榮气不足。榮衛不能相將。三焦无所仰。身体痹不仁。榮气不足。則煩疼。口難言。衛气虚。則惡寒數欠。三焦不歸其部。上焦不歸者。噫而酢吞。中焦不歸者。不能消穀引食。下焦不歸者。則遺溲。

趺陽脉沈而數。沈为実。數消穀。緊者病難治。

寸口脉微而濇。微者衛气衰。濇者榮气不足。衛气衰。面色黄。榮气不足。面色青。榮为根。衛为葉。榮衛俱微。則根葉枯槁。而寒慄欬逆。唾腥吐涎沫也。

趺陽脉浮而芤。浮者衛气衰。芤者榮气傷。其身体瘦。肌肉甲錯。浮芤相摶。宗气衰微。四屬斷絶。

寸口脉微而緩。微者衛气疏。疏則其膚空。緩者胃气実。実則穀消而水化也。穀入於胃。脉道乃行。而入於經。其血乃成。榮盛則其膚必疏。三焦絶經。名曰血崩。

趺陽脉微而緊。緊为寒。微則为虚。微緊相摶。則为短气。

少陰脉弱而濇。弱者微煩。濇者厥逆。

趺陽脉不出。脾不上下。身冷膚堅。

少陰脉不至。腎气微。少精血。奔气促迫。上入胸膈。宗气反聚。血結心下。陽气退下。熱歸陰股。与陰相動。令身不仁。此为尸厥。当刺期門。巨闕。

寸口脉微。尺脉緊。其人虚損多汗。知陰常在。絶不見陽也。

寸口諸微亡陽。諸濡亡血。諸弱发熱。諸緊为寒。諸乘寒者。則为厥。鬱冒不仁。以胃无穀气。脾濇不通。口急不能言。戰而慄也。

問曰。濡弱何以反適十一頭。\\
師曰。五臓六腑相乘。故令十一。

問曰。何以知乘腑。何以知乘臓。\\
師曰。諸陽浮數为乘腑。諸陰遲濇为乘臓也。

\chapter{傷寒例}

陰陽大論云。春气温和。夏气暑熱。秋气清涼。冬气冷冽。此則四时正气之序也。冬时嚴寒。万類深藏。君子固密。則不傷於寒。觸冒之者。乃名傷寒耳。其傷於四时之气。皆能为病。以傷寒为毒者。以其最成殺厉之气也。

中而即病者。名曰傷寒。不即病者。寒毒藏於肌膚。至春變为温病。至夏變为暑病。暑病者。熱極重於温也。是以辛苦之人。春夏多温熱病。皆由冬时觸寒所致。非时行之气也。

凡时行者。春时應暖而復大寒。夏时應大熱而反大涼。秋时應涼而反大熱。冬时應寒而反大温。此非其时而有其气。是以一歲之中。长幼之病多相似者。此則时行之气也。

夫欲候知四时正气为病。及时行疫气之法。皆当按斗曆占之。九月霜降節後。宜漸寒。向冬大寒。至正月雨水節後。宜觧也。所以谓之雨水者。以冰雪觧而为雨水故也。至驚蟄二月節後。气漸和暖。向夏大熱。至秋便涼。

從霜降以後。至春分以前。凡有觸冒霜露。体中寒即病者。谓之傷寒也。九月十月。寒气尚微。为病則輕。十一月十二月。寒冽已嚴。为病則重。正月二月。寒漸將觧。为病亦輕。此以冬时不調。適有傷寒之人。即为病也。其冬有非節之暖者。名曰冬温。冬温之毒。与傷寒大異。冬温復有先後。更相重沓。亦有輕重。为治不同。證如後章。

從立春節後。其中无暴大寒。又不冰雪。而有人壯熱为病者。此屬春时陽气。发於冬时伏寒。變为温病。

從春分以後至秋分節前。天有暴寒者。皆为时行寒疫也。三月四月。或有暴寒。其时陽气尚弱。为寒所折。病熱猶輕。五月六月。陽气已盛。为寒所折。病熱則重。七月八月。陽气已衰。为寒所折。病熱亦微。其病与温及暑病相似。但治有殊耳。

十五日得一气。於四时之中。一时有六气。四六名为二十四气也。然气候亦有應至而不至。或有未應至而至者。或有至而太過者。皆成病气也。但天地動靜。陰陽鼓擊者。各正一气耳。是以彼春之暖。为夏之暑。彼秋之忿。为冬之怒。是故冬至之後。一陽爻升。一陰爻降也。夏至之後。一陽气下。一陰气上也。斯則冬夏二至。陰陽合也。春秋二分。陰陽離也。

陰陽交易。人變病焉。此君子春夏養陽。秋冬養陰。順天地之剛柔也。小人觸冒。必嬰暴疹。須知毒烈之气留在何經。而发何病。詳而取之。是以春傷於風。夏必飱泄。夏傷於暑。秋必病瘧。秋傷於濕。冬必咳嗽。冬傷於寒。春必病温。此必然之道。可不審明之。

傷寒之病。逐日淺深。以施方治。今世人傷寒。或始不早治。或治不對病。或日數久淹。困乃告醫。醫人又不依次第而治之。則不中病。皆宜臨时消息制方。无不效也。今搜采仲景舊論。錄其證候。診脉。聲色。對病真方有神驗者。擬防世急也。

又土地温涼。高下不同。物性剛柔。飡居亦異。是黄帝興四方之問。岐伯舉四治之能。以訓後賢。開其未悟者。臨病之工。宜須兩審也。

凡傷於寒則为病熱。熱雖甚不死。若兩感於寒而病者必死。

尺寸俱浮者。太陽受病也。当一二日发。以其脉上連風府。故頭項痛。腰脊强。

尺寸俱长者。陽明受病也。当二三日发。以其脉俠鼻。絡於目。故身熱。目疼。鼻乾。不得卧。

尺寸俱弦者。少陽受病也。当三四日发。以其脉循脇絡於耳。故胸脇痛而耳聋。此三經皆受病。未入於府者。可汗而已。

尺寸俱沈細者。太陰受病也。当四五日发。以其脉布胃中絡於嗌。故腹滿而嗌乾。

尺寸俱沈者。少陰受病也。当五六日发。以其脉貫腎絡於肺。系舌本。故口燥舌乾而渴。

尺寸俱微緩者。厥陰受病也。当六七日发。以其脉循陰器絡於肝。故煩滿而囊縮。此三經皆受病。已入於府。可下而已。

若兩感於寒者。一日太陽受之。即与少陰俱病。則頭痛。口乾。煩滿而渴。二日陽明受之。即与太陰俱病。則腹滿。身熱。不欲食。譫語。三日少陽受之。即与厥陰俱病。則耳聋。囊縮而厥。水漿不入。不知人者。六日死。若三陰三陽。五臓六腑皆受病。則榮衛不行。腑臓不通。則死矣。其不兩感於寒。更不傳經。不加異气者。至七日太陽病衰。頭痛少愈也。八日陽明病衰。身熱少歇也。九日少陽病衰。耳聋微聞也。十日太陰病衰。腹減如故。則思飲食。十一日少陰病衰。渴止舌乾。已而嚏也。十二日厥陰病衰。囊縱。少腹微下。大气皆去。病人精神爽慧也。若過十三日以上不間。尺寸陷者。大危。若更感異气變为他病者。当依舊壞證病而治之。

若脉陰陽俱盛。重感於寒者。變为温瘧。陽脉浮滑。陰脉濡弱者。更遇於風。變为風温。陽脉洪數。陰脉実大者。遇温熱。變为温毒。温毒为病最重也。陽脉濡弱。陰脉弦緊者。更遇温气。變为温疫。以此冬傷於寒。发为温病。脉之變證。方治如説。

凡人有疾。不时即治。隱忍冀差。以成痼疾。小兒女子。益以滋甚。时气不和。便当早言。尋其邪由。及在腠理。以时治之。罕有不愈者。患人忍之。數日乃説。邪气入藏。則難可制。此为家有患。備慮之要。

凡作湯藥。不可避晨夜。覺病須臾。即宜便治。不等早晚。則易愈矣。若或差遲。病即傳變。雖欲除治。必難为力。服藥正如方法。縱意違師。不須治之。

凡傷寒之病。多從風寒得之。始表中風寒。入裏則不消矣。未有温覆而当不消散者。若病不察證。擬欲攻之。猶当先觧表。乃可下之。若表已觧而内不消。大滿。大実。腹堅者。必内有燥屎。自可徐徐下之。雖經四五日。不能为害也。若病不宜下而强攻之。内虚熱入。[則为]協熱遂利。煩燥諸變。不可勝數。輕者困篤。重者必死。
	\footnote{宋本「乃可下之」後有「若表已觧而内不消非大滿猶生寒熱則病不除」。}

世上之士。但務彼翕習之榮。而莫見此傾危之敗。惟明者居然能護其本。近取諸身。夫何遠之有焉。

凡发汗。温服湯藥。其方雖言日三服。若病劇不觧。当促其間。可半日中尽三服。若与病相阻。即便有所覺。重病者。一日一夜当晬时觀之。如服一剂。病證猶在。故当復作本湯服之。至有不肯汗出。服三剂乃觧。若汗不出者。死病也。

凡得时气病。至五六日。而渴欲飲水。飲不能多。不当与也。何者?以腹中熱尚少。不能消之。便更与人作病也。至七八日。大渴欲飲水者。猶当依證与之。与之常令不足。勿極意也。言能飲一斗。与五升。若飲而腹滿。小便不利。若喘若噦。不可与之。忽然大汗出。是为自愈也。

凡得病。反能飲水。此为欲愈之病。其不曉病者。但聞病飲水自愈。小渴者乃强与飲之。因成其禍。不可復數。

凡得病。厥脉動數。服湯藥更遲。脉浮大減小。初躁後靜。此皆愈證也。

凡治温病。可刺五十九穴。又身之穴三百六十有五。其三十九穴灸之有害。七十九穴刺之为災。并中髓也。

凡脉四損。三日死。平人四息。病人脉一至。名曰四損。脉五損。一日死。平人五息。病人脉一至。名曰五損。脉六損。一时死。平人六息。病人脉一至。名曰六損。

脉盛身寒。得之傷寒。脉虚身熱。得之傷暑。脉陰陽俱盛。大汗出不觧者死。脉陰陽俱虚。熱不止者死。脉至乍疏乍數者死。脉至如轉索者。其日死。譫言妄語。身微熱。脉浮大。手足温者生。逆冷。脉沈細者。不過一日死矣。此以前是傷寒熱病證候也。

\chapter{辨太陽病}

太陽之为病。頭項强痛而惡寒。1

太陽病。其脉浮。1

太陽病。发熱。汗出。惡風。脉緩者。为中風。2

太陽中風。发熱而惡寒。0

太陽病。或已发熱。或未发熱。必惡寒。体痛。嘔逆。脉陰陽俱緊者。为傷寒。3

傷寒一日。太陽脉弱。至四日。太陰脉大。0

傷寒一日。太陽受之。脉若靜者。为不傳。頗欲吐。或躁煩。脉數急者。乃为傳。4

傷寒[二三日]。陽明少陽證不見者。为不傳。5
	\footnote{「陽明少陽證」除宋本外其它版本均作「其二陽證」。}

傷寒三日。陽明脉大[者。为欲傳]。186

傷寒三日。少陽脉小者。为欲已。271

太陽病三四日。不吐下。見芤乃汗之。0

太陽病。发熱而渴。不惡寒者。为温病。若发汗已。身灼熱者。为風温。6

風温[之]为病。脉陰陽俱浮。自汗出。身重。多眠睡。鼻息必鼾。語言難出。若被下者。小便不利。直視。失溲。若被火者。微发黄[色]。劇則如驚癇。时瘛瘲。若火熏之。一逆尚引日。再逆促命期。6
	\footnote{「瘛瘲」宋本作「痸瘲」,玉函作「掣縱发作」。}

病有发熱而惡寒者。发於陽也。不熱而惡寒者。发於陰也。发於陽者七日愈。发於陰者六日愈。以陽數七。陰數六故也。7

太陽病。頭痛。至七日自当愈。以其經尽故也。若欲作再經者。当針足陽明。使經不傳則愈。8
	\footnote{「自当愈」宋本、千金翼作「以上自愈者」。}

太陽病欲觧时。從巳尽未。9

風家。表觧而不了了者。十二日愈。10

病人身大熱。反欲得近衣者。熱在皮膚。寒在骨髓也。身大寒。反不欲近衣者。寒在皮膚。熱在骨髓也。11

太陽中風。[脉]陽浮而陰弱。陽浮者熱自发。陰弱者汗自出。嗇嗇惡寒。淅淅惡風。翕翕发熱。鼻鳴。乾嘔。桂枝湯主之。12
	\footnote{孫世揚:「嗇嗇」今諺稱「冷瑟瑟」。}

太陽病。发熱。汗出。此为榮弱衛强。故使汗出。欲救邪風。宜桂枝湯。95

太陽病。頭痛。发熱。汗出。惡風。桂枝湯主之。13

太陽病。項背强几几。反汗出。惡風。桂枝[加葛根]湯主之。14
	\footnote{據孫世揚考證「几几」当作「掔掔」,説文觧字段玉裁註云:「掔之言緊也。」。}

太陽病。下之。其气上衝者。可与桂枝湯。不衝者。不可与之。15

太陽病三日。已发汗吐下温針而不觧。此为壞病。桂枝湯不復中与也。觀其脉證。知犯何逆。隨證治之。16

桂枝湯本为觧肌。若其人脉浮緊。发熱。无汗。不可与也。常須識此。勿令誤也。16

酒客不可与桂枝湯。得之則嘔。以酒客不喜甘故也。17

喘家作桂枝湯。加厚朴杏仁佳。18

服桂枝湯吐者。其後必吐膿血。19

太陽病。发汗。遂漏不止。其人惡風。小便難。四肢微急。難以屈伸。桂枝加附子湯主之。20

太陽病。下之。脉促。胸滿者。桂枝去芍藥湯主之。若微[惡]寒者。桂枝去芍藥加附子湯主之。21.22
	\footnote{「若微惡寒者」除玉函卷二第三、成本外,其它版本均无「惡」字。}


太陽病。得之八九日。如瘧狀。发熱。惡寒。熱多寒少。其人不嘔。清便續自可。一日再三发。脉微緩者。为欲愈也。脉微而惡寒者。此为陰陽俱虚。不可復[吐下]发汗也。面反有熱色者。未欲觧也。以其不能得汗出。身必癢。宜桂枝麻黄各半湯。23
	\footnote{「續自可」玉函作「自調」。\\「面反有熱色」聖惠方作「面色赤有熱」。}

太陽病。初服桂枝湯。反煩不觧者。当先刺風池風府。卻与桂枝湯即愈。24

服桂枝湯。大汗出。若脉[但]洪大者。与桂枝湯。若形如瘧。一日再发者。汗出便觧。宜桂枝二麻黄一湯。25

服桂枝湯。大汗出。大煩渴不觧。若脉洪大。与白虎[加人参]湯。26

太陽病。发熱。惡寒。熱多寒少。脉微弱者。此无陽也。不可[復]发汗。[宜桂枝二越婢一湯。]27

服桂枝湯。[或]下之。仍頭項强痛。翕翕发熱。无汗。心下滿。微痛。小便不利。桂枝去桂加茯苓[白]术湯主之。28
	\footnote{「白术」脉經无「白」字。}

傷寒。脉浮。自汗出。小便數。心煩。微惡寒。腳攣急。反与桂枝湯。欲攻其表。得之便厥。咽乾。煩躁。吐逆者。当作甘草乾薑湯。以復其陽。若厥愈。足温者。更作芍藥甘草湯与之。其腳即伸。若胃气不和。譫語者。少与[調胃]承气湯。若重发汗。復加燒針者。四逆湯主之。29

太陽病。項背强几几。无汗。惡風。葛根湯主之。31

太陽与陽明合病。而自利者。葛根湯主之。不下利。但嘔者。葛根加半夏湯主之。32.33
	\footnote{「而自利者」宋本作「者必自下利」,脉經作「而自利不嘔者」。}

太陽病。桂枝證。醫反下之。遂利不止。脉促者。表未觧也。喘而汗出者。宜葛根黄連[黄芩]湯。34

太陽病。頭痛。发熱。身疼。腰痛。骨節疼痛。惡風。无汗而喘。麻黄湯主之。35

太陽与陽明合病。喘而胸滿者。不可下。宜麻黄湯。36

太陽病。十日已去。脉浮細而嗜卧者。外已觧也。設胸滿脇痛者。与小柴胡湯。脉[但]浮者。与麻黄湯。37
	\footnote{「脉但浮」脉經、玉函作「脉浮」。}

太陽中風。脉浮緊。发熱。惡寒。身体疼痛。不汗出而煩躁者。大青龙湯主之。若脉微弱。汗出。惡風者。不可服之。服之則厥。筋愓肉瞤。此为逆也。38
	\footnote{「煩躁者」脉經、玉函作「煩躁頭痛」。}

傷寒。脉浮緩。身不疼。但重。乍有輕时。无少陰證者。大青龙湯发之。39
	\footnote{「乍」聖惠方作「或」。}

*水之为病。其脉沈小。屬少陰。浮者为風。无水。虚胀者为气。水。发其汗即已。脉沈者。宜麻黄附子湯。浮者。宜杏子湯。(金匱)

*飲水流行。歸於四肢。当汗出而不汗出。身体疼重。謂之溢飲。病溢飲者。当发其汗。大青龙湯主之。(金匱)

傷寒。表不觧。心下有水气。乾嘔。发熱而欬。或渴。或利。或噎。或小便不利。少腹滿。或[微]喘。小青龙湯主之。40

傷寒。心下有水气。欬而微喘。发熱。不渴。服湯已而渴者。此寒去。为欲觧。小青龙湯主之。41

太陽病。外證未觧。脉浮弱者。当以汗觧。宜桂枝湯。42

太陽病。下之。微喘者。表未觧故也。桂枝[加厚朴杏仁]湯主之。43
	\footnote{「桂枝加厚朴杏仁湯」一云「麻黄湯」。}

太陽病。外證未觧者。不可下。下之为逆。欲觧外者。宜桂枝湯。44
	\footnote{「欲觧外者宜桂枝湯」脉經无。}

太陽病。先发汗不觧而下之。其脉浮者不愈。浮为在外。而反下之。故令不愈。今脉浮。故在外。当觧其外則愈。宜桂枝湯。45

*太陽病。下之不愈。其脉浮者为在外。汗之則愈。宜桂枝湯。45(聖惠方)

太陽病。脉浮緊。无汗。发熱。身疼痛。八九日不觧。表證續在。此当发其汗。服藥已。微除。其人发煩目暝。劇者必衄。衄乃觧。所以然者。陽气重故也。麻黄湯主之。46

太陽病。脉浮緊。发熱。身无汗。自衄者愈。47

二陽并病。太陽初得病时。发其汗。汗先出[。復]不徹。因轉屬陽明。續自微汗出。不惡寒。若太陽病證不罷者。不可下。下之为逆。如此者可小发汗。設面色緣緣正赤者。陽气怫鬱不得越。当觧之熏之。当汗不汗。其人躁煩。不知痛処。乍在腹中。乍在四肢。按之不可得。其人短气。但坐。以汗出不徹故也。更发汗則愈。何以知汗出不徹。以脉濇故知之。48

脉浮數者。法当汗出而愈。若下之。身体重。心悸者。不可发汗。当自汗出而觧。所以然者。尺中脉微。此裏虚。須表裏実。津液和。即自汗出愈。49

脉浮而緊。法当身体疼痛。当以汗觧之。假令尺中脉遲者。不可发汗。何以知然。以榮气不足。血少故也。50

*凡脉尺中遲。不可发汗。榮衛不足。血少故也。(聖惠方)50

脉浮者。病在表。可发汗。宜麻黄湯。51
	\footnote{「麻黄湯」一云「桂枝湯」。}

[太陽病。]脉浮而數者。可发汗。宜麻黄湯。52
	\footnote{「太陽病」三字宋本、千金翼无。\\「麻黄湯」一云「桂枝湯」。}

病常自汗出者。此为榮气和。衛气不和也。榮行脉中。衛行脉外。復发其汗。衛和則愈。宜桂枝湯。53
	\footnote{「此为榮气和衛气不和也」玉函作「此为營气与衛气不和也」,脉經作「此为榮气和榮气和而外不觧此衛不和也」,千金作「此为榮气和榮气和而外不觧此为衛不和也」,千金翼作「此为榮气和衛气不和故也」,聖惠方作「此为榮气和衛气不和」,宋本作「此为榮气和榮气和者外不谐以衛气不共榮气谐和故尔」。}

病人臓无他病。时发熱。自汗出。而不愈者。此衛气不和也。先其时发汗則愈。宜桂枝湯。54

傷寒。脉浮緊。不发汗。因致衄者。宜麻黄湯。55

[寸口]脉浮而緊。浮則为風。緊則为寒。風則傷衛。寒則傷榮。榮衛俱病。骨節煩疼。当发其汗。宜麻黄湯。0

傷寒。不大便六七日。頭痛。有熱者。与承气湯。其小便清者。此为不在裏。續在表也。当发其汗。頭痛者必衄。宜桂枝湯。56
	\footnote{玉函此條出現兩次,「与承气湯」四字上分别有「未可」、「不可」二字。\\「其小便清者」脉經、千金翼作「其大便反青」,玉函作「其小便反清」,外臺作「其人小便反清者」。}

傷寒。发汗已觧。半日許復煩。脉浮數者。可復发汗。宜桂枝湯。57
	\footnote{「半日許復煩」聖惠方作「半日後復煩躁」。}

凡病。或发汗。或吐。或下。或亡血。[内]无津液而陰陽自和者。必自愈。58

大下後。復发汗。其人小便不利。此亡津液。勿治之。得小便利。必自愈。59

下之後。復发汗。必振寒。脉微細。所以然者。内外俱虚故也。60

下之後。復发汗。晝日煩躁不得眠。夜而安靜。不嘔。不渴。无表證。脉沈微。身无大熱。乾薑附子湯主之。61

发汗後。身体疼痛。其脉沈遲。桂枝加芍藥生薑人参湯主之。62

发汗後。不可更行桂枝湯。汗出而喘。无大熱者。可与麻杏石甘湯。63

*治傷寒。发汗出而喘。无大熱。麻黄杏仁石膏甘草湯方。(千金要方)

发汗過多。其人叉手自冒心。心下悸。欲得按者。桂枝甘草湯主之。64

发汗後。其人脐下悸。欲作奔豚。苓桂甘棗湯主之。65

发汗後。腹胀滿者。厚朴[生薑半夏甘草人参]湯主之。66

傷寒吐下发汗後。心下逆滿。气上衝胸。起則頭眩。其脉沈緊。发汗則動經。身为振搖。苓桂术甘湯主之。67
	\footnote{「吐下发汗後」玉函作「若吐若下若发汗後」,宋本作「若吐若下後」。\\「白术」脉經作「术」。}

发汗不觧。反惡寒者。虚故也。芍藥甘草附子湯主之。不惡寒。但熱者。実也。当和胃气。宜調胃承气湯。68.70
	\footnote{「調胃承气湯」除宋本外其它版本均作「小承气湯」。}

发汗或下之。病仍不觧。煩躁。茯苓四逆湯主之。69

*发汗吐下以後。不觧。煩躁。茯苓四逆湯主之。(脉經。玉函。千金翼)69

太陽病发汗後。大汗出。胃中乾。煩躁不得眠。其人欲飲水。当稍飲之。令胃气和即愈。若脉浮。小便不利。微熱。消渴者。五苓散主之。71
	\footnote{「其人欲飲水当稍飲之」宋本作「欲得飲水者少少与飲之」。}

发汗已。脉浮數。煩渴者。五苓散主之。72

傷寒。汗出而渴者。五苓散主之。不渴者。茯苓甘草湯主之。73

中風。发熱。六七日不觧而煩。有表裏證。渴欲飲水。水入則吐。此为水逆。五苓散主之。74

未持脉时。病人叉手自冒心。師因教试令欬。而不即欬者。此必兩耳聋无聞也。所以然者。以重发汗。虚故也。75

发汗後。飲水多者必喘。以水灌之亦喘。75

发汗後。水藥不得入口。为逆。[若更发汗。必吐下不止。]76

发汗吐下後。虚煩不得眠。若劇者。反覆顛倒。心中懊憹。梔子[豉]湯主之。若少气者。梔子甘草[豉]湯主之。若嘔者。梔子生薑[豉]湯主之。76

发汗或下之。煩熱。胸中窒者。梔子[豉]湯主之。77
	\footnote{「胸中窒者」脉經「窒」作「塞」,千金要方作「胸中窒气逆搶心者」。}

傷寒五六日。大下之後。身熱不去。心中結痛者。未欲觧也。梔子[豉]湯主之。78

傷寒下後。煩而腹滿。卧起不安。梔子厚朴湯主之。79
	\footnote{「煩而」宋本作「心煩」。}

傷寒。醫以丸藥大下之。身熱不去。微煩。梔子乾薑湯主之。80

凡用梔子湯證。其人微溏者。不可与服之。81

*凡用梔子湯。病人舊微溏者。不可与服之。(宋本)81

太陽病。发汗。汗出不觧。其人仍发熱。心下悸。頭眩。身瞤動。振振欲擗地。玄武湯主之。82

咽喉乾燥者。不可发汗。83

淋家不可发汗。发汗必便血。84

瘡家。雖身疼痛。不可发汗。汗出則痙。85

衄家不可发汗。汗出必額上促急[而緊]。直視不能眴。不得眠。86
	\footnote{「額上促急而緊」宋本作「額上陷脉急緊」,脉經、玉函作「額陷脉上促急而緊」,千金翼作「額上促急」。}

亡血家不可发汗。汗出則寒慄而振。87

汗家。重发汗。必恍惚心亂。小便已。陰疼。与禹餘糧丸。88

*凡失血者。不可发汗。发汗必恍惚心亂。(聖惠方)88

病者有寒。復发汗。胃中冷。必吐蛔。89
	\footnote{「吐蛔」一云「吐逆」。}

本发汗而復下之。为逆。若先发汗。治不为逆。本先下之而反汗之。为逆。若先下之。治不为逆。90

傷寒。醫下之。續得下利。清穀不止。身体疼痛。急当救裏。後身体疼痛。清便自調。急当救表。救裏宜四逆湯。救表宜桂枝湯。91

病发熱。頭痛。脉反沈。若不差。身体疼痛。当救其裏。宜四逆湯。92

太陽病。先下而不愈。因復发汗。表裏俱虚。其人因冒。冒家当汗出自愈。所以然者。汗出表和故也。表和。然後下之。93
	\footnote{「表和然後下之」玉函、千金翼作「表和故下之」,宋本、玉函作「裏未和然後復下之」。}

太陽病未觧。脉陰陽俱微。必先振汗出而觧。但陽[脉]微者。先汗出而觧。但陰[脉]微者。先下之而觧。汗之宜桂枝湯。下之宜[調胃]承气湯。94
	\footnote{「調胃承气湯」一云「大柴胡湯」。}

血弱气尽。腠理開。邪气因入。与正气相摶。結於脇下。正邪分爭。往來寒熱。休作有时。默默不欲飲食。臓腑相連。其痛必下。邪高痛下。故使嘔也。小柴胡湯主之。服柴胡湯已而渴者。屬陽明。以法治之。97

得病六七日。脉遲浮弱。惡風寒。手足温。醫再三下之。不能食。其人脇下滿[痛]。面目及身黄。頸項强。小便難。与柴胡湯。後必下重。本渴。飲水而嘔。柴胡[湯]不復中与也。食穀者噦。98
	\footnote{按:史記曰:「寧为雞口,不为牛後」,「後」是指「肛門」。此條中的「後必下重」当为裏急後重之意。}

傷寒五六日。中風。往來寒熱。胸脇苦滿。默默不欲飲食。心煩。喜嘔。或胸中煩而不嘔。或渴。或腹中痛。或脇下痞堅。或心下悸。小便不利。或不渴。外有微熱。或欬。小柴胡湯主之。96
	\footnote{「傷寒五六日中風往來寒熱」脉經、玉函作「中風往來寒熱傷寒五六日以後」,玉函此條重出作「中風五六日傷寒往來寒熱」。\\「外有微熱」宋本作「身有微热」。}

傷寒四五日。身体熱。惡風。頸項强。脇下滿。手足温而渴。小柴胡湯主之。99

傷寒。陽脉濇。陰脉弦。法当腹中急痛。先与小建中湯。不差者。与小柴胡湯。100
	\footnote{「小柴胡湯」聖惠方作「大柴胡湯」。}

傷寒中風。有柴胡證。但見一證便是。不必悉具。101
	\footnote{「有柴胡證」玉函作「有小柴胡證」。}

凡柴胡湯證而下之。柴胡證不罷者。復与柴胡湯。必蒸蒸而振。卻发熱汗出而觧。101

傷寒二三日。心中悸而煩者。小建中湯主之。102

太陽病。過經十餘日。反再三下之。後四五日。柴胡證續在。先与小柴胡湯。嘔不止。心下急。其人鬱鬱微煩者。为未觧。与大柴胡湯下之則愈。103
	\footnote{「嘔不止心下急」除宋本外其它版本均作「嘔止小安」。}

傷寒十三日不觧。胸脇滿而嘔。日晡所发潮熱[。已]而微利。此本当柴胡湯下之。不得利。今反利者。知醫以丸藥下之。非其治也。潮熱者。実也。先宜服小柴胡湯以觧其外。後以柴胡加芒硝湯主之。104
	\footnote{「已而微利」外臺祕要作「熱畢而微利」。}

柴胡加大黄芒硝桑螵蛸湯。
	\footnote{此方宋本无,千金翼、玉函有方无證。}

傷寒十三日。過經。譫語者。内有熱也。当以湯下之。若小便利者。大便当堅。而反[下]利。脉調和者。知醫以丸藥下之。非其治也。若自利者。脉当微厥。今反和者。此为内実也。[調胃]承气湯主之。105
	\footnote{「内有熱也」宋本作「以有熱也」。}

太陽病不觧。熱結膀胱。其人如狂。血自下。下之即愈。其外不觧者。尚未可攻。当先觧其外。[宜桂枝湯。]外觧已。[但]少腹急結者。乃可攻之。宜桃仁承气湯。106
	\footnote{「下之即愈」除脉經外其它本皆作「下者即愈」或「下者愈」。}

傷寒八九日。下之。胸滿。煩。驚。小便不利。譫語。一身尽重。不可轉側。柴胡加龙骨牡蛎湯主之。107

傷寒。腹滿。譫語。寸口脉浮而緊。此为肝乘脾。名曰縱。当刺期門。108

傷寒。发熱。嗇嗇惡寒。其人大渴。欲飲水者。其腹必滿。自汗出。小便利。其病欲觧。此为肝乘肺。名曰横。当刺期門。109

傷寒。脉浮。醫以火迫劫之。亡陽。[必]驚狂。卧起不安。桂枝去芍藥加蜀漆牡蛎龙骨救逆湯主之。112

*傷寒。脉浮。而以火逼劫。汗即亡陽。必驚狂。卧起不安。(聖惠方)112

傷寒。其脉不弦緊而弱[。弱]者必渴。被火必譫語。[弱者。发熱。脉浮。觧之当汗出愈。]113

太陽病。以火熏之。不得汗。其人必躁。到經不觧。必清血。114

*太陽病。以火蒸之。不得汗者。其人必燥結。若不結。必下清血。其脉躁者。必发黄也。(聖惠方)114

脉浮。熱甚。而反灸之。此为実。実以虚治。因火而動。咽燥。必吐血。115 

微數之脉。慎不可灸。因火为邪。則为煩逆。追虚逐実。血散脉中。火气雖微。内攻有力。焦骨傷筋。血難復也。116

*凡微數之脉。不可灸。因熱为邪。必致煩逆。内有損骨傷筋血枯之患。(聖惠方)116

脉浮。当以汗觧。而反灸之。邪无從出。因火而盛。病從腰以下必重而痹。此为火逆。若欲自觧。当先煩。煩乃有汗。隨汗而觧。何以知之。脉浮。故知汗出当觧。116

*脉当以汗觧。反以灸之。邪无所去。因火而盛。病当必重。此为逆治。若欲觧者。当发其汗而觧也。(聖惠方)116

燒針令其汗。針処被寒。核起而赤者。必发奔豚。气從少腹上衝心者。灸其核上各一壯。与桂枝加桂湯。117

火逆。下之。因燒針。煩躁者。桂枝甘草龙骨牡蛎湯主之。118

傷寒。加温針必驚。119

太陽病。当惡寒。发熱。今自汗出。反不惡寒。发熱。関上脉細數。此醫吐之過也。一二日吐之者。腹中飢。口不能食。三四日吐之者。不喜糜粥。欲食冷食。朝食暮吐。此醫吐之所致也。此为小逆。120

病人脉數。數为熱。当消穀引食。而反吐者。以醫发其汗。令陽气微。膈气虚。脉則为數。數为客熱。不能消穀。胃中虚冷。故吐也。122

太陽病。過經十餘日。心下嗢嗢欲吐。而胸中痛。大便反溏。腹微滿。鬱鬱微煩。先[此]时自極吐下者。与[調胃]承气湯。若不尔者。不可与。但欲嘔。胸中痛。微溏者。此非柴胡湯證。以嘔。故知極吐下也。123

太陽病六七日。表證續在。脉微而沈。反不結胸。其人发狂。此熱在下焦。少腹当堅滿。小便自利者。下血乃愈。所以然者。以太陽隨經。瘀熱在裏故也。抵当湯主之。124
	\footnote{「瘀熱在裏」千金翼保元堂本、千金翼世補齋本作「瘀血在裏」。\\「抵当湯主之」千金翼作「宜下之以抵当湯」。}

太陽病。身黄。脉沈結。少腹堅。小便不利者。为无血也。小便自利。其人如狂者。血證諦也。抵当湯主之。125

傷寒。有熱。少腹滿。應小便不利。今反利者。为有血也。当下之。宜抵当丸。126
	\footnote{「当下之」三字下宋本、玉函、千金翼有「不可餘藥」四字。}

問曰。病有結胸。有臓結。其狀何如。\\
答曰。按之痛。寸口脉浮。関上自沈。为結胸。\\
問曰。何谓臓結。\\
答曰。如結胸狀。飲食如故。时时下利。寸口脉浮。関上細沈而緊。为臓結。舌上白胎滑者。難治。128.129

臓結无陽證。不往來寒熱。其人反靜。舌上胎滑者。不可攻也。130

病发於陽而反下之。熱入因作結胸。病发於陰而反下之。因作痞。所以成結胸者。以下之太早故也。131

結胸者。項亦强。如柔痙狀。下之則和。宜大陷胸丸。131

結胸證。脉浮大者。不可下。下之則死。132

結胸證悉具。而煩躁者死。133

太陽病。脉浮而動數。浮則为風。數則为熱。動則为痛。數則为虚。頭痛。发熱。微盜汗出。而反惡寒者。表未觧也。醫反下之。動數變遲。膈内拒痛。胃中空虚。客气動膈。短气。躁煩。心中懊憹。陽气内陷。心下因堅。則为結胸。大陷胸湯主之。若不結胸。但頭汗出。餘処无汗。齐頸而還。小便不利。身必发黄。134
	\footnote{「膈内拒痛」脉經、千金翼作「頭痛即眩」,玉函作「頭痛則眩」。}

傷寒六七日。結胸熱実。其脉沈緊。心下痛。按之如石堅。大陷胸湯主之。135

傷寒十餘日。熱結在裏。復往來寒熱者。与大柴胡湯。但結胸。无大熱者。此为水結在胸脇。[但]頭微汗出。大陷胸湯主之。136
	\footnote{「熱結在裏」千金翼作「邪气結在裏」。\\宋本、外臺「但頭微汗出」五字下有「者」字。}

太陽病。重发汗而復下之。不大便五六日。舌上燥而渴。日晡所小有潮熱。從心下至少腹堅滿而痛不可近。大陷胸湯主之。137

小結胸者。正在心下。按之則痛。其脉浮滑。小陷胸湯主之。138

太陽病二三日。[終]不能卧。但欲起者。心下必結。脉微弱者。此本寒也。而反下之。利止者。必結胸。未止者。四五日復下之。此[作]挾熱利也。139
	\footnote{「此本寒也」宋本作「此本有寒分也」,外臺作「本有久寒也」。}

太陽病。下之。其脉促。不結胸者。此为欲觧。脉浮者。必結胸。脉緊者。必咽痛。脉弦者。必兩脇拘急。脉細數者。頭痛未止。脉沈緊者。必欲嘔。脉沈滑者。協熱利。脉浮滑者。必下血。140

病在陽。当以汗觧。反以水潠之或灌之。其熱被劫不得去。益煩。皮上粟起。意欲飲水。反不渴。宜服文蛤散。若不差。与五苓散。若寒実結胸。无熱證者。与三物白散。141
	\footnote{「其熱被劫不得去」脉經、千金翼、玉函、外臺作「其熱卻不得去」。\\「益煩」二字上,宋本、外臺有「弥更」二字。}

太陽与少陽并病。頭項强痛。或眩冒。时如結胸。心下痞堅者。当刺大椎第一間。肺腧。肝腧。慎不可发汗。发汗則譫語。譫語則脉弦。譫語五日不止者。当刺期門。142

婦人中風。发熱。惡寒。經水適來。得之七八日。熱除。脉遲。身涼。胸脇下滿。如結胸狀。譫語。此为熱入血室。当刺期門。隨其[虚]実而取之。143
	\footnote{「隨其虚実而取之」脉經、玉函、千金翼同,宋本、玉函作「隨其実而取之」,成本作「隨其実而瀉之」。}

婦人中風七八日。續得寒熱。发作有时。經水適斷。此为熱入血室。其血必結。故使如瘧狀。发作有时。小柴胡湯主之。144

婦人傷寒。发熱。經水適來。晝日明了。暮則譫語。如見鬼狀。此为熱入血室。无犯胃气及上二焦。必自愈。145
	\footnote{「明了」宋本、玉函作「明瞭」。\\「二焦」脉經作「三焦」。}

傷寒六七日。发熱。微惡寒。肢節煩疼。微嘔。心下支結。外證未去者。柴胡桂枝湯主之。146

发汗多。亡陽。狂語者。不可下。[可]与柴胡桂枝湯。和其榮衛。以通津液。後自愈。

傷寒五六日。已发汗而復下之。胸脇滿。微結。小便不利。渴而不嘔。但頭汗出。往來寒熱。心煩。此为未觧。柴胡桂枝乾薑湯主之。147

*傷寒六日。已发汗及下之。其人胸脇滿。大腸微結。小腸不利而不嘔。但頭汗出。往來寒熱而煩。此为未觧。宜小柴胡桂枝湯。(聖惠方)147

傷寒五六日。頭汗出。微惡寒。手足冷。心下滿。口不欲食。大便堅。其脉細。此为陽微結。必有表。復有裏。沈亦为病在裏。汗出为陽微。假令純陰結。不得有外證。悉入在裏。此为半在外半在裏。脉雖沈緊。不得为少陰病。所以然者。陰不得有汗。今頭汗出。故知非少陰也。可与[小]柴胡湯。設不了了者。得屎而觧。148

傷寒五六日。嘔而发熱。柴胡湯證具。而以他藥下之。柴胡證仍在者。復与柴胡湯。此雖已下之。不为逆。必蒸蒸而振。卻发熱汗出而觧。若心下滿而堅痛者。此为結胸。宜大陷胸湯。若但滿而不痛者。此为痞。柴胡[湯]不復中与也。宜半夏瀉心湯。149

太陽与少陽并病。而反下之。[成]結胸。心下堅。下利不[復]止。水漿不[肯]下。其人[必]心煩。150

脉浮緊而反下之。緊反入裏。則作痞。按之自濡。但气痞耳。151

太陽中風。下利。嘔逆。表觧乃可攻之。其人漐漐汗出。发作有时。頭痛。心下痞堅滿。引脇下痛。乾嘔。短气。汗出。不惡寒。此为表觧。裏未和。十棗湯主之。152

太陽病。醫发其汗。遂发熱。惡寒。復下之。則心下痞。此表裏俱虚。陰陽气并竭。无陽則陰獨。復加燒針。因胸煩。面色青黄。膚瞤者。難治。今色微黄。手足温者。易愈。153

心下痞。按之濡。其脉関上浮者。大黄[黄連]瀉心湯主之。154

心下痞。而復惡寒。汗出者。附子瀉心湯主之。155

本以下之。故心下痞。与瀉心湯。痞不觧。其人渴而口燥[煩]。小便不利者。五苓散主之。156

傷寒。汗出。觧之後。胃中不和。心下痞堅。乾噫食臭。脇下有水气。腹中雷鳴。下利者。生薑瀉心湯主之。157
	\footnote{「傷寒汗出觧之後」千金方作「傷寒发汗後」,聖惠方作「太陽病汗出後」。}

傷寒中風。醫反下之。其人下利日數十行。穀不化。腹中雷鳴。心下痞堅而滿。乾嘔。心煩。不能得安。醫見心下痞。谓病不尽。復下之。其痞益甚。此非結熱。但以胃中虚。客气上逆。故使之堅。甘草瀉心湯主之。158
	\footnote{「故使之堅」千金要方作「使之然也」。}

傷寒。服湯藥。下利不止。心下痞堅。服瀉心湯已。復以他藥下之。利不止。醫以理中与之。利益甚。理中者。理中焦。此利在下焦。赤石脂禹餘糧湯主之。復不止者。当利小便。159

傷寒。吐下[後]发汗。虚煩。脉甚微。八九日。心下痞堅。脇下痛。气上衝咽喉。眩冒。經脉動愓者。久而成痿。160

傷寒。发汗[或]吐[或]下。觧後。心下痞堅。噫气不除者。旋復代赭湯主之。161

下後。不可更行桂枝湯。汗出而喘。无大熱者。可与麻杏石甘湯。162

太陽病。外證未除。而數下之。遂挾熱而利。利下不止。心下痞堅。表裏不觧。桂枝人参湯主之。163

傷寒。大下後。復发汗。心下痞。惡寒者。表未觧也。不可攻痞。当先觧表。表觧乃可攻痞。觧表宜桂枝湯。攻痞宜大黄黄連瀉心湯。164

傷寒。发熱。汗出不觧。心中痞堅。嘔吐。下利。大柴胡湯主之。165

病如桂枝證。頭不痛。項不强。寸[口]脉微浮。胸中痞堅。气上衝咽喉。不得息。此为胸有寒。当吐之。宜瓜蒂散。166
	\footnote{「脉微浮」脉經作「脉微細」。\\「气上衝」脉經、玉函作「气上撞」。}

諸亡血。虚家。不可与瓜蒂散。
	\footnote{按。此條諸本均在瓜蒂散煎服法中,未入正文。根據宋本獨有的子目,第166條瓜蒂散證後註云:「下有不可与瓜蒂散證」,也就是説第166條後当有一條不可与瓜蒂散的條文,故此條当为正文。}

病者脇下素有痞。連在脐傍。痛引少腹。入陰筋者。此为臓結。死。167
	\footnote{「入陰筋者」玉函作「入陰俠陰筋者」。}

傷寒或吐或下後。七八日不觧。熱結在裏。表裏俱熱。时时惡風。大渴。舌上乾燥而煩。欲飲水數升。白虎[加人参]湯主之。168

*傷寒六日不觧。熱結在裏。但熱。时时惡風。大渴。舌乾。煩躁。宜白虎湯。(聖惠方)168

凡用白虎湯。立夏後至立秋前得用之。立秋後不可服。
	\footnote{「凡用白虎湯」五字千金翼无,宋本作「此方」,千金作「若」。}

春三月。病常苦裏冷。白虎湯亦不可与。与之則嘔利而腹痛。
	\footnote{「春三月」宋本作「正月二月三月」。\\「病常苦裏冷」宋本、千金作「尚凜冷」。\\「白虎湯亦不可与」宋本作「亦不可与服之」,千金作「亦不可与」。}

諸亡血。虚家。亦不可与白虎湯。得之則腹痛而利。但当温之。
	\footnote{此條千金无。\\「亦不可与白虎湯」宋本作「亦不可与之」。\\「得之則腹痛而利」玉函作「得之腹痛而利者」,宋本作「得之則腹痛利者」\\「但当温之」玉函作「急当温之」,宋本作「但可温之可愈」。\\按。以上三條,宋本在168條後的白虎加人参湯的煎服法中,未入正文,千金翼在176條後的白虎湯煎服法中,未入正文,玉函在170條後,为正文,而千金方我手邊沒有資料。根據宋本獨有的子目,168條後註云:「下有不可与白虎湯證」,也就是説第168條後当有一條不可与白虎湯的條文,故此三條当为正文。}

傷寒。无大熱。口燥渴。心煩。背微惡寒。白虎[加人参]湯主之。169

傷寒。脉浮。发熱。无汗。其表不觧。不可与白虎湯。渴欲飲水。无表證者。白虎[加人参]湯主之。170

太陽与少陽合病。自下利者。与黄芩湯。若嘔者。与黄芩加半夏生薑湯。172

傷寒。胸中有熱。胃中有邪气。腹中痛。欲嘔吐。黄連湯主之。173

傷寒八九日。風濕相摶。身体疼煩。不能自轉側。不嘔。不渴。脉浮虚而濇者。桂枝附子湯主之。若其人大便堅。小便自利者。术附子湯主之。174

風濕相摶。骨節疼煩。掣痛不得屈伸。近之則痛劇。汗出。短气。小便不利。惡風。不欲去衣。或身微腫。甘草附子湯主之。175

傷寒。脉浮滑。此以表有熱。裏有寒。白虎湯主之。176

*傷寒。脉浮滑。而表熱裏寒者。白通湯主之。(玉函)176

傷寒。脉結代。心動悸。炙甘草湯主之。177

*傷寒。脉結代。心中驚悸。炙甘草湯主之。(玉函)177

\chapter{辨陽明病}

陽明之为病。胃家実是也。180

*傷寒二日。陽明受病。陽明者。胃中寒是也。宜桂枝湯。(聖惠方)180

問曰。病有太陽陽明。有正陽陽明。有微陽陽明。何谓也。\\
答曰。太陽陽明者。脾約是也。正陽陽明者。胃家実是也。微陽陽明者。发汗或利小便。胃中燥。大便難是也。179
	\footnote{「微陽」宋本作「少陽」。\\宋本「胃中燥」下有「煩実」二字。}

問曰。何緣得陽明病。\\
答曰。太陽病。发汗或下之。亡其津液。胃中乾燥。因轉屬陽明。不更衣。内実。大便難者。为陽明病也。181

問曰。陽明病外證云何。\\
答曰。身熱。汗出。而不惡寒。[但]反惡熱。182

*陽明病外證。身熱。汗出。而不惡寒。但惡熱。宜柴胡湯。(聖惠方)182

問曰。病有得之一日。不发熱而惡寒者。何。\\
答曰。然雖一日。惡寒自罷。即汗出。惡熱也。183\\
問曰。惡寒何故自罷。\\
答曰。陽明居中。主土。万物所歸。无所復傳。始雖惡寒。二日自止。此为陽明病也。184

本太陽。初得病时。发其汗。汗先出[。復]不徹。因轉屬陽明。185

*太陽病而发汗。汗雖出。復不觧。不觧者。轉屬陽明也。宜麻黄湯。(聖惠方)185

傷寒。发熱。无汗。嘔不能食。而反汗出濈濈然。是为轉屬陽明。185

傷寒。脉浮而緩。手足自温。是为系在太陰。太陰[身]当发黄。若小便自利者。不能发黄。至七八日。大便堅者。为屬陽明。187

傷寒轉系陽明者。其人濈濈然微汗出也。188

陽明中風。口苦。咽乾。腹滿。微喘。发熱。惡寒。脉浮緊。若下之。則腹滿。小便難也。189

陽明病。能食为中風。不能食为中寒。190

陽明病。中寒。不能食。小便不利。手足濈然汗出。此欲作堅瘕。必大便頭堅後溏。所以然者。胃中冷。水穀不别故也。191

*陽明中寒。不能食。小便不利。手足濈然汗出。欲作堅症也。所以然者。胃中水穀不化故也。宜桃仁承气湯。(聖惠方)191

陽明病。初欲食。小便反不利。大便自調。其人骨節疼。翕翕如有熱狀。奄然发狂。濈然汗出而觧。此为水不勝穀气。与汗共并。脉緊則愈。192

陽明病欲觧时。從申尽戍。193

陽明病。不能食。下之不觧。攻其熱必噦。所以然者。胃中虚冷故也。[以其人本虚。故攻其熱必噦。]194

*陽明病。能食。下之不觧。其人不能食。攻其熱必噦者。胃中虚冷也。宜半夏湯。(聖惠方)194

陽明病。脉遲。食難用飽。飽則发煩。頭眩。必小便難。此欲作穀疸。雖下之。腹滿如故。所以然者。脉遲故也。195

*陽明病。脉遲。发熱。頭眩。小便難。此欲作榖疸。下之必腹滿。宜柴胡湯。(聖惠方)195

陽明病当多汗。而反无汗。其身如虫行皮中狀。此以久虚故也。196

陽明病。反无汗。但小便利。二三日。嘔而欬。手足厥者。其人頭必痛。若不嘔。不欬。手足不厥者。頭不痛。197

陽明病。但頭眩。不惡寒。故能食而欬。其人咽必痛。若不欬者。咽不痛。198

陽明病。脉浮緊者。必潮熱。发作有时。[脉]但浮者。必盜汗出。201

陽明病。无汗。小便不利。心中懊憹者。必发黄。199

*陽明病。无汗。小便不利。心中熱壅。必发黄也。宜茵陳湯。(聖惠方)199

陽明病。被火。額上微汗出。小便不利者。必发黄。200

*陽明病。被火灸。其額上微有汗出。小便不利者。必发黄也。宜茵陳湯。(聖惠方)200

陽明病。口燥。但欲漱水。不欲咽者。必衄。202

陽明病。本自汗出。醫復重发汗。病已差。其人微煩不了了者。此必大便堅故也。以亡津液。胃中乾燥。故令其堅。当問其小便日幾行。若本日三四行。今日再行者。知必大便不久出。今为小便數少。津液当還入胃中。故知不久必大便也。203

夫病陽多者熱。下之則堅。汗出多極。发其汗亦堅。

*夫病陽多熱。下之則堅。汗出多極。发其汗亦堅。(玉函)

*欬而小便利。若失小便。不可攻其表。汗出則厥逆冷。汗出多極。发其汗亦堅。(脉經)

傷寒。嘔多。雖有陽明證。不可攻之。204

陽明病。心下堅滿者。不可攻之。攻之遂利不止者死。利止者愈。205

陽明病。面合赤色者。不可攻之。[攻之]必发熱。色黄。小便不利也。206

陽明病。不吐下而心煩者。可与[調胃]承气湯。207

陽明病。脉遲。雖汗出。不惡寒。其身必重。短气。腹滿而喘。有潮熱。如此者。其外为觧。可攻其裏。若手足濈然汗出者。此大便已堅。[大]承气湯主之。若汗多。微发熱。惡寒者。为外未觧。[桂枝湯主之。]其熱不潮。未可与承气湯。若腹大滿而不大便者。可与小承气湯。微和其胃气。勿令至大下。208
	\footnote{「如此者其外为觧可攻其裏」宋本、千金要方、外臺作「者此外欲觧可攻裏也」。\\千金要方「外未觧也」四字下有「桂枝湯主之」五字。}

陽明病。潮熱。大便微堅者。可与大承气湯。不堅者。不可与之。若不大便六七日。恐有燥屎。欲知之法。可少与小承气湯。若腹中轉失气者。此有燥屎。乃可攻之。若不轉失气者。此但頭堅後溏。不可攻之。攻之必腹滿。不能食也。欲飲水者。与水即噦。其後发熱者。必大便復堅而少也。以小承气湯和之。若不轉失气者。慎不可攻之。209

*陽明病。有潮熱。大便堅。可与承气湯。若有結燥。乃可徐徐攻之。若无壅滯。不可攻之。攻之者。必腹滿。不能食。欲飲水者即噦。其候发熱。必腹堅胀。宜与小承气湯。(聖惠方)209

夫実則譫語。虚則鄭聲。鄭聲者。重語也。直視。譫語。喘滿者死。下利者亦死。210
	\footnote{「鄭聲者重語也」外臺作「鄭聲重語也」五字寫作雙行小字注釋。}

发汗多。重发汗。[此为]亡陽。[若]譫語。脉短者死。脉自和者不死。211

傷寒。吐下後未觧。不大便五六日。至十餘日。其人日晡所发潮熱。不惡寒。獨語。如見鬼神之狀。若劇者。发則不識人。順衣妄撮。怵惕不安。微喘。直視。脉弦者生。濇者死。[若]微者。但发熱。譫語。[大]承气湯主之。若一服利。止後服。212
	\footnote{「若微者」各本均无「若」字,为編者所加。}

陽明病。其人多汗。津液外出。胃中燥。大便必堅。堅則譫語。[小]承气湯主之。[若一服譫語止。莫復服。]213

陽明病。譫語。发潮熱。脉滑疾者。[小]承气湯主之。因与承气湯一升。腹中轉失气者。復与一升。若不轉失气者。勿更与之。明日又不大便。脉反微濇者。此为裏虚。为難治。不可復与承气湯。214
	\footnote{「譫語」二字下玉函、千金翼、聖惠方有「妄言」二字。}

陽明病。譫語。有潮熱。反不能食者。[胃中]必有燥屎五六枚。若能食者。但堅耳。[大]承气湯主之。215
	\footnote{「胃中」二字除宋本外其它版本均无。\\「大承气湯主之」宋本作「宜大承气湯下之」。}

陽明病。下血。譫語者。此为熱入血室。但頭汗出者。当刺期門。隨其実而瀉之。濈然汗出則愈。216

汗出。譫語者。以有燥屎在胃中。此風也。[須下者。]過經乃可下之。下之若早。語言必亂。以表虚裏実故也。下之則愈。宜[大]承气湯。217
	\footnote{「大承气湯」一云「大柴胡湯」。}

傷寒四五日。脉沈而喘滿。沈为在裏。反发其汗。津液越出。大便为難。表虚裏実。久則譫語。218

三陽合病。腹滿。身重。難以轉側。口不仁。面垢。譫語。遺尿。发汗則譫語[甚]。下之則額上生汗。手足厥冷。自汗。白虎湯主之。219
	\footnote{「发汗則譫語甚」除脉經外均无「甚」字\\「自汗」宋本作「若自汗出者」。}

二陽并病。太陽證罷。但发潮熱。手足漐漐汗出。大便難而譫語者。下之則愈。宜[大]承气湯。220

陽明病。脉浮而緊。咽乾。口苦。腹滿而喘。发熱。汗出。不惡寒。反惡熱。身重。若发汗則躁。心憒憒。反譫語。若加温針。必怵惕。煩躁。不得眠。若下之。則胃中空虚。客气動膈。心中懊憹。舌上胎者。梔子[豉]湯主之。若渴欲飲水。口乾舌燥者。白虎[加人参]湯主之。若脉浮。发熱。渴欲飲水。小便不利者。豬苓湯主之。221.222.223
	\footnote{據孫世揚考證,「舌上胎者」之「胎」当作「菭」,説文觧字:「菭,水衣也。」}

*陽明病。脉浮。咽乾。口苦。腹滿。汗出而喘。不惡寒。反惡熱。心躁。譫語。不得眠。胃虚。客熱。舌燥。宜梔子湯。(聖惠方)221

陽明病。汗出多而渴者。不可与豬苓湯。以汗多。胃中燥。豬苓湯復利其小便故也。224

*陽明病。汗出多而渴者。不可与豬苓湯。汗多者。胃中燥也。汗少者。宜与豬苓湯利其小便。(聖惠方)224

脉浮而遲。表熱裏寒。下利清穀者。四逆湯主之。225

[陽明病。]若胃中虚冷。不能食者。飲水即噦。226

脉浮。发熱。口乾。鼻燥。能食者。則衄。227

*脉浮。发熱。口鼻中燥。能食者。必衄。宜黄芩湯。(聖惠方)227

陽明病。下之。其外有熱。手足温。不結胸。心中懊憹。飢不能食。但頭汗出。梔子[豉]湯主之。228

陽明病。发潮熱。大便溏。小便自可。胸脇滿不去。小柴胡湯主之。229

*陽明病。发潮熱。大便溏。小便自利。胸脇煩滿不止。宜小柴胡湯。(聖惠方)229

陽明病。脇下堅滿。不大便而嘔。舌上白胎者。可与小柴胡湯。上焦得通。津液得下。胃气因和。身濈然汗出而觧。230

*陽明病。脇下堅滿。大便祕而嘔。口燥。宜柴胡湯。(聖惠方)230

陽明中風。脉弦浮大而短气。腹都滿。脇下及心痛。久按之。气不通。鼻乾。不得汗。嗜卧。一身及目悉黄。小便難。有潮熱。时时噦。耳前後腫。刺之小差。外不觧。病過十日。脉續浮者。与[小]柴胡湯。脉但浮。无餘證者。与麻黄湯。若不尿。腹滿加噦者。不治。231.232

*陽明病。中風。其脉浮大。短气心痛。鼻乾嗜卧。不得汗。一身悉黄。小便難。有潮熱而噦。耳前後腫。刺之雖小差。外若不觧。宜柴胡湯。(聖惠方)231.232

陽明病。自汗出。若发汗。小便自利者。此为[津液]内竭。雖堅。不可攻之。当須自欲大便。宜蜜煎。導而通之。若土瓜根及豬膽汁。皆可以導。233

陽明病。脉遲。汗出多。微惡寒者。表未觧也。可发汗。宜桂枝湯。234

陽明病。脉浮。无汗而喘者。发汗則愈。宜麻黄湯。235
	\footnote{「而喘者」除宋本外其它版本均作「其人必喘」。}

陽明病。发熱。汗出者。此为熱越。不能发黄也。但頭汗出。身无汗。齐頸而還。小便不利。渴引水漿者。此为瘀熱在裏。身必发黄。茵陳[蒿]湯主之。236
	\footnote{「熱越」聖惠方作「熱退」。}

陽明證。其人喜忘者。必有畜血。所以然者。本有久瘀血。故令喜忘。屎雖堅。大便反易。其色必黑。抵当湯主之。237

陽明病。下之。心中懊憹而煩。胃中有燥屎者。可攻。其人腹微滿。頭堅後溏者。不可攻之。若有燥屎者。宜[大]承气湯。238

病者五六日不大便。繞脐痛。躁煩。发作有时。此有燥屎。故使不大便也。239
	\footnote{「躁煩」除宋本作「煩躁」外,其它版本均同。}

病者煩熱。汗出則觧。復如瘧狀。日晡所发者。屬陽明。脉実者。当下之。脉浮虚者。当发其汗。下之宜[大]承气湯。发汗宜桂枝湯。240
	\footnote{「日晡所发」脉經卷七第七、千金翼同,脉經卷七第二、玉函、宋本作「日晡所发熱」。\\「大承气湯」一云「大柴胡湯」。}

*汗出後則暫觧。日晡則復发。脉実者。当宜下之。(聖惠方)240

大下後。六七日不大便。煩不觧。腹滿痛者。此有燥屎。所以然者。本有宿食故也。宜[大]承气湯。241
	\footnote{「此有燥屎所以然者」玉函作「有燥屎者」四字。}

病者小便不利。大便乍難乍易。时有微熱。喘冒。不能卧者。有燥屎故也。宜[大]承气湯。242
	\footnote{「喘冒」玉函、千金翼作「怫鬱」。}

食穀欲嘔者。屬陽明。[吳]茱萸湯主之。得湯反劇者。屬上焦。243
	\footnote{「得湯反劇者屬上焦」八字千金翼在煎服法中。}

太陽病。寸[口]緩。関[上小]浮。尺[中]弱。其人发熱。汗出。復惡寒。不嘔。但心下痞者。此为醫下之故也。若不下。其人不惡寒而渴者。此轉屬陽明。小便數者。大便必堅。不更衣十日。无所苦也。[渴]欲飲水者。少少与之。但以法救之。渴者。宜五苓散。244
	\footnote{「太陽病」千金翼作「陽明病」。}

脉陽微而汗出少者。为自和。汗出多者。为太過。陽脉実。因发其汗。出多者。亦为太過。太過者。陽絶於内。亡津液。大便因堅。245

脉浮而芤。浮則为陽。芤則为陰。浮芤相摶。胃气生熱。其陽則絶。246

跌陽脉浮而濇。浮則胃气强。濇則小便數。浮濇相摶。大便則堅。其脾为約。麻子仁丸主之。247

太陽病三日。发汗不觧。蒸蒸发熱者。屬胃也。[調胃]承气湯主之。248

傷寒吐後。腹胀滿者。与[調胃]承气湯。249

太陽病吐下发汗後。微煩。小便數。大便因堅。可与小承气湯。和之則愈。250

得病二三日。脉弱。无太陽柴胡證。煩躁。心下堅。至四日。雖能食。以[小]承气湯少与。微和之。令小安。至六日。与承气湯一升。若不大便六七日。小便少者。雖不大便。但頭堅後溏。未定成堅。攻之必溏。当須小便利。屎定堅。乃可攻之。宜[大]承气湯。251
	\footnote{「雖不大便」宋本作「雖不受食」,玉函作「雖不能食」。}

傷寒六七日。目中不了了。睛不和。无表裏證。大便難。身微熱者。此为実也。急下之。宜[大]承气湯。252
	\footnote{「大承气湯」一云「大柴胡湯」。}

*傷寒六七日。目中瞳子不明。无外證。大便難。微熱者。此为実。宜急下之。(聖惠方)252

陽明病。发熱。汗多者。急下之。宜[大]承气湯。253
	\footnote{「大承气湯」一云「大柴胡湯」。}

发汗不觧。腹滿痛者。急下之。宜[大]承气湯。254
	\footnote{「大承气湯」一云「大柴胡湯」。}

*傷寒病。腹中滿痛者为実。当宜下之。(聖惠方)254

腹滿不減。減不足言。当下之。宜[大]承气湯。255
	\footnote{「大承气湯」一云「大柴胡湯」。}

陽明与少陽合病而利。脉不負者为順。負者。失也。互相克賊。名为負。256

脉滑而數者。有宿食也。当下之。宜[大]承气湯。256
	\footnote{「大承气湯」一云「大柴胡湯」。}

*傷寒。脉數而滑者。有宿食。当下之則愈。(聖惠方)256

*脉數而滑者。実也。此有宿食也。当下之。宜大承气湯。(金匱要略)256

病人无表裏證。发熱七八日。雖脉浮數。可下之。宜大柴胡湯。假令下已。脉數不觧。合熱則消穀善飢。至六七日。不大便者。有瘀血。宜抵当湯。若脉數不觧而下不止。必挾熱。便膿血。257.258
	\footnote{「宜大柴胡湯」宋本、千金翼无。}

傷寒发汗已。身目为黄。所以然者。寒濕相摶。在裏不觧故也。[以为非瘀熱而不可下。当於寒濕中求之。]259

傷寒七八日。身黄如橘。小便不利。腹微滿者。茵陳[蒿]湯主之。260
	\footnote{「腹微滿者」脉經、玉函作「少腹微滿」。}

傷寒。身黄。发熱。梔子蘗皮湯主之。261
	\footnote{「身黄发熱」千金翼作「其人发黄」。}

傷寒。瘀熱在裏。身必发黄。麻黄連軺赤小豆湯主之。262
	\footnote{「連軺」千金翼作「連翹」。}

\chapter{辨少陽病}

少陽之为病。口苦。咽乾。目眩也。263

*傷寒三日。少陽受病。口苦乾燥。目眩。宜柴胡湯。(聖惠方)263

少陽中風。兩耳无所聞。目赤。胸中滿而煩。不可吐下。吐下則悸而驚。264

傷寒。脉弦細。頭痛。发熱者。屬少陽。少陽不可发汗。发汗則譫語。此屬胃。胃和即愈。胃不和。煩而悸。265

太陽病不觧。轉入少陽。脇下堅滿。乾嘔。不能食。往來寒熱。尚未吐下。脉沈緊者。可与小柴胡湯。若已吐下发汗温針。譫語。柴胡湯證罷。此为壞病。知犯何逆。以法治之。266.267

三陽合病。脉浮大。上関上。但欲寐。目合則汗。268

傷寒六七日。无大熱。其人躁煩。此为陽去入陰故也。269

傷寒三日。三陽为尽。三陰当受邪。其人反能食。而不嘔。此为三陰不受邪也。270

傷寒三日。少陽脉小者。[为]欲已。271

少陽病欲觧时。從寅尽辰。272

\chapter{辨太陰病}

太陰之为病。腹滿而吐。食不下。下之益甚。时腹自痛。胸下堅結。273

*太陰之为病。腹滿而吐。食不下。自利益甚。时腹自痛。若下之。必胸下結硬。(宋本)273

*傷寒四日。太陰受病。腹滿。吐食。下之益甚。时时腹痛。心胸堅滿。若脉浮者。可发其汗。沈者宜攻其裏也。发汗者宜桂枝湯。攻裏者宜承气湯。(聖惠方)273.276

太陰病。脉浮者。可发汗。宜桂枝湯。276

太陰中風。四肢煩疼。[脉]陽微陰濇而长者。为欲愈。274

太陰病欲觧时。從亥尽丑。275

自利。不渴者。屬太陰。以其臓有寒故也。当温之。宜四逆輩。277

*太陰病。利而不渴者。其臟有寒。当温之以四逆湯。(聖惠方)277

傷寒。脉浮而緩。手足自温者。系在太陰。太陰[身]当发黄。若小便自利者。不能发黄。至七八日。雖暴煩。下利日十餘行。必自止。所以自止者。脾家実。腐穢已去故也。278

*傷寒。脉浮而緩。手足自温。是为系在太陰。小便不利。其人当发黄。宜茵陳湯。太陰病不觧。雖暴煩。下利十餘行而自止。所以自止者。脾家実。腐穢已去故也。宜橘皮湯。278(聖惠方)

[本]太陽病。醫反下之。因尔腹滿时痛者。屬太陰。桂枝加芍藥湯主之。大実痛者。桂枝加大黄湯主之。279

*太陰病。下之後。腹滿时痛。宜桂心芍藥湯。若太実腹痛者。宜承气湯下之。(聖惠方)279

太陰为病。脉弱。其人續自便利。設当行大黄芍藥者。宜減之。以其人胃气弱。易動故也。280

\chapter{辨少陰病}

少陰之为病。脉微細。但欲寐。281

少陰病。欲吐不吐。心煩。但欲寐。五六日。自利而渴者。屬少陰。虚故引水自救。若小便色白者。少陰病形悉具。所以然者。以下焦虚寒。不能制水。故白也。282
	\footnote{「欲吐不吐心煩」千金翼作「欲吐而不煩」,聖惠方作「其人欲吐而不煩」。}

*傷寒五日。少陰受病。其脉微細。但欲寐。其人欲吐而不煩。五日自利而渴者。屬陰虚。故引水以自救。小便白而利者。下焦有虚寒。故不能制水。而小便白也。宜龙骨牡蛎湯。(聖惠方)281.282

病人脉陰陽俱緊。反汗出者。为亡陽。此屬少陰。法当咽痛。而復吐利。283

少陰病。欬而下利。譫語者。被火气劫故也。小便必難。以强責少陰汗也。284

*少陰病。欬而下利。譫語。是为心臓有積熱故也。小便必難。宜服豬苓湯。(聖惠方)284

少陰病。脉細沈數。病为在裏。不可发汗。285

少陰病。脉微。不可发汗。亡陽故也。陽已虚。尺中弱濇者。復不可下之。286

少陰病。脉緊。至七八日。[自]下利。脉暴微。手足反温。脉緊反去。此为欲觧。雖煩。下利。必自愈。287

少陰病。下利。若利自止。惡寒而踡。手足温者。可治。288

少陰病。惡寒而踡。时自煩。欲去衣被者。可治。289
	\footnote{「可治」千金翼作「不可治」三字。}

少陰中風。脉陽微陰浮者。为欲愈。290

少陰病欲觧时。從子尽寅。291

少陰病八九日。一身手足尽熱者。以熱在膀胱。必便血。293

少陰病。吐利。手足不逆[冷]。反发熱者。不死。脉不至者。灸少陰七壯。292

*少陰病。吐利。手足逆而發熱。脉不足者。灸其少陰。(聖惠方)

少陰病。但厥。无汗。而强发之。必動其血。未知從何道出。或從口鼻。或從[耳]目出。是为下厥上竭。为難治。294

少陰病。惡寒。身踡而利。手足逆[冷]者。不治。295

少陰病。下利止而頭眩。时时自冒者死。297

少陰病。吐利。躁煩。四逆者死。296

少陰病。四逆。惡寒而踡。脉不至。不煩而躁者死。298

少陰病六七日。息高者死。299

少陰病。脉微細沈。但欲卧。汗出不煩。自欲吐。[至]五六日。自利。復煩燥。不得卧寐者死。300

少陰病。始得之。反发熱。脉沈者。麻黄細辛附子湯主之。301

少陰病。得之二三日。麻黄附子甘草湯微发汗。以二三日无證。故微发汗。302
	\footnote{「无證」玉函作「无裏證」。}

少陰病。得之二三日以上。心中煩。不得卧。黄連阿膠湯主之。303

少陰病。得之一二日。口中和。其背惡寒者。当灸之。附子湯主之。304

*少陰病。其人雖裏和。其病惡寒者。宜灸之。(聖惠方)304

少陰病。身体痛。手足寒。骨節痛。脉沈者。附子湯主之。305

少陰病。下利。便膿血。桃花湯主之。306

少陰病二三日至四五日。腹痛。小便不利。下利不止。便膿血。桃花湯主之。307

少陰病。下利。便膿血者。可刺。308

少陰病。吐利。手足逆[冷]。煩躁欲死者。[吳]茱萸湯主之。309

少陰病。下利。咽痛。胸滿。心煩。豬膚湯主之。310

少陰病二三日。咽痛者。可与甘草湯。不差者。与桔梗湯。311

少陰病。咽中傷。生瘡。不能語言。聲不出者。苦酒湯主之。312

少陰病。咽中痛。半夏散及湯主之。313

少陰病。下利。白通湯主之。314

少陰病。下利。脉微。服白通湯。利不止。厥逆。无脉。乾嘔。煩者。白通加豬膽汁湯主之。服湯脉暴出者死。微[微]續[出]者生。315
	\footnote{「利不止」脉經作「下利止」,聖惠方作「止後」。\\「服湯脉暴出者死微微續出者生」脉經作「服湯藥其脉暴出者死微細者生」。}

*少陰病。下利。服白通湯。止後。厥逆。无脉。煩躁者。宜白通豬苓湯。其脉暴出者死。微微續出者生。(聖惠方)315

少陰病。二三日不已。至四五日。腹痛。小便不利。四肢沈重疼痛而利。此为有水气。其人或欬。或小便[自]利。或不利。或嘔。玄武湯主之。316
	\footnote{「不利」各本均作「下利」,为編者所改。}

少陰病。下利清穀。裏寒外熱。手足厥逆。脉微欲絶。身反不惡寒。其人面赤。或腹痛。或乾嘔。或咽痛。或利止而脉不出。通脉四逆湯主之。317
	\footnote{「身反不惡寒」千金翼、聖惠方作「身反惡寒」。\\「或利止而脉不出」聖惠方作「或时利止而脉不出者」。}

少陰病。四逆。其人或欬。或悸。或小便不利。或腹中痛。或泄利下重。四逆散主之。318

少陰病。下利六七日。欬而嘔。渴。心煩不得眠。豬苓湯主之。319

*少陰病。下利。咳而嘔。煩渴。不得眠卧。宜豬苓湯。(聖惠方)319

少陰病。得之二三日。口燥。咽乾者。急下之。宜[大]承气湯。320

少陰病。下利清水。色青者。心下必痛。口乾燥者。急下之。宜[大]承气湯。321
	\footnote{「色青者」宋本、玉函作「色純青」。\\「急下之」玉函、聖惠方、外臺同,脉經、宋本、玉函、千金翼作「可下之」。}

少陰病六七日。腹滿。不大便者。急下之。宜[大]承气湯。322

少陰病。脉沈者。急温之。宜四逆湯。323

少陰病。其人飲食入則吐。心中嗢嗢欲吐。復不能吐。始得之。手足寒。脉弦遲。此胸中実。不可下也。当吐之。若膈上有寒飲。乾嘔者。不可吐。当温之。宜四逆湯。324

少陰病。下利。脉微濇。嘔而汗出。必數更衣。反少者。当温其上。灸之。325
	\footnote{「灸之」二字下宋本、脉經有「一云灸厥陰可五十狀」小注。}

\chapter{辨厥陰病}

厥陰之为病。消渴。气上撞[心]。心中疼熱。飢而不欲食。[甚者]食則吐[蛔]。下之利不止。326

厥陰中風。其脉微浮为欲愈。不浮为未愈。327

厥陰病欲觧时。從丑尽卯。328

厥陰病。渴欲飲水者。少少与之即愈。329

*傷寒六日。渴欲飲水者。宜豬苓湯。(聖惠方)329

\chapter{辨厥利嘔噦}

諸四逆。厥者。不可下之。虚家亦然。330

*諸四逆。厥者。不可吐之。虚家亦然。330

傷寒。先厥後发熱而利者。必自止。見厥復利。331

傷寒。始发熱六日。厥反九日而利。凡厥利者。当不能食。今反能食。恐为除中。食以黍餅。不发熱者。知胃气尚在。必愈。恐暴熱來出而復去也。後日脉之。其熱續在者。期之旦日夜半愈。所以然者。本发熱六日。厥反九日。復发熱三日。并前六日。亦为九日。与厥相應。故期之旦日夜半愈。後三日脉之而脉數。其熱不罷者。此为熱气有餘。必发癰膿。332

傷寒。脉遲六七日。而反与黄芩湯徹其熱。脉遲为寒。而与黄芩湯復除其熱。腹中應冷。当不能食。今反能食。此为除中。必死。333

傷寒。先厥後发熱。下利必自止。而反汗出。咽中痛者。其喉为痹。发熱。无汗。而利必自止。若不止。必便膿血。便膿血者。其喉不痹。334

傷寒一二日至四五日。厥者。必发熱。前熱者後必厥。厥深者熱亦深。厥微者熱亦微。厥應下之。而反发汗者。必口傷爛赤。335
	\footnote{「前熱者後必厥」脉經、玉函、千金翼作「前厥者後必熱」。}

凡厥者。陰陽气不相順接。便为厥。厥者。手足逆冷是也。337

傷寒。病厥五日。熱亦五日。設六日当復厥。不厥者自愈。厥終不過五日。以熱五日。故知自愈。336

傷寒。脉微而厥。至七八日。膚冷。其人躁无暫安时者。此为臓厥。非蛔厥也。蛔厥者。其人当吐蛔。今病者靜。而復时煩。此为臓寒。蛔上入膈。故煩。須臾復止。得食而嘔。又煩者。蛔聞食臭出。其人常自吐蛔。蛔厥者。烏梅丸主之。338

傷寒。熱少。厥微。指頭寒。默默不欲食。煩躁。數日。小便利。色白者。此熱除也。欲得食。其病为愈。若厥而嘔。胸脇煩滿者。其後必便血。339

病者手足厥冷。言我不結胸。小腹滿。按之痛。此冷結在膀胱関元也。340
	\footnote{「小腹」千金翼作「少腹」。}

傷寒。发熱四日。厥反三日。復[发]熱四日。厥少熱多。其病当愈。四日至七日。熱不除者。必便膿血。341

傷寒。厥四日。熱反三日。復厥五日。其病为進。寒多熱少。陽气退。故为進。342

傷寒六七日。脉微。手足厥[冷]。煩躁。灸其厥陰。厥不還者死。343
	\footnote{「脉微」千金翼、聖惠方作「其脉數」。}

傷寒。[发熱。]下利。厥逆。躁不得卧者死。344
	\footnote{「发熱」二字宋本、玉函有,脉經、千金翼无。}

傷寒。发熱。下利至[甚。]厥不止者死。345

傷寒六七日。不利。忽发熱而利。其人汗出不止者死。有陰无陽故也。346

*傷寒。厥逆六七日。不利。便发熱而利者生。其人汗出。利不止者死。但有陰无陽故也。(脉經)346

*傷寒六七日。不利。便发熱而利。其人汗出不止者死。有陰无陽故也。(千金翼。宋本)346

傷寒五六日。不結胸。腹濡。脉虚。復厥者。不可下。此为亡血。[下之]死。347

傷寒。发熱而厥七日。下利者。为難治。348

傷寒。脉促。手足厥逆。可灸之。349

傷寒。脉滑而厥者。裏有熱也。白虎湯主之。350

手足厥寒。脉細欲絶者。当歸四逆湯主之。若其人内有久寒者。当歸四逆加吳茱萸生薑湯主之。351.352

大汗出。熱不去。内拘急。四肢疼。又下利。厥逆而惡寒。四逆湯主之。353
	\footnote{「又下利」脉經作「下利」,千金翼作「若下利」。}

大汗出或大下利。而厥冷者。四逆湯主之。354
	\footnote{「或」各本均作「若」。}

病者手足厥冷。脉乍緊。邪結在胸中。心下滿而煩。飢不能食。病在胸中。当吐之。宜瓜蒂散。355

傷寒。厥而心下悸。宜先治水。当与茯苓甘草湯。卻治其厥。不尔。水漬入胃。必作利也。356

傷寒六七日。大下後。[寸]脉沈遲。手足厥逆。下部脉不至。咽喉不利。唾膿血。泄利不止者。为難治。麻黄升麻湯主之。357

傷寒四五日。腹中痛。若轉气下趨少腹者。为欲自利也。358

傷寒。本自寒下。醫復吐下之。寒格。更逆吐下。食入即出。乾薑黄芩黄連人参湯主之。359

下利。有微熱而渴。脉弱者自愈。360

下利。脉數。有微熱。汗出者。自愈。設[脉]復緊。为未觧。361
	\footnote{「有微熱汗出者」千金翼作「若微发熱汗出者」。\\「設脉復緊」除千金翼外其它版本均无「脉」字。}

下利。手足厥[冷]。无脉。[当灸其厥陰。]灸之不温[而脉不還]。反微喘者死。少陰負趺陽者为順。362

下利。寸脉反浮數。尺中自濇者。必清膿血。363

下利清穀。不可攻表。汗出必胀滿。364

下利。脉沈弦者下重。脉大者为未止。脉微弱數者为欲自止。雖发熱。不死。365

下利。脉沈而遲。其人面少赤。身有微熱。下利清穀者。必鬱冒。汗出而觧。其人必微厥。所以然者。其面戴陽。下虚故也。366

下利。脉反數而渴者。今自愈。設不差。必清膿血。以有熱故也。367

下利後。脉絶。手足厥[冷]。晬时脉還。手足温者生。不還[不温]者死。368

傷寒。下利日十餘行。脉反実者死。369

下利清穀。裏寒外熱。汗出而厥。通脉四逆湯主之。370

熱利下重者。白頭翁湯主之。371

下利。欲飲水者。为有熱也。白頭翁湯主之。373

下利。腹[胀]滿。身体疼痛者。先温其裏。乃攻其表。温裏宜四逆湯。攻表宜桂枝湯。372

下利。譫語者。有燥屎也。宜[小]承气湯。374

下利後更煩。按之心下濡者。为虚煩也。梔子[豉]湯主之。375

嘔家有癰膿。不可治嘔。膿尽自愈。376

嘔而发熱者。小柴胡湯主之。379

嘔而脉弱。小便復利。身有微熱。見厥者。難治。四逆湯主之。377

乾嘔。吐涎沫。頭痛者。吳茱萸湯主之。378
	\footnote{「頭痛者」玉函、千金翼作「而復頭痛」。}

傷寒。大吐大下之。極虚。復極汗者。其人外气怫鬱。復与之水。以发其汗。因得噦。所以然者。胃中寒冷故也。380

傷寒。噦而腹滿。視其前後。知何部不利。利之即愈。381

\chapter{辨霍亂}

問曰。病有霍亂者何。\\
答曰。嘔吐而利。此为霍亂。382

問曰。病发熱。頭痛。身疼。惡寒。吐利者。当屬何病。\\
答曰。当为霍亂。霍亂吐利止。復更发熱也。383
	\footnote{「霍亂吐利止」千金翼作「霍亂吐下利止」,玉函作「吐下利止」,宋本作「霍亂自吐下又利止」。}

傷寒。其脉微濇。本是霍亂。今是傷寒。卻四五日。至陰經上。轉入陰。必利。本素嘔下利者不治。若其人似欲大便。但反失气而不利者。此屬陽明。便必堅。十三日愈。所以然者。經尽故也。384
	\footnote{「必利」玉函、千金翼作「当利」,脉經作「必吐利」。}

下利後当便堅。堅則能食者愈。今反不能食。到後經中。頗能食。復過一經能食。過之一日当愈。若不愈。不屬陽明也。384

惡寒。脉微。而復利。利止必亡血。四逆加人参湯主之。385
	\footnote{「利止必亡血」宋本、玉函作「利止亡血也」。}

霍亂。頭痛。发熱。身疼痛。熱多。欲飲水者。五苓散主之。寒多。不用水者。理中湯主之。386
	\footnote{「理中湯」宋本作「理中丸」。}

吐利止而身痛不休者。当消息和觧其外。宜桂枝湯小和之。387

吐利。汗出。发熱。惡寒。四肢拘急。手足厥冷。四逆湯主之。388

既吐且利。小便復利。而大汗出。下利清穀。裏寒外熱。脉微欲絶。四逆湯主之。388

吐已下斷。汗出而厥。四肢拘急不觧。脉微欲絶。通脉四逆加豬膽汁湯主之。390

吐利发汗後。其人脉平。小煩者。以新虚不勝穀气故也。391

\chapter{辨陰易病已後勞復\footnote{「陰易」除千金翼外其它版本均作「陰陽易」。}}

傷寒陰易之为病。其人身体重。少气。少腹裏急。或引陰中拘攣。熱上衝胸。頭重不欲舉。眼中生眵。[眼胞赤。]膝脛拘急。燒裩散主之。392
	\footnote{「眼中生眵」除千金要方外其它版本均作「眼中生花」。\\「眼胞赤」三字宋本、千金要方、外臺均无,千金翼作「痂胞赤」。}

大病差後勞復者。枳実梔子湯主之。393

傷寒差已後。更发熱者。小柴胡湯主之。脉浮者。以汗觧之。脉沈実者。以下觧之。394

大病差後。從腰以下有水气者。牡蛎澤瀉散主之。395

傷寒觧後。虚羸少气。气逆欲吐。竹葉石膏湯主之。397

大病差後。其人喜唾。久不了了者。胃上有寒。当温之。宜理中丸。396
	\footnote{「胃上有寒」千金翼作「胸上有寒」。}

病人脉已觧。而日暮微煩者。以病新差。人强与穀。脾胃气尚弱。不能消穀。故令微煩。損穀即愈。398

\chapter{发汗吐下後}

发汗後。身熱。又重发汗。胃中虚冷。必反吐也。0

大下後。口燥者。裏虚故也。0

\chapter{不可发汗}

咽中閉塞。不可发汗。发汗即吐血。气微絶。厥冷。

*凡咽中閉塞。不可发汗。(聖惠方)

厥[而脉緊]。不可发汗。发汗即聲亂。咽嘶。舌萎。聲不能出。

冬时。不可发汗。发汗必吐利。口中爛。生瘡。0

*凡積熱在臓。不宜发汗。汗則必吐。口中爛。生瘡。(聖惠方)

欬而小便利。若失小便者。不可攻表。汗出則厥。逆冷。

*欬嗽小便利者。不可攻表。汗出即逆。(聖惠方)

太陽病。发其汗。因致痙。

*太陽病。发汗太多。因致痙。(宋本。金匱)

\chapter{可发汗}

大法。春夏宜发汗。

凡发汗。欲令手足皆周至。汗出漐漐然。一时間許益佳。不可令如水流離。若病不觧。当重发汗。汗多則亡陽。陽虚不得重发汗也。

凡服湯发汗。中病便止。不必尽剂也。

凡云可发汗而无湯者。丸散亦可用。然不如湯隨證良驗。

凡脉浮者。病在外。可发汗。0

陽明病。脉浮虚者。可发汗。(千金翼)0

*陽明病。脉浮數者。可发汗。(聖惠方)0

\chapter{可吐}

大法。春宜吐。

凡服湯吐。中病便止。不必尽剂也。

病胸上諸実。胸中鬱鬱而痛。不能食。欲使人按之。而反有涎唾。下利日十餘行。其脉反遲。寸口[脉]微滑。此可吐之。吐之利即止。

*夫胸心滿実。胸中鬱鬱而痛。不能食。多涎唾。下利。其脉遲反逆。寸口脉數。此可吐也。(聖惠方)

宿食在上脘。宜吐之。
	\footnote{「上脘」脉經、宋本作「上管」,聖惠方作「胃管」}

\chapter{不可下}

咽中閉塞。不可下。下之則上輕下重。水漿不下。卧則欲踡。身体急痛。下利日數十行。

諸外実。不可下。下之則发微熱。亡脉則厥。当脐握熱。

諸虚。不可下。下之則渴。引水者易愈。惡水者劇。

脉數者。久數不止。止則邪結。正气不能復。正气卻結於臓。故邪气浮之。与皮毛相得。脉數者。不可下。下之必煩。利不止。

脉浮大。應发汗。醫反下之。此为大逆。

太陽与少陽并病。心下痞堅。頸項强而眩。[当刺大椎第一間。肺腧。肝腧。]不可下。171

*太陽与少陽并病。頭項强痛。或眩冒。时如結胸。心下痞堅。当刺大椎第一間。肺腧。肝腧。慎不可发汗。发汗即譫語。譫語則脉弦。譫語五日不止。当刺期門。142

病欲吐者。不可下。

\chapter{可下}

大法。秋宜下。

凡可下者。用湯勝丸散。

凡服湯下。中病則止。不必尽剂也。

下利。三部脉皆平。按其心下。堅者。急下之。宜[大]承气湯。

*傷寒下痢。三部脉皆和。按其心下堅。宜急下之。(聖惠方)

下利。脉遲而滑者。[内]実也。利未欲止。当下之。宜[大]承气湯。

問曰。人病有宿食。何以别之。\\
師曰。寸口脉浮大。按之反濇。尺中亦微而濇。故知有宿食。当下之。宜[大]承气湯。

下利。不欲食者。有宿食也。当下之。宜[大]承气湯。

下利[已]差。至其[年月日]时復发者。此为病不尽。当復下之。宜[大]承气湯。

病腹中滿痛者。为実。当下之。宜大承气湯。

*病腹中滿痛者。为実。当下之。宜大柴胡湯。

脉雙弦而遲。心下堅。脉大而緊者。陽中有陰也。可下之。宜[大]承气湯。

\chapter{可温}

大法。冬宜服温熱藥及灸。

*大法。冬宜熱藥。(聖惠方)

下利。脉遲緊。为痛未欲止。当温之。得冷者。滿而便腸垢。0

*下利。脉遲緊。为痛未止。(聖惠方)

下利。脉浮大者。此为虚。以强下之故也。当温之。与水必噦。[宜当歸四逆湯。]0

*下利。脉浮大者。此皆为虚。宜温之。(聖惠方)

下利。欲食者。当温之。0

\chapter{可火}

下利。穀道中痛。当温之。宜灸枳実。或熬鹽等熨之。0

*凡下利後。下部中痛。当温之。宜炒枳実。若熬鹽等熨之。(聖惠方)

\chapter{不可刺}

大怒勿刺。[已刺勿怒。]新内勿刺。[已刺勿内。]大勞勿刺。[已刺勿勞。]大醉勿刺。[已刺勿醉。]大飽勿刺。[已刺勿飽。大飢勿刺。已刺勿飢。]大渴勿刺。[已刺勿渴。]大驚勿刺。

勿刺熇熇之熱。勿刺漉漉之汗。勿刺渾渾之脉。

身熱甚。陰陽皆爭者。勿刺也。其可刺者。急取之。不汗則泄。所谓勿刺者。有死徵也。

勿刺病与脉相逆者。上工刺未生。其次刺未盛。其次刺已衰。工逆此者。是谓伐形。0

\chapter{可刺}

婦人傷寒。懷娠。腹滿。不得小便。從腰以下重。如有水气狀。懷娠七月。太陰当養不養。此心气実。当刺。瀉勞宮及関元。小便利則愈。0

*婦人傷寒。懷娠。腹滿。不得大便。從腰以下重。如有水气狀。懷娠七月。太陰当養不養。此心气実。当刺。瀉勞宮及関元。小便利則愈。0

傷寒。喉痹。刺手少陰。少陰在腕当小指後動脉是也。針入三分補之。0

\chapter{不可水}

下利。其脉浮大。此为虚。以强下之故也。設脉浮革。因尔腸鳴。当温之。与水即噦。0

太陽病。小便利者。为水多。心下必悸。0

\chapter{可水}

嘔吐而病在膈上。急思水者。与五苓散飲之。即可飲水也。0

*嘔吐而病在膈上。後必思水者。与五苓散飲之。水亦得也。(脉經。玉函)0

*若嘔吐。熱在膈上。思水者。与五苓散。即可飲水也。(聖惠方)0

\chapter{衍文}

問曰。證象陽旦。按法治之而增劇。厥逆。咽中乾。兩脛拘急而讝語。師曰。言夜半手足当温。兩腳当伸。後如師言。何以知此。荅曰。寸口脉浮而大。浮則为風。大則为虚。風則生微熱。虚則兩脛攣。病證象桂枝。因加附子参其間。增桂令汗出。附子温經。亡陽故也。厥逆。咽中乾。煩燥。陽明內結。讝語。煩亂。更飲甘草乾薑湯。夜半陽气還。兩足当熱。脛尚微拘急。重与芍藥甘草湯。尔乃脛伸。以承气湯微溏。則止其讝語。故知病可愈。30

太陽病二日。反躁。反熨其背。而大汗出。大熱入胃。胃中水竭。躁煩必发譫語。十餘日。振慄自下利者。此为欲觧也。故其汗從腰以下不得汗。欲小便不得。反嘔。欲失溲。足下惡風。大便硬。小便当數而反不數及多。大便已。頭卓然而痛。其人足心必熱。穀气下流故也。110

太陽病。中風。以火劫发汗。邪風被火熱。血气流溢。失其常度。兩陽相熏灼。其身发黄。陽盛則欲衄。陰虚則小便難。陰陽俱虚竭。身体則枯燥。但頭汗出。齐頸而還。腹滿微喘。口乾咽爛。或不大便。久則譫語。甚者至噦。手足躁擾。捻衣摸床。小便利者。其人可治。111

太陽病。吐之。但太陽病当惡寒。今反不惡寒。不欲近衣。此为吐之内煩也。121

太陽病。小便利者。以飲水多。必心下悸。小便少者。必苦裏急也。127

脉按之來緩。而时一止復來者。名曰結。又脉來動而中止。更來小數。中有還者反動。名曰結。陰也。脉來動而中止。不能自還。因而復動名曰代。陰也。得此脉者。必難治。178

\part{金匱要略}

\chapter{臓腑經絡先後}

	\footnote{「問曰上工治未病」一段,吳本在臓腑經絡先後篇之前,未入正文,且此段文字都是无用不実之词,故刪。}

問曰。病人有气色見於面部。願聞其説。\\
師曰。鼻頭色青。腹中痛。苦冷者死。鼻頭色微黑者。有水气。色黄者。胸上有寒。色白者。亡血也。設微赤。非时者死。其目正圓者。痙。不治。又色青为痛。色黑为勞。色赤为風。色黄者便難。色鲜明者有留飲。

師曰。病人語聲寂然。喜驚呼者。骨節間病。語聲喑喑然不徹者。心膈間病。語聲啾啾然細而长者。頭中病。

師曰。息搖肩者。心中堅。息引胸中上气者。欬。息张口短气者。肺痿唾沫。

師曰。吸而微數。其病在中焦。実也。当下之即愈。虚者不治。在上焦者。其吸促。在下焦者。其吸遠。此皆難治。呼吸動搖振振者。不治。

師曰。寸口脉動者。因其王时而動。假令肝王色青。四时各隨其色。肝色青而反色白。非其时色脉。皆当病。

問曰。有未至而至。有至而不至。有至而不去。有至而太過。何谓也。\\
師曰。冬至之後。甲子夜半少陽起。少陽之时陽始生。天得温和。以未得甲子。天因温和。此为未至而至也。以得甲子而天未温和。此为至而不至也。以得甲子而天大寒不觧。此为至而不去也。以得甲子而天温如盛夏五六月时。此为至而太過也。

師曰。病人脉浮者在前。其病在表。浮者在後。其病在裏。腰痛背强不能行。必短气而極也。

問曰。經云。厥陽獨行。何谓也。\\
師曰。此为有陽无陰。故稱厥陽。

問曰。寸脉沈大而滑。沈則为実。滑則为气。実气相摶。血气入臓即死。入腑即愈。此为卒厥。何谓也。\\
師曰。唇口青。身冷。为入臓即死。如身和。汗自出。为入腑即愈。

問曰。脉脱。入臓即死。入腑即愈。何谓也。\\
師曰。非为一病。百病皆然。譬如浸淫瘡。從口起流向四肢者。可治。從四肢流來入口者。不可治。[諸]病在外者可治。入裏者即死。

問曰。陽病十八。何谓也。\\
師曰。頭痛。項。腰。脊。臂。腳掣痛。\\
問曰。陰病十八。何谓也。\\
師曰。欬。上气。喘。噦。咽。腸鳴。胀滿。心痛。拘急。五臓病各有十八。合为九十病。人又有六微。微有十八病。合为一百八病。五勞。七傷。六極。婦人三十六病。不在其中。清邪居上。濁邪居下。大邪中表。小邪中裏。䅽飪之邪。從口入者。宿食也。五邪中人。各有法度。風中於前。寒中於暮。濕傷於下。霧傷於上。風令脉浮。寒令脉急。霧傷皮腠。濕流関節。食傷脾胃。極寒傷經。極熱傷絡。

問曰。病有急当救裏救表者。何谓也。\\
師曰。病。醫下之。續得下利。清穀不止。身体疼痛者。急当救裏。後身体疼痛。清便自調者。急当救表也。

夫病痼疾。加以卒病。当先治其卒病。後乃治其痼疾也。

師曰。五臓病各有所得者愈。五臓病各有所惡。各隨其所不喜者为病。病者素不應食。而反暴思之。必发熱也。

夫諸病在臓。欲攻之。当隨其所得而攻之。如渴者。与豬苓湯。餘皆仿此。

\chapter{痙濕暍}

太陽病。发熱。无汗。反惡寒者。为剛痙。

太陽病。发熱。汗出。不惡寒者。为柔痙。

太陽病。发熱。脉沈細者。为痙。
	\footnote{「为痙」金匱要略作「名曰痙为難治」。}

太陽病。发汗太多。因致痙。
	\footnote{脉經、玉函、千金翼「发汗太多」作「发其汗」。}

病者身熱足寒。頸項强急。惡寒。时頭熱。面赤。目赤。獨頭動搖。卒口噤。背反张者。痙病也。
	\footnote{「目赤」鄧本同,脉經、玉函、千金翼、吳本均作「目脉赤」。}

痙病。发其汗者。寒濕相得。其表益虚。即惡寒甚。发其汗已。其脉如蛇。暴腹胀大者。为欲觧。脉如故。反伏弦者。痙

夫風病。下之則痙。復发汗。必拘急。

夫痙脉。按之緊如弦。直上下行。

痙病有灸瘡。難治。

瘡家。雖身疼痛。不可发汗。汗出則痙。85

太陽病。其證備。身体强。几几然。脉反沈遲。此为痙。栝蔞桂枝湯主之。

太陽病。无汗而小便反少。气上衝胸。口噤不得語。欲作剛痙。葛根湯主之。

[剛]痙为病。胸滿。口噤。卧不著席。腳攣急。其人必齘齒。可与大承气湯。

太陽病。関節疼煩。脉沈緩者。此名濕痹。濕痹之候。其人小便不利。大便反快。但当利其小便。
	\footnote{「関節疼煩」宋本、金匮「関節疼痛而煩」,脉經作「関節疼痛」。\\「脉沈緩者」宋本、金匱作「脉沈而細者」。\\「此名濕痹」脉經、玉函、千金翼作「为中濕」。}

濕家之为病。一身尽疼。发熱。身色如熏黄也。

濕家。其人但頭汗出。背强。欲得被覆向火。若下之早則噦。或胸滿。小便[不]利。舌上如胎。以丹田有熱。胸上有寒。渴欲得飲而不能飲。則口燥[煩]也。

濕家下之。額上汗出。微喘。小便利者死。下利不止者亦死。

問曰。風濕相摶。一身尽疼痛。法当汗出而觧。值天陰雨不止。師云。此可发汗。汗之病不愈者。何也。\\
答曰。发其汗。汗大出者。但風气去。濕气續在。是故不愈。若治風濕者。发其汗。但微微似欲出汗者。則風濕俱去也。

濕家。病身疼。发熱。面黄而喘。頭痛。鼻塞而煩。其脉大。自能飲食。腹中和。无病。病在頭。中寒濕。故鼻塞。内藥鼻中即愈。

濕家身煩疼。可与麻黄加术湯。发其汗为宜。慎不可以火攻之。

病者一身尽疼。发熱。日晡所劇者。名風濕。此病傷於汗出当風。或久傷取冷所致也。可与麻杏薏甘湯。

風濕。脉浮。身重。汗出。惡風者。防己黄耆湯主之。

傷寒八九日。風濕相摶。身体疼煩。不能自轉側。不嘔。不渴。脉浮虚而濇者。桂枝附子湯主之。若其人大便堅。小便自利者。术附子湯主之。

風濕相摶。骨節疼煩。掣痛。不得屈伸。近之則痛劇。汗出短气。小便不利。惡風。不欲去衣。或身微腫者。甘草附子湯主之。

太陽中熱者。暍是也。其人汗出。惡寒。身熱而渴。白虎[加人参]湯主之。

太陽中暍。身熱疼重而脉微弱。此以夏月傷冷水。水行皮膚中所致也。瓜蒂湯主之。

太陽中暍。发熱。惡寒。身重而疼痛。其脉弦細芤遲。小便已。洒洒然毛聳。手足逆冷。小有勞。身即熱。口開。前板齒燥。若发其汗。則惡寒甚。加温針。則发熱甚。數下之。則淋甚。

\chapter{百合狐惑陰陽毒}

論曰。百合病者。百脉一宗。悉致其病也。意欲食復不能食。常默默。欲卧復不得眠。欲行復不能行。飲食或有美时。或有不用聞食臭时。如寒无寒。如熱无熱。口苦。小便赤。諸藥不能治。得藥則劇吐利。如有神靈者。身形如和。其脉微數。每尿时頭痛者。六十日乃愈。若尿时頭不痛。淅然者。四十日愈。若尿快然。但頭眩者。二十日愈。其證或未病而預見。或病四五日而出。或病二十日或一月微見者。各隨證治之。

百合病发汗後。更发者。百合知母湯主之。

百合病下之後。更发者。百合滑石代赭湯主之。

百合病吐之後。更发者。百合雞子湯主之。

百合病。不經发汗吐下。病形如初者。百合地黄湯主之。

百合病。經月不觧。變成渴者。百合洗方主之。

百合病。渴不差者。栝蔞牡蛎散主之。
	\footnote{千金要方此條为上條注文。}

百合病。變发熱者。百合滑石散主之。

百合病。變腹中滿痛者。但取百合根隨多少熬令黄色搗篩为散飲服方寸匕。日三。滿消痛止。
	\footnote{此條金匱要略无,從千金要方補入}

百合病。見於陰者。以陽法救之。見於陽者。以陰法救之。見陽攻陰。復发其汗。此为逆。見陰攻陽。乃復下之。此亦为逆。

狐惑之为病。狀如傷寒。默默欲眠。目不得閉。卧起不安。蝕於喉为惑。蝕於陰为狐。不欲飲食。惡聞食臭。其面目乍赤。乍黑。乍白。蝕於上部則聲喝。甘草瀉心湯主之。蝕於下部則咽乾。苦参湯洗之。蝕於肛者。雄黄熏之。

病者脉數。无熱。微煩。默默。但欲卧。汗出。初得之三四日。目赤如鳩眼。七八日目四眥黑。若能食者。膿已成也。赤[小]豆当歸散主之。

陽毒之为病。面赤斑斑如锦文。咽喉痛。唾膿血。五日可治。七日不可治。升麻鱉甲湯主之。
	\footnote{「升麻鱉甲湯主之」吳本无。}

陰毒之为病。面目青。身痛如被杖。咽喉痛。五日可治。七日不可治。升麻鱉甲湯去雄黄蜀椒主之。
	\footnote{「如被杖」吳本作「狀如被打」。}

\chapter{瘧}

師曰。瘧脉自弦。弦數者多熱。弦遲者多寒。弦小緊者下之差。弦遲者可温之。弦緊者可发汗。針灸也。浮大者可吐之。弦數者風发也。以飲食消息止之。
	\footnote{「風发」吳本作「風疾」。}

問曰。瘧以月一日发。当以十五日愈。設不差。当月尽觧。如其不差。当如何。\\
師曰。此結为癥瘕。名曰瘧母。急治之。宜鱉甲煎丸。

師曰。陰气孤絶。陽气獨发。則熱而少气煩寃。手足熱而欲嘔。名曰癉瘧。若但熱不寒者。邪气内藏於心。外舍分肉之間。令人消鑠脱肉。
	\footnote{「煩寃」吳本作「煩滿」。}

温瘧者。其脉如平。身无寒。但熱。骨節疼煩。时嘔。白虎加桂枝湯主之。

瘧多寒者。名曰牡瘧。蜀漆散主之。

附方

牡蛎湯。治牡瘧。

瘧病发渴者。与小柴胡去半夏加栝蔞湯。

柴胡桂薑湯。此方治寒多微有熱。或但寒不熱。服一剂如神。故錄之。
	\footnote{「此方」至「錄之」二十一字为小字注文。}

\chapter{中風歷節}

夫風之为病。当半身不遂。或但臂不遂者。此为痹。脉微而數。中風使然。

寸口脉浮而緊。緊則为寒。浮則为虚。寒虚相摶。邪在皮膚。浮者血虚。絡脉空虚。賊邪不瀉。或左或右。邪气反緩。正气即急。正气引邪。喎僻不遂。邪在於絡。肌膚不仁。邪在於經。即重不勝。邪入於腑。即不識人。邪入於臓。舌即難言。口吐涎。

大風。四肢煩重。心中惡寒不足者。侯氏黑散主之。

寸口脉遲而緩。遲則为寒。緩則为虚。榮緩則为亡血。衛緩則为中風。邪气中經。則身癢而癮疹。心气不足。邪气入中。則胸滿而短气。
	\footnote{此條吳本无。}

風引湯。除熱。主癱癇。
	\footnote{此條鄧本作「風引湯除熱癱癇」,吳本作「風引除熱主癱癇湯方」。}

病如狂狀。妄行。獨語不休。惡寒熱。其脉浮。防己地黄湯主之。

頭風摩散方。

寸口脉沈而弱。沈即主骨。弱即主筋。沈即为腎。弱即为肝。汗出入水中。如水傷心。歷節黄汗出。故曰歷節。

跌陽脉浮而滑。滑則穀气実。浮則汗自出。

少陰脉浮而弱。浮則为風。弱則血不足。風血相摶。即疼痛如掣。

盛人脉濇小。短气。自汗出。歷節疼。不可屈伸。此皆飲酒。汗出当風所致。

諸肢節疼痛。身体魁瘰。腳腫如脱。頭眩短气。嗢嗢欲吐。桂枝芍藥知母湯主之。

味酸則傷筋。筋傷則緩。名曰泄。鹹則傷骨。骨傷則痿。名曰枯。枯泄相摶。名曰斷泄。榮气不通。衛不獨行。榮衛俱微。三焦无所御。四屬斷絶。身体羸瘦。獨足腫大。黄汗出。脛冷。假令发熱。便为歷節也。
	\footnote{此條吳本无。}

病歷節。疼痛。不可屈伸。烏頭湯主之。

烏頭湯。治腳气。疼痛。不可屈伸。
	\footnote{此條吳本无。}

礬石湯。治腳气衝心。

附方

續命湯。治中風痱。身体不能自收。口不能言。冒昧不知痛処。或拘急不得轉側。

三黄湯。治中風。手足拘急。百節疼痛。煩熱心亂。惡寒。經日不欲飲食。

术附子湯。治風虚頭重眩。苦極不知食味。暖肌補中。益精气。

崔氏八味丸。治腳气上入。少腹不仁。

越婢加术湯。治肉極熱則身体津脱。腠理開。汗大泄。厉風气。下焦腳弱。
	\footnote{「肉」俞本作「内」。}

\chapter{血痹虚勞}

問曰。血痹病從何得之。\\
師曰。夫尊榮人。骨弱肌膚盛。重因疲勞汗出。卧不时動搖。加被微風。遂得之。[形如風狀。]但以脉自微濇。在寸口。関上小緊。宜針引陽气。令脉和緊去則愈。

血痹。陰陽俱微。寸口関上微。尺中小緊。外證身体不仁。如風痹狀。黄耆桂枝五物湯主之。

夫男子平人。脉大为勞。極虚亦为勞。

男子面色薄者。主渴及亡血。卒喘悸。脉浮者。裏虚也。

男子脉虚沈弦。无寒熱。短气。裏急。小便不利。面色白。时目瞑。兼衄。少腹滿。此为勞使之然。

勞之为病。其脉浮大。手足煩。春夏劇。秋冬差。陰寒精自出。痠削不能行。

男子脉浮弱而濇。为无子。精清泠。
	\footnote{「精清泠」鄧本作「精气清冷」。}

夫失精家。少腹弦急。陰頭寒。目眩。髮落。脉極虚芤遲。为清穀。亡血。失精。脉得諸芤動微緊。男子失精。女子夢交。桂枝加龙骨牡蛎湯主之。天雄散亦主之。
	\footnote{「天雄散亦主之」鄧本作「天雄散」,獨立成一條。}

男子平人。脉虚弱細微者。善盜汗也。

人年五六十。其病脉大者。痹俠背行。苦腸鳴。馬刀俠癭者。皆为勞得之。

脉沈小遲。名脱气。其人疾行則喘喝。手足逆寒。腹滿。甚則溏泄。食不消化也。

脉弦而大。弦則为減。大則为芤。減則为寒。芤則为虚。虚寒相摶。此名为革。婦人則半產。漏下。男子則亡血。失精。

虚勞。裏急。悸。衄。腹中痛。夢失精。四肢痠疼。手足煩熱。咽乾口燥。小建中湯主之。

虚勞。裏急。諸不足。黄耆建中湯主之。

虚勞。腰痛。少腹拘急。小便不利者。八味腎气丸主之。

虚勞。諸不足。風气百疾。薯蕷丸主之。

虚勞。虚煩。不得眠。酸棗湯主之。

*虚勞。煩。悸。不得眠。酸棗湯主之。(千金方)

五勞。虚極。羸瘦。腹滿。不能飲食。食傷。憂傷。飲傷。房室傷。飢傷。勞傷。經絡榮衛气傷。内有乾血。肌膚甲錯。兩目黯黑。緩中補虚。大黄䗪虫丸主之。

附方

虚勞不足。汗出而悶。脉結。心悸。行動如常。不出百日。危急者。十一日死。炙甘草湯主之。

獺肝散。治冷勞。又主鬼疰。一門相染。

\chapter{肺痿肺癰欬嗽上气}

問曰。熱在上焦者。因欬为肺痿。肺痿之病。何從得之。\\
師曰。或從汗出。或從嘔吐。或從消渴。小便利數。[或從便難。]又被快藥下利。重亡津液。故得之。
	\footnote{「或從便難」吳本无。}

問曰。寸口脉數。其人欬。口中反有濁唾涎沫者何。\\
師曰。[此]为肺痿之病。若口中辟辟燥。欬即胸中隱隱痛。脉反滑數。此为肺癰。欬唾濃血。脉數虚者为肺痿。數実者为肺癰。

問曰。病欬逆。脉之何以知此为肺癰。当有膿血。吐之則死。其脉何類。\\
師曰。寸口脉微而數。微則为風。數則为熱。微則汗出。數則惡寒。風中於衛。呼气不入。熱過於榮。吸而不出。風傷皮毛。熱傷血脉。風舍於肺。其人則欬。口乾。喘滿。咽燥。不渴。时唾濁沫。时时振寒。熱之所過。血为凝滯。畜結癰膿。吐如米粥。始萌可救。膿成則死。

上气。面浮腫。肩息。其脉浮大。不治。又加利尤甚。

上气。喘而躁者。屬肺胀。欲作風水。发汗則愈。

肺痿。吐涎沫而不欬者。其人不渴。必遺尿。小便數。所以然者。以上虚不能制下故也。此为肺中冷。必眩。多涎唾。甘草乾薑湯以温之。若服湯已渴者。屬消渴。
	\footnote{「温之若服湯已渴者屬消渴」吳本作「温其病」。}

欬而上气。喉中水雞聲。射干麻黄湯主之。

欬逆上气。时时唾濁。但坐不得卧。皂莢丸主之。

*欬逆。气上衝。唾濁。但坐不得卧。皂莢丸主之。(吳本)

欬而脉浮者。厚朴麻黄湯主之。脉沈者。澤漆湯主之。
	\footnote{「欬而」吳本作「上气」。}

大逆上气。咽喉不利。止逆下气者。麦門冬湯主之。

肺癰。喘不得卧。葶藶大棗瀉肺湯主之。

欬而胸滿。振寒。脉數。咽乾。不渴。时出濁唾腥臭。久久吐膿如米粥者。为肺癰。桔梗湯主之。

欬而上气。此为肺胀。其人喘。目如脱狀。脉浮大者。越婢加半夏湯主之。
	\footnote{「欬而上气」吳本作「欬逆倚息」。}

肺胀。欬而上气。煩躁而喘。脉浮者。心下有水。小青龙加石膏湯主之。
	\footnote{「躁」鄧本作「燥」。}

附方

肺痿。涎唾多。心中嗢嗢液液者。炙甘草湯主之。

肺痿。欬唾涎沫不止。咽燥而渴。生薑甘草湯主之。

肺痿。吐涎沫。桂枝去芍藥加皂莢湯主之。

欬而胸滿。振寒。脉數。咽乾。不渴。时出濁唾腥臭。久久吐膿如米粥者。为肺癰。桔梗白散主之。

葦莖湯。治欬。有微熱。煩滿。胸中甲錯。是为肺癰。

肺癰。胸滿胀。一身面目浮腫。鼻塞。清涕出。不聞香臭酸辛。欬逆上气。喘鳴迫塞。葶藶大棗瀉肺湯主之。

欬而上气。肺胀。其脉浮。心下有水气。脇下痛引缺盆。小青龙加石膏湯主之。

\chapter{奔豚气}

師曰。病有奔豚。有吐膿。有驚怖。有火邪。此四部病。皆從驚发得之。

師曰。奔豚病。從少腹起。上衝咽喉。发作欲死。復還止。皆從驚恐得之。

奔豚。气上衝胸。腹痛。往來寒熱。奔豚湯主之。

发汗後。燒針令其汗。針処被寒。核起而赤者。必发奔豚。气從少腹上衝心者。灸其核上各一壯。与桂枝加桂湯。

发汗後。其人脐下悸者。欲作奔豚。苓桂甘棗湯主之。

\chapter{胸痹心痛短气}

師曰。夫脉当取太過不及。陽微陰弦。即胸痹而痛。所以然者。責其極虚也。今陽虚。知在上焦。所以胸痹心痛者。以其陰弦故也。

平人无寒熱。短气不足以息者。実也。

胸痹之病。喘息欬唾。胸背痛。短气。寸口脉沈而遲。関上小緊數。栝蔞薤白白酒湯主之。

胸痹不得卧。心痛徹背者。栝蔞薤白半夏湯主之。

胸痹。心中痞。留气結在胸。胸滿。脇下逆搶心。枳実薤白桂枝湯主之。理中湯亦主之。

胸痹。胸中气塞。短气。茯苓杏仁甘草湯主之。橘[皮]枳[実生]薑湯亦主之。

胸痹緩急者。薏苡附子散主之。

心中痞。諸逆。心懸痛。桂枝生薑枳実湯主之。

心痛徹背。背痛徹心。烏頭赤石脂丸主之。

九痛丸。治九種心痛。

\chapter{腹滿寒疝宿食}

趺陽脉微弦。法当腹滿。不滿者必便難。兩胠疼痛。此虚寒從下上也。当以温藥服之。

病者腹滿。按之不痛[者]为虚。痛者为実。可下之。舌黄未下者。下之黄自去。

*傷寒。腹滿。按之不痛者为虚。痛者为実。当下之。舌黄未下者。下之黄自去。宜大承气湯。(玉函)

腹滿时減。復如故。此为寒。当与温藥。

病者痿黄。躁而不渴。胸中寒実而利不止者死。

寸口脉弦者。即脇下拘急而痛。其人嗇嗇惡寒也。

夫中寒家。喜欠。其人清涕出。发熱。色和者。善嚏。

中寒。其人下利。以裏虚也。欲嚏不能。此人肚中寒。

夫瘦人繞脐痛。必有風冷。穀气不行。而反下之。其气必衝。不衝者。心下則痞。

病腹滿。发熱十日。脉浮而數。飲食如故。厚朴七物湯主之。

腹中寒气。雷鳴。切痛。胸脇逆滿。嘔吐。附子粳米湯主之。

痛而閉者。厚朴三物湯主之。
	\footnote{「痛而閉者」吳本作「腹滿脉數」。}

按之心下滿痛者。此为実也。当下之。宜大柴胡湯主之。
	\footnote{「按之心下」吳本作「病腹中」。}

腹滿不減。減不足言。当須下之。宜大承气湯。

心胸中大寒痛。嘔。不能飲食。腹中寒。上衝皮起。出見有頭足。上下痛而不可觸近。大建中湯主之。

脇下偏痛[发熱]。其脉緊弦。此寒也。以温藥下之。宜大黄附子湯。
	\footnote{脉經无「发熱」二字。\\按。根據臨床表現,寒実内結,腹痛便祕證,有时可見发熱症狀,但发熱不一定是全身性的,可以在某一局部出現,故「发熱」可与上句連讀为一句。}

寒气厥逆。赤丸主之。

腹痛。脉弦而緊。弦則衛气不行。[衛气不行]即惡寒。緊則不欲食。邪正相摶。即为寒疝。寒疝遶脐痛。若发則白汗出。手足厥冷。其脉沈弦者。大烏頭煎主之。

寒疝。腹中痛。及脇痛裏急者。当歸生薑羊肉湯主之。

寒疝。腹中痛。逆冷。手足不仁。若身疼痛。灸刺諸藥不能治。抵当烏頭桂枝湯主之。

其脉數而緊乃弦。狀如弓弦。按之不移。脉數弦者。当下其寒。脉雙弦而遲者。必心下堅。脉大而緊者。陽中有陰。可下之。
	\footnote{「雙弦」鄧本作「緊大」。}

附方

烏頭湯。治寒疝。腹中絞痛。賊風入攻五臟。拘急不得轉側。发作有时。使人陰縮。手足厥逆。

寒疝。腹中痛者。柴胡桂枝湯主之。(吳本)

*柴胡桂枝湯方。治心腹卒中痛者。(鄧本)

卒疝。走馬湯主之。(吳本)

*走馬湯。治中惡。心痛。腹胀。大便不通。(鄧本)

問曰。人病有宿食。何以别之。\\
師曰。寸口脉浮而大。按之反濇。尺中亦微而濇。故知有宿食。大承气湯主之。

脉緊如轉索无常者。有宿食也。

脉緊。頭痛。風寒。腹中有宿食不化也。

脉數而滑者。実也。此有宿食。下之愈。宜大承气湯。

下利。不欲食者。有宿食也。当下之。宜大承气湯。

宿食在上脘。当吐之。宜瓜蒂散。

\chapter{五臓風寒積聚}

肺中風者。口燥而喘。身運而重。冒而腫胀。

肺中寒者。吐濁涕。

肺死臓。浮之虚。按之弱如葱葉。下无根者。死。

肝中風者。頭目瞤。兩脇痛。行常傴。令人嗜甘。

肝中寒者。兩臂不舉。舌本燥。喜太息。胸中痛。不得轉側。食則吐而汗出也。

肝死臓。浮之弱。按之如索不來。或曲如蛇行者。死。

肝著。其人常欲蹈其胸上。先未苦时。但欲飲熱。旋復花湯主之。

心中風者。翕翕发熱。不能起。心中飢[而欲食]。食即嘔吐。
	\footnote{「心中飢」下吳本有「而欲食」三字。}

心中寒者。其人苦病心如噉蒜狀。劇者心痛徹背。背痛徹心。譬如蠱注。其脉浮者。自吐乃愈。

心傷者。其人勞倦即頭面赤而下重。心中痛而自煩。发熱。当脐跳。其脉弦。此为心臓傷所致也。

心死臓。浮之実如麻豆。按之益躁疾者。死。

邪哭使魂魄不安者。血气少也。血气少者。屬於心。心气虚者。其人則畏。合目欲眠。夢遠行而精神離散。魂魄妄行。陰气衰者为癲。陽气衰者为狂。

脾中風者。翕翕发熱。形如醉人。腹中煩重。皮肉瞤瞤而短气。

脾死臓。浮之大堅。按之如覆杯潔潔。狀如搖者死。

趺陽脉浮而濇。浮則胃气强。濇則小便數。浮濇相摶。大便則堅。其脾为約。麻子仁丸主之。

腎著之病。其人身体重。腰中冷。如坐水中。形如水狀。反不渴。小便自利。飲食如故。病屬下焦。身勞汗出。衣裏冷濕。久久得之。腰以下冷痛。腹重如帶五千錢。甘[草乾]薑[茯]苓[白]术湯主之。

腎死臓。浮之堅。按之亂如轉丸。益下入尺中者死。

問曰。三焦竭部。上焦竭善噫。何谓也。\\
師曰。上焦受中焦气未和。不能消穀。故令噫耳。下焦竭。即遺尿失便。其气不和。不能自禁制。不須治。久自愈。

師曰。熱在上焦者。因欬为肺痿。熱在中焦者。則为堅。熱在下焦者。則尿血。亦令淋祕不通。大腸有寒者。多鶩溏。有熱者。便腸垢。小腸有寒者。其人下重便血。有熱者必痔。

問曰。病有積。有聚。有䅽气。何谓也。\\
師曰。積者。臓病也。終不移。聚者。腑病也。发作有时。展轉痛移。为可治。䅽气者。脇下痛。按之則愈。復发为䅽气。諸積大法。脉來細而附骨者。乃積也。寸口。積在胸中。微出寸口。積在喉中。関上。積在脐傍。上関上。積在心下。微下関。積在少腹。尺中。積在气衝。脉出左。積在左。脉出右。積在右。脉兩出。積在中央。各以其部処之。

\chapter{痰飲欬嗽}

問曰。夫飲有四。何谓也。\\
師曰。有痰飲。有懸飲。有溢飲。有支飲。

問曰。四飲何以为異。\\
師曰。其人素盛今瘦。水走腸間。瀝瀝有聲。谓之痰飲。飲後水流在脇下。欬唾引痛。谓之懸飲。飲水流行。歸於四肢。当汗出而不汗出。身体疼重。谓之溢飲。欬逆倚息。短气不得卧。其形如腫。谓之支飲。

水在心。心下堅築。短气。惡水。不欲飲。

水在肺。吐涎沫。欲飲水。

水在脾。少气身重。

水在肝。脇下支滿。嚏而痛。

水在腎。心下悸。

夫心下有留飲。其人背寒冷如手大。

留飲者。脇下痛引缺盆。欬嗽則輒已。

胸中有留飲。其人短气而渴。四肢歷節痛。脉沈者。有留飲。

膈上病痰。滿。喘。欬。吐。发則寒熱。背痛。腰疼。目泣自出。其人振振身瞤劇。必有伏飲。
	\footnote{「病痰」吳本作「之病」。}

夫病人飲水多。必暴喘滿。凡食少飲多。水停心下。甚者則悸。微者短气。

脉雙弦者。寒也。皆大下後善虚。脉偏弦者。飲也。

肺飲不弦。但苦喘。短气。

支飲。亦喘而不能卧。加短气。其脉平也。

病痰飲者。当以温藥和之。

心下有痰飲。胸脇支滿。目胘。苓桂术甘湯主之。

夫短气。有微飲。当從小便去之。苓桂术甘湯主之。腎气丸亦主之。

病者脉伏。其人欲自利。利反快。雖利。心下續堅滿。此为留飲欲去故也。甘遂半夏湯主之。

脉浮而細滑。傷飲。

脉弦數。有寒飲。冬夏難治。

脉沈而弦者。懸飲内痛。

病懸飲者。十棗湯主之。

病溢飲者。当发其汗。大青龙湯主之。小青龙湯亦主之。

膈間支飲。其人喘滿。心下痞堅。面色黎黑。其脉沈緊。得之數十日。醫吐下之不愈。木防己湯主之。虚者即愈。実者三日復发。復与不愈者。宜去石膏加茯苓芒硝湯。

心下有支飲。其人苦冒眩。澤瀉湯主之。

支飲。胸滿者。厚朴大黄湯主之。

支飲。不得息。葶藶大棗瀉肺湯主之。

嘔家本渴。渴者为欲觧。今反不渴。心下有支飲故也。小半夏湯主之。

腹滿。口舌乾燥。此腸間有水气。己椒藶黄丸主之。

卒嘔吐。心下痞。膈間有水。眩悸者。[小]半夏加茯苓湯主之。

假令瘦人脐下悸。吐涎沫而癲眩。此水也。五苓散主之。
	\footnote{「脐下悸」各本均作「脐下有悸」,編者刪「有」字。}

附方

茯苓飲。治心胸中有停痰宿水。自吐出水後。心胸間虚。气滿。不能食。消痰气。令能食。

欬家。其脉弦。为有水。十棗湯主之。

夫有支飲家。欬煩。胸中痛者。不卒死。至一百日[或]一歲。宜十棗湯。

久欬數歲。其脉弱者可治。実大數者死。其脉虚者必苦冒。其人本有支飲在胸中故也。治屬飲家。

欬逆倚息。[不得卧。]小青龙湯主之。

青龙湯下已。多唾。口燥。寸脉沈。尺脉微。手足厥逆。气從少腹上衝胸咽。手足痹。其面翕熱如醉狀。因復下流陰股。小便難。时復冒者。与茯苓桂枝五味甘草湯。治其气衝。

衝气即低。而反更欬。胸滿者。用桂苓五味甘草湯。去桂加乾薑。細辛。以治其欬滿。

欬滿即止。而復更渴。衝气復发者。以細辛。乾薑为熱藥也。服之当遂渴。而渴反止者。为支飲也。支飲者。法当冒。冒者必嘔。嘔者復内半夏。以去其水。

水去。嘔止。其人形腫者。可内麻黄。以其欲逐痹。故不内麻黄。乃内杏仁也。若逆而内麻黄者。必厥。所以然者。以其人血虚。麻黄发其陽故也。(吳本)

*水去。嘔止。其人形腫者。加杏仁主之。其證應内麻黄。以其人遂痹。故不内之。若逆而内之者。必厥。所以然者。以其人血虚。麻黄发其陽故也。(鄧本)

若面熱如醉[狀者]。此为胃熱上衝熏其面。加大黄以利之。
	\footnote{「加大黄以利之」吴本作「加大黄湯和之」。}

先渴後嘔。为水停心下。此屬飲家。小半夏[加]茯苓湯主之。

\chapter{消渴小便[不]利淋}

厥陰之为病。消渴。气上衝心。心中疼熱。飢而不欲食。食即吐。下之不肯止。

寸口脉浮而遲。浮即为虚。遲即为勞。虚則衛气不足。勞則榮气竭。跌陽脉浮而數。浮即为气。數即消穀而大[便]堅。气盛則溲數。溲數即堅。堅數相摶。即为消渴。
	\footnote{「大便堅」吳本作「矢堅」,鄧本作「大堅」,「便」字为編者所加。}

男子消渴。小便反多。以飲一斗。小便一斗。腎气丸主之。

脉浮。小便不利。微熱。消渴者。宜利小便。发汗。五苓散主之。

渴欲飲水。水入則吐者。名曰水逆。五苓散主之。

渴欲飲水不止者。文蛤散主之。

淋之为病。小便如粟狀。小腹弦急。痛引脐中。

趺陽脉數。胃中有熱。即消穀引食。大便必堅。小便即數。

淋家不可发汗。发汗必便血。84

小便不利者。有水气。其人若渴。栝蔞瞿麦丸主之。
	\footnote{「若渴」徐本作「苦渴」。}

小便不利。蒲灰散主之。滑石白魚散。茯苓戎鹽湯并主之。

渴欲飲水。口乾舌燥者。白虎加人参湯主之。

脉浮。发熱。渴欲飲水。小便不利者。豬苓湯主之。

\chapter{水气}

師曰。病有風水。有皮水。有正水。有石水。有黄汗。風水。其脉自浮。外證骨節疼痛。惡風。皮水。其脉亦浮。外證胕腫。按之沒指。不惡風。其腹如鼓。不渴。当发其汗。正水。其脉沈遲。外證自喘。石水。其脉自沈。外證腹滿。不喘。黄汗。其脉沈遲。身发熱。胸滿。四肢頭面腫。久不愈。必致癰膿。

脉浮而洪。浮則为風。洪則为气。風气相擊。身体洪腫。汗出則愈。惡風則虚。此为風水。不惡風者。小便通利。上焦有寒。其口多涎。此为黄汗。(吳本)

*脉浮而洪。浮則为風。洪則为气。風气相摶。風强則为癮疹。身体为癢。癢为泄風。久为痂癩。气强則为水。難以俛仰。風气相擊。身体洪腫。汗出則愈。惡風則虚。此为風水。不惡風者。小便通利。上焦有寒。其口多涎。此为黄汗。(鄧本)

*脉浮而大。浮为風虚。大为气强。風气相摶。必成癮疹。身体为癢。癢者名泄風。久久为痂癩。(平脉法)

寸口脉沈滑者。中有水气。面目腫大。有熱。名曰風水。視人之目窠上微擁。如蠶新卧起狀。其頸脉動。时时欬。按其手足上。陷而不起者。風水。

太陽病。脉浮而緊。法当骨節疼痛。反不痛。身体反重而痠。其人不渴。汗出即愈。此为風水。惡寒者。此为極虚。发汗得之。渴而不惡寒者。此为皮水。身腫而冷。狀如周痹。胸中窒。不能食。反聚痛。暮躁不得眠。此为黄汗。痛在骨節。欬而喘。不渴者。此为脾胀。其狀如腫。发汗即愈。然諸病此者。渴而下利。小便數者。皆不可发汗。

裏水者。一身面目洪腫。其脉沈。小便不利。故令病水。假如小便自利。此亡津液。故令渴也。越婢加术湯主之。

趺陽脉当伏。今反緊。本自有寒。疝瘕。腹中痛。醫反下之。下之即胸滿短气。

趺陽脉当伏。今反數。本自有熱。消穀。小便數。今反不利。此欲作水。

寸口脉浮而遲。浮脉則熱。遲脉則潛。熱潛相摶。名曰沈。趺陽脉浮而數。浮脉即熱。數脉即止。熱止相摶。名曰伏。沈伏相摶。名曰水。沈則絡脉虚。伏則小便難。虚難相摶。水走皮膚。即为水矣。

寸口脉弦而緊。弦則衛气不行。[衛气不行]即惡寒。水不沾流。走於腸間。

少陰脉緊而沈。緊則为痛。沈則为水。小便即難。脉得諸沈。当責有水。身体腫重。水病脉出者死。

夫水病人。目下有卧蠶。面目鲜澤。脉伏。其人消渴。病水腹水。小便不利。其脉沈絶者。有水。可下之。

問曰。病下利後。渴飲水。小便不利。腹滿因腫者。何也。\\
答曰。此法当病水。若小便自利及汗出者。自当愈。

心水者。其身重而少气。不得卧。煩而躁。其人陰腫。

肝水者。其腹大。不能自轉則。脇下腹痛。时时津液微生。小便續通。

肺水者。其身腫。小便難。时时鴨溏。

脾水者。其腹大。四肢苦重。津液不生。但苦少气。小便難。

腎水者。其腹大。脐腫。腰痛。不得尿。陰下濕如牛鼻上汗。其足逆冷。面反瘦。

師曰。諸有水者。腰以下腫。当利小便。腰以上腫。当发汗乃愈。

師曰。寸口脉沈而遲。沈則为水。遲則为寒。寒水相摶。趺陽脉伏。水穀不化。脾气衰則鶩溏。胃气衰則身腫。少陽脉卑。少陰脉細。男子則小便不利。婦人則經水不通。經为血。血不利則为水。名曰血分。
	\footnote{「少陽脉卑」吳本无。}

問曰。病者苦水。面目身体四肢皆腫。小便不利。脉之不言水。反言胸中痛。气上衝咽。狀如炙肉。当微欬喘。審如師言。其脉何類。\\
師曰。寸口脉沈而緊。沈为水。緊为寒。沈緊相摶。結在関元。始时当微。年盛不覺。陽衰之後。榮衛相干。陽損陰盛。結寒微動。腎气上衝。喉咽塞噎。脇下急痛。醫以为留飲。而大下之。气擊不去。其病不除。後重吐之。胃家虚煩。咽燥欲飲水。小便不利。水穀不化。面目手足浮腫。又与葶藶丸下水。当时如小差。食飲過度。腫復如前。胸脇苦痛。象若奔豚。其水揚溢。則浮欬喘逆。当先攻擊衛气令止。乃治欬。欬止。其喘自差。先治新病。病当在後。

風水。脉浮。身重。汗出。惡風者。防己黄耆湯主之。腹痛加芍藥。

風水。惡風。一身悉腫。脉浮。不渴。續自汗出。无大熱。越婢湯主之。

皮水为病。四肢腫。水气在皮膚中。四肢聶聶動者。防己茯苓湯主之。

裏水。越婢加术湯主之。甘草麻黄湯亦主之。

水之为病。其脉沈小。屬少陰。浮者为風。无水。虚胀者为气。水。发其汗即已。脉沈者。宜麻黄附子湯。浮者。宜杏子湯。
	\footnote{按。杏子湯方書中未見,可能是大青龙湯。}

厥而皮水者。蒲灰散主之。

問曰。黄汗之为病。身体腫。发熱。汗出而渴。狀如風水。汗沾衣。色正黄如檗汁。脉自沈。何從得之。\\
師曰。以汗出入水中浴。水從汗孔入得之。
	\footnote{「身体腫」千金要方作「身体洪腫」。\\「汗出而渴」千金要方作「汗出不渴」。}

黄汗。黄耆芍藥桂枝苦酒湯主之。

黄汗之病。兩脛自冷。假令发熱。此屬歷節。食已汗出。又身常暮[卧]盜汗出者。此勞气也。若汗出已。反发熱者。久久其身必甲錯。发熱不止者。必生惡瘡。若身重。汗出已輒輕者。久久必身瞤。即胸中痛。又從腰以上必汗出。下无汗。腰髖弛痛。如有物在皮中狀。劇者不能食。身疼重。煩躁。小便不利。此为黄汗。桂枝加黄耆湯主之。

師曰。寸口脉遲而濇。遲則为寒。濇为血不足。趺陽脉微而遲。微則为气。遲則为寒。寒气不足。則手足逆冷。手足逆冷。則榮衛不利。榮衛不利。則腹滿脇鳴相逐。气轉膀胱。榮衛俱勞。陽气不通即身冷。陰气不通即骨痛。陽前通則惡寒。陰前通則痹不仁。陰陽相得。其气乃行。大气一轉。其气乃散。実則失气。虚則遺尿。名曰气分。

气分。心下堅。大如盤。邊如旋杯。水飲所作。桂枝去芍藥加麻[黄細]辛附子湯主之。

心下堅。大如盤。邊如旋盤。水飲所作。枳[実]术湯主之。

附方

夫風水。脉浮为在表。其人或頭汗出。表无他病。病者但下重。故知從腰以上为和。腰以下当腫及陰。難以屈伸。防己黄耆湯主之。

\chapter{黄疸}

論曰。黄有五種。有黄汗。黄疸。穀疸。酒疸。女勞疸。黄汗者。身体四肢微腫。胸滿。不渴。汗出如黄蘗汁。良由大汗出卒入水中所致。黄疸者。一身面目悉黄如橘。由暴得熱。以冷水洗之。熱因留胃中。食生黄瓜熏上所致。若成黑疸者多死。穀疸者。食畢頭眩。心忪怫鬱不安而发黄。由失飢大食。胃气衝熏所致。酒疸者。心中懊痛。足脛滿。小便黄。面发赤斑黄黑。由大醉當風入水所致。女勞疸者。身目皆黄。发熱。惡寒。小腹滿急。小便難。由大勞大熱而交接竟入水所致。但依後方治之。
	\footnote{此條金匱要略无,從千金要方補入。}

寸口脉浮而緩。浮則为風。緩則为痹。痹非中風。四肢苦煩。脾色必黄。瘀熱以行。

趺陽脉緊而數。數則为熱。熱則消穀。緊則为寒。食即为滿。尺脉浮为傷腎。趺陽脉緊为傷脾。風寒相摶。食穀即眩。穀气不消。胃中苦濁。濁气下流。小便不通。陰被其寒。熱流膀胱。身体尽黄。名曰穀疸。額上黑。微汗出。手足中熱。薄暮即发。膀胱急。小便自利。名曰女勞疸。腹如水狀。不治。心中懊憹而熱。不能食。时欲吐。名曰酒疸。

陽明病。脉遲者。食難用飽。飽則发煩。頭眩。必小便難。此欲作穀疸。雖下之。腹滿如故。所以然者。脉遲故也。

夫病酒黄疸。必小便不利。其候心中熱。足下熱。是其證也。

酒黄疸者。或无熱。靖言了[了]。腹滿欲吐。鼻燥。其脉浮者。先吐之。沈弦者。先下之。

酒疸。心中熱。欲嘔者。吐之愈。

酒疸下之。久久为黑疸。目青。面黑。心中如噉蒜虀狀。大便正黑。皮膚爪之不仁。其脉浮弱。雖黑。微黄。故知之。

師曰。病黄疸。发熱。煩喘。胸滿。口燥者。以病发时火劫其汗。兩熱所得。然黄家所得。從濕得之。一身尽发熱而黄。肚熱。熱在裏。当下之。

脉沈。渴欲飲水。小便不利者。皆发黄。
	\footnote{「脉沈」吳本作「脉浮」。}

腹滿。舌痿黄燥。不得睡。屬黄家。

黄疸之病。当以十八日为期。治之十日以上差。反劇。为難治。

疸而渴者。其疸難治。疸而不渴者。其疸可治。发於陰部。其人必嘔。[发於]陽部。其人振寒而发熱也。

穀疸之为病。寒熱不食。食即頭眩。心胸不安。久久发黄。为穀疸。茵陳蒿湯主之。

黄家。日晡所发熱。而反惡寒。此为女勞得之。膀胱急。少腹滿。身尽黄。額上黑。足下熱。因作黑疸。其腹胀如水狀。大便必黑。时溏。此女勞之病。非水也。腹滿者難治。硝石礬石散主之。

酒黄疸。心中懊憹。或熱痛。梔子[枳実豉]大黄湯主之。

諸病黄家。但利其小便。假令脉浮。当以汗觧之。宜桂枝加黄耆湯主之。

諸黄。豬膏髮煎主之。

黄疸病。茵陳五苓散主之。

黄疸。腹滿。小便不利而赤。自汗出。此为表和裏実。当下之。宜大黄[黄蘗梔子]硝石湯。

黄疸病。小便色不變。欲自利。腹滿而喘。不可除熱。熱除必噦。噦者。小半夏湯主之。

諸黄。腹痛而嘔者。宜柴胡湯。

男子黄。小便自利。当与虚勞小建中湯。

附方

諸黄。瓜蒂湯主之。

黄疸。麻黄淳酒湯主之。

\chapter{驚悸吐衄下血胸滿瘀血}

寸口脉動而弱。動即为驚。弱即为悸。

師曰。尺脉浮。目睛暈黄。衄未止。暈黄去。目睛急了。知衄今止。
又曰。從春至夏衄者。太陽。從秋至冬衄者。陽明。

衄家不可发汗。汗出必額上陷。脉緊急。直視不能眴。不得眠。

病人面无色。无寒熱。脉沈弦者。衄。[脉]浮弱。手按之絶者。下血。煩欬者。必吐血。

夫吐血。欬逆上气。其脉數而有熱。不得卧者死。

夫酒客。欬者。必致吐血。此因極飲過度所致也。

寸口脉弦而大。弦則为減。大則为芤。減則为寒。芤則为虚。寒虚相擊。此名曰革。婦人則半產漏下。男子則亡血。

亡血不可攻其表。汗出則寒慄而振。

病人胸滿。唇痿舌青。口燥。但欲嗽水。不欲咽。无寒熱。脉微大來遲。腹不滿。其人言我滿。为有瘀血。

病者如熱狀。煩滿。口乾燥而渴。其脉反无熱。此为陰狀。是瘀血也。当下之。

火邪者。桂枝去芍藥加蜀漆牡蛎龙骨救逆湯主之。

心下悸者。半夏麻黄丸主之。

吐血不止者。蘗葉湯主之。

下血。先便後血。此遠血也。黄土湯主之。

下血。先血後便。此近血也。赤小豆当歸散主之。

心气不足。吐血。衄血。瀉心湯主之。

\chapter{嘔吐噦下利}

夫嘔家有癰膿。不可治嘔。膿尽自愈。

先嘔卻渴者。此为欲觧。先渴卻嘔者。为水停心下。此屬飲家。

嘔家本渴。今反不渴者。以心下有支飲故也。此屬支飲。

問曰。病人脉數。數为熱。当消數引食。而反吐者。何也。\\
師曰。以发其汗。令陽微。膈气虚。脉乃數。數为客熱。不能消穀。胃中虚冷故也。脉弦者虚也。胃气无餘。朝食暮吐。變为胃反。寒在於上。醫反下之。令脉反弦。故名曰虚。
	\footnote{「胃中虚冷故也」吳本作「胃中虚冷故吐也」。}

寸口脉微而數。微則无气。无气則榮虚。榮虚則血不足。血不足則胸中冷。

趺陽脉浮而濇。浮則为虚。濇則傷脾。脾傷則不磨。朝食暮吐。暮食朝吐。宿穀不化。名曰胃反。脉緊而濇。其病難治。

病人欲吐者。不可下之。

噦而腹滿。視其前後。知何部不利。利之即愈。

嘔而胸滿者。茱萸湯主之。

乾嘔。吐涎沫。頭痛者。吳茱萸湯主之。
	\footnote{「吳茱萸湯」鄧本作「茱萸湯」。}

嘔而腸鳴。心下痞者。半夏瀉心湯主之。

乾嘔而利者。黄芩加半夏生薑湯主之。

諸嘔吐。穀不得下者。小半夏湯主之。

嘔吐而病在膈上。後思水者觧。急与之。思水者。豬苓散主之。

嘔而脉弱。小便復利。身有微熱。見厥者難治。四逆湯主之。

嘔而发熱者。小柴胡湯主之。

胃反嘔吐者。大半夏湯主之。

食已即吐者。大黄甘草湯主之。

胃反。吐而渴欲飲水者。茯苓澤瀉湯主之。

吐後。渴欲得水而貪飲者。文蛤湯主之。兼主微風。脉緊。頭痛。

乾嘔。吐逆。吐涎沫。半夏乾薑散主之。

病人胸中似喘不喘。似嘔不嘔。似噦不噦。徹心中憒憒然无奈。生薑半夏湯主之。

乾嘔。噦。若手足厥者。橘皮湯主之。

噦逆者。橘皮竹茹湯主之。

夫六腑气絶於外者。手足寒。上气。腳縮。五臓气絶於内者。利不禁。下甚者。手足不仁。

下利。脉沈弦者。下重。脉大者。为未止。脉微弱數者。为欲自止。雖发熱。不死。

下利。手足厥[冷]。无脉。[当灸其厥陰。]灸之不温[而脉不還]。反微喘者死。少陰負趺陽者为順。362

*下利。手足厥冷。无脉者。灸之不温。若脉不還。反微喘者死。少陰負趺陽者。为順也。(金匱)362

下利。有微熱而渴。脉弱者。今自愈。

下利。脉數。有微熱。汗出。今自愈。設脉緊。为未觧。

下利。脉數而渴者。今自愈。設不差。必清膿血。以有熱故也。

下利。脉反弦。发熱。身汗者。自愈。

下利气者。当利其小便。

下利。寸脉反浮數。尺中自濇者。必清膿血。

下利清穀。不可攻其表。汗出必胀滿。

下利。脉沈而遲。其人面少赤。身有微熱。下利清穀者。必鬱冒。汗出而觧。病人必微熱。所以然者。其面戴陽。下虚故也。
	\footnote{「必微熱」吳本作「必微厥」。}

下利後。脉絶。手足厥[冷]。晬时脉還。手足温者生。不還[不温]者死。368

*下利後。脉絶。手足厥冷。晬时脉還。手足温者生。脉不還者死。(金匱)

下利。腹胀滿。身体疼痛者。先温其裏。乃攻其表。温裏宜四逆湯。攻表宜桂枝湯。372

下利。三部脉皆平。按之心下堅者。急下之。宜大承湯。

下利。脉遲而滑者。実也。利未欲止。急下之。宜大承气湯。

下利。脉反滑者。当有所去。下乃愈。宜大承气湯。

下利已差。至其[年月日]时復发者。此为病不尽。当復下之。宜[大]承气湯。

*下利已差。至其年月日时復发者。以病不尽故也。当下之。宜大承气湯。(金匱)

下利。譫語者。有燥屎也。小承气湯主之。

下利。便膿血者。桃花湯主之。

熱利重下者。白頭翁湯主之。

下利後更煩。按之心下濡者。为虚煩也。梔子豉湯主之。

下利清穀。裏寒外熱。汗出而厥。通脉四逆湯主之。370

下利。肺痛。紫参湯主之。

气利。诃梨勒散主之。

附方

小承气湯。治大便不通。噦。數譫語。

乾嘔。下利。黄芩湯主之。

\chapter{瘡癰腸癰浸淫}

諸浮數脉。應当发熱。而反洒淅惡寒。若有痛処。当发其癰。

師曰。諸癰腫。欲知有膿无膿。以手掩腫上。熱者为有膿。不熱者为无膿。

腸癰之为病。其身甲錯。腹皮急。按之濡。如腫狀。腹无積聚。身无熱。脉數。此为腸内有癰膿。薏苡附子敗醬散主之。

腸癰者。少腹腫痞。按之即痛如淋。小便自調。时时发熱。自汗出。復惡寒。脉遲緊者。膿未成。可下之。当有血。脉洪數者。膿已成。不可下也。大黄牡丹湯主之。

問曰。寸口脉浮微而濇。法当亡血。若汗出。設不汗者云何。\\
答曰。若身有瘡。被刀斧所傷。亡血故也。

病金瘡。王不留行散主之。

浸淫瘡。從口流向四肢者。可治。從四肢流來入口者。不可治。黄連粉主之。
	\footnote{「黄連粉主之」鄧本作「浸淫瘡黄連粉主之」且單獨列为一條。}

\chapter{趺蹶手指臂腫轉筋陰狐疝蛔虫}

師曰。病趺蹶。其人但能前。不能卻。刺腨入二寸。此太陽經傷也。

病人常以手指臂腫動。此人身体瞤瞤者。藜蘆甘草湯主之。
	\footnote{「臂腫動」吳本作「臂脛動」。}

轉筋之为病。其人臂腳直。脉上下行。微弦。轉筋入腹者。雞屎白散主之。

陰狐疝气者。偏有大小。时时上下。蜘蛛散主之。

問曰。病腹痛有虫。其脉何以别之。\\
師曰。腹中痛。其脉当沈若弦。反洪大。故有蛔虫。

蛔虫之为病。令人吐涎。心痛。发作有时。毒藥不止。甘草粉蜜湯主之。

蛔厥者。其人当吐蛔。今病者靜。而復时煩。此为臓寒。蛔上入膈。故煩。須臾復止。得食而嘔。又煩者。蛔聞食臭出。其人当自吐蛔。蛔厥者。烏梅丸主之。
	\footnote{「今病者靜」吳本、鄧本均作「令病者靜」。}

\chapter{婦人妊娠病}

師曰。婦人得平脉。陰脉小弱。其人渴。不能食。无寒熱。名妊娠。桂枝湯主之。於法六十日当有此證。設有醫治逆者。卻一月。加吐下者。則絶之。

*師曰。脉婦人得平脉。陰脉小弱。其人渴。不能食。无寒熱。名为軀。桂枝湯主之。法六十日当有娠。設有醫治逆者。卻一月。加吐下者。則絶之。(吳本)

婦人妊娠。經斷三月而得漏下。下血四十日不止。胎欲動。在於脐上。此为妊娠。六月動者。前三月經水利时。胎也。下血者。後斷三月。衃也。所以下血不止者。其癥不去故也。当下其癥。宜桂枝茯苓丸。(吳本)

*婦人宿有癥病。經斷未及三月。而得漏下不止。胎動在脐上者。为癥痼害。妊娠六月動者。前三月經水利时。胎下血者。後斷三月。衃也。所以血不止者。其癥不去故也。当下其癥。桂枝茯苓丸主之。(鄧本)

婦人懷娠六七月。脉弦。发熱。其胎愈胀。腹痛。惡寒者。少腹如扇。所以然者。子臓開故也。当以附子湯温其臓。

師曰。婦人有漏下者。有半產後因續下血都不絶者。有妊娠下血者。假令妊娠腹中痛。为胞阻。膠艾湯主之。

婦人懷娠。腹中㽲痛。当歸芍藥散主之。

妊娠。嘔吐不止。乾薑人参半夏丸主之。

妊娠。小便難。飲食如故。当歸貝母苦参丸主之。

妊娠。有水气。身重。小便不利。洒淅惡寒。起即頭眩。葵子茯苓散主之。

婦人妊娠。宜常服当歸散。

附方

妊娠養胎。白术散主之。

婦人傷胎。懷身。腹滿。不得小便。從腰以下重。如有水气狀。懷身七月。太陰当養不養。此心气実。当刺瀉勞宮及関元。小便[微]利則愈。
	\footnote{吳本「傷胎」作「傷寒」,「小便微利」作「小便利」。}

\chapter{婦人產後病}

問曰。新產婦人有三病。一者病痙。二者病鬱冒。三者大便難。何谓也。\\
師曰。新產血虚。多汗出。喜中風。故令病痙。亡血復汗。寒多。故令鬱冒。亡津液。胃燥。故大便難。產婦鬱冒。其脉微弱。不能食。大便反堅。但頭汗出。所以然者。血虚而厥。厥而必冒。冒家欲觧。必大汗出。以血虚下厥。孤陽上出。故頭汗出。所以產婦喜汗出者。亡陰血虚。陽气獨盛。故当汗出。陰陽乃復。大便堅。嘔不能食。小柴胡湯主之。病觧。能食。七八日更发熱者。此为胃実。大承气湯主之。
	\footnote{「胃実」吳本作「胃熱气実」。\\「病觧能食」以下,金匱獨作一條。}

產後腹中㽲痛。当歸生薑羊肉湯主之。并治腹中寒疝。虚勞不足。

產後腹痛。煩滿不得卧。枳実芍藥散主之。

師曰。產婦腹痛。法当与枳実芍藥散。假令不愈者。此为腹中有乾血著脐下。宜下瘀血湯主之。

產後七八日。无太陽證。少腹堅痛。此惡露不尽。不大便。煩躁发熱。切脉微実。再倍发熱。日晡时煩躁者。不食。食則譫語。至夜即愈。宜大承气湯主之。熱在裏。結在膀胱也。(鄧本)

*婦人產後七八日。无太陽證。少腹堅痛。此惡露不尽。不大便四五日。趺陽脉微実。再倍其人发熱。日晡所煩躁者。不食。食即譫語。利之即愈。宜大承气湯。熱在裏。結在膀胱也。(吳本)

產後風。續之數十日不觧。頭微痛。惡寒。时时有熱。心下悶。乾嘔。汗出。雖久。陽旦證續在耳。可与陽旦湯。

產後中風。发熱。面正赤。喘而頭痛。竹葉湯主之。

婦人乳中虚。煩亂。嘔逆。安中益气。竹皮大丸主之。

產後下利。虚極。白頭翁加甘草阿膠湯主之。

附方

婦人在草蓐得風。四肢苦煩熱。皆自发露所为。頭痛者。与小柴胡湯。頭不痛。但煩者。与三物黄芩湯。

内補当歸建中湯。治婦人產後虚羸不足。腹中刺痛不止。吸吸少气。或苦少腹拘急。攣痛引腰背。不能食飲。產後一月。日得服四五剂为善。令人强壯。

\chapter{婦人雜病}

婦人中風七八日。續得寒熱。发作有时。經水適斷。此为熱入血室。其血必結。故使如瘧狀。发作有时。小柴胡湯主之。144

婦人傷寒。发熱。經水適來。晝日明了。暮則譫語。如見鬼狀。此为熱入血室。无犯胃气及上二焦。必自愈。145

婦人中風。发熱。惡寒。經水適來。得之七八日。熱除。脉遲。身涼。胸脇下滿。如結胸狀。其人譫語。此为熱入血室。当刺期門。隨其[虚]実而取之。143

陽明病。下血。譫語者。此为熱入血室。但頭汗出。当刺期門。隨其実而瀉之。濈然汗出者愈。

婦人咽中如有炙臠。半夏厚朴湯主之。

婦人臓躁。喜悲傷。欲哭。象如神靈所作。數欠伸。甘[草小]麦大棗湯主之。

婦人吐涎沫。醫反下之。心下即痞。当先治其吐涎沫。宜小青龙湯。涎沫止。乃治痞。宜瀉心湯。

婦人之病。因虚積冷結气。为諸經水斷絶。至有歷年。血寒積結。胞門寒傷。經絡凝堅。在上嘔吐涎唾。久成肺癰。形体損分。在中盤結。繞脐寒疝。或兩脇疼痛。与臓相連。或結熱[在]中。痛在関元。脉數无瘡。肌若魚鱗。时著男子。非止女身。在下未多。經候不勻。令陰掣痛。少腹惡寒。或引腰脊。下根气街。气衝急痛。膝脛疼煩。奄忽眩冒。狀如厥癲。或有憂惨。悲傷多嗔。此皆帶下。非有鬼神。久則羸瘦。脉虚多寒。三十六病。千變万端。審脉陰陽。虚実緊弦。行其針藥。治危得安。其雖同病。脉各異源。子当辯記。勿谓不然。

問曰。婦人年五十所。病下血。數十日不止。暮即发熱。少腹裏急。腹滿。手掌煩熱。唇口乾燥。何也。\\
師曰。此病屬帶下。何以故。曾經半產。瘀血在少腹不去。何以知之。其證唇口乾燥。故知之。当以温經湯主之。
	\footnote{「病下血」諸本均作「病下利」,为編者所改。\\吳本「少腹裏急」四字下有「痛」字。}

[婦人]帶下。經水不利。少腹滿痛。經一月再見者。土瓜根散主之。

寸口脉弦而大。弦則为減。大則为芤。減則为寒。芤則为虚。寒虚相摶。此名曰革。婦人則半產漏下。旋復花湯主之。

婦人陷經。漏下黑不觧。膠薑湯主之。

婦人少腹滿如敦狀。小便微難而不渴。生後者。此为水与血并結在血室也。大黄甘遂湯主之。

婦人經水不利。抵当湯主之。
	\footnote{鄧本「經水不利」四字下有「下」字。}

婦人經水閉不利。臓堅癖不止。中有乾血。下白物。礬石丸主之。

婦人六十二種風。及腹中血气刺痛。紅藍花酒主之。

婦人腹中諸疾痛。当歸芍藥散主之。

婦人腹中痛。小建中湯主之。

問曰。婦人病。飲食如故。煩熱不得卧。而反倚息者。何也。\\
師曰。此名轉胞。不得尿也。以胞系了戾。故致此病。但利小便則愈。宜腎气丸。
	\footnote{「宜腎气丸」後鄧本有「主之」二字,吳本有「以中有茯苓故也」七字。}

蛇床子散。温陰中坐藥。

少陰脉滑而數者。陰中即生瘡。[婦人]陰中蝕瘡爛者。狼牙湯洗之。

胃气下泄。陰吹而正喧。此穀气之実也。膏髮煎導之。

\end{document}

%字形
%
%実为与洒虚术体气処无当脉沈温别卧麦盖内弃泪时尽觧发强
%覺學舉
%納約結細緣緩縮純紅絶絞縱經絡續綱終繞總綠紙
%證許譫語設諸診訴謝辯訣記談諺謬誠靄諦訓認譚識説調論議誤
%齐脐剂
%腫種
%針鑠鎮鐘銓錯銖鋸
%門瞤潤悶闔聞問閉開関間闕癇爛
%輿輗輒暈渾漸連軺載暫陳運轉輕輩
%尔弥
%飲飢飪飽餘蝕餅饐
%龙聋
%万厉蛎
%帶滯
%黄横
%长胀张
%参惨
